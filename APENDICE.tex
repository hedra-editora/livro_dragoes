
\part[Sobre as histórias e os autores]{Sobre as histórias\\ e os autores}

\chapter*{\ }
\markboth{sobre as histórias e os autores}{marcos maffei}

\section{“Perseu e Andrômeda”}

{\centering
Ovídio (43 a.C.--17 d.C.) --- \textit{Metamorphoses},\\ Livro \textsc{iv}, 663-764 (2--8 d.C.)
\par\smallskip}

Públio Ovídio Naso nasceu em Sulmo, próximo a Roma. Seu pai, que queria
que ele seguisse uma carreira política, enviou"-o a Roma para estudar
com os melhores mestres de retórica, mas Ovídio, seguindo sua inclinação “natural”,  tornou"-se poeta. 
Sua primeira obra publicada foi \textit{Amores}, uma série de poemas curtos. 
Em seguida escreveu \textit{Heroínas}, uma série
de cartas de mulheres célebres da mitologia, como Penélope,
Ariadne e Helena, a seus amados ausentes. Depois, escreve \textit{Arte
do amor}, um manual de sedução e um retrato da Roma da época, e
\textit{Remédios do amor}, em que fingia se retratar do que
dissera na obra anterior. Escreveu ainda uma tragédia, \textit{Medeia}, que se
perdeu; mas a obra"-prima que lhe traria fama imortal (como aliás ele
mesmo profetiza no fim do livro) seria \textit{Metamorfoses}. Um enorme poema
dividido em quinze partes e quase doze mil versos, em que reconta uma infinidade
de histórias, sobretudo da mitologia grega, tendo como elemento
unificador a transformação: do caos em harmonia, de animais em
pedras, de homens e mulheres em animais ou estrelas. Seu talento 
poético e narrativo fez desse livro uma das versões
mais fascinantes e lidas dessas histórias, mesmo que a ideia de
metamorfose em muitos momentos sirva só como um pretexto para que ele
vá emendando os episódios que quer contar uns nos outros e tenha de
recorrer a uma variedade de artifícios, dos mais elegantes aos mais
artificiais, para justificar sua inclusão --- um deles é o de se demorar
num detalhe irrelevante para o que está sendo contado, só por conter
alguma metamorfose, como é o caso do coral no trecho aqui apresentado,
embora seja justo admitir que Perseu estava voltando de uma aventura
que, de fato, envolvia portentosas transformações: dos cabelos da
Medusa em cobras e de quem a via em pedra. No ano 8, quando esse
poema estava quase pronto e \textit{Fastos}, em que descrevia os
festivais romanos de cada mês e os mitos que os fundavam, pela
metade, Ovídio foi condenado ao exílio de Roma pelo imperador Augusto. Não se conhece
exatamente a razão da desgraça do poeta, que passaria no exílio a maior parte dos
últimos nove anos de sua vida. Mas
sua influência, e em especial das \textit{Metamorfoses}, seria enorme em toda
a literatura ocidental; do século \textsc{xii} ao \textsc{xvii} não há quase nenhum
grande autor que a ele não se refira ou nele não tenha se inspirado. 

\chapter*{\ }

\section[“Stan Bolovan” (de Rumanischen Märchen)]{\vspace*{-.4em}“Stan Bolovan”\break (de Rumanischen Märchen)}
\enlargethispage{.8em}

{\centering
Andrew Lang (1844-1912) --- \textit{The Crimson Fairy Book}\\ (O livro carmim dos contos de fadas) (1903); 
\textit{The Violet Fairy Book} (O livro lilás dos contos de fadas) (1901)
\par\smallskip}

O escritor, crítico, antropólogo e folclorista Andrew Lang nasceu na
Escócia e, quando criança, além de ouvir avidamente todas as
histórias e lendas da região, “lia tudo quanto era conto de fada que
encontrava; conhecia bem todas as fadas de \textit{Sonho de uma noite de
verão}, todos os fantasmas de Walter Scott e detestava máquinas de
qualquer espécie”. Odiou ter que aprender grego na escola, até
descobrir Homero, que seria uma de suas paixões pelo resto da vida:
traduções da \textit{Odisseia} e da \textit{Ilíada} estão entre seus primeiros
trabalhos publicados e, em 1890, colaboraria com H.~Rider Haggard (o
autor das \textit{Minas do rei Salomão}) em um romance sobre as últimas
aventuras de Odisseu, em busca de Helena depois da morte de Penélope
e Telêmaco. Em vez de uma carreira acadêmica em Oxford, optou por
escrever para \mbox{jornais} e revistas em Londres, tornando"-se um influente
crítico literário. O interesse que sempre teve por lendas e contos do
folclore o direcionou para a antropologia, em que foi um dos
pioneiros nos estudos de mitologia comparada, publicando duas obras
importantes sobre o assunto, \textit{Custom and Myth} (Costume e mito, 1884) e \textit{Myth, Ritual
and Religion} (Mito, ritual e religião, 1887). E resultou também na série de livros pela qual
ficou mais conhecido, as doze coletâneas de contos de fada que
iniciaria em 1889 com \textit{The Blue Fairy Book} (O livro azul dos contos de fadas). 
Com 37 histórias, o livro incluía os contos mais famosos de Perrault, Aulnoy, dos Grimm e das
\textit{Mil e uma noites}, mas também outros menos conhecidos do folclore inglês,
escocês e escandinavo. Não havia originalmente a intenção de fazer
uma série, mas o sucesso foi tão grande que em 1890 aparecia \textit{The Red
Fairy Book} (O livro vermelho dos contos de fadas), com mais histórias das mesmas fontes e também do folclore
russo, e em 1892 \textit{The Green Fairy Book} (O livro verde dos contos de fadas). 
Seriam mais nove, terminando em 1910 com \textit{The Lilac Fairy Book} (O livro
lilás dos contos de fadas), e além dos autores conhecidos e do folclore
europeu, acabariam incluindo histórias tradicionais de todos
os continentes. Lang atuava principalmente como editor dos livros,
selecionando as histórias e encomendando as traduções ou adaptações
para outros, em especial sua mulher Leonore, que fez a maior parte delas.
Lang escreveu e publicou também alguns contos de fadas próprios, e
editou mais treze coletâneas de histórias: de aventura, de animais,
de heróis, de fantasmas e de mistério, a partir da História
propriamente dita, da mitologia grega e das \textit{Mil e uma noites}; a última
delas foi \textit{The Strange Story Book} (O livro das histórias estranhas), 
que sairia um ano após sua morte. 

\chapter*{\ }

\section{“São Jorge e o dragão”}

{\centering
Jacobus de Voragine (1228/9--1298) --- \textit{Legenda Aurea} (Lenda dourada) (c. 1253--70)
\par\smallskip}

Com o cristianismo, matar dragões tornou"-se um serviço com frequência
atribuído a santos; pelo menos uma dúzia deles, em algum
momento de sua vida, teria enfrentado e vencido tais encarnações do
mal. O mais popular é certamente São Jorge; seu combate com o dragão
disseminou"-se em lendas por toda a Europa e foi tema de inúmeros
pintores desde a Idade Média. Do verdadeiro Jorge, pouco se sabe:
teria sido um mártir cristão no Oriente Médio e sua suposta
sepultura se encontra em Lydda, na Palestina. Do século \textsc{vi} em diante,
lendas dos feitos heroicos desse santo guerreiro começaram a se
espalhar pela Europa, ganhando especial ímpeto com as Cruzadas, das
quais era tido como um inspirador; em torno dos séculos \textsc{xiii} e \textsc{xiv},
não se sabe exatamente por quê, tornou"-se o santo padroeiro da
Inglaterra. Sua sepultura fica perto do local onde teria ocorrido o
mitológico combate entre Perseu e o monstro marinho; supõe"-se que
dessa proximidade teria surgido a lenda de São Jorge e o dragão, já
que há similaridades entre as duas histórias. Essa e outras lendas
sobre ele encontram"-se na \textit{Legenda Aurea}, escrita em latim entre 1253
e 1270 pelo arcebispo de Gênova, Jacobus de Voragine (ou Jacopo de
Varazze), uma coletânea de narrativas das vidas de 175 santos, de
relatos de eventos das vidas de Cristo e da Virgem Maria, e de
informações sobre as datas do calendário litúrgico. Imensamente
popular na Idade Média, a obra foi traduzida (e gradualmente
ampliada) para todas as línguas europeias. Foi um dos primeiros livros
a serem publicados em inglês, por William Caxton, em 1483 (\textit{The Golden
Legend}; essa tradução inglesa de Caxton foi a utilizada aqui). Entre
outras histórias famosas de santos e dragões, Jacopus conta também a
de Santa Margarida, padroeira dos partos. Aprisionada por se recusar
a casar com o governador de Antioquia, em sua cela aparece um dragão,
que ela domina com o sinal da cruz. Em outra versão que Jacopus
descarta como apócrifa, ela é engolida pelo dragão e o poder da cruz
faz com que ela saia da barriga dele. Quanto a São Jorge, outro livro
que popularizou sua lenda foi \textit{The Seven Champions of Christendom} (Os
sete campeões da cristandade, 1596--97, de Richard Johnson), 
mencionado na história de Edith Nesbit “Os salvadores da pátria”. 

\chapter*{\ }

\section{“A história de Sigurd”} 

{\centering
Anônimo islandês do século \textsc{xiii} --- \textit{Saga dos volsungos}
\par\smallskip}

Saga, em islandês, significa simplesmente história em prosa e várias
foram escritas no chamado período clássico da literatura islandesa,
os séculos \textsc{xii} e \textsc{xiii}, tratando da vida e dos feitos de reis e
famílias que de fato existiram, com variável fidelidade histórica. Ou
então, de heróis lendários, de procedência tanto escandinava quanto
germânica, as \textit{Fornaldar Sögur} (sagas, ou histórias, da Antiguidade). A
\textit{Saga dos volsungos} é a mais conhecida
delas. Escrita por um autor anônimo na segunda metade do século \textsc{xii},
conta lendas sobre heróis que aparecem também na \textit{Edda
Poética}, o manuscrito da mesma época que é uma compilação da \mbox{poesia}
islandesa. Começa com os ancestrais de Sigurd, relata a
história deste e de Brynhild, e então o destino de sua mulher Gudrun
depois de sua morte. Também da mesma época é a versão germânica
destas lendas, o \textit{Nibelungenlied} (Canção dos nibelungos), mais centrado na figura de Siegfried
(Sigurd) e já mais próximo do romance de cavalaria medieval cristão,
deixando de lado muitos de seus elementos pagãos e fantásticos. Foi
utilizada aqui a adaptação de Andrew Lang (uma das poucas que fez
pessoalmente) da tradução em inglês de William
Morris e Eirikr Magnusson para \textit{O livro vermelho dos 
contos de fadas}, publicado em 1890.

\chapter*{\ }

\section[“Os salvadores da pátria”; “O último dragão”]{“Os salvadores da pátria”;\break “O último dragão”} 

{\centering
Edith Nesbit (1858-1924) --- \textit{The Book of Dragons}\\ (O livro dos dragões) (1900)
\par\smallskip}

Edith Nesbit passou os primeiros nove anos de sua vida no Colégio
Agrícola de Kennington (então um subúrbio rural de Londres), do
qual seu pai era diretor, cargo assumido por sua mãe depois da
morte dele, quando Edith tinha três anos. Em 1867, sua mãe decidiu ir
para a França, em busca de um clima mais apropriado para a saúde de sua filha
mais velha, Mary. Seguiu"-se um período nômade, que incluiu escolas
internas que Edith detestou (de uma delas teria tentado fugir), mas
também uma casa enorme e deliciosa na Bretanha. Foi para morar em uma casa
parecida no campo que a família voltou para a Inglaterra, quando ela
tinha treze anos. As explorações e aventuras dela e de seus irmãos
nos arredores de ambas inspirariam mais tarde as das crianças de seus
livros. Três anos depois, quando sua mãe não tinha mais condições de manter
a casa, mudaram"-se para Londres. Edith publicou um poema e logo
começou a produção contínua de ficção popular para revistas, o que
garantiria também por muito tempo o sustento de sua família após seu
casamento em 1880. Seu marido, Hubert Bland, depois de uma
malsucedida tentativa de ter um negócio próprio, devotaria todas
as suas energias à Fabian Society, sociedade socialista que fundaram 
e de que fizeram parte George Bernard
Shaw e H.G.~Wells, entre outros. A casa de Nesbit tornou"-se um
animado ponto de encontro de intelectuais. Edith ignorava as convenções 
comportamentais de sua época: fumava, recusava"-se a seguir a desconfortável moda feminina e
usava o cabelo sempre curto. Produziu todo
tipo de ficção para garantir a sobrevivência, mas foi só em 1898 que
experimentou escrever para crianças. Com \textit{The Story of the Treasure
Seekers} (História dos caçadores de tesouro), 
as aventuras de cinco irmãos tentando restaurar a fortuna da
família Bastable, convincentemente narrada por um deles, Nesbit
iniciaria sua obra verdadeiramente original, que a faria famosa e a
tornaria uma das mais importantes e influentes autoras inglesas de
literatura para crianças. Uma série de histórias de dragões — entre
elas as presentes neste livro —  seriam reunidas em \textit{O livro dos dragões},
publicado em 1900; no ano seguinte sairia \textit{The Wouldbegoods},
continuação das aventuras da família Bastable; e em 1902 \textit{Five
Children and it} (Cinco crianças e aquilo), o livro que consolidaria esse gênero tão
reproduzido na literatura infantil, em que o cotidiano contemporâneo das
crianças se combina com elementos fantásticos (no caso, cinco irmãos
de férias numa casa de campo, e Psammead, um duende da areia que
realiza desejos). Com ele, \textit{The Phoenix and the Carpet} (A fênix e o tapete, 1904) e \textit{The
Story of the Amulet} (A história do amuleto, 1905), um dos primeiros livros infantis a
incluírem viagens no tempo, formariam uma trilogia; e além de um
terceiro com os Bastable, Nesbit escreveu mais sete livros, e
publicou também algumas coletâneas de histórias curtas. 

\chapter*{\ }

\section{“O dragão relutante”} 

{\centering
Kenneth Grahame (1859-1932) --- \textit{Dream Days}\\ (Dias de sonho) (1898)
\par\smallskip}

Kenneth Grahame nasceu em Edimburgo, na Escócia; antes de fazer cinco
anos, sua mãe morreu e ele e seus irmãos ficaram aos cuidados de uma
avó fria e distante e de um pai incompetente, que abandonaria
definitivamente a família em 1867. Sem condições financeiras para ir
à universidade de Oxford (seu sonho era tornar"-se escritor e
seguir uma carreira acadêmica), em 1879 começou a trabalhar no Banco
da Inglaterra, onde ficou até 1908.
Manteve, todavia, inclinações e amizades literárias e, em
1887, começou a publicar em revistas. Em 1893, publicou uma
coletânea de seus ensaios com o título \textit{Pagan Papers} (Papéis pagãos), e dois anos
depois \textit{The Golden Age} (A era dourada), uma narrativa em primeira pessoa sobre a
infância de cinco irmãos numa casa de campo, que fez um enorme
sucesso. Apesar de ter sido escrito para adultos, o livro capta com
precisão o universo e as preocupações das crianças e se tornaria uma influência significativa
na literatura infantil da época. Em 1898, Grahame publicou uma
continuação, \textit{Dias de sonho}; nela, aparece a história “O dragão
relutante”, contada para as cinco crianças. Mesmo com o enorme sucesso 
dos dois livros, ele só voltaria a escrever oito anos
depois, a partir das histórias que começara a contar para seu filho,
que acabariam se tornando \textit{The Wind in the Willows} (O vento nos
salgueiros), um clássico da literatura infantil inglesa. Publicado em
1908, este foi seu último livro.

