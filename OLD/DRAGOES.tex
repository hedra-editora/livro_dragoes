\documentclass[a5,11pt]{hedrabook}
\usepackage[brazilian]{babel}
\usepackage{ucs}
\usepackage[utf8x]{inputenc}
\usepackage[T1]{fontenc}
\usepackage[svn,cam,center]{hedracrop}
\usepackage{hedraextra,hedralogo}
\usepackage[protrusion=true,expansion]{microtype}
\usepackage[chapterdot]{hedratoc}
\usepackage{graphicx,color}
\usepackage{comment,lipsum}
\usepackage{kerkis}
%\usepackage[colorlinks]{hyperref} % cria hyperlinks e realça número da nota.
%\usepackage{makeidx,hedraindice}  % cria índice
%\makeindex			   % ...  índice

\usepackage{fltpoint}
\usepackage{bez123}
\fpDecimalSign{.}
\newcounter{bck}
\newcounter{bcx} \newcounter{bcy}
\newcounter{bcxmr} \newcounter{bcxpr}
\newcounter{bcymr} \newcounter{bcypr}
\newcounter{bcxmk} \newcounter{bcxpk}
\newcounter{bcymk} \newcounter{bcypk}
\newcommand{\bigcircle}[3]{ \fpMul{\bck}{0.5522847498}{#3}
\fpSub{\bcxmr}{#1}{#3} \fpAdd{\bcxpr}{#1}{#3}
\fpSub{\bcymr}{#2}{#3} \fpAdd{\bcypr}{#2}{#3}
\fpSub{\bcxmk}{#1}{\bck} \fpAdd{\bcxpk}{#1}{\bck}
\fpSub{\bcymk}{#2}{\bck} \fpAdd{\bcypk}{#2}{\bck}
\fpAdd{\bcx}{#1}{0.0} \fpAdd{\bcy}{#2}{0.0}
\cbezier(\bcxmr,\bcy)(\bcxmr,\bcypk)(\bcxmk,\bcypr)(\bcx,\bcypr)
\cbezier(\bcx,\bcypr)(\bcxpk,\bcypr)(\bcxpr,\bcypk)(\bcxpr,\bcy)
\cbezier(\bcxpr,\bcy)(\bcxpr,\bcymk)(\bcxpk,\bcymr)(\bcx,\bcymr)
\cbezier(\bcx,\bcymr)(\bcxmk,\bcymr)(\bcxmr,\bcymk)(\bcxmr,\bcy)
}

\newcommand{\circulo}{\begin{center}
$\bigcircle{0}{0}{75}$
\end{center}}


\begin{document}
\SVN $Id: DRAGOES.tex 9903 2011-09-29 14:29:44Z oliveira $

% Fazer uma classe para front simples da hedra:
%     Aprender a criar proporção:
%\pagestyle{plain}
\ \\
\ \\
\ \\
\ \\
{\centering\LARGE\thispagestyle{empty}
Livro dos dragões
\par}

\pagebreak

\newcommand{\hedracopy}{\raisebox{-1.2mm}{$^\copyright$}}

\noindent\hedracopy\ Hedra, 2011\\
\noindent\hedracopy\ Marcos Maffei, 2011
\bigskip

%Revisão:
%				\thispagestyle{empty}
{
\noindent\begin{minipage}{80mm}\thispagestyle{empty}
				{\baselineskip=0.7\baselineskip\footnotesize
				Dados Internacionais de Catalogação  na Publicação (CIP)\\
                                \vspace{-.3em}
				\hrule
				\vspace{1ex} 
				Maffei, Marcos (Organização e tradução). \textsc{Livro dos dragões}\\
				-- São Paulo: Hedra, 2011
				\vspace{1ex}
				\setlength{\parindent}{3ex}\\
				\textsc{isbn} 978-85-7715-114-1\\ %Jorge: Corrigi o ISBN, que foi impresso errado.
                                \vspace{1ex}
				1. Literatura brasileira I. Novela III. Título\\
				\vspace{1ex}
				\noindent   \hspace{\stretch{1}} \textsc{cdd} 028.5
				\vspace{1ex}\hrule\vspace{1ex}
				Índice para catálogo sistemático:\\
				1. Literatura infantil : \quad 028.5\hfill\\ 
				2. Literatura intanfo-juvenil : \quad 028.5\hfill 
                                }
\end{minipage}

\vspace{0.5cm}
\noindent Capa: Ronaldo Alves\\\EP[2]%
%Jorge: Aguardar Renan
%Ilustração de capa: Rosa Marques\\
Revisão: \\
\vfil


{\footnotesize 
\noindent Direitos reservados em l\'ingua\\ portuguesa somente para o Brasil\\%
\ \\
	\textsc{EDITORA HEDRA LTDA.}\\%
	R. Fradique Coutinho, 1139 (subsolo)\\
	05416-011 S\~ao Paulo SP Brasil\\
	Telefone/Fax (011) 3097-8304\\
	editora@hedra.com.br\\
	www.hedra.com.br%
\ \\
	Foi feito o dep\'osito legal.
}

\pagebreak
\ \\
\ \\
\ \\
\ \\
{\centering\LARGE\thispagestyle{empty}
Livro dos dragões
\par}
\ \\
{\centering
\small Organização e tradução\\ \Large Marcos Maffei
\par}
\bigskip


{\centering
%\normalsize ************\\
%\small\textit{Ilustrações}
\par}
\vfill

{\centering
\hskip-1ex\logoum\\
{\small
São Paulo 2011
\par}}

%\fontsize{11.2pt}{\baselineskip}\selectfont
%\baselineskip=0.97\baselineskip   % Não passar de 0.9 do \baselineskip

\setcounter{tocdepth}{2}     % amplitude da presença das partes no índice
\setcounter{secnumdepth}{-2} % amplitude da numeração das partes

%\hedratoc



\pagestyle{plain}
\part{Livro dos dragões}

\chapter{Perseu e Andrômeda\subtitulo{Ovídio}}

Perseu pegou suas sandálias aladas e as pôs nos pés. Prendeu sua
espada curva à cintura, e com o movimento das asas em suas sandálias
alçou vôo e atravessou os céus. Era já o segundo dia em que estava
retornando de seu combate com a Medusa, trazendo como troféu a cabeça
daquela monstra de cabelos de serpente. Voou sobre inúmeros povos,
cujas terras se estendiam em todas as direções, até avistar as tribos
etíopes e o reino de Cepheus. Lá, a jovem e inocente Andrômeda estava
a ponto de pagar injustamente pela insensata vaidade de sua mãe, a
rainha Cassiopeia, que se gabara de serem ela e sua filha muito mais
belas que as Nereidas, despertando a ira de Netuno, que enviara uma
inundação e um monstro marinho para devastar a costa do reino;
Cepheus consultara então o oráculo de Ammon, que lhe respondera que
seu reino só seria salvo se ele entregasse sua filha ao monstro
marinho.

Quando Perseu viu a princesa, com seus braços acorrentados à firme
rocha, teria achado que era uma estátua de mármore, não fosse pela
brisa que agitava os cabelos dela, e as cálidas lágrimas que
escorriam de seus olhos. Sem se dar conta, ele imediatamente se
apaixonou por ela. Encantado de ver tamanha beleza, deteve-se
maravilhado, e quase esqueceu de manter suas asas se movendo no ar.
Assim que pousou, disse:

-- Você não devia estar assim acorrentada; os laços que lhe cabem são
aqueles que unem os corações daqueles que se amam! Diga-me seu nome,
eu lhe peço, e o nome de seu país, e por que está acorrentada.

A princípio, ela nada disse, pois sendo uma virgem, não ousava
dirigir-se a um homem. Teria escondido por pudor sua face com as
mãos, se elas não estivessem acorrentadas. O que podia fazer, ela
fez; e era deixar seus olhos se encherem ainda mais de lágrimas. Ante
a insistência de Perseu, que repetia suas perguntas, ficou com medo
que sua recusa em falar fosse entendida como admissão de culpa; então
ela lhe disse o nome do país, o dela, e também como sua mãe, uma bela
mulher, deixara sua beleza subir-lhe à cabeça.

Antes que ela terminasse, as águas se agitaram em ruidosos vagalhães,
e das profundezas do oceano veio o ameaçador monstro, tão enorme que
cobria toda a extensão das ondas. A jovem gritou. Seu desconsolado
pai estava bem perto, e sua mãe também. Os dois estavam tomados por
um profundo desespero, embora sua mãe ainda com mais razão. Nenhuma
ajuda podiam lhe oferecer, apenas as lágrimas e os lamentos que
cabiam nas circunstâncias. E assim ficaram, até que o recém-chegado
forasteiro, Perseu, se dirigisse a eles, dizendo:

-- Haverá tempo de sobra para as lágrimas depois, mas para ajudá-la o
tempo é curto. Meu nome é Perseu, sou filho de Júpiter e Danae, a
qual era prisioneira numa torre quando Júpiter a fecundou sob a forma
de uma chuva de ouro. Sou aquele Perseu que venceu a Górgona de
cabelos de serpente, e que ousa viajar pelas brisas do ar com asas em
seus pés. Já seria o bastante para vocês me quererem como seu genro
se eu pedisse a mão de sua filha; mas se os deuses me ajudarem, vou
tentar acrescentar ainda outro serviço aos que já fiz. Quero esse
compromisso de vocês: que ela seja minha, se com a minha coragem eu
puder salvá-la.

Os pais dela concordaram -- quem, de fato teria hesitado? Imploraram
pela ajuda dele, prometendo que além da filha deles, teria também o
reino como dote. 

Mas eis que, como um navio veloz cortando as ondas com sua proa aguda,
impulsionado pelos braços fortes da suada tripulação em seus remos, o
monstro avançou, rompendo as ondas com seu peito. Não estava mais
longe dos rochedos do que o quanto alcança o projétil de uma fronda
baleárica, quando de repente o herói, saltando da terra, lançou-se
muito alto nas nuvens. A sombra dele se projetou na superfície do
oceano, e o monstro a atacou com toda sua fúria. Então Perseu
investiu lá do alto; como age a águia de Júpiter ao ver do céu uma
serpente ao sol num deserto, e a agarra por trás, cravando suas
garras ávidas nas escamas do pescoço do réptil, para evitar o ataque
de suas cruéis presas, foi o que Perseu velozmente vindo do ar fez,
atacando o monstro pelas costas e, ao som de seus urros, enterrando
fundo no ombro direito dele sua espada curva. Atormentado por tão
profunda ferida, o monstro arqueou para o alto seu corpo, voltou a
afundá-lo nas águas, e começou a se virar de um lado para o outro
como um feroz porco selvagem cercado e aterrorizado por uma matilha
de cães ladrando. O herói, valendo-se de suas velozes asas,
desviava-se das terríveis mandíbulas do monstro, desferindo golpes
com sua espada curva sempre que tinha oportunidade; acertando ou as
costas cobertas de conchas ocas da criatura, ou suas costelas, ou
ainda o ponto em que sua cauda se transformava no rabo de um peixe.
Da sua boca o monstro lançava golfadas de água do mar tingidas com o
vermelho do sangue, e logo as asas de Perseu ficaram úmidas e pesadas
com a espuma. Sem se atrever a continuar confiando em suas enxarcadas
asas, ele percebeu uma pedra cujo topo ficava acima da superfície
quando as águas estavam paradas, mas que o movimento das ondas cobria
inteira. Nela ele se apoiou, e segurando firme nas escarpas com a mão
esquerda, com a outra cravou sua espada duas, três, quatro vezes nos
flancos do monstro; tantas foram as vezes em que repetidamente o
atingiu, que logo das costas do mar até a morada dos deuses no céu só
o que se ouvia eram aplausos e gritos de júbilo.

Cassiopeia e Cepheus ficaram felicíssimos, e acolheram Perseu como
genro, dizendo que a casa deles lhe devia sua salvação e
continuidade. A jovem, causa e recompensa desse feito heróico, foi
libertada de suas correntes. O herói lavou suas mãos vitoriosas na
água do mar, e para que a áspera areia não prejudicasse a cabeça
cheia de serpentes da Medusa, antes de colocá-la no chão forrou-o com
folhas e depois com algas-marinhas. As algas, colhidas frescas e
ainda vivas, mostraram-se sensíveis ao poder da monstruosa cabeça, e
se endureceram, adquirindo uma nova e estranha rigidez em suas folhas
e ramos. As ninfas do mar testaram então esse milagre, aplicando-o a
vários ramos, e ficaram encantadas ao ver que ele sempre se repetia;
e espalharam pelas ondas suas sementes, produzindo mais e mais dessa
nova substância. E até hoje o coral mantém essa mesma característica,
enrijecendo-se em contato com o ar; o que é uma planta debaixo da
água se torna rocha acima da superfície.

Perseu erigiu então três altares de turfa em honra a três deuses: o da
esquerda para Mercúrio, o da direita para Minerva, e no centro, entre
os dois, um para Júpiter. Para a deusa sacrificou uma vaca, para
Mercúrio um bezerro, e para o mais poderoso dos deuses um touro. E
sem mais demora pediu Andrômeda como recompensa por sua grande
proeza, e a aceitou sem dote algum. Cupido e Hymen agitaram em frente
ao par as torchas nupciais; incenso em abundância tornou perfurmadas
as chamas, guirlandas foram penduradas no teto, e por toda a parte se
ouviram as liras e as flautas, e os cantos testemunhando e celebrando
a felicidade dos corações. Os grandes portões foram por inteiro
abertos, revelando o dourado luxo do palácio, e todos os nobres da
corte de Cepheus participaram do fausto banquete que foi servido.

\chapter{Stan Bolovan\subtitulo{Andrew Lang}}

O que aconteceu uma vez, aconteceu mesmo; pois se não tivesse
acontecido essa história nunca teria sido contada.

Nos arredores de uma aldeia, bem onde os bois são deixados para pastar
e os porcos perambulam enfiando seus focinhos nas raízes das árvores,
havia uma casinha. 

Nela morava um homem com sua mulher, e a mulher passava o dia todo
triste.

-- Cara esposa, o que há de errado para você ficar com sua cabeça
pendendo como uma rosa murcha? -- perguntou o marido uma manhã. -- Você
tem tudo o que quer; por que não pode viver contente como as outras
mulheres?

-- Deixe-me em paz, e não tente descobrir a razão -- ela respondeu,
explodindo em lágrimas; o homem achou que não era uma boa hora para
ficar fazendo perguntas a ela, e saiu para trabalhar.

Ele não conseguia, entretanto, esquecer o assunto, e alguns dias
depois voltou a perguntar a razão da tristeza dela, mas recebeu
apenas a mesma resposta. Por fim ele não podia mais aguentar, e
tentou uma terceira vez, e então sua mulher voltou-se para ele e
respondeu:

-- Por Deus! -- ela exclamou -- Por que você não pode deixar as coisas
como estão? Se eu lhe contasse, você iria ficar tão infeliz quanto
eu. Se ao menos você acreditasse que é melhor não saber de nada!

Mas nenhum homem jamais ficaria contente com uma resposta assim.
Quanto mais se implora para não perguntar, maior fica a curiosidade
de saber tudo.

-- Bom, se você precisa tanto saber -- disse enfim sua mulher -- eu vou
dizer. Não temos sorte nessa casa, nenhuma sorte!

-- Mas não é nossa vaca a que mais dá leite da aldeia? Não ficam nossas
árvores tão cheias de frutos quanto as colmeias de abelhas? Alguém
tem um milharal tão bom quanto o nosso? Olhe, você está falando
bobagem ao dizer isso.

-- Sim, tudo o que você disse é verdade, mas não temos filhos.

Então Stan comprendeu, e quando um homem comprende e tem seus olhos
abertos não há mais o que fazer. Desse dia em diante a casinha
continha um homem infeliz além de uma mulher infeliz. E ver a
tristeza de seu marido deixou a mulher mais desconsolada que nunca.

E assim ficaram as coisas por um tempo.

Algumas semanas se passaram, e Stan resolveu consultar um sábio que
vivia a um dia de viagem de sua casa. O sábio estava sentado em
frente à porta de sua casa quando ele chegou, e Stan se ajoelhou
diante dele.

-- Dê-me filhos, senhor, dê-me filhos.

-- Cuidado com o que você está pedindo -- respondeu o sábio. -- E se
filhos forem um fardo para você? Você é rico o suficiente para
alimentá-los e vesti-los? 

-- Apenas faça com que eu os tenha, e eu me virarei de algum jeito! -- e
o sábio lhe fez um sinal para ele partir.

Ele voltou para casa naquela noite cansado e sujo da viagem, mas com
esperança no coração. Quando estava se aproximando de sua casa, o som
de vozes chegou até seus ouvidos. Ele olhou donde vinham, e percebeu
o lugar todo cheio de crianças. Crianças no jardim, crianças no
quintal, crianças olhando de todas as janelas -- pareceu ao homem que
todas as crianças do mundo haviam se juntado ali. E nenhuma era maior
que a outra, mas cada uma era menor que a outra, e cada uma era mais
barulhenta, mais petulante e mais atrevida que as outras, e Stan
parou ali e gelou de horror ao se dar conta que eram todas dele.

-- Meu Deus! Quantas que há! Quantas! -- murmurou para si mesmo.

-- Ah, mas nem uma demais -- sua mulher sorriu, aparecendo com mais uma
multidão de crianças nas barras da saia.

Mas mesmo ela descobriu que não era tão fácil cuidar de cem crianças,
e depois de alguns dias, quando elas tinham comido toda a comida que
havia na casa, começaram a gritar:

-- Pai! Estou com fome! Estou com fome! 

E Stan coçou a cabeça e se perguntou o que ia fazer agora. Não que ele
achasse que havia crianças demais, pois sua vida parecia mais cheia
de alegria desde que elas apareceram, mas tinham chegado ao ponto em
que não sabia mais como alimentá-las. A vaca parara de dar leite, e
não estava ainda na época das árvores darem frutos.

-- Sabe, minha velha -- ele disse um dia a sua mulher -- preciso correr
mundo e tentar trazer comida de algum jeito, embora não saiba dizer
donde vou tirá-la.

Para o homem com fome toda estrada é comprida, e ainda havia sempre a
lembrança de que tinha a fome de cem crianças para satisfazer, além
da sua própria.

Stan andou, andou e andou, até chegar ao fim do mundo, onde o que
existe se mistura com o que não existe, e lá ele viu, a uma pequena
distância, um ovil, com sete ovelhas dentro. Na sombra de umas
árvores estava o resto do rebanho.

Stan se escondeu, esperando que conseguiria fazer algumas escapulirem
com ele discretamente, para levá-las para alimentar sua família, mas
logo descobriu que isso não ia dar certo. Pois à meia-noite ouviu um
barulho: era um dragão que viera voando e levou embora um carneiro,
uma ovelha e um cordeiro, e três vacas que estavam ali por perto. E
além disso tirou também o leite de 77 ovelhas, que levou para sua
velha mãe nele se banhar para recuperar sua juventude. E isso
acontecia toda noite.

O pastor se lamentava em vão: o dragão apenas ria, e Stan viu que
aquele não era um bom lugar para conseguir comida para sua família.

Mas apesar de entender que era quase impossível lutar contra um
monstro tão poderoso, a lembrança de suas crianças famintas em casa
se agarrava a ele feito um carrapicho, do qual não dá para se livrar,
e por fim ele disse ao pastor:

-- O que você me daria se eu o livrasse do dragão?

-- Um de cada três carneiros, uma de cada três ovelhas, um de cada três
cordeiros -- respondeu o pastor.

-- É uma barganha -- disse Stan, embora naquele momento não soubesse
como, supondo que ele vencesse o dragão, iria conseguir levar um
rebanho tão grande para casa.

Todavia, esse problema podia ser resolvido depois. No momento, a noite
estava chegando, e ele precisava pensar como seria o melhor jeito de
lutar com o dragão.

Precisamente à meia-noite, Stan foi tomado por uma sensação horrível,
nova e estranha para ele -- uma sensação que ele não encontrou
palavras para descrever a si mesmo, mas que quase o forçou a desistir
da batalha e pegar o caminho mais curto para casa. Ele se virou para
partir; mas aí lembrou das crianças, e se virou de novo.

-- É você ou eu -- disse Stan para si mesmo, e se posicionou junto ao
rebanho.

-- Pare! -- ele gritou de repente, quando o ar se encheu com o barulho
das asas do dragão descendo.

-- Ora essa! -- exclamou o dragão, olhando em volta. -- Quem é você, e de
onde vem? 

-- Eu sou Stan Bolovan, que come pedras toda noite, e durante o dia as
flores da montanha; e se você bulir com essas ovelhas eu vou entalhar
uma cruz em seu lombo.

Ao ouvir essas palavras o dargão parou bem quieto no meio da estrada,
pois sabia que havia encontrado um adversário a sua altura.

-- Mas você vai ter antes que lutar comigo -- ele disse com voz trêmula,
pois quando alguém o enfrentava de verdade ele não era nem um pouco
corajoso.

-- Lutar com você? -- retrucou Stan. -- Ora, se posso matá-lo com um
sopro…

Então, pegando um grande queijo que estava a seus pés, acrescentou:

-- Vá buscar uma pedra como essa no rio, para que não percamos tempo em
descobrir quem é o melhor.

O dragão fez o que Stan lhe pediu, e trouxe uma pedra do riacho.

-- Você consegue tirar leite da sua pedra? -- Stan perguntou.

O dragão catou sua pedra com uma mão, e espremeu-a até virar pó, mas
nenhum leite escorreu dela.

-- Claro que não! -- ele disse, com uma certa raiva.

-- Bom, se você não consegue, eu consigo -- respondeu Stan, e apertou o
queijo até o leite escorrer entre seus dedos.

Quando o dragão viu aquilo, achou que já era mais que hora de voltar
para sua casa, mas Stan ficou no caminho dele.

-- Ainda temos contas a acertar -- ele disse -- sobre o que você anda
fazendo por aqui.

E o pobre dragão estava com muito medo para se mexer, temendo que Stan
o matasse com um sopro e o enterrasse no meio das flores dos pastos
das montanhas.

-- Escute -- ele disse enfim. -- Posso ver que você é uma pessoa muito
útil, e minha mãe está precisando de alguém como você. Digamos que
você preste serviços a ela por três dias, que duram tanto quanto um
de seus anos, e ela lhe pague com sete sacos cheio de ducados por
cada dia.

Três vezes sete sacos cheios de ducados! A oferta era muito tentadora,
e Stan não conseguiu resistir a ela. Não gastou palavras, apenas
acenou que concordava para o dragão, e os dois partiram pela estrada.

Foi uma jornada muito, muito longa, mas quando chegaram ao fim dela
encontraram a mãe do dragão, que era tão velha quanto o próprio
tempo, esperando por eles. Stan viu de longe os olhos dela brilhando
como lanternas, e quando entraram na casa viram uma enorme chaleira
no fogo, cheia de leite. Quando a velha mãe viu que seu filho voltara
de mãos vazias ficou muito brava, e fogo e chamas saíram de suas
narinas, mas antes de ela dizer qualquer coisa o dragão voltou-se
para Stan.

-- Fique aqui -- disse -- e me espere; vou explicar as coisas para minha
mãe.

Stan já estava se arrependendo amargamente de ter vindo para tal
lugar, mas como já estava lá, não havia nada a fazer senão enfrentar
tudo calmamente, e não demonstrar que estava com medo.

-- Olhe, mãe -- disse o dragão assim que ficaram sozinhos. -- Eu trouxe
esse homem para me livrar dele. Ele é um sujeito terrível, que come
rochas e pode tirar leite de pedras -- e lhe contou o que acontecera
na noite passada.

-- Ah, deixe-o comigo! -- ela disse. -- Nunca deixei um homem escapar por
entre meus dedos.

Então Stan teve que ficar prestando serviços à velha mãe.

No dia seguinte ela lhe disse que ele e seu filho deviam ver quem era
mais forte, e catou uma enorme clava, envolta sete vezes com ferro.

O dragão pegou-a como se fosse uma pena, e depois de girá-la sobre sua
cabeça, atrirou-a a três milhas de distância, dizendo a Stan para
jogá-la mais longe se pudesse.

Andaram até o lugar onde caíra a clava. Stan se inclinou e
experimentou pegá-la; e então um grande medo lhe assomou, pois sabia
que ele e todas suas crianças juntas jamais conseguiriam levantar do
chão aquela clava. 

-- O que você está fazendo? -- perguntou o dragão.

-- Eu estava pensando o quanto é bela essa clava, e me deu pena de que
seja ela a causar a sua morte.

-- Minha morte? O que você quer dizer? -- perguntou o dragão.

-- Apenas que receio que, se atirá-la, você nunca mais verá outro
amanhecer. Você não faz ideia do quanto eu sou forte!

-- Ah, não se preocupe, atire-a de uma vez.

-- Se você quer isso mesmo, vamos festejar por três dias; ao menos
assim você terá três dias a mais de vida.

Stan falou com tanta calma que dessa vez o dragão ficou com um pouco
de medo, embora não acreditasse que a coisa ia ser tão ruim quanto
Stan dissera.

Eles voltaram para a casa, pegaram toda a comida que acharam na
despensa da velha mãe, e voltaram para onde estava a clava. Então
Stan sentou-se no saco de mantimentos, e ficou tranquilamente
observando a lua que se punha.

-- O que você está fazendo? -- perguntou o dragão.

-- Esperando a lua sair do meu caminho.

-- Como assim? Não entendi.

-- Você não está vendo que a lua está exatamente no meu caminho? Mas
claro, se você quiser, posso atirar a clava na lua.

Essas palavras deixaram o dragão inquieto pela segunda vez.

Ele tinha grande estima por aquela clava, que havia sido herança de
seu avô, e não tinha a menor vontade de que ela fosse parar na lua.

-- Vou lhe dizer uma coisa -- ele disse, depois de pensar um pouco. --
Não precisa atirar a clava. Eu a atiro uma segunda vez, e fica por
isso mesmo.

-- Não, de jeito nenhum! -- respondeu Stan. -- Basta esperar a lua se
pôr.

Mas o dragão, temendo que Stan cumprisse suas ameaças, tentou toda
espécie de suborno para evitá-las, e no fim teve que prometer a Stan
sete sacos de ducados antes de enfim conseguir jogar de volta a clava
ele mesmo.

-- Ah, nossa, ele é mesmo um homem forte -- disse o dragão para sua mãe.
-- Você acredita que foi a maior dificuldade evitar que ele atirasse a
clava na lua? 

Então a velha ficou inquieta também, só de pensar nisso! Atirar coisas
na lua não é brincadeira! Então não se falou mais na clava, e no diz
seguinte todos tinham outro assunto com que se preocupar.

-- Vão buscar água! -- disse a mãe, assim que amanheceu, e deu a eles
doze odres de pele de búfalo com a ordem de enchê-los até de noite.

Eles partiram na mesma hora para o riacho, e num piscar de olhos o
dragão enchera todos os doze, levou-os até a casa, e os trouxe de
volta para Stan. Stan estava cansado: mal conseguia levantar os odres
vazios, e teve um arrepio só de pensar o que iria acontecer qaundo
estivessem cheios. Mas a única coisa que fez foi tirar uma faca de
seu bolso e começar a cavar a terra perto do riacho.

-- O que você está fazendo aí? Não vai carregar a água para casa? --
perguntou o dragão.

-- O quê? Ora, isso é fácil demais! Eu vou é levar o riacho inteiro!

Essas palavras fizeram o queixo do dragão cair. Era a última coisa que
passaria por sua cabeça, pois o riacho sempre estivera ali, desde os
tempos de seu avô.

-- Vou lhe dizer uma coisa -- disse. -- Deixe que eu carrego os odres
para você.

-- De jeito nenhum -- respondeu Stan, continuando a cavar, e o dragão,
temendo que ele cumprisse sua ameaça, tentou toda espécie de suborno,
e no fim teve que prometer de novo sete sacos de ducados para fazer
Stan concordar em largar o riacho ali em paz e deixar o próprio
dragão levar a água para a casa.

No terceiro dia a velha mãe mandou Stan buscar lenha na floresta e,
como sempre, o dragão foi junto com ele.

Antes de você contar até três ele já tinha derrubado mais árvores que
Stan poderia ter cortado em toda a sua vida, e as arrumara
devidamente em fileiras. Quando o dragão tinha terminado, Stan
começou a olhar a sua volta e, escolhendo a maior das árvores, subiu
nela, onde cortou um longo cipó e com ele amarrou a ponta dela à
ponta da árvore seguinte ao lado. E assim ele fez com toda uma
fileira de árvores.

-- O que você está fazendo aí? -- perguntou o dragão.

-- Basta você olhar para saber -- repsondeu Stan, continuando calmamente
seu trabalho.

-- Por que você está amarrando as árvores juntas?

-- Para não ter trabalho à toa; quando eu arrancar uma, todas as outras
já virão junto. 

-- Mas como você vai carregá-las para casa?

-- Ora essa! Você ainda não entendeu que eu vou levar a floresta
inteira de volta comigo? -- disse Stan, amarrando mais duas árvores
enquanto falava.

-- Vou lhe dizer uma coisa -- exclamou o dragão, tremendo de medo só de
pensar nisso -- deixe que eu carrego a lenha para você, e eu lhe dou
sete vezes sete sacos cheios de ducados.

-- Você é um bom sujeito, e eu concordo com sua proposta -- Stan
respondeu, e o dragão carregou a lenha.

Então os três dias de serviço que era para ser contados como um ano já
haviam terminado, e a única coisa que preocupava Stan era: como levar
todos aqueles ducados de volta para casa?

Naquela noite o dragão e sua mãe tiveram uma longa conversa, mas Stan
ouviu-a inteirinha por um buraco no teto.

-- Que infelicidade a nossa, mãe -- disse o dragão -- esse homem logo vai
nos dominar. Dê a ele o dinheiro, para podermos nos livrar dele.

Mas a velha mãe gostava de dinheiro, e a ideia não a agradava.

-- Escute -- ela disse -- você precisa matá-lo esta noite.

-- Tenho medo -- disse ele.

-- Não há nada a temer -- retrucou a velha mãe. -- Quando ele estiver
dormindo, pegue a clava e acerte-o na cabeça com ela. Vai ser fácil.

E teria sido mesmo, se Stan não tivesse ouvido tudo. Quando o dragão e
sua mãe apagaram as luzes, ele pegou a gamela dos porcos, encheu-a de
terra e a pôs em sua cama, cobrindo-a com suas roupas. Então ele se
escondeu sob a cama, e se pôs a roncar bem alto.

Logo o dragão entrou silenciosamente no quarto e deu um tremendo golpe
no lugar onde a cabeça de Stan deveria estar. Stan gemeu bem alto
debaixo da cama, e o dragão saiu tão silenciosamente quanto entrara.
Assim que ele fechou a porta, Stan tirou dali a gamela dos porcos, e
deitou-se em seu lugar, depois de ter deixado tudo limpo e arrumado;
mas foi esperto o bastante para não pregar os olhos naquela noite.

Na manhã seguinte ele veio para a sala quando o dragão e sua mãe
estavam tomando café da manhã.

-- Bom dia -- disse.

-- Bom dia. Dormiu bem?

-- Ah, muito bem, mas sonhei que uma pulga havia me mordido, e ainda a
estou sentindo.

O dragão e sua mãe se entreolharam.

-- Você ouviu só? -- ele sussurrou. -- Ele falou de uma pulga. E eu
quebrei minha clava na cabeça dele.

Dessa vez a mãe ficou com tanto medo quanto seu filho. Não havia nada
a fazer com um homem como esse, e ela se apressou a encher os sacos
com os ducados, para se livrar de Stan o mais rápido possível. Mas,
por sua parte, Stan estava tremendo feito vara verde, pois não
conseguia levantar nem mesmo um saco do chão. Então ele ficou ali
parado olhando para eles.

-- O que você está esperando aí? -- perguntou o dragão.

-- Ah, eu estava esperando aqui porque acabou de me ocorrer que
gostaria de ficar a serviço de vocês mais um ano. Tenho vergonha de
chegar em casa e verem que eu trouxe tão pouco. Tenho certeza que vão
dizer “olha só o Stan Bolovan, que em um ano acabou ficando tão fraco
quanto um dragão”.

Nesse momento uma exclamação de pasmo escapou tanto do dragão quanto
de sua mãe, que declarou que ela ia lhe dar sete vezes ou mesmo sete
vezes sete vezes o número de sacos se ele fosse embora.

-- Vou lhe dizer uma coisa -- enfim Stan disse. -- Estou vendo que não
querem que eu fique, e não quero incomodá-los. Eu irei embora
imediatamente, mas sob a condição do dragão carregar para mim o
dinheiro até em casa; assim não passarei vergonha frente a meus
amigos.

Mal acabara de falar e o dragão já agarrara os sacos e os empilhara em
suas costas. Então ele e Stan partiram.

O caminho de volta, se na verdade não era muito comprido, ainda assim
demorou um bocado para Stan, mas enfim ele ouviu as vozes de suas
crianças, e parou de repente. Não queria que o dragão ficasse sabendo
onde ele morava, temendo que algum dia ele viesse recuperar seu
tesouro. Não haveria nada que ele pudesse dizer para se livrar do
monstro? De repente uma ideia lhe veio à cabeça, e ele se virou.

-- Não sei bem o que fazer -- disse. -- Tenho cem filhos, e tenho medo
que eles possam machucá-lo, pois estão sempre prontos a entrar numa
briga. Em todo o caso, vou fazer todo o possível para protegê-lo.

Cem crianças! Isso não era brincadeira mesmo! O dragão até deixou cair
os sacos, tanto o seu terror, mas tratou de catá-los de novo. Foi
então que as crianças, que não tinham comido nada desde que seu pai
partira, vieram correndo na direção dele, brandindo facas na mão
direita e garfos na esquerda, e gritando “Papai! Dê-nos carne de
dragão! Queremos carne de dragão!”

Ao ver essa terrível cena o dragão não esperou nem mais um instante:
largou os sacos e saiu voando o mais rápido que podia, e tão
aterrorizado ficou com o destino do qual escapou por pouco que desse
dia em diante nunca mais teve coragem de mostrar a cara pelo mundo de
novo.

\chapter{São Jorge e o dragão\subtitulo{Jacobus de Voragine e William Caxton}}

São Jorge era um cavaleiro que nascera na Capadócia. Um dia ele foi
para a província da Líbia, onde havia uma cidade chamada Sylene. E
perto dessa cidade havia uma lagoa feito um mar, e nela um dragão que
ameaçava envenenar todo o país. Um dia o povo se juntou para matá-lo,
mas ao vê-lo se apavoraram e fugiram. E quando chegou a noite, o
dragão ia envenenar o povo com o ar que expirava, de modo que eles
decidiram dar todo dia duas ovelhas para alimentá-lo, para que assim
ele não fizesse mal ao povo. 

Quando as ovelhas acabaram, chegou a vez dos homens, e por um decreto
na cidade se estabeleceu que seriam entregues primeiro as crianças e
os jovens, e quem se recusasse a entregar seus filhos iria em lugar
deles. E assim foi que muitas crianças e jovens foram entregues ao
dragão, tantas que acabou chegando a vez da filha do rei. O rei
entristecido disse a seu povo:

-- Pelo amor dos deuses, levem ouro e prata e tudo o que tenho, mas
deixem eu ficar com minha filha. 

Eles responderam:

-- Como assim? Foi Vossa Majestade quem fez a lei e mandou cumprir, de
tal modo que agora todos nossos filhos estão mortos, e agora quer
fazer diferente? Sua filha deverá ser entregue, ou então levaremos
Vossa Majestade. 

Quando o rei viu que não havia nada a fazer começou a chorar e disse
para a filha dele que jamais iria ver as bodas dela. Então ele voltou
ao povo e pediu oito dias de prorrogação, e o povo lhe concedeu. E
quando os oito dias se passaram voltaram a ele dizendo que já era a
hora, pois ele podia ver que a cidade estava perecendo.

Então o rei fez vestirem sua filha como se ela fosse se casar, e a
abraçou e beijou, e a abençoou, e a conduziu para o lugar em que
estava o Dragão. 

Quando ela estava lá, São Jorge passou por ali. Vendo a dama,
perguntou o que ela fazia ali, e ela disse:

-- Siga seu caminho, belo jovem, para que não morra você também.

Então ele disse:

-- Diga-me o que a aflige e por que você chora, e nada tema.

Quando ela viu que ele tanto queria saber, ela disse que havia sido
deixada para o dragão. São Jorge disse:

-- Bela jovem, nada tema, pois eu vou ajudá-la em nome de Jesus Cristo.


-- Pelos deuses -- ela disse -- siga seu caminho, não fique aqui comigo,
pois você pode não conseguir me salvar.

Enquanto eles falavam o dragão apareceu e veio correndo até eles. São
Jorge montou em seu cavalo, desembainhou a espada, fez o sinal da
cruz, cavalgou bravamente em direção ao Dragão e o acertou com sua
lança e o feriu gravemente, jogando-o por terra. Então ele disse para
a jovem:

-- Pegue seu cinto e amarre-o em volta do pescoço do dragão. Não tenha
medo.Quando ela fez isso o Dragão se pôs a segui-la como se fosse um
animal domesticado e inofensivo. Ela levou-o até a cidade, e o povo
fugiu para as montanhas e vales, dizendo que seriam todos mortos.
Então São Jorge disse: 

-- Nada temam, creiam em Jesus Cristo, e não tardem a ser batizados,
que eu matarei o dragão. 

Então o rei e todo seu povo foi batizado, e São Jorge matou o dragão
cortando-lhe a cabeça, e ordenou que ele fosse jogado no campo, e foi
preciso três carros de boi para tirá-lo da cidade. 

Haviam sido batizados quinze mil homens, sem contar mulheres e
crianças, e o rei mandou erguer uma igreja para Nossa Senhora e São
Jorge, onde ainda hoje brota uma fonte de água da vida que cura os
doentes que dela bebem.

Depois disso o rei ofereceu a São Jorge tanto dinheiro quanto tinha,
mas ele recusou tudo e pediu que fosse dado aos pobres em nome de
Deus, e exigiu do rei quatro coisas, a saber, que ele se encarregasse
das igrejas, que ele honrasse os padres, que ouvisse diligentemente
seus sermões, e que ele tivesse piedade dos pobres; e depois de
beijar o rei ele partiu.

\chapter{A história de Sigurd\subtitulo{Andrew Lang}}

Havia uma vez um rei no Norte que ganhara muitas guerras, e que agora
já estava velho. Mas ele se casou com uma outra mulher, e então um
certo príncipe, que também queria casar com ela, veio atacá-lo com um
grande exército. O velho rei lutou bravamente, mas sua espada acabou
se quebrando, ele se feriu e todos os seus homens fugiram. De noite,
depois que a batalha tinha terminado, sua jovem esposa veio
procurá-lo entre os mortos e feridos. Enfim o encontrou, e perguntou
se ele podia ser curado. Mas ele respondeu que não, sua sorte
acabara, sua espada quebrara, e ele ia morrer. E disse ainda que ela
ia ter um filho, que esse filho seria um grande guerreiro e iria
vingar-se do outro rei seu inimigo. E pediu a ela que guardasse os
pedaços quebrados da espada, para que com eles uma nova espada fosse
feita para o seu filho; e essa nova arma deveria ser chamada de Gram.


Então ele morreu. E sua mulher chamou a criada e disse:

-- Vamos trocar de roupa; e você será chamada pelo meu nome, e eu pelo
seu, caso os inimigos nos encontrem.

Foi o que fizeram, e se esconderam na floresta, mas então uns
forasteiros as encontraram e as levaram num navio para a Dinamarca. E
quando elas foram levadas até o rei, este achou que a criada parecia
uma rainha, e a rainha uma criada. Então ele perguntou à rainha:

-- Como você sabe no meio da noite que falta pouco para amanhecer? 

E ela disse:

-- Eu sempre sei porque, quando era mais jovem, costumava levantar para
acender os fogos, e ainda acordo na mesma hora.

“Estranho, uma rainha que acende os fogos”, o rei pensou.

Então ele perguntou à rainha que estava vestida de criada:

-- Como você sabe no meio da noite que a aurora se aproxima? 

-- Meu pai me deu um anel de ouro -- ela disse -- e sempre antes do
amanhecer ele esfria em meu dedo.

-- Rica essa casa em que as criadas usam ouro -- disse o rei. -- Na
verdade você não é nenhuma criada, mas a filha de um rei. 

Então ele a tratou como convinha a uma rainha, e com o passar do tempo
ela teve um filho que chamou de Sigurd, um menino belo e muito forte.
Ele tinha um tutor a acompanhá-lo, e um dia o tutor disse a ele para
ir pedir um cavalo para o rei. 

-- Escolha você mesmo o seu cavalo -- disse o rei; e Sigurd foi até a
floresta, onde encontrou um velho com uma barba branca, e disse a
ele:

-- Venha! Ajude-me a escolher um cavalo.

O velho disse:

-- Leve todos os cavalos para o rio, e escolha aquele que o atravessar
nadando. 

Então Sigurd levou-os até o rio, e só um deles o atravessou. Sigurd o
escolheu: seu nome era Grani, ele vinha da linhagem de Sleipnir e era
o melhor cavalo do mundo. Pois Sleipnir era o cavalo de Odin, o Deus
do Norte, e era tão rápido quanto o vento. 

Um dia ou dois depois disso, o tutor disse a Sigurd:

-- Há um grande tesouro em ouro escondido não muito longe daqui, e
seria apropriado para você se o encontrasse. 

Mas Sigurd respondeu:

-- Já ouvi falar desse tesouro, e sei que o dragão Fafnir o guarda, e
ele é tão enorme e terrível que nenhum homem ousa se aproximar dele.

-- Ele não é maior que outros dragões -- disse o tutor -- e se você fosse
tão corajoso quanto seu pai não teria medo dele. 

-- Não sou um covarde -- disse Sigurd. -- Por que você quer que eu lute
com esse dragão? 

Então seu tutor, que se chamava Regin, contou a ele que todo aquele
tesouro de ouro vermelho pertencera antes a seu pai. E seu pai teve
três filhos: o primeiro foi Fafnir, o dragão; o segundo foi Lontra,
que podia tomar a forma de uma lontra sempre que quisesse; e o último
foi ele, Regin, que era um grande ferreiro e fabricante de espadas. 

Havia então um anão chamado Andvari, que morava junto a uma lagoa sob
uma cachoeira, e lá ele escondera um tesouro. E um dia Lontra estava
pescando ali, e tinha pego um salmão e o comido, e estava dormindo
numa pedra na forma de lontra. Alguém que passava por ali jogou uma
pedra na lontra e a matou; e tirou a pele dela, e a levou para a casa
do pai de Lontra. Então ele soube que seu filho estava morto, e para
punir a pessoa que o matara exigiu que a pele da lontra fosse enchida
com ouro, e coberta com ouro, ou senão teria que se ver com ele. A
pessoa que matara Lontra foi até a cachoeira e capturou o anão que
tinha o tesouro e o tirou dele. 

Só sobrara um anel, que o anão estava usando; mas até este foi tirado
dele. 

O pobre anão ficou muito bravo, e rogou uma praga: aquele ouro só iria
trazer má sorte para quem o possuísse, para sempre. 

Então a pele da lontra foi enchida com ouro e coberta com ouro
inteira, exceto por um pelo, e nele foi enfiado o último anel do
pobre anão. 

Mas não trouxe sorte para ninguém. Primeiro Fafnir, o dragão, matou
seu próprio pai; e então ele foi até o ouro e sobre ele se estendeu,
não deixando seu irmão ficar com nem um pouco; e nenhum homem ousava
se aproximar dele. 

Ao ouvir a história Sigurd disse a Regin:

-- Faça-me uma boa espada para eu matar esse dragão. 

Regin fez uma espada para ele, mas ele a testou com um golpe num bloco
de ferro, e ela se quebrou. 

Outra espada foi feita, e Sigurd também a quebrou. 

Então Sigurd foi ter com sua mãe, e pediu os pedaços da espada de seu
pai, e os deu para Regin. E ele os forjou e martelou numa espada
nova, tão afiada que as bordas de sua lâmina pareciam incandescentes.

Sigurd experimentou essa espada num bloco de ferro e ela não quebrou,
mas partiu ao meio o ferro. Então ele jogou um floco de lã no rio, e
quando ele flutuou até a lâmina, cortou-se em dois. Sigurd disse que
aquela espada servia. Mas antes de ir lutar com o dragão ele liderou
um exército para lutar com os homens que haviam matado seu pai. Ele
executou o Rei deles, ficou com toda sua fortuna, e voltou para casa.


Depois de alguns dias, ele foi a cavalo com Regin até a charneca onde
o dragão costumava ficar. Então viu o rastro que o dragão deixava
quando ia num rochedo beber; e o rastro era como se um grande rio
tivesse passado e deixado um vale profundo. 

Então Sigurd foi até esse valo profundo, e cavou muitos buracos nele,
e num deles se escondeu com sua espada na mão. Lá ficou esperando, e
logo a terra começou a tremer com o peso do dragão rastejando até a
água. E uma nuvem de veneno era lançada à sua frente quando ele
bufava e urrava, de modo que seria morte certa ficar diante dele. 

Mas Sigurd esperou até a metade do corpo do dragão ter rastejado sobre
o buraco, e então enfiou a espada Gram bem no coração dele. 

O dragão vergastou sua cauda em volta, despedaçando rochas e
derrubando árvores. E quando viu que ia morrer, disse:

-- Quem quer que seja você que me matou, esse ouro será sua ruína, e a
ruína de todos aqueles que o possuírem. 

Sigurd disse:

-- Eu não o tocaria se isso me fizesse nunca morrer. Mas todos os
homens morrem, e nenhum homem corajoso deixa a morte amedrontá-lo de
ter aquilo que deseja. Morra, Fafnir! 

E Fafnir morreu. 

E Sigurd passou a ser chamado de o Algoz de Fafnir, e Matador de
Dragões. 

Então Sigurd voltou, e encontrou com Regin, que disse a ele para assar
o coração de Fafnir e deixá-lo provar. 

Sigurd pôs o coração de Fafnir num espeto, e o pôs para assar. Mas
aconteceu de ele tocá-lo por acidente, queimando o dedo. Ele pôs o
dedo na boca, e assim provou o coração de Fafnir. 

E imediatamente ele compreendeu a linguagem dos pássaros, e ouviu o
que diziam os pica-paus:

-- Eis Sigurd, assando o coração de Fafnir para outro, quando devia ser
ele mesmo a prová-lo e aprender toda a sabedoria. 

O pássaro ao lado disse:

-- Eis Regin, pronto para trair Sigurd, que confia nele. 

O terceiro pássaro disse:

-- Que ele corte a cabeça de Regin, e fique com todo o ouro para si
mesmo.

O quarto pássaro disse:

-- Que ele faça isso, e então vá para Hindfell, o lugar onde Brynhild
dorme. 

Ao ouvir essas palavras, e como Regin tramava traí-lo, Sigurd cortou a
cabeça dele com um só golpe da espada Gram.

Então todos os pássaros cantaram, e era uma canção sobre uma bela
jovem que dormia num lugar cercado por fogo, à espera dele para
acordá-la. 

E Sigurd lembrou-se que havia uma história sobre uma bela dama
enfeitiçada num lugar muito distante dali. Ela estava sob um encanto,
que a fazia dormir num castelo cercado por chamas; lá ela dormiria
até que chegasse um cavaleiro capaz de atravessar o fogo para
acordá-la. Ele decidiu ir até lá, mas antes seguiu o horrível rastro
de Fafnir. E encontrou sua caverna, que era bem profunda e tinha
portas de ferro, e estava cheia de braceletes, coroas e anéis de
ouro; e lá Sigurd achou também o Elmo do Pavor, que era de ouro e
deixava invisível quem o usava. Ele carregou tudo isso no bom cavalo
Grani, e então partiu para Hindfell, ao sul. 

Já era de noite, e no cimo de uma colina Sigurd viu um fogo brilhando
vermelho no céu, e dentro das chamas um castelo, com um estandarte na
torre mais alta. Então ele arremeteu no fogo com seu cavalo Grani,
que o saltou com leveza, como se não passasse de um arbusto. Sigurd
passou pelo portão do castelo e viu alguém dormindo, usando uma
armadura. Ele tirou o elmo da pessoa adormecida, e para sua surpresa
viu que era uma dama das mais belas. Ela acordou e disse:

-- Ah, é Sigurd, o filho de Sigmund, que quebrou a maldição, e veio
enfim me acordar?

A maldição tinha sido posta nela quando o espinho da árvore do sono
picou sua mão, tempos atrás, como uma punição por ter desagradado ao
deus Odin. E tempos atrás também ela tinha jurado nunca casar com um
homem que tivesse medo, e não ousasse atravessar a cerca de chamas.
Pois ela era uma guerreira, e ia para as batalhas armada como um
homem. Mas agora ela e Sigurd se apaixonaram, e prometeram ser fiéis
uma ao outro. Ele deu a ela um anel, que era o último anel do anão
Andvari. Então Sigurd partiu, e encontrou o castelo de um rei que
tinha uma bela filha. O nome dela era Gudrun, e sua mãe era uma
feiticeira. Gudrun se apaixonou por Sigurd, mas ele só falava em
Brynhild, em como ela era bela e o quanto ele a amava. Um dia então a
mãe feiticeira de Gudrun pôs sementes de papoula e outras drogas de
esquecimento numa taça mágica, e fez Sigurd beber à saúde dela. No
mesmo instante ele esqueceu a pobre Brynhild e se apaixonou por
Gudrun, e eles se casaram com grandes festejos.

A feiticeira, mãe de Gudrun, quis então que seu filho Gunnar casasse
com Brynhild, e disse a ele para ir com Sigurd fazer a corte a ela.
Eles partiram para o castelo do pai dela, pois Brynhild tinha saído
completamente da cabeça de Sigurd por causa da poção da feiticeira,
mas ela ainda lembrava dele e o amava. O pai de Brynhild disse a
Gunnar que ela só casaria com alguém capaz de atravessar o fogo em
frente a sua torre encantada, e para lá eles foram. Gunnar tentou
passar pelas chamas, mas seu cavalo não quis enfrentá-las. Então
Gunnar tentou usar o cavalo de Sigurd, mas montado por ele, Grani não
se movia. Daí Gunnar lembrou-se dos feitiços que sua mãe havia lhe
ensinado, e com eles fez Sigurd ficar exatamente igual a ele, e ele
exatamente igual a Sigurd. Então Sigurd, sob a forma de Gunnar,
montou em Grani, que saltou a cerca de fogo, e o rapaz foi até
Brynhild, mas ainda nada lembrava dela, por causa da poção de
esquecimento da feiticeira.

E Brynhild não teve alternativa a não ser prometer casar com ele, ser
a esposa de Gunnar, pois Sigurd estava sob a forma de Gunnar, e ela
prometera casar com quem quer que atravessasse o fogo. E ele deu a
ela um anel, e ela devolveu a ele o anel que ele tinha lhe dado antes
sob sua própria forma de Sigurd, aquele que era o último anel do
pobre anão Andvari. Ele voltou, e trocou de aparência com Gunnar, e
cada um sendo ele mesmo de novo, voltaram ao castelo da rainha
feiticeira, e Sigurd deu o anel do anão a sua esposa Gudrun. Brynhild
foi ter com seu pai e disse que um príncipe chamado Gunnar havia
atravessado o fogo, e ela teria de se casar com ele. 

-- E no entanto eu achava -- ela disse -- que nenhum homem conseguiria
tal proeza a não ser Sigurd, o Algoz de Fafnir, que era o meu
verdadeiro amor. Mas ele se esqueceu de mim, e tenho de manter minha
promessa. 

Então Gunnar e Brynhild se casaram, embora não tivesse sido Gunnar que
atravessara o fogo, mas Sigurd disfarçado. 

Quando acabou o casamento e todos os festejos, a mágica da poção da
feiticeira deixou a mente de Sigurd, e ele se lembrou de tudo.
Lembrou que libertara Brynhild do encanto, e que era ela seu
verdadeiro amor, como a esquecera e casara com outra mulher, e
conquistara Brynhild para ser a mulher de outro homem. 

Mas ele era corajoso, e nada disse aos outros, para não fazê-los
infelizes. Não conseguia escapar, entretanto, da maldição sobre todo
aquele que possuísse o tesouro do anão Andavari, e seu anel de ouro
fatal. 

E logo a maldição voltou a se abater sobre todos eles. Pois um dia,
quando Brynhild e Gudrun estavam se banhando num rio, Brynhild foi
mais fundo nas águas, e disse que fizera isso para provar sua
superioridade sobre Gudrun, porque o marido dela, disse, tinha
atravessado o fogo que nenhum outro homem ousara enfrentar. Gudrun
ficou muito brava, e disse que tinha sido Sigurd, e não Gunnar, que
atravessara o fogo, e recebera de volta de Brynhild aquele anel
fatal, o anel do anão Andvari. 

Então Brynhild viu o anel que Sigurd dera a Gudrun e soube de tudo,
ficou tão pálida como morta, e foi para casa. Durante aquela noite
ficou em silêncio. Na manhã seguinte, ela disse a Gunnar, seu marido,
que ele era um covarde e um mentiroso, pois jamais atravessara o
fogo, mas mandara Sigurd no lugar dele, e fingira que tinha sido ele.
E disse que ele jamais voltaria a vê-la feliz em sua casa, bebendo
vinho, jogando xadrez, bordando com fios de ouro ou trocando palavras
carinhosas. Então ela rasgou todos seus bordados e chorou bem alto,
para que todos na casa a ouvissem. Pois seu coração se partira, e no
mesmo instante também seu orgulho. Perdera seu verdadeiro amor,
Sigurd, o Algoz de Fafnir, e casara com um homem mentiroso. 

Sigurd então veio tentar consolá-la, mas ela não quis ouvi-lo, e disse
que queria uma espada atravessando o coração dela. 

-- Não terá que esperar muito -- ele disse -- até a espada atravessar o
meu coração, pois você não viverá muito depois que eu morrer. Mas,
Brynhild querida, console-se e siga vivendo, e ame Gunnar, seu
marido, e eu lhe darei todo o ouro, o tesouro do dragão Fafnir. 

Brynhild disse:

-- Tarde demais.

Sigurd se viu tomado de tamanha tristeza que seu coração inchou em seu
peito e os anéis de sua malha de ferro se romperam.

Ele então se foi, e Brynhild decidiu matá-lo. Misturou veneno de
serpente com carne de lobo, e ofereceu-os num prato para o irmão mais
novo de seu marido; quando ele comeu ficou enlouquecido, foi até o
quarto de Sigurd enquanto ele dormia e o prendeu a cama
atravessando-o com uma espada. Mas Sigurd acordou, agarrou a espada
Gram e atirou-a no homem que fugia, partindo-o ao meio. E assim
morreu Sigurd, o Algoz de Fafnir, que nem dez homens seriam capazes
de matar num combate justo. Então Gudrun acordou, viu-o morto e ficou
aos prantos; Brynhild a ouviu e riu, mas o bom cavalo Grani se deitou
e morreu de tanto pesar. Depois Brynhild se pôs a chorar amarga e
desesperadamente, até que seu coração se partiu de vez. Então os
outros puseram em Sigurd sua armadura dourada, acenderam uma fogueira
a bordo de seu navio, e de noite nele deitaram os mortos Sigurd e
Brynhild, e o bom cavalo Grani, atearam fogo e o lançaram na água. E
o vento levou-o ardendo para o mar, as chamas brilhando na escuridão.
Sygurd e Brynhild foram cremados juntos, e a maldição do anão Andvari
se cumpriu. 

\chapter{Os salvadores da pátria\subtitulo{Edith Nesbit}}

Tudo começou quando caiu um cisco no olho de Effie. Doía muito mesmo,
dava a sensação de ter uma fagulha incandescente no olho -- só que
parecia ter pernas e asas também, feito uma mosca. Effie esfregou os
olhos e chorou -- não um choro de verdade, mas do tipo a que o olho se
entrega por conta própria, sem você precisar se sentir horrível por
dentro -- e então ela foi atrás de seu pai para que ele tirasse o
cisco de seu olho. O pai de Effie era médico, de modo que ele sabia
como tirar ciscos dos olhos -- e ele o fez muito habilmente, usando um
pincel macio embebido em óleo de rícino.

Quando tirou o cisco, ele disse:

-- Isso é muito curioso.

Effie frequentemente tivera ciscos no olho antes, e seu pai sempre
pareceu achar normal -- um tanto aborrecido e descuidado, talvez, mas
ainda assim normal. Ele nunca tinha achado curioso.

Effie ficou lá com seu lenço no olho, dizendo:

-- Não acredito que enfim saiu. -- As pessoas sempre dizem isso quando
tiram um cisco dos olhos.

-- Ah sim, saiu -- disse o doutor. -- Está aqui, no pincel. E é muito
interessante.

Effie nunca ouvira seu pai falar isso sobre qualquer coisa em que ela
tivesse parte. Ela disse:

-- O quê?

O doutor levou o pincel com muito cuidado para o outro lado da sala, e
pôs a ponta dele sob seu microscópio, e então ajustou os botões, e
espiou pela parte de cima do microscópio com um olho só.

-- Minha nossa -- ele disse. -- Minha nossa! Quatro membros bem
desenvolvidos; um longo apêndice caudal; cinco dedos, de comprimento
desigual; quase igual a um dos Lacertidae, e no entanto há traços de
asas. 

A criatura retorceu-se um pouco no óleo de rícino, e ele continuou:

-- Sim, uma asa similar à dos morcegos. Um novo espécime, sem a menor
dúvida. Effie, corra até o professor e peça para ele fazer a
gentileza de vir aqui por alguns minutos.

-- Você podia me dar seis pence, papai -- disse Effie -- porque fui eu
que trouxe para você o novo espécime. Eu tomei bastante cuidado com
ele dentro de meu olho; e meu olho está mesmo doendo.

O doutor estava tão contente com o novo espécime que ele deu a Effie
um xelim; e logo o professor apareceu. Ele ficou para o almoço, e ele
e o doutor discutiram todos contentes a tarde inteira sobre o nome e
a família da coisa que saíra do olho de Effie.

Mas na hora do jantar outra coisa aconteceu. Harry, o irmão de Effie,
pescou alguma coisa dentro de seu chá, que ele a princípio achou que
era uma pequena centopeia. Ele estava a ponto de deixá-la cair no
chão, e acabar com a vida dela da maneira usual, quando ela se
sacudiu na colher -- abriu duas asas molhadas, e se deixou cair na
toalha da mesa. E lá ficou, se esfregando com suas patas e esticando
as asas, e Harry disse:

-- Ora, é uma salamandra minúscula!

O professor debruçou-se antes que o doutor pudesse dizer uma palavra.

-- Dou-lhe meia coroa por ele, Harry, meu rapaz -- disse bem rápido; e
então catou-o cuidadosamente em seu lenço.

-- É um novo espécime -- disse. -- E superior ao seu, doutor.

Era um lagarto minúsculo, de um centímetro e meio de comprimento, com
escamas e asas.

De modo que agora tanto o doutor quanto o professor tinham o seu
espécime, e estavam ambos muito satisfeitos. Mas não demorou muito
para esses espécimes ficarem bem menos valiosos, porque na manhã
seguinte, quando o engraxate lustrava as botas do doutor, ele, de
repente, largou a escova e a graxa, gritando que tinha sido queimado.

E de dentro da bota saiu rastejando um lagarto do tamanho de um gato,
com asas grandes e brilhantes.

-- Ora -- disse Effie -- eu sei o que é isso. É um dragão, igual ao que
São Jorge matou.

E Effie estava certa. Naquela tarde Towser foi mordido no jardim por
um dragão do tamanho de um coelho, que ele tentara caçar, e na manhã
seguinte todos os jornais só falavam dos “lagartos com asas” que
estavam aparecendo por todo o país. Os jornais não os chamavam de
dragões porque, claro, ninguém acredita em dragões hoje em dia -- e
obviamente os jornais não seriam tolos a ponto de acreditar em contos
de fadas. A príncipio, havia apenas uns poucos, mas em uma ou duas
semanas o país estava simplesmente infestado de dragões de todos os
tamanhos, e era possível às vezes ver muitos deles no ar, feito um
enxame de abelhas. Eram todos iguais, exceto no tamanho. Eram verdes
com escamas, tinham quatro patas, uma cauda comprida, e grandes asas
parecidas com as dos morcegos, exceto pelo fato de que eram de um
amarelo claro, semitransparente, como as caixas de câmbio das
bicicletas.

Exalavam fogo e fumaça, como convém a todo dragão legítimo, mas ainda
assim os jornais continuaram fingindo que eram lagartos, até que o
editor do Standard foi pego por um deles e levado embora, e então o
resto do pessoal do jornal ficou sem quem lhes dissesse no que deviam
ou não acreditar. De modo que, quando o maior elefante do zoológico
foi carregado embora por um dragão, os jornais desistiram de fingir,
e saíram com a manchete “Alarmante Praga de Dragões” na primeira
página.

Você nem imagina o quanto era alarmante, e ao mesmo tempo o quanto era
irritante. Os dragões de tamanho grande eram certamente terríveis,
mas depois que se descobriu que eles iam para a cama cedo porque
tinham medo do ar frio da noite, bastava passar o dia inteiro dentro
de casa, para ficar a salvo deles. Mas os de tamanhos menores eram um
perfeito incômodo. Os do tamanho de centopeias ficavam caindo na sopa
ou na manteiga. Os do tamanho de cachorros pulavam nas banheiras, e o
fogo e a fumaça dentro deles fazia com que a água fria da torneira
virasse vapor instantaneamente, de modo que pessoas descuidadas
podiam se escaldar seriamente. Os do tamanho de pombos entravam nas
cestas de costura e nas gavetas e mordiam quem estava com pressa de
pegar uma agulha ou um lenço. Os do tamanho de uma ovelha eram mais
fáceis de se evitar, porque era possível vê-los chegando; mas quando
eles voavam pelas janelas e se aninhavam debaixo das cobertas, e não
se notava antes de entrar na cama, era sempre um choque. Os desse
tamanho não comiam gente, só alface, mas eles sempre chamuscavam os
lençóis e as fronhas horrivelmente. 

Claro, o Conselho do Condado e polícia fizeram tudo o que havia para
fazer; mas oferecer a mão da princesa a quem matasse o dragão não
adiantava. Essa solução era muito boa nos velhos tempos, quando havia
apenas um dragão e uma princesa, mas agora havia bem mais dragões do
que princesas, mesmo a Família Real sendo bem grande. Além disso,
seria um desperdício de princesas oferecê-las como recompensa a quem
matasse dragões, porque todo mundo matava tantos dragões quanto
podia, inteiramente por conta própria e sem pensar em recompensas,
apenas para tirar da frente aqueles bichos tão desagradáveis. O
Conselho do Condado encarregara-se de cremar todos os dragões
entregues entre as dez e as catorze horas, e carrinhos, carroças e
caminhões cheios de dragões mortos podiam ser vistos todos os dias da
semana fazendo uma longa fila na rua em que ficava o prédio do
Conselho. Meninos traziam carrinhos de mão cheios de dragões, e
crianças na volta da escola, no fim da manhã, paravam para deixar um
ou dois punhados de dragões que traziam em suas malas, ou no bolso,
embrulhados em seus lenços. E no entanto parecia continuar a haver
tantos dragões quanto antes. Então a polícia ergueu torres de pano e
madeira cobertas de cola. Quando os dragões ao voar batiam nessas
torres, ficavam grudados como moscas e vespas no papel mata-moscas da
cozinha; quando as torres ficavam todas cobertas de dragões, o
inspetor da polícia punha fogo nelas, queimando os dragões e o resto
junto.

E no entanto parecia haver ainda mais dragões que antes. As lojas
estavam cheias de veneno para dragão, e sabão antidragão, e cortinas
a prova de dragão para as janelas; e de fato, tudo o que era possível
foi feito.

E no entanto parecia haver ainda mais dragões que antes.

Não era muito fácil descobrir o que envenava um dragão, porque eles
comiam as coisas mais variadas. Os maiores comiam elefantes, enquanto
havia elefantes, e depois passaram a comer cavalos e vacas. Um outro
tamanho não comia nada a não ser lírios do vale, e um terceiro
tamanho comia apenas primeiros ministros se os havia disponíveis e,
em não havendo, se alimentavam generosamente de meninos que
trabalhavam de uniforme como criados. Outro tamanho vivia de tijolos,
e três deles comeram dois terços da enfermaria de South Lambeth numa
tarde.

Mas aqueles de que Effie tinha mais medo eram os tão grandes quanto a
sala de jantar; os desse tamanho comiam menininhas e menininhos.

A princípio Effie e seu irmão ficaram muito satisfeitos com as
mudanças na vida deles. Era tão divertido ficar acordado a noite
inteira em vez de ir dormir, e brincar no jardim iluminado com luz
elétrica. E soava tão engraçado ouvir a mãe dizer, quando iam para a
cama:

-- Boa noite, meus queridos, durmam bem o dia todo, e não levantem
muito cedo. Vocês não podem sair da cama até ficar bem escuro. Não
vão querer que os horríveis dragões os peguem.

Mas depois de um tempo cansaram de tudo aquilo. Queriam ver as flores
e as árvores crescendo no campo e o sol brilhando do lado de fora, em
vez de através do vidro e da cortina a prova de dragão das janelas. E
queriam brincar na grama, o que não era permitido no jardim iluminado
com luz elétrica por causa da umidade do orvalho.

E eles queriam tanto sair lá fora, uma vez que fosse, na bela,
brilhante e perigosa luz do dia, que começaram a tentar achar alguma
razão para terem de sair. Só que não gostavam de desobedecer a mãe
deles.

Mas uma manhã a mãe deles estava ocupada preparando algum novo veneno
para dragão para pôr na adega, e o pai estava fazendo um curativo na
mão do engraxate que fora arranhado por um dos dragões que gostava de
comer primeiros ministros quando disponíveis, de modo que ninguém
lembrou de dizer às crianças “não levantem até ficar escuro”.

-- Vamos agora -- disse Harry. -- Não vai ser desobediência. E eu sei
exatamente o que temos que fazer, só não sei como vamos fazer.

-- O que temos que fazer? -- disse Effie.

-- Temos que acordar São Jorge, claro -- disse Harry. -- Ele era a única
pessoa da cidade que sabia como lidar com dragões; o pessoal dos
contos de fadas não conta. Mas São Jorge é uma pessoa de verdade, ele
só está dormindo, à espera de ser acordado. Só que ninguém acredita
mais em São Jorge. Ouvi papai dizer isso.

-- Nós acreditamos -- disse Effie.

-- Claro que sim. E você não percebe, Ef, qual a razão para eles não
conseguirem acordá-lo? Não dá para acordar alguém em quem não se
acredita, dá?

Effie disse que não, mas onde eles iam encontrar São Jorge?

-- Precisamos procurá-lo -- disse Harry decididamente. -- Você vai usar
um vestido a prova de dragão, feito do mesmo pano que as cortinas. E
eu vou passar no corpo o melhor veneno para dragão, e…

Effie apertou as mãos e pulou de alegria dizendo:

-- Oh, Harry! Eu sei onde podemos achar São Jorge! Na igreja de São
Jorge, claro.

-- Hum -- disse Harry, querendo que tivesse sido ele a pensar nisso. --
Você às vezes até que é inteligente para uma menina.

Então na tarde seguinte, bem cedo, muito antes que os raios do
pôr-do-sol anunciassem a noite chegando, quando todo mundo iria
acordar para ir trabalhar, as duas crianças saíram da cama. Effie
enrolou em volta dela um xale de musselina a prova de dragões -- não
havia tempo para fazer um vestido -- e Harry fez dele uma meleca só
com veneno para dragão. Era seguramente inofensivo para crianças e
inválidos, de modo que ele não precisava se preocupar.

Eles se deram as mãos e saíram para ir até a igreja de São Jorge. Como
você sabe, há muitas igrejas de São Jorge, mas por sorte eles viraram
a esquina que levava à certa, e lá se foram sob o sol brilhante,
sentindo-se muito corajosos e aventureiros.

Não havia ninguém nas ruas a não ser dragões, e a cidade estava
simplesmente infestada deles. Por sorte nenhum era do tamanho certo
para comer menininhos e menininhas, se não talvez esta história
terminasse aqui. Havia dragões na calçada, e dragões na rua, e
dragões tomando sol nas escadarias dos prédios públicos, e dragões
alisando as asas nos telhados. A cidade estava toda verde deles.
Mesmo quando as crianças saíram da cidade e seguiram pela estrada,
elas perceberam que os campos dos dois lados estavam mais verdes que
o normal, com todas aquelas escamas e caudas; e alguns dos menores
haviam feito ninhos de asbestos nas cercas-vivas de espiriteiro
florido.

Effie segurava a mão de seu irmão apertando muito, e quando um dragão
gordo bateu as asas perto de sua orelha ela deu um berro, fazendo com
que uma revoada de dragões verdes levantasse vôo do campo, se
espalhando pelo céu. As crianças podiam ouvir o ruído das asas deles
no ar.

-- Oh, quero ir para casa -- disse Effie.

-- Não seja boba -- disse Harry. -- Com certeza você não esqueceu dos
Sete Campeões e suas princesas. As pessoas que vão ser os salvadores
da pátria nunca berram e dizem que querem ir para casa. 

-- E nós… somos? -- Effie perguntou. -- Salvadores, quero dizer.

-- Você vai ver -- disse o irmão dela, e eles seguiram em frente.

Quando chegaram à igreja de São Jorge encontraram a porta aberta e
entraram; mas São Jorge não estava lá dentro, então eles saíram para
o cemitério do lado de fora da igreja, e logo acharam a grande tumba
de pedra de São Jorge, com a figura dele esculpida em mármore do lado
de fora, com sua armadura e capacete, e as mãos cruzadas sobre o
peito.

-- Como vamos acordá-lo? -- disseram. Então Harry falou com São Jorge,
mas ele não respondia; e aí ele tentou acordar o grande matador de
dragões chacoalhando seus ombros de mármore. Mas São Jorge nem notou.

Então Effie começou a chorar, e pôs os braços em volta do pescoço de
São Jorge o melhor que pôde no mármore, que ficava muito no caminho
nas costas; ela beijou o rosto de mármore e disse:

-- Oh, caro, bom, gentil São Jorge, por favor acorde e nos ajude.

E com isso São Jorge abriu os olhos sonolentamente, se espreguiçou e
disse:

-- Qual é o problema, menininha?

Então as crianças contaram tudo o que havia para contar; ele se virou
em seu mármore e apoiou-se num cotovelo para escutar. Mas quando
soube que havia tantos dragões balançou a cabeça.

-- Assim não dá -- ele disse. -- São dragões demais para o velho Jorge.
Vocês deviam ter me acordado antes. Sempre fui a favor de lutas
justas: um homem, um dragão, era meu lema.

Bem naquele momento uma revoada de dragões passou por cima deles, e
São Jorge começou a desembainhar a espada. Mas ele balançou a cabeça
de novo e empurrou a espada de volta para o lugar dela, enquanto os
dragões iam ficando pequenos ao se a distanciarem.

-- Não posso fazer nada -- ele disse. -- As coisas mudaram desde a minha
época. Santo André me contou. Ele foi acordado na greve dos
maquinistas, e veio conversar comigo. Disse que hoje em dia tudo se
faz com máquinas; deve haver algum maneira de dar um jeito nesses
dragões. Falando nisso, como tem estado o tempo ultimamente?

A pergunta soou tão descabida e indelicada que Harry se recusou a
responder, mas Effie disse pacientemente:

-- Muito bom. Papai disse que é o verão mais quente que já houve nesse
país.

-- Ah, foi o que imaginei -- disse o campeão, pensativo. -- Bom, a única
coisa que podia ajudar… Dragões não suportam o frio e a umidade, essa
é a única coisa. Se ao menos vocês conseguissem achar as torneiras…

São Jorge estava começando a se ajeitar de novo em sua lápide de
pedra.

-- Boa noite, sinto muito não poder ajudá-los -- disse, bocejando por
trás de sua mão de mármore.

-- Ah, mas você pode -- exclamou Effie. -- Diga: que torneiras?

-- Ah, igual no banheiro -- disse São Jorge, ainda mais sonolento. -- E
tem um espelho, também: mostra o mundo todo e o que acontece nele.
São Dionísio que me contou; disse que era uma coisa muito bonita.
Sinto muito não… Boa noite.

Ele voltou a seu mármore e num instante dormia profundamente.

-- Nós nunca vamos achar as torneiras -- disse Harry. -- Escuta, não ia
ser terrível se São Jorge acordasse justo quando houvesse um dragão
por perto, um do tamanho dos que comem campeões?

Effie tirou seu xale a prova de dragão.

-- Não encontramos nenhum dos do tamanho da sala de jantar -- ela disse.
-- Acho que estamos seguros.

Então ela cobriu São Jorge com o pano, e Harry esfregou o máximo que
conseguiu de veneno para dragão na armadura de São Jorge, de modo a
deixá-lo bem seguro.

-- Podemos nos esconder na igreja até escurecer -- ele disse -- e então…

Mas naquele momento uma sombra escura se abateu sobre eles, e eles
viram que era um dragão exatamente do tamanho da sala de jantar de
casa. E viram que tudo estava perdido. O dragão arremeteu para o solo
e pegou os dois com suas garras, Effie pelo cinto verde de seda dela,
e Harry pela pontinha da parte de trás de sua jaqueta de Eton -- e
então, abrindo suas enormes asas amarelas, alçou vôo, fazendo um
barulhão igual a um vagão de terceira classe com o breque puxado.

-- Oh, Harry -- disse Effie -- me pergunto quando ele vai nos comer!

O dragão estava voando por cima de florestas e campos com o lento
bater de suas enormes asas, atravessando um quarto de milha a cada
batida delas.

Harry e Effie podiam ver o campo lá embaixo, cercas e rios e igrejas e
casas de fazendas ficando para trás embaixo deles, muito mais rápido
do que passavam no mais rápido dos trens expressos.

E o dragão continuava a voar. As crianças viram outros dragões no ar
por onde passavam, mas o dragão que era do tamanho da sala de jantar
nunca parou para falar com nenhum deles, apenas continuou voando
adiante num ritmo constante.

-- Ele sabe aonde está indo -- disse Harry. -- Ah, se a menos ele nos
largasse antes de chegar lá!

Mas o dragão segurava-os firme, e ele voou e voou e voou, até que
finalmente, quando as crianças estavam já bem tontas, aterrisou no
topo de uma montanha, com todas suas escamas retinindo. Ele então
deitou seu corpo escamoso e verde, ofegando, completamente sem fôlego
de tanto que voara. Mas suas garras estavam firmes no cinto de Effie
e na jaqueta de Harry.

Então Effie catou o canivete que Harry lhe dera de presente de
aniversário. Custara só seis pence, e ela já o tinha há um mês e até
agora tudo o que conseguira fazer com ele foi apontar lápis de
ardósia, mas de algum jeito ela fez com que aquele canivete cortasse
o cinto dela, e ela se livrou dele, deixando o dragão apenas com uma
tira de seda verde em suas garras. Mas aquele canivete jamais
cortaria a jaqueta de Harry. Depois de tentar um bocado Effie viu que
não tinha jeito e desistiu. Mas com a ajuda dela Harry conseguiu
esgueirar-se para fora das mangas, de modo que o dragão ficou apenas
com uma jaqueta de Eton na sua outra garra. Então as crianças foram
na ponta dos pés até uma rachadura nas pedras e entraram nela. Era
estreita demais para o dragão entrar também, e lá elas ficaram,
esperando para fazer caretas para o dragão quando ele tivesse
descansado o bastante para se sentar e começar a pensar em comer
eles. Ele ficou muito bravo mesmo quando eles fizeram caretas, e
exalou fogo e fumaça, mas eles correram mais para dentro da caverna
para não serem alcançados, e o dragão acabou cansando e indo embora.

Mas os dois estavam com medo de sair da caverna, então eles
continuaram andando para dentro, e a caverna foi se alargando e
ficando maior, e o chão dela era de areia macia, e quando eles
chegaram ao fim dela havia uma porta, na qual estava escrito: Sala
Universal das Torneiras. Privativo. Entrada Proibida a Todos.

Eles abriram a porta na mesma hora para espiar dentro, lembrando do
que São Jorge dissera.

-- A nossa situação não pode ficar pior do que já está -- disse Harry --
com um dragão esperando do lado de fora. Vamos entrar.

Entraram então decididamente na sala das torneiras, e fecharam a porta
atrás de si.

E agora estavam numa espécie de uma sala cavada na rocha sólida, e
havia torneiras ao longo de toda uma das paredes, e todas as
torneiras tinham rótulos feitos de plaquinhas de porcelana como os
das estâncias de águas mineirais. E como ambos eram capazes de ler
palavras de duas sílabas e às vezes até mesmo de três, eles
entenderam no mesmo instante que tinham achado o lugar onde se liga o
tempo. Havia três torneiras grandes com os rótulos: “sol”, “vento”,
“chuva”, “neve”, “granizo”, “geada”; e várias menores, dizendo: “leve
a moderada”, “chuvoso”, “brisa do sul”, “bom para as plantas
crescerem”, “patinação”, “vento sul”, “vento leste”, e por aí afora.
E a torneira grande com o rótulo “sol” estava inteira aberta. Não
dava para ver nenhuma luz do sol -- a caverna era iluminada por uma
claraboia de vidro azul -- de modo que eles acharam que a luz do sol
devia estar saindo por algum outro lugar, como acontece com a
torneira que lava as partes de baixo de certas pias de cozinha.

Então eles viram que do outro lado da sala havia apenas um grande
espelho, e olhando por ele podia se ver tudo que estava acontecendo
no mundo -- e tudo de uma vez só, também, bem diferente da maioria dos
outros espelhos. Eles viram as carroças entregando dragões mortos no
prédio do Conselho do Condado, e viram São Jorge dormindo sob o
tecido à prova de dragão. E eles viram a mãe deles em casa chorando
porque seus filhos tinham saído na terrível e perigosa luz do dia, e
estava com medo de um dragão tê-las comido. E eles viram a Inglaterra
inteira, feito um grande mapa de quebra-cabeça, verde nas partes do
campo, marrom nas cidades, e preto nos lugares onde se faz carvão e
cerâmica e facas e substâncias químicas. Cobrindo tudo, as partes
pretas, verdes e marrons, havia uma rede de dragões verdes. E eles
puderam ver que ainda era dia, e os dragões ainda não tinham ido para
a cama. Effie disse:

-- Dragões não gostam do frio.

E ela tentou fechar o sol, mas a torneira estava com defeito, e era
essa a razão de estar fazendo tanto calor, e dos dragões terem sido
chocados a ponto de quebrarem seus ovos. Então eles deixaram a
torneira do sol em paz, mas abriram a de neve e a deixaram inteira
aberta enquanto iam olhar no espelho. Lá eles viram os dragões
correndo para todos os lados como as formigas fazem se você é cruel o
suficiente para derramar água num formigueiro, o que você nunca é,
claro. E cada vez caía mais neve.

Então Effie abriu inteira a torneira da chuva, e logo os dragões
estavam se mexendo menos, e aos poucos alguns deles foram ficando
completamente imóveis, e as crianças sabiam que a água tinha apagado
o fogo dentro deles e eles estavam mortos. Eles abriram então a de
granizo -- só até a metade, com medo de quebrar as janelas das pessoas
-- e depois de um tempo não havia mais dragões se movendo para ver.

As crianças souberam então que eram de fato os salvadores da pátria.

-- Vão erguer um monumento para a gente -- disse Harry -- tão alto quanto
o do Almirante Nelson! Todos os dragões morreram.

-- Espero que o que estava esperando pela gente do lado de fora esteja
morto! -- disse Effie. -- E quanto ao monumento, Harry, não sei não. O
que eles vão fazer com esse monte de dragões mortos? Iria levar anos
e anos para queimar todos, e nem dá para queimá-los agora que eles
estão todos ensopados. Gostaria que a chuva levasse todos eles para o
mar.

Mas isso não aconteceu, e as crianças começaram a achar que não tinham
sido tão terrivelmente espertas assim, no final das contas.

-- Para que será que essa coisa velha serve? -- disse Harry. Tinha
achado uma torneira velha e enferrujada, que parecia não ser usada há
um monte de tempo. O rótulo de porcelana dela estava coberto de
poeira e teias de aranha. Quando Effie a limpou com a ponta de sua
saia (pois curiosamente ambas as crianças haviam saído sem seus
lenços) ela descobriu que estava escrito “ralo”.

-- Vamos abri-la -- ela disse. -- Quem sabe leva os dragões.

A torneira estava emperrada por não ter sido usada durante tanto
tempo, mas juntos eles conseguiram abri-la, e correram para o espelho
para ver o que acontecera.

Um enorme e redondo buraco negro já havia se aberto bem no meio do
mapa da Inglaterra, e os lados do mapa se inclinaram, de modo que a
água da chuva corria toda para o buraco.

-- Viva, viva, viva! -- gritou Effie, e correu de volta para as
torneiras para abrir tudo o que parecia molhado: “chuvoso”, “bom para
as plantas crescerem”, e até “vento sul” e “vento sudeste”, pois
ouvira seu pai dizer que esses ventos traziam chuva.

E então a chuva caía torrencial em todo o país, e grandes vagas de
água corriam para o centro do mapa, e cataratas caíam no grande
buraco redondo no meio do mapa, e os dragões estavam sendo levados
para desaparecerem cano abaixo pelo ralo, em compactas massas verdes
e espalhados grumos verdes, dragões sozinhos e dragões às dezenas,
dos que carregavam elefantes aos que caíam no chá.

Logo não havia nenhum dragão sobrando. Então eles fecharam a torneira
“ralo”, e fecharam até a metade a que estava marcada “sol” -- estava
quebrada, de modo que não puderam fechá-la de vez. Abriram “leve a
moderada” e “chuvoso”, e as duas torneiras emperraram, e não dava
mais para fechá-las, o que explica o clima do país.


\bigskip

Como eles voltaram para casa? Pela estrada de ferro de Snowdon, claro.

E a pátria ficou agradecida? Bom, a pátria estava muito molhada. E
quando enfim a pátria ficou seca de novo, estava mais interessda numa
nova invenção que usava eletricidade para assar muffins, e todos os
dragões já tinham sido quase esquecidos. Dragões não parecem assim
tão importantes depois de terem morrido e sumido todos e, você sabe,
nunca foi oferecida uma recompensa.

E o que disseram o pai e a mãe quando Effie e Harry chegaram em casa?
É o tipo da pergunta boba que vocês crianças sempre fazem. No
entanto, só dessa vez não vou me importar de responder. A mãe disse:

-- Oh meus queridos, meus queridos, vocês estão salvos! Suas crianças
travessas, como puderam ser tão desobedientes? Já para a cama!

E o pai deles, o doutor, disse:

-- Se eu soubesse o que vocês iam fazer! Queria de ter preservado um
espécime. Joguei fora o que tirei do olho da Effie. Pretendia pegar
um espécime em melhores condições. Não previ essa tão imediata
extinção dos dragões.

O professor nada disse, mas esfregou as mãos. Tinha guardado o seu
espécime -- o do tamanho de uma pequena centopeia, pelo qual dera meia
coroa a Harry -- e o tem até hoje. Você precisa conseguir que ele o
mostre para você!

\chapter{O dragão relutante\subtitulo{Kenneth Grahame }}

Tempos atrás -- pode ter sido centenas de anos atrás --, num chalé a
meio caminho entre uma aldeia e as encostas dos Downs, no sul da
Inglaterra, vivia um pastor com sua mulher e seu filho. O pastor
passava os dias -- e em certas épocas do ano as noites também -- lá em
cima nas vastas encostas, tendo apenas o sol, as estrelas e as
ovelhas de companhia, bem longe da cordial tagarelice do mundo dos
homens e das mulheres. Mas seu filho, quando não estava ajudando o
pai, e às vezes quando estava também, passava muito de seu tempo
imerso em grossos volumes que ele emprestava da gente culta afável e
dos vigários letrados da região. E seus pais gostavam muito dele, e
também tinham bastante orgulho dele, embora não o dissessem na frente
dele, de modo que deixavam-no viver sua vida e ler o quanto quisesse;
e em vez de frequentemente levar uns sopapos na cabeça, como bem
poderia ter acontecido, ele era tratado mais ou menos como um igual
por seus pais, os quais achavam sensatamente que era uma divisão de
trabalho muito justa aquela: eles forneciam o conhecimento prático, e
ele o que se encontrava nos livros. Sabiam que o conhecimento dos
livros não raro podia se revelar muito útil numa emergência, apesar
do que os seus vizinhos diziam. O que sobretudo interessava o menino
era história natural e contos de fadas, e ele ia lendo os dois
assuntos conforme apareciam, de um jeito meio ensanduichado, sem
fazer quaisquer distinções; e realmente esse seu modo de ler parece
bastante sensato.

Um dia, ao entardecer, o pastor, que fazia algumas noites andava
incomodado e preocupado, fora de seu usual equilíbrio mental, voltou
para casa tremendo todo e, sentando à mesa onde sua mulher e filho
estavam tranquilamente ocupados, ela com sua costura, ele com as
aventuras do Gigante sem coração em seu corpo, exclamou muito
agitado:

-- Está tudo acabado para mim, Maria! Nunca mais de jeito nenhum eu vou
poder subir lá nas encostas, nem uma vez mais!

-- Não fique assim -- disse sua mulher, que era muito sensata. --
Primeiro conte-nos tudo o que aconteceu, o que foi que lhe causou
toda essa agitação, e então comigo e com o filho aqui, talvez a gente
consiga esclarecer o assunto.

-- Começou algumas noites atrás -- disse o pastor. -- Sabe aquela caverna
que tem lá em cima? Eu nunca gostei dela, por algum motivo, e as
ovelhas também não, e quando as ovelhas não gostam de alguma coisa em
geral há uma razão. Bom, já faz algum tempo que dela têm vindo uns
ruídos fracos, ruídos como suspiros profundos, com uns grunhidos no
meio, e às vezes um ronco, bem lá do fundo, ronco de verdade, mas de
algum jeito não um ronco honesto, como o meu e o seu de noite, você
sabe como é.

-- Eu sei -- observou o menino, discretamente.

-- Claro, fiquei com muito medo -- o pastor continuou -- e no entanto não
consegui ficar longe. Então hoje, no fim da tarde, antes de vir
embora, fui de mansinho dar uma olhada ali pela caverna. E lá, meu
Deus!, lá estava ele, enfim o vi, tão bem quanto te vejo agora!

-- Viu quem? -- disse sua mulher, começando a se contagiar com o
aterrorizado nervosismo do marido.

-- Ora, ele, estou lhe dizendo! -- disse o pastor. -- Ele estava parado,
metade fora da caverna, e parecia estar apreciando o friozinho do fim
da tarde de um jeito meio poético. Era tão grande quanto quatro
cavalos de puxar carroça, e todo coberto de escamas brilhantes;
escamas azul-escuras na parte de cima dele, passando para um verde
suave embaixo. Quando ele respirava, sobre suas narinas havia esse
tipo de ar tremido que se vê sobre as estradas num dia de sol quente
e sem vento no verão. Ele estava com o queixo apoiado nas patas, e eu
diria que estava meditando sobre as coisas. Oh, sim, era um tipo
bastante pacífico de animal, sem ameaçar atacar ou aprontar ou fazer
qualquer coisa que não fosse certa ou direita. Admito tudo isso.
Ainda assim, o que posso fazer? Escamas, sabe, e garras, e com
certeza uma cauda, embora essa parte dele eu não tenha visto; não
estou acostumado com essas coisas e não me dou bem com elas e isso é
um fato.

O menino, que aparentemente ficara concentrado em seu livro durante a
récita de seu pai, fechou o volume, bocejou, cruzou as mãos atrás da
cabeça e disse sonolento:

-- Está tudo certo, pai. Não precisa se preocupar. É só um dragão.

-- Só um dragão? -- seu pai gritou. -- O que você quer dizer, sentado aí,
você e seus dragões? Só um dragão, essa é boa! E o que você sabe
disso?

-- Porque é, e porque eu sei -- respondeu o menino, calmamente. --
Escute, pai, você sabe que cada um de nós tem a sua
especialidade.Você sabe sobre ovelhas, e o tempo, e coisas assim; eu
sei sobre dragões. Eu sempre disse, você sabe, que aquela caverna lá
em cima era uma caverna de dragão. Eu sempre disse que em alguma
época ela devia ter sido de um dragão, e deveria ser de um dragão
agora, se as regras valem alguma coisa. Bom, agora você me diz que
ela tem um dragão, e isso que é o certo. Eu fiquei bem menos surpreso
agora do que quando você me disse que não tinha um dragão nela. As
regras sempre dão certo se você espera com calma. Agora, por favor,
deixe que eu cuide disso. Vou dar um passeio por lá amanhã de manhã,
não, de manhã não dá, tenho uma pilha de coisas para fazer, bom,
talvez no fim da tarde, se eu tiver tempo, vou até lá ter uma
conversa com ele, e você verá que vai dar tudo certo. Só o que peço,
por favor, é que você não fique se preocupando por ali sem mim. Você
não entende nada deles, e eles são muito sensíveis, sabe?

-- Ele está muito certo, pai -- disse a mãe sensata. -- Como ele disse,
dragões são a especialidade dele e não a nossa. Ele é ótimo nisso de
animais de livros, como todo mundo admite. E para falar a verdade, eu
mesma não fico nem um pouco contente, pensando no pobre daquele
animal sozinho lá em cima, sem um jantar quentinho ou alguém com quem
trocar ideias; e quem sabe poderemos fazer alguma coisa por ele; e se
ele não for respeitável, nosso menino vai logo descobrir. Ele tem um
certo jeito simpático que faz todo mundo contar tudo para ele.

No dia seguinte, depois do jantar, o menino subiu a estrada de
cascalho que leva até o topo dos Downs; e lá, claro, encontrou o
dragão, deitado preguiçosamente na relva na frente da caverna dele. A
vista dali é das mais magníficas. Para a direita e esquerda, léguas e
léguas das ondulantes encostas sem árvores dos Downs; à frente, o
vale, com as aglomerações das fazendas, e a trama de estradinhas
brancas passando por pomares e terras bem aradas e, bem longe no
horizonte, sinais das velhas cidades cinzentas. Uma brisa fresca
brincava com a superfície da relva e um pedaço prateado de uma enorme
lua estava aparecendo sobre distantes zimbros. Não era nenhuma
surpresa que o dragão parecesse estar num estado de espírito
tranquilo e satisfeito; e de fato, ao chegar perto o menino ouviu-o
ronronando com uma feliz regularidade. “Bom, vivendo e aprendendo!”
ele disse para si mesmo. “Nenhum dos meus livros jamais disse que
dragões ronronavam!” 

-- Olá, dragão -- disse o menino tranquilamente, quando chegou até onde
ele estava.

O dragão, ao ouvir alguém se aproximando, começara a fazer um esforço
para se levantar, por cortesia. Mas quando viu que era um menino,
franziu as sobrancelhas com severidade.

-- Você não venha me bater -- ele disse -- ou jogar pedras, ou espirrar
água, ou qualquer coisa assim. Não vou tolerar, já vou logo avisando.

-- Não vou jogar nada em você -- disse o menino, aborrecido, largando-se
sentado na grama perto do bicho. -- E, pelo amor de Deus, não fique me
dizendo “não isso” e “não aquilo”; eu já ouço tanto disso, e é
monótono, enjoa. Eu só passei por aqui para perguntar como vai você e
esse tipo de coisa; mas se estou atrapalhando, eu posso muito bem ir
embora. Tenho um monte de amigos, e nenhum deles pode dizer que eu
tenho o hábito de ficar insistindo quando não querem a minha
presença!

-- Não, não, não vá embora bravo -- o dragão apressou-se a dizer. -- O
fato é que eu estou muito bem mesmo aqui em cima; nunca sem alguma
ocupação, meu camarada, nunca sem alguma ocupação! Ainda assim, cá
entre nós, é um pouquinho chato às vezes.

O menino mordeu um talo de grama e ficou chupando-o. 

-- Pretende ficar muito tempo por aqui? -- perguntou, polidamente. 

-- No momento, não saberia dizer -- respondeu o dragão. -- Parece um
lugar bom o bastante; mas faz pouco tempo que estou aqui, é preciso
ver mais e refletir e considerar bem antes de se decidir por ficar
num lugar. É uma coisa muito séria, escolher onde morar. Além disso
(e agora eu vou lhe contar uma coisa que você jamais teria
adivinhado, mesmo se tivesse tentado) o fato é que eu sou um
vagabundo danado de preguiçoso!

-- Fico muito surpreso -- disse o menino, educadamente.

-- É a triste verdade -- o dragão continuou, se acomodando entre suas
patas e evidentemente encantado de ter enfim achado um ouvinte. -- E
imagino que seja isso que me fez vir parar aqui, na verdade. Veja
você, todos os outros camaradas são tão ativos e sérios e todo esse
tipo de coisa, sempre agressivos, arranjando briga, varrendo as
areias do deserto, rondando as margens do mar, e perseguindo
cavaleiros por toda a parte, devorando donzelas, e aprontando em
geral… já eu sempre gostei de fazer minhas refeições na hora certa, e
então deitar as costas em alguma rocha e tirar uma sonequinha, e daí
acordar e pensar como as coisas estão e como elas sempre estão
iguais, sabe? De modo que quando aconteceu eu fui pego de surpresa.

-- Quando o quê aconteceu, posso saber? -- perguntou o menino.

-- É precisamente isso que eu não sei -- disse o dragão. -- Suponho que a
terra espirrou, ou chacoalhou, ou o fundo caiu de algum lugar. De
qualquer modo, houve um tremor, um estrondo e uma total barafunda, e
eu fui parar milhas debaixo da terra, e fiquei completamente preso.
Bom, graças a Deus, não preciso de muito, e de qualquer maneira tinha
paz e tranquilidade, e não era sempre chamado para ir junto fazer
alguma coisa. E eu tenho uma mente tão ativa: sempre ocupada, eu lhe
asseguro! Mas o tempo foi passando, e minha vida começou a ficar numa
certa mesmice, e afinal achei que ia ser divertido abrir um caminho
para o andar de cima e ver o que os outros andavam fazendo. Então eu
cavei e escavei, e trabalhei assim e assado, e enfim consegui sair
através dessa caverna aqui. E eu gostei da região, e da vista, e das
pessoas (as poucas que eu vi) e no geral estou inclinado a ficar por
aqui.

-- Com o que sua mente está sempre ocupada? -- perguntou o menino. --
Gostaria de saber. 

O dragão ficou ligeiramente vermelho e desviou os olhos. Enfim disse,
envergonhado:

-- Você alguma vez, só de farra, já tentou fazer poesia… versos, sabe?

-- Claro que sim -- disse o menino. -- Pilhas e pilhas. E algumas delas
são bem boas, tenho certeza, só que não há ninguém por aqui que ligue
para essas coisas. Minha mãe é muito simpática e etcétera, quando eu
as leio para ela, e também meu pai, quanto a isso. Mas de algum jeito
eles não…

-- Exatamente -- exclamou o dragão -- exatamente o meu caso. Eles não… e
não há o que você possa fazer com isso. Agora, você eu vejo que tem
cultura, vi no mesmo instante, e eu gostaria de ouvir sua opinião
sincera sobre algumas coisinhas que eu fui pondo no papel, enquanto
estava por aqui. Fiquei incrivelmente satisfeito em conhecê-lo, e
espero que os outros vizinhos sejam igualmente agradáveis. Ontem à
noite mesmo esteve aqui um velho senhor muito simpático, mas ele
pareceu não querer incomodar.

-- Era o meu pai -- disse o menino. Ele é um velho senhor simpático, e
qualquer dia eu posso apresentá-lo a ele se você quiser.

-- Será que vocês dois não poderiam subir aqui amanhã para um jantar ou
algo assim? -- perguntou o dragão ansiosamente. E acrescentou
polidamente -- Claro, se vocês não tiverem nada melhor para fazer. 

-- Muito obrigado mesmo -- disse o menino -- mas não saímos para ir em
lugar nenhum sem minha mãe e, para falar a verdade, receio que ela
possa não aprová-lo muito. Veja você, não há como escapar da dura
realidade de que você é um dragão, há? E quando você fala de ficar
morando aqui, e os vizinhos, e por aí afora, não consigo evitar a
impressão de que você não percebe exatamente a sua situação. Você é
um inimigo da raça humana, afinal!

-- Não tenho um inimigo no mundo -- disse o dragão, alegremente. -- Sou
muito preguiçoso para fazer algum, para começo de conversa. E se eu
insisto em ler minhas poesias para os outros, estou sempre pronto
para ouvir as deles!

-- Ora essa! -- exclamou o menino. -- Gostaria que você tentasse
realmente entender a sua situação. Quando as outras pessoas ficarem
sabendo sobre você, vão vir todas atrás de você com lanças e espadas
e todo tipo de coisa. Você terá que ser exterminado, de acordo com a
maneira deles de ver o mundo! Você é um flagelo, uma praga, um
monstro assassino!

-- Não há uma palavra de verdade nisso tudo -- disse o dragão,
balançando a cabeça solenemente. -- Meu caráter provar-se-á íntegro
mesmo sob a mais estrita das investigações. E agora, eis um pequeno
sonetinho no qual eu estava trabalhando quando você apareceu…

-- Ih, se você se recusa a ser sensato -- exclamou o menino,
levantando-se -- eu vou embora para casa. Não, não posso dar só uma
olhadinha no soneto; minha mãe está me esperando. Vou vir vê-lo
amanhã, alguma hora, e você pelo amor de Deus veja se tenta entender
que é um flagelo pestilencial, ou vai acabar entrando na pior fria.
Boa noite!

Foi fácil para o menino deixar seus pais tranquilos em relação a seu
novo amigo. Eles sempre tinham confiado ao menino esse ramo, e
aceitaram o que ele disse sem um murmúrio. O pastor foi formalmente
apresentado, e muitos elogios e informações foram gentilmente
trocados. Sua mulher, no entanto, mesmo se dizendo disposta a fazer o
que estivesse a seu alcance -- remendar coisas, pôr ordem na caverna
ou cozinhar alguma coisinha quando o dragão ficasse absorto em seus
sonetos e esquecesse das refeições, como os machos sempre fazem -- não
pôde ser convencida a aceitá-lo completamente. O fato de que ele era
um dragão e que “eles não sabiam quem ele era” parecia contar mais
que tudo para ela. No entanto, ela não fez objeção a que seu filhinho
passasse tranquilamente todo começo de noite com o dragão, desde que
voltasse para casa até as nove; e muitas noites agradáveis eles
tiveram, sentados na relva, enquanto o dragão contava histórias dos
velhos, bem velhos tempos, quando havia dragões de monte e o mundo
era um lugar mais animado que agora, e a vida cheia de frêmitos,
sobressaltos e surpresas.

Mas o que o menino temia, todavia, não demorou a acontecer. Mesmo o
mais reservado e discreto dos dragões do mundo, se é tão grande
quanto quatro cavalos e coberto de escamas azuis, acaba não tendo
como escapar do conhecimento público. De modo que nas noites na
taverna da aldeia, o fato de que um dragão de verdade ficava cismando
numa caverna nos Downs era naturalmente o assunto das conversas.
Embora os aldeões estivessem com muito medo, estavam também bastante
orgulhosos. Era uma marca de distinção ter um dragão próprio,
acrescentava um charme à aldeia. Ainda assim, todos concordavam que
era o tipo da coisa que não se podia permitir que continuasse. O
terrível monstro devia ser exterminado, o campo devia ficar livre
daquela praga, daquele terror, daquele flagelo destruidor. O fato de
que nem mesmo um único poleiro de galinhas havia sido prejudicado
desde que o dragão chegara não era levado em conta, nem se permitia
que tivesse algo a ver com o assunto. Ele era um dragão, não se podia
negar, e se preferia não se comportar como um, isso era lá problema
dele. Mas apesar de muita valente falação, nenhum herói aparecia
disposto a pegar espada e lança para libertar a aldeia de seu
padecimento e adquirir fama imortal; e a inflamada discussão de todas
as noites acabava em nada. Enquanto isso, o dragão, um boêmio feliz,
refestelava-se na relva, apreciava os pôres-do-sol, contava anedotas
antediluvianas para o menino, e polia seus velhos versos ou cogitava
novos.

Um dia o menino, ao entrar na aldeia, percebeu que estava tudo com uma
aparência de festa, embora nenhuma constasse do calendário. Tapetes e
panos de cores alegres estavam pendurados nas janelas, os sinos da
igreja repicavam ruidosamente, a rua estreita estava cheia de flores,
e a população inteira estava se empurrando dos dois lados dela,
tagarelando, empurrando, e mandando uns aos outros ficarem mais para
trás. O menino viu um amigo da mesma idade dele e o chamou. 

-- O que está acontecendo? -- gritou. -- São atores, ou ursos, ou o
circo, ou o quê?

-- Está tudo bem -- seu amigo gritou de volta. -- Ele está vindo.

-- Quem está vindo? -- quis saber o menino, sendo empurrado na multidão.

-- Ora, quem; São Jorge, claro -- respondeu seu amigo. -- Ele ouviu falar
do dragão, e está vindo com o propósito de matar o mortífero monstro,
e nos libertar de seu horrível jugo. Nossa vai ser uma luta daquelas!

Aquela era uma novidade e tanto! O menino achou que devia se
certificar ele mesmo, e se insinuou pelo meio das pernas dos mais
velhos, xingando-os o tempo todo pelo seu mal-educado hábito de ficar
empurrando. Quando conseguiu chegar na fila da frente, ficou
esperando ansiosamente a chegada.

Logo veio da ponta mais distante da multidão o som das ovações. Em
seguida, o barulho compassado dos cascos do cavalo de batalha fez seu
coração bater mais rápido, e logo ele se descobriu gritando junto com
o resto quando, em meio aos brados de boas-vindas, os gritinhos
agudos das mulheres, o erguimento de bebês e a agitação de lenços,
São Jorge avançou lentamente pela rua. O coração do menino parou
quieto e ele respirava aos soluços, a beleza e a graça do herói iam
muito além de qualquer coisa que ele já tinha visto. Sua armadura era
incrustada em ouro, seu elmo pendia da sela, e seus densos cabelos
louros emolduravam um rosto de uma delicadeza inexprimível, até que
se via a firmeza dos olhos. Ele puxou as rédeas em frente à estalagem
e os aldeões se aglomeraram em volta com saudações, agradecimentos e
enumerações loquazes de injúrias, injustiças e opressões. O menino
ouviu a voz séria e gentil do Santo, assegurando a todos que tudo
ficaria bem agora, e que ele ia lutar por eles, reparar as injustiças
e livrá-los de seu inimigo; então ele desceu do cavalo e entrou pela
porta, e a multidão se amontoou atrás dele. Mas o menino saiu em
disparada para o morro o mais rápido que suas pernas conseguiam.

-- Está tudo acabado, dragão! -- ele gritou assim que viu o bicho. -- Ele
está vindo! Ele já está aqui! Você vai ter que criar vergonha e enfim
fazer alguma coisa!

O dragão estava lambendo suas escamas e esfregando-as com um pedaço de
flanela que a mãe do menino lhe emprestara, até ficar brilhante feito
uma grande turquesa.

-- Não seja violento, menino -- ele disse sem se virar. -- Sente e
recupere o fôlego, e tente lembrar que o sujeito antecede o
predicado, e então talvez você podessa me fazer a gentileza de dizer
quem está vindo.

-- Está certo, fique frio -- disse o menino. -- Só quero ver você
conseguir ficar na metade dessa tranquilidade depois que eu contar a
novidade. É só o São Jorge que está vindo, não passa disso; ele
chegou na aldeia faz uma meia hora. Claro que você pode dar uma surra
nele, um sujeito grandão como você! Mas achei que eu devia
preveni-lo, porque com certeza ele vai vir para cá logo cedo, e ele
tem a lança mais comprida e mais mal-encarada que você já viu!

E o menino ficou de pé e se pôs a pular para lá e para cá de puro
prazer com a perespectiva da batalha.

-- Essa não -- gemeu o dragão. -- Isso é muito desagradável. Eu não quero
vê-lo, e isso não se discute. Não tenho o menor interesse em conhecer
o sujeito. Tenho certeza que ele não é boa gente. Você precisa ir lá
dizer a ele para ir embora imediatamente, por favor. Diga a ele que
pode escrever se quiser, mas eu não vou recebê-lo. Não estou para
ninguém, no momento. 

-- Ah, dragão, dragão -- disse o menino, implorando -- não seja insensato
e teimoso. Você tem que lutar com ele alguma hora, você sabe, porque
ele é o São Jorge e você é o dragão. Melhor se livrar disso logo, e
então você pode continuar com seus sonetos. E você precisava levar um
pouco em consideração os outros também. Se por aqui anda meio chato
para você, imagine só o quanto tem sido chato para mim!

-- Meu caro homenzinho -- disse o dragão solenemente --, faça o favor de
entender, de uma vez por todas, que eu não posso lutar e eu não vou
lutar. Eu jamais lutei em toda a minha vida, e não vou começar agora,
só para você ter um espetáculo de feira. Nos velhos tempos eu sempre
deixava os outros, os que eram sérios, se encarregarem das lutas
todas, e sem dúvida essa é a razão de eu ter o prazer de estar aqui
agora.

-- Mas se você não lutar ele vai cortar fora sua cabeça! -- o menino
disse quase num soluço, desconsolado de perder tanto a luta quanto
seu amigo.

-- Ah, acho que não -- disse o dragão, com seu jeito indolente. -- Você
vai ser capaz de dar um jeito. Tenho inteira confiança em você, você
é tão bom para cuidar das coisas. Seja um bom sujeito, desça até lá e
resolva tudo. Deixo inteiramente por sua conta.

O menino percorreu todo o caminho de volta à aldeia num estado de
grande desânimo. Antes de mais nada, não haveria combate algum;
depois, seu bom e honrado amigo, o dragão, não se mostrou nem um
pouco heróico como ele teria gostado; e por fim, fosse o dragão no
fundo um herói ou não, não fazia a menor diferença, pois sem a menor
dúvida São Jorge iria cortar fora a cabeça dele. “Dar um jeito,
sei!”, ele disse amargamente para si mesmo. “O dragão trata a coisa
toda como se tivesse sido convidado para um chá e uma partida de
croqué.”

Os aldeões estavam indo para suas casas quando ele passou pela rua,
todos muito animados, e alegremente discutindo a esplêndida luta que
estava para acontecer. O menino seguiu até a estalagem e foi até o
salão principal, onde São Jorge estava sentado sozinho, considerando
suas chances na luta, e as tristes histórias de saques e violências
tão recentemente despejadas em seus ouvidos solidários.

-- Posso entrar, São Jorge? -- o menino perguntou polidamente, da porta
onde parara. -- Eu gostaria de conversar com o senhor sobre esse
assunto do dragão, se o senhor não estiver cansado dele a essa
altura.

-- Sim, pode entrar, menino -- disse o Santo gentilmente. -- Outra
história de maldade e infortúnio, receio. Sua boa mãe ou seu honrado
pai, foi algo assim, de que o tirano o privou? Ou de alguma terna
irmãzinha, ou irmão caçula? Bom, logo você será vingado.

-- Nada assim -- disse o menino. -- Há um mal-entendido em algum lugar, e
eu queria esclarecê-lo. O fato é que se trata de um bom dragão.

-- Exatamente -- disse São Jorge, sorrindo satisfeito. -- Entendo
perfeitamente. Um bom dragão. Creia-me, eu nem por um momento lamento
que ele seja um adversário à altura da minha espada, em vez de um
espécime fraco da sua tribo nociva.

-- Mas ele não é uma tribo nociva -- disse o menino todo aflito. --
Caramba, como ficam estúpidas as pessoas quando metem uma ideia na
cabeça! O que eu estou dizendo é que ele é um bom dragão, e meu
amigo, e conta as histórias mais bonitas que você já ouviu, todas
sobre os velhos tempos, quando ele era pequeno. E ele é muito gentil
com a minha mãe, e ela faria qualquer coisa por ele. E meu pai gosta
dele também, embora não tenha lá muito interesse em arte e poesia, e
sempre durma quando o dragão começa a falar sobre estilo. Mas o fato
é que ninguém consegue deixar de gostar dele assim que o conhece. Ele
é tão cativante e confiável, tão simples quanto uma criança!

-- Sente-se, e aproxime sua cadeira -- disse São Jorge. -- Gosto de
alguém que defende seus amigos, e tenho certeza que o dragão tem seus
pontos positivos, se tem um amigo feito você. Mas essa não é a
questão. Durante toda esta noite eu fiquei ouvindo, com inexprimível
angústia e pesar, histórias de assassinatos, roubos e demais
maldades; talvez um pouco exageradas demais, nem sempre inteiramente
convincentes, mas no geral constituindo uma lista de crimes dos mais
sérios. A história nos ensina que o pior dos malfeitores não raro se
mostra todo virtuoso em seu lar; e eu receio que seu culto amigo,
apesar das qualidades pelas quais ele mereceu (e devidamente) sua
afeição, tem que ser exterminado o quanto antes.

-- Ah, o senhor engoliu todas as lorotas que esse fulanos ficaram lhe
contando -- disse o menino, sem paciência. -- Ora, os aldeões daqui são
os maiores contadores de lorota de toda a região. É um fato sabido e
notório. O senhor é um forasteiro por aqui, se não já teria ouvido
falar. Tudo o que eles querem é uma luta. Eles são capazes de
qualquer coisa para conseguir lutas; é do que eles mais gostam.
Cachorros, touros, dragões; qualquer coisa, desde que seja uma luta.
Pois nesse momento mesmo há um pobre de um inocente texugo no
estábulo aqui atrás; eles iam se divertir com ele hoje, mas
resolveram guardá-lo para depois que o seu caso terminar. E não tenho
a menor dúvida que eles ficaram lhe dizendo que grande herói o senhor
é, e como tem tudo para vencer, em defesa do que é certo e justo, e
por aí afora; mas posso lhe assegurar, acabo de vir da rua, e eles
estavam apostando seis contra quatro que o dragão ganha!

-- Seis contra quatro no dragão -- São Jorge murmurou tristemente,
apoiando o rosto na mão. -- É um mundo perverso este, e às vezes eu me
ponho a achar que toda a maldade dele não fica inteiramente
engarrafada dentro dos dragões. Ainda assim, será que esse ardiloso
animal não o iludiu quanto ao verdadeiro caráter dele, para que a sua
boa impressão dele servisse de cobertura às malvadas façanhas dele?
Não, poderia até mesmo haver, neste preciso instante, alguma
desafortunada princesa aprisionada na lugúbre caverna dele?

Mas São Jorge se arrependeu do que disse assim que acabou de falar, ao
ver o quanto o menino ficara genuinamente incomodado.

-- Posso lhe assegurar, São Jorge -- ele disse --, que não há nada nem
sequer parecido na caverna dele. O dragão é um cavalheiro de verdade,
cada centímetro dele, e posso dizer que ninguém ficaria mais chocado
ou magoado do que ele, se ouvisse o senhor falando… falando desse
jeito de assuntos em relação aos quais ele tem os mais elevados
princípios!

-- Bom, talvez eu tenha sido exageradamente crédulo -- disse São Jorge.
-- Talvez eu tenha me enganado quanto ao animal. Mas o que podemos
fazer? Aqui estamos, o dragão e eu, quase frente a frente,
supostamente ávidos pelo sangue um do outro. Não consigo ver uma
saída. O que você sugere? Você não tem como dar um jeito, de algum
modo?

-- Foi exatamente o que o dragão disse -- retrucou o menino, bastante
exasperado. -- Sério, isso de vocês deixarem para eu resolver tudo…
Bom, suponho que não dá para convencê-lo a ir embora discretamente,
dá?

-- Impossível, receio -- disse o Santo. -- Completamente contra as
regras. Você sabe disso tanto quanto eu. 

-- Bom, então, escute aqui -- disse o menino. -- Ainda está cedo, será
que não daria para o senhor vir comigo encontrar com o dragão, e
tentar resolver com ele? Não é muito longe, e qualquer amigo meu será
muito bem recebido. 

-- Bem, é inteiramente contra as regras -- disse São Jorge, se
levantando -- mas realmente parece ser a coisa mais sensata a fazer.
Você está se dando a um trabalhão por conta de seu amigo -- ele
acrescentou afavelmente, ao passarem juntos pela porta. -- Mas
anime-se! Talvez afinal não precise haver luta alguma.

-- É, mas eu espero que haja sim -- disse o menino, nostalgicamente. 

-- Trouxe um amigo que quer conhecê-lo, dragão -- disse o menino, um
tanto alto.

O dragão acordou num sobressalto.

-- Eu estava só, hum, meditando sobre as coisas -- disse, do seu jeito
simples. -- Muito prazer em ser apresentado ao senhor. O tempo está
excelente, não?

-- Esse é São Jorge -- disse o menino, secamente. -- São Jorge, deixe-me
apresentá-lo ao dragão. Subimos até aqui para resolver discretamente
as coisas, dragão, e agora, pelo amor de Deus, veja se consegue ter
um pouco de simples bom senso, para ver se chegamos a alguma solução
prática e eficiente, porque já estou enjoado de opiniões e teorias
sobre a vida e tendências pessoais, e todo esse tipo de coisa. Talvez
eu deva acrescentar que minha mãe está me esperando.

-- Fico muito feliz em conhecê-lo, São Jorge -- começou o dragão um
tanto nervoso --, porque ouvi dizer que é um grande viajante, e eu
sempre fui mais do tipo caseiro. Mas posso lhe mostrar muitas
relíquias, muitos aspectos interessantes de nossa região, se fôr
ficar por algum tempo…

-- Eu acho -- disse São Jorge, de seu jeito franco e simpático -- que o
melhor a fazer é seguirmos o conselho de nosso jovem amigo, e tentar
chegar a algum entendimento, um acordo sério, sobre esse nosso
pequeno problema. Agora, você não acha que o plano mais simples,
afinal, seria apenas lutarmos, de acordo com as regras, e que vença o
melhor? Estão apostando em você, devo lhe dizer, lá na aldeia, mas eu
não ligo para isso.

-- Oh, sim, dragão, lute -- disse o menino. -- Iria nos poupar tantos
aborrecimentos!

-- Meu jovem amigo, você cale a boca -- disse o dragão, severamente. E
prosseguiu: -- Creia-me, São Jorge, não há ninguém no mundo que eu
gostaria de contentar mais do que este jovem cavalheiro aqui. Mas a
coisa toda é bobagem, e convencionalismo, e burrice popular. Não há
absolutamente nada pelo que lutar, do começo ao fim. E de qualquer
forma eu não vou, e isso encerra o assunto.

-- Mas supondo que eu o obrigue a lutar ---- disse São Jorge, um tanto
exasperado.

-- Não há como -- disse o dragão, triunfantemente. -- Eu simplesmente
entraria na minha caverna e voltaria por um tempo ao buraco donde
vim. E você logo enjoaria de ficar sentado do lado de fora esperando
eu sair para lutar. E assim que tivesse ido embora, eu sairia todo
contente, pois, para falar a verdade, gosto desse lugar e pretendo
ficar aqui!

São Jorge observou por um instante a bela paisagem em torno deles.

-- Mas esse seria um lugar maravilhoso para uma luta -- ele recomeçou,
persuasivo. -- Essas encostas do Downs de arena, eu em minha armadura
dourada fazendo contraste com suas escamas azuis! Pense só que bela
pintura daria!

-- Agora você está tentando me pegar pela minha sensibilidade artística
-- disse o dragão. -- Mas não vai funcionar. Não que não desse uma
pintura muito bela, como você disse -- acrescentou, cedendo um pouco.

-- Parece que estamos chegando mais perto de tratar de negócios -- o
menino opinou. -- Você precisa entender, dragão, que vai ser
necessário haver uma luta de algum tipo, porque você não vai querer
se enfiar de novo naquele velho buraco sujo e lá ficar até sabe-se lá
quando.

-- Podia ser arranjada -- disse São Jorge, pensativo. -- Eu tenho que
enfiar minha lança em algum lugar, claro, mas não precisa ser um
lugar que doa muito. Você é tão grande que com certeza deve haver
alguns pedaços de sobra em alguma parte. Aqui, por exemplo, bem atrás
da coxa. Não há de doer muito, só aí!

-- Assim você me faz cócegas, Jorge -- disse o dragão,
envergonhadamente. -- Não, esse lugar não vai dar de jeito nenhum.
Mesmo se não doesse, e tenho certeza que vai, e muito, iria me fazer
rir, e ia estragar tudo.

-- Vamos tentar outro lugar, então -- disse São Jorge, pacientemente. --
Debaixo do pescoço, por exemplo; todas essas dobras de pele grossa,
se eu enfiar minha lança aí você nem vai notar…

-- Sim, mas você tem certeza que consegue enfiá-la bem no lugar certo?
-- o dragão perguntou, ansiosamente.

-- Claro que tenho -- disse São Jorge, com confiança. -- Deixe comigo! 

-- É exatamente porque eu tenho que deixar com você que estou
perguntando -- disse o dragão, com uma certa irritação. -- Sem dúvida
você vai lamentar profundamente qualquer erro que fizer no calor da
hora; mas não vai lamentar nem a metade do que eu vou! No entanto,
suponho que é preciso confiar nos outros, ao longo da vida, e seu
plano parece, no geral, o melhor que há.

-- Escute aqui, dragão -- interrompeu o menino, um pouco ciumento por
seu amigo, que parecia estar ficando com a pior parte da barganha. --
Eu não estou entendendo direito como você entra nisso! Vai haver uma
luta, aparentemente, e você vai levar uma surra; o que eu quero saber
é: o que você vai ganhar com isso?

-- São Jorge -- disse o dragão --, conte a ele, por favor, o que irá
acontecer depois de eu ter sido vencido no mortal combate.

-- Bom, de acordo com as regras suponho que eu vou levá-lo em triunfo
até a praça do mercado ou o que servir como tal -- disse São Jorge.

-- Precisamente -- disse o dragão. -- E então?

-- E então haverá ovações e discursos e o resto -- continuou São Jorge.
-- E eu explicarei que você foi convertido, e viu o erro em sua
conduta, e assim por diante.

-- Certo -- disse o dragão. -- E aí?

-- Ah, e aí… -- disse São Jorge -- ora, suponho que vai haver o usual
banquete.

-- Exatamente -- disse o dragão. -- E é aí que eu entro. Olhe aqui --
continuou, dirigindo-se ao menino. -- Eu morro de tédio aqui em cima,
e ninguém realmente gosta de mim. Vou ser apresentado à sociedade,
graças à ajuda gentil de nosso amigo aqui, que está se dando a tanto
trabalho por mim; e você vai ver que tenho todas as qualidades para
que gostem de mim as pessoas que realmente contam! Bom, agora que
está tudo resolvido, e se vocês não se importarem, sou um sujeito de
modos antiquados, não gostaria de mandá-los embora, mas…

-- Lembre-se, você vai ter que fazer direito a sua parte na luta,
dragão! -- disse São Jorge, ao perceber a insinuação e se levantando
para partir. -- Quero dizer, dar investidas, exalar fogo, e por aí
afora!

-- Sei investir muito bem -- respondeu o dragão, todo confiante. -- Já
quanto a exalar fogo, é surpreendente como se perde fácil a prática;
mas farei o melhor que puder. Boa noite!

Já haviam descido o morro e estavam quase de volta na aldeia quando
São Jorge parou de repente.

-- Sabia que tinha esquecido de alguma coisa! -- disse. -- Precisa haver
uma princesa. Aterrorizada e acorrentada a uma rocha, essa coisa
toda. Menino, será que você não podia arranjar uma princesa?

O menino estava no meio de um bocejo tremendo.

-- Estou morto de cansaço -- ele reclamou -- e não posso arranjar uma
princesa, ou qualquer outra coisa, a essa hora da noite. E minha mãe
está me esperando, e faça o favor de parar de ficar me pedindo para
arranjar coisas até amanhã!


\bigskip

Na manhã seguinte as pessoas começaram a afluir para o morro bem cedo,
com suas roupas de domingo e carregando cestas com garrafas
aparecendo, todos querendo assegurar bons lugares para o combate. Não
era exatamente uma questão simples, porque era bem possível, claro,
que o dragão ganhasse, e nesse caso nem mesmo os que tinham apostado
seu dinheiro nele podiam esperar que ele tratasse os seus apoiadores
de maneira muito diferente do que o resto. Os lugares eram
escolhidos, portanto, com muito critério e com a garantia de uma fuga
rápida em caso de emergência; e a fila da frente era composta na
maioria de meninos que haviam escapado ao controle dos pais e agora
se espalhavam e rolavam na relva, ignorando as estridentes ameaças e
avisos a eles disparados por suas ansiosas mães lá atrás.

O menino tinha garantido um bom lugar na frente, bastante perto da
caverna, e estava tão ansioso quanto um diretor de palco na estreia.
Seria possível contar com o dragão? Ele podia mudar de ideia e
arruinar completamente a performance; ou então, percebendo que a
coisa fora planejada muito às pressas, sem nem mesmo um ensaio, podia
estar muito nervoso para aparecer. O menino olhou atentamente a
caverna, mas não havia nela sinal de vida ou ocupação. Teria o dragão
escapado no meio da noite?

As partes mais altas do morro estavam agora cobertas de espectadores,
e logo um som de aplausos e os acenos de lenços indicaram que havia
algo a vista que o menino, do tanto que estava para cima no lado do
dragão, não podia ainda ver. Um minuto mais e as plumas vermelhas de
São Jorge apareceram no topo do morro, quando ele foi chegando
lentamente à grande parte plana que ia até a soturna entrada da
caverna. Muito galante e belo ele estava, em seu alto cavalo de
batalha, sua armadura dourada refletindo o sol, sua enorme lança
ereta, a pequena flâmula branca, com uma cruz carmesim, tremulando na
ponta dela. Ele puxou as rédeas e ficou imóvel. As filas de
espectadores recuaram um pouco, nervosamente; e mesmo os meninos na
frente pararam de puxar o cabelo e se cutucar uns aos outros, e se
inclinaram para a frente, na expectativa.

-- Agora, dragão -- murmurou o menino impaciente, sem conseguir ficar
parado no lugar. Ele não precisava ter-se preocupado, se soubesse. As
possibilidades dramáticas da coisa haviam interessado imensamente o
dragão, e ele estava acordado desde a madrugada, preparando-se para
sua primeira aparição em público com inteira dedicação, como se os
anos tivessem voltado para trás e ele ainda fosse um pequeno
dragãozinho brincando de santo-e-dragão no chão da caverna de sua mãe
com suas irmãs; uma brincadeira em que o dragão sempre ganhava.

Um surdo múrmurio e um intermitente resfolegar fizeram-se ouvir então;
aumentando para se tornar um ensurdecedor rugido que pareceu encobrir
todo o platô. Em seguida uma nuvem de fumaça encobriu a entrada da
caverna, e do meio dela o dragão em pessoa, brilhante, azul-marinho,
magnífico, avançou solene e esplêndido; e todo mundo fez “oooh!” como
se tivessem visto um potente fogo de artifício! Suas escamas
resplandeciam, sua longa e espinhosa cauda serpenteava no chão, suas
garras arrancavam tufos de relva e os lançavam por cima de suas
costas, e fumaça e fogo eram incessantemente expelidos de suas iradas
narinas. 

-- Oh, muito bem, dragão -- o menino gritou, entusiasmado. E para si
mesmo acrescentou: “não achava que ele tivesse tanto jeito para a
coisa”.

São Jorge abaixou sua lança, inclinou a cabeça, cravou os calcanhares
no cavalo e investiu tronitruante pela relva. O dragão atacou com um
rugido e um guincho -- uma enorme e azul mistura serpenteante e
resfolegante de escamas, garras, espinhos e fogo.

-- Errou! -- gritou a multidão. Houve um instante de emaranhamento de
armadura dourada com escamas azul-turquesa e cauda espinhuda, e então
o grande cavalo, puxando por seu lado, carregou o Santo, sua lança
balançando no ar, quase até a entrada da caverna.

O dragão sentou e rosnou malevolamente, enquanto São Jorge com
dificuldade manobrava seu cavalo de volta à posição.

“Fim do primeiro round!”, pensou o menino. “Como eles se saíram bem!
Mas espero que o Santo não se entusiasme demais. No dragão dá para
confiar. Que excelente ator ele é!”

São Jorge enfim conseguiu fazer com que seu cavalo ficasse firme, e
enquanto enxugava a testa deu uma olhada em volta. Vendo o menino,
sorriu e fez um gesto afirmativo com a cabeça, e mostrou três dedos
por um instante.

“Parece que está tudo planejado”, disse o menino para si mesmo. “O
terceiro round vai ser o final, evidentemente. Gostaria que durasse
um pouco mais. E o que diabos aquele tonto do dragão está aprontando
agora?”

O dragão estava aproveitando o intervalo para para fazer uma exibição
de sua investida para a multidão. A investida dele, é necessário
explicar, consistia de correr num amplo círculo, mandando vagas e
ondas de movimento por toda a extensão de sua espinha, de suas
orelhas pontudas até o último espinho na ponta de sua comprida cauda.
Quando se é recoberto de escamas azuis, o efeito é particularmente
notável; e o menino lembrou o desejo recentemente expresso do dragão
de se tornar um sucesso na sociedade. 

São Jorge então juntou suas rédeas e começou a se mover para a frente,
abaixando a ponta de sua lança e se firmando na sela.

-- Segundo round! -- todo mundo gritou, entusiasmadamente; e o dragão
parou sua exibição, sentou-se e começou a pular de um lado para outro
com enormes saltos desajeitados, bradando feito um pele-vermelha.
Isso naturalmente desconcertou o cavalo, e ele deu uma brusca
guinada, o Santo só não caiu por ter se agarrado à crina. Quando eles
passaram a toda por ele, o dragão deu uma safada mordida na cauda do
cavalo que fez o pobre animal desembestar ensandecido encosta abaixo,
de modo que as palavras do Santo, com um dos pés fora do estribo, por
sorte ficaram inaudíveis para a plateia.

O segundo round produziu uma sonora prova de um sentimento favorável
ao dragão. Os espectadores não tardaram em apreciar um combatente que
mantinha tão bem sua posição e que claramente queria uma exibição
limpa; e muitos comentários encorajadores chegaram aos ouvidos de
nosso amigo enquanto ele se pavoneava para lá e para cá, o peito
empinado e a cauda no ar, apreciando enormemente sua nova
popularidade.

São Jorge tinha apeado e estava apertando as cilhas, e dizendo ao seu
cavalo, com um fluxo bastante oriental de metáforas, exatamente o que
pensava dele, sua família, e sua conduta na presente situação; então
o menino foi até o lado do Santo, e segurou a lança para ele.

-- Está sendo uma ótima luta, São Jorge! -- disse com um suspiro. -- Será
que não daria para fazê-la durar mais? 

-- Bem, acho melhor não -- respondeu o Santo. -- O fato é que seu velho
amigo simplório está ficando metido, agora que começaram a
aplaudi-lo, e ele bem pode esquecer o combinado e se meter a besta, e
aí não há como saber onde isso vai dar. Eu vou terminar com ele neste
round.

Ele subiu na sela e pegou a lança que o menino lhe deu.

-- Mas não fique com medo -- ele acrescentou gentilmente. -- Marquei o
lugar certo, e ele com certeza vai me ajudar o melhor que puder,
porque sabe que é sua única chance de ser convidado para o banquete.

São Jorge então encurtou sua lança, trazendo a sua empunhadura bem
debaixo do braço; e, em vez de galopar como antes, foi trotando em
direção dragão, que se agachou com a aproximação dele, sacudindo sua
cauda até fazê-la estalar no ar como um chicote. O Santo foi virando
ao chegar perto dele e circundou-o cautelosamente, mantendo os olhos
fixos no lugar marcado, enquanto o dragão, adotando tática similar,
moveu-se com cuidado no mesmo círculo, ocasionalmente fintando com a
cabeça. Os dois ficaram assim medindo o adversário, enquanto os
espectadores acompanhavam com a respiração em suspenso.

Embora o round tenha durado alguns minutos, o fim foi tão veloz que
tudo que o menino viu foi um movimento rápido como um relâmpago do
braço do Santo, e então um rodamoinho e uma confusão de espinhas,
garras, cauda, e tufos de grama arrancados. A poeira se assentou, os
espectadores correram para lá dando vivas e aplaudindo, e o menino
conseguiu ver que o dragão estava caído, preso ao solo pela lança.
São Jorge apeara e estava com um pé sobre ele.

Tudo parecia tão genuíno que o menino correu esbaforido para lá,
esperando que o bom e velho dragão não estivesse ferido de verdade.
Quando ele se aproximava, o dragão ergueu uma de suas enormes
pálpebras, piscou solenemente, e fechou-a de novo. Ele estava
firmemente preso à terra pela lança, mas o Santo a acertara no lugar
combinado, onde havia sobra, e não parecia nem fazer cócegas.

-- Não vai cortar a cabeça, Santo? -- perguntou um fulano da multidão
que aplaudia. Tinha apostado no dragão, e naturalmente estava um
pouco ressentido.

-- Bom, acho que hoje não -- respondeu São Jorge, todo simpático. --
Veja, isso pode ser feito qualquer dia. Não há a menor pressa. Acho
que é melhor voltarmos à aldeia antes, para os comes e bebes, e então
eu vou ter uma conversa séria com ele, e vocês verão que ele vai se
tornar um dragão muito diferente.

Com as palavras mágicas comes e bebes, a multidão formou uma procissão
e aguardou silenciosamente o sinal de partida. A hora de falar,
aplaudir e apostar já tinha passado, chegava a hora de agir. São
Jorge, puxando a lança com as duas mãos, soltou o dragão, que se
levantou e se sacudiu e passou os olhos por suas escamas, espinhos e
o demais, para ver se estava tudo em ordem. Então o Santo montou em
seu cavalo e tomou a dianteira da procissão, com o dragão seguindo
humildemente ao lado do menino, e com os sedentos espectadores
mantendo um respeitoso intervalo atrás dele.

Houve momentosos eventos quando todos chegaram à aldeia e se
organizaram em frente à estalagem. Depois dos comes e bebes, São
Jorge fez um discurso, no qual informou à plateia que havia removido
o terrível flagelo, com muito trabalho e sacrifício de sua parte, e
que agora não era mais para eles ficarem resmungando e imaginando que
tinham infortúnios, porque não tinham. E que eles não deviam gostar
tanto assim de lutas, porque da próxima vez podiam bem ter que
lutarem eles mesmos, que de jeito nenhum ia ser a mesma coisa. E que
havia um certo texugo no estábulo da estalagem que deveria ser
imediatamente libertado, e que ele pessoalmente ia garantir que isso
fosse feito. Então ele disse que o dragão andara refletindo sobre as
coisas, e vira que há dois lados em toda questão, e não ia mais fazer
isso, e se eles fossem bons talvez ele ficasse morando ali. De modo
que eles deviam ficar amigos, não serem preconceituosos, nem ficarem
por aí achando que sabem tudo o que há para saber, porque não sabem,
nem de longe. E ele advertiu-os quanto ao pecado de fantasiar e
inventar histórias e fazer os outros acreditarem só por serem
plausíveis e cheias de detalhes. Então ele sentou-se de novo, em meio
a muitos vivas arrependidos, e o dragão cutucou o menino e cochichou
que nem ele mesmo teria dito melhor. Então todo mundo foi embora se
arrumar para o banquete.

Banquetes são sempre agradáveis, consistindo em sua maior parte em
comer e beber; mas a melhor coisa de um banquete é que ele acontece
no fim de alguma coisa, e não é preciso mais se preocupar, e o dia
seguinte parece ficar muito longe. São Jorge estava feliz porque
tinha havido uma luta e ele não precisara matar ninguém; porque ele
realmente não gostava de matar, embora em geral tivesse que fazê-lo.
O dragão estava feliz porque tinha havido uma luta, e além de não ter
se machucado nem um pouco nela, ganhara popularidade e uma posição
garantida na sociedade. O menino estava feliz porque tinha havido uma
luta e apesar disso seus dois amigos estavam no maior dos
entendimentos. E todos os outros estavam felizes porque tinha havido
uma luta e… bem, eles não precsiavam de nenhuma outra razão para
estarem felizes. O dragão esforçou-se para dizer a coisa certa para
todo mundo, e provou-se a alma da festa; enquanto o Santo e o menino
olhavam, com a impressão de estarem assistindo a uma festa em que a
honra e a glória eram inteiramente do dragão. Mas eles não se
incomodaram, sendo boa gente, e o dragão não tinha ficado orgulhoso
ou ingrato. Ao contrário, a cada dez minutos ele se inclinava na
direção do menino e dizia, comovido:

-- Olhe, você vai me levar para casa, não vai? -- E o menino sempre
fazia que sim, embora tivesse prometido a sua mãe não voltar muito
tarde.

Por fim o banquete terminou, os convidados foram embora com muitos
boas-noites, congratulações e convites, e o dragão, que tinha ficado
para se despedir até o último deles, emergiu na rua seguido pelo
menino, enxugou a testa, suspirou, sentou na calçada e olhou para as
estrelas.

-- Foi uma noite excelente! -- murmurou. -- Excelentes estrelas!
Excelente lugarzinho! Acho que vou ficar aqui mesmo. Não estou com a
menor vontade de subir aquele maldito morro. O menino prometeu me
levar para casa. Melhor o menino fazer isso! Responsabilidade não é
minha. Responsabilidade é do menino!

E seu queixo afundou no peito largo e ele pegou tranquilamente no
sono.

-- Ah, levanta, dragão -- gritou o menino, desconsolado. -- Vocês sabe
que minha mãe está me esperando, e eu estou tão cansado. Você me fez
prometer que iria levá-lo para casa, mas eu não sabia o que você
queria dizer, senão não teria prometido! 

E o menino sentou na calçada ao lado do dragão dormindo e começou a
chorar.

A porta atrás deles se abriu, um recorte de luz iluminou a rua, e São
Jorge, que havia saído para dar uma caminhada no ar frio da noite,
viu as duas figuras sentadas ali, o enorme e imóvel dragão e o
menininho chorando.

-- Qual é o problema, menino? -- ele perguntou gentilmente, do lado
dele.

-- Ah, é esse grandessíssimo desse porco dorminhoco desse dragão! --
soluçou o menino. -- Primeiro me fez prometer levá-lo para casa, aí
disse que era melhor eu fazer isso mesmo, e então se pôs a dormir!
Mais fácil levar um monte de feno para casa! E estou tão cansado, e
minha mãe…

E ele começou a chorar de novo.

-- Não fique assim -- disse São Jorge. -- Eu vou ajudá-lo, e nós dois
vamos levá-lo para casa. Acorda, dragão! -- ele disse incisivamente,
chacoalhando o dragão pelo ombro.

O dragão abriu os olhos todo sonolento.

-- Que noite, Jorge! -- murmurou -- Que…

-- Escute aqui, dragão -- disse o Santo, firmemente. -- Aqui está esse
camaradinha esperando para levá-lo para casa, e você sabe muito bem
que ele já devia estar na cama há duas horas atrás, e o que a mãe
dele vai dizer eu nem imagino, e qualquer um menos porco e egoísta
teria feito ele ir para cama faz tempo…

-- E ele vai para a cama! -- exclamou o dragão, levantando-se. --
Pobrezinho, imagine só, ainda acordado a essa hora! É uma vergonha, é
isso o que é, e eu não acho, São Jorge, que você teve muita
consideração… Mas vamos embora imediatamente, e chega de discussões
ou papo furado. Dê-me sua mão, menino; e obrigado, Jorge, um braço
amigo morro acima é só o que eu queria!

E lá se foram os três abraçados morro acima, o Santo, o dragão e o
menino. As luzes na aldeia começaram a se apagar; mas havia as
estrelas, e uma lua no fim da noite, enquanto eles subiam os Downs
juntos. E, quando eles viraram a última esquina e desapareceram de
vista, pedaços de uma velha canção chegaram trazidos pela brisa da
noite. Não sei com certeza quem estava cantando, mas eu acho que era
o dragão!

\chapter{O último dragão\subtitulo{Edith Nesbit}}

Claro que você sabe que houve uma época em que os dragões eram tão
comuns quanto os ônibus são hoje, e quase tão perigosos. Mas como se
esperava de todo príncipe que tivesse tido uma boa educação, que
matasse um dragão e salvasse uma princesa, começou a haver cada vez
menos dragões, até ficar frequentemente bem difícil para uma princesa
achar um dragão do qual ser salva. E por fim não havia mais dragões
na França, nem na Alemanha, na Espanha, na Itália ou na Rússia.
Alguns sobraram na China, e ainda estão lá, mas são frios e de
bronze, e nunca houve nenhum, claro, na América. Mas o último dragão
de verdade vivo que sobrou estava na Inglaterra, e claro que isso foi
há muito tempo atrás, antes do que se chama História da Inglaterra
ter começado. Esse dragão vivia na Cornualha, em cavernas enormes no
meio dos rochedos, e era um excelente e enorme dragão, com nada menos
que 21 metros da ponta de seu temível focinho ao fim de sua terrível
cauda. Exalava fogo e fumaça, e fazia barulho ao andar, pois suas
escamas eram feitas de ferro. Suas asas eram iguais a meios
guarda-chuvas -- ou como asas de morcego, só que milhares de vezes
maiores. Todo mundo tinha muito medo dele, e com razão.

Acontece que o rei da Cornualha tinha uma filha, e claro que quando
ela fizesse 16 anos teria que ir enfrentar o dragão. Essas histórias
sempre se contam ao anoitecer nos quartos das crianças das famílias
reais, de modo que a Princesa sabia o que lhe esperava. O dragão não
iria comê-la, claro -- porque o príncipe ia chegar para salvá-la. Mas
ela não conseguia deixar de pensar que seria muito mais agradável não
se relacionar com o dragão, de jeito nenhum -- nem mesmo ser salva
dele.

-- Todos os príncipes que eu conheço são uns menininhos completamente
bobos -- ela disse a seu pai. -- Por que eu tenho de ser salva por um
príncipe?

-- É como sempre se faz, minha querida -- disse o Rei, tirando a coroa e
pondo-a na grama, pois estavam sozinhos no jardim, e mesmo reis
precisam relaxar às vezes.

-- Papai querido, -- disse a Princesa, quando terminou de fazer uma
coroa de margaridas e a pôs na cabeça do Rei, onde devia estar a
outra. -- Papai querido, não podíamos amarrar um dos principezinhos
bobos para o dragão ir comer, e então seria eu a matar o dragão e
salvar o Príncipe? Eu esgrimo muito melhor que qualquer um dos
príncipes que conhecemos. 

-- Que ideia inapropriada para uma dama! -- disse o Rei, e pôs de volta
a coroa, ao ver o primeiro ministro vindo com uma cesta de decretos
novinhos para ele assinar. -- Tire essa ideia da cabeça, minha filha.
Eu salvei sua mãe de um dragão, e você não pretende se achar melhor
que ela, espero.

-- Mas esse é o último dragão. É diferente de todos os outros.

-- Como assim? -- perguntou o Rei.

-- Porque ele é o último -- disse a Princesa, e foi para sua aula de
esgrima, a qual ela levava muito a sério. Ela levava a sério todas as
aulas dela; porque não conseguia desistir da ideia de lutar com o
dragão. Ela as levou tão a sério que se tornou a princesa mais forte,
corajosa, hábil e sensata da Europa. A mais bonita e a mais simpática
ela sempre fora. 

E os dias e os anos se passaram, até que enfim chegou a véspera do dia
em que a Princesa deveria ser salva do dragão. O príncipe encarregado
dessa valorosa façanha era um príncipe pálido, de olhos grandes e com
a cabeça cheia de matemática e filosofia, mas que infelizmente
negligenciara suas aulas de esgrima. Ele ia passar a noite no
palácio, e houve um banquete.

Depois do jantar a Princesa mandou seu papagaio de estimação para o
príncipe com um bilhete. Dizia: “Por favor, Príncipe, venha até o
terraço. Quero falar com você sem ninguém ouvindo. -- A Princesa.”

E lá foi ele, claro, e viu de longe o vestido prateado dela brilhando
entre as sombras das árvores, como água sob a luz das estrelas. E
quando chegou bem perto, disse:

-- Princesa, a seu serviço -- e dobrou seu joelho coberto de
tecido-bordado-em-ouro e pôs a mão em seu coração coberto de
tecido-bordado-em-ouro.

-- Você acha -- disse a Princesa sinceramente -- que você vai ser capaz
de matar o dragão?

-- Matarei o dragão -- disse o Príncipe firmemente -- ou sucumbirei
tentando. 

-- Você sucumbir não vai servir para nada -- disse a Princesa. 

-- É o mínimo que posso fazer -- disse o Príncipe.

-- O meu medo é que acabe sendo o máximo que você vai poder fazer --
disse a Princesa.

-- É a única coisa que posso fazer -- disse ele --, a menos que eu mate o
dragão.

-- Por que você deveria fazer alguma coisa por mim é o que eu não
entendo -- ela disse.

-- Mas eu quero -- ele disse. -- Você precisa saber que eu a amo mais do
que qualquer outra coisa no mundo. 

Ao dizer isso, ele pareceu tão gentil que a Princesa começou a gostar
um pouquinho dele.

-- Escute aqui -- ela disse --, ninguém mais vai sair amanhã. Você sabe
que eles me amarram numa rocha, me deixam lá; e então todo mundo sai
em disparada para casa e fecha as venezianas e as deixa fechadas até
você entrar na cidade triunfante, gritando que matou o dragão, e eu
vou atrás de você no cavalo chorando de alegria.

-- Ouvi dizer que é assim que se faz -- ele disse.

-- Bom, você me ama o suficiente para vir muito rápido e me soltar, e
então lutamos juntos contra o dragão?

-- Não seria seguro para você.

-- É muito mais seguro para nós dois se eu estiver solta com uma espada
na mão do que amarrada e indefesa. Concorda, vai.

Ele não podia negar nada a ela. Então concordou. E no dia seguinte
tudo aconteceu como ela dissera. 

Quando ele cortou as cordas que a amarravam à rocha, os dois ficaram
na montanha deserta olhando um para o outro.

-- Tenho a impressão -- o Príncipe disse -- que essa cerimônia podia ter
sido arranjada sem o dragão.

-- Sim, -- disse a Princesa -- mas como ela foi arranjada com o dragão…

-- É uma pena matar o dragão, o último do mundo -- disse o Príncipe.

-- Bom, então, não vamos -- disse a Princesa. -- Vamos ensiná-lo a comer
na nossa mão, em vez de comer princesas. Dizem que tudo pode ser
amansado através da bondade.

-- Amansá-lo assim implica dar-lhe algo para comer -- disse o Príncipe.
-- Você tem alguma coisa para comer?

Ela não tinha, mas o príncipe admitiu que tinha alguns biscoitos. 

-- O café da manhã foi tão cedo -- ele disse -- que eu achei que você ia
ficar com fome depois da luta.

-- Bem pensado -- disse a princesa, e eles pegaram um biscoito com cada
mão. E olharam aqui e olharam acolá, mas nada de dragão à vista.

-- Mas eis o rastro dele -- disse o Príncipe, apontando para onde a
rocha estava marcada e arranhada de modo a fazer uma pista até a
entrada de uma caverna escura. Era como as marcas de uma carroça numa
estrada de Sussex, misturadas com as pegadas de uma gaivota na areia
da praia. -- Veja, aqui é ele arrastando sua cauda de metal e essas
são as marcas de suas garras de aço.

-- Melhor não pensar quão duras são a cauda e as garras dele -- disse a
Princesa --, senão vou começar a ficar com medo, e você sabe que não
dá para amansar nada, não importa a sua bondade, se você estiver com
medo. Vamos. É agora ou nunca.

Ela pegou o Príncipe pela mão e os dois correram pela pista que levava
à entrada escura da caverna. Mas não correram até dentro dela. Era de
fato escura demais.

Então eles ficaram do lado de fora, e o Príncipe gritou: 

-- Ó de casa! Ei, dragão! Tem alguém em casa?

E da caverna eles ouviram uma voz respondendo e um monte de estrondos
e estalos. Soava como se um cotonifício estivesse se espreguiçando e
se levantando de seu sono.

O Príncipe e a Princesa tremeram, mas fiacaram ali firmes.

-- Dragão, dragão! -- disse a Princesa -- Por favor saia e fale com a
gente. Trouxemos um presente para você.

-- Ah, sei; conheço os presentes de vocês -- grunhiu o dragão com uma
voz trovejante. -- Uma dessas preciosas princesas, eu suponho. E eu
tenho que sair e lutar por ela. Bom, vou logo lhes dizendo, eu não
pretendo fazer isso. A um combate justo eu não diria não, uma luta
justa e sem favorecimentos; mas essas lutas arranjadas que você tem
de perder, não. É o que eu lhes digo. Se eu quisesse uma princesa, eu
iria atrás de uma, quando me desse vontade; mas eu não quero. O que
vocês acham que eu iria fazer com uma princesa, se eu pegasse uma?

-- Comê-la… não? -- a Princesa disse com um ligeiro tremor na voz. 

-- Eca! Comer uma gororoba dessas? -- disse o dragão rudemente. -- Eu nem
tocaria nesta horrorosa coisa. 

A voz da Princesa ficou mais firme.

-- Você gosta de biscoitos? -- ela perguntou.

-- Não -- grunhiu o dragão.

-- Nem mesmo aqueles caros com cobertura de chocolate?

-- Não -- grunhiu o dragão.

-- Mas então do que você gosta? -- perguntou o Príncipe. 

-- Vão embora e não me encham mais -- grunhiu o dragão, e eles puderam
ouvir ele voltando para dentro, e os rangidos e estalos de seus
movimentos ecoaram pela caverna como o barulho dos martelos a vapor
[?] do arsenal de Woolwich.

O Príncipe e a Princesa olharam um para o outro. O que eles iam fazer?
Claro que não adiantaria voltar para casa e dizer ao Rei que o dragão
não queria princesas -- porque Sua Majestade era muito antiquada e
jamais acreditaria  que um dragão moderno pudesse de alguma forma ser
diferente de um dragão à moda antiga. Eles não podiam entrar na
caverna e matar o dragão. De fato, a menos que ele atacasse a
Princesa, não ia ser nada justo matá-lo.

-- De alguma coisa ele deve gostar -- a Princesa sussurrou, e chamou-o
de novo com uma voz tão doce quanto mel e cana-de-açúcar: -- Dragão!
Dragãzinho querido!

-- O quê? -- berrou o dragão. -- Repita isso!

E eles oviram o dragão vindo na direção deles no escuro da caverna. A
princesa teve um arrepio, e disse num fiapo de voz:

-- Dragão! Dragãozinho querido!

E então o dragão saiu. O Príncipe puxou sua espada e a Princesa a dela
-- a bonita espada com cabo de prata que o Príncipe trouxera em seu
automóvel. Mas eles não atacaram; recuaram lentamente enquanto o
dragão saía, com todo seu imenso e escamoso tamanho, e se estendia
sobre a rocha, suas enormes asas meio abertas e os reflexos prateados
dele resplandescendo feito diamantes no sol. Por fim não tinham mais
para onde recuar -- o rochedo escuro atrás dele barrava a passagem -- e
com as costas contra a pedra eles ficaram com as espadas em punho,
esperando.

O dragão foi chegando cada vez mais perto, e agora eles podiam ver que
ele não estava exalando fogo e fumaça como esperavam; veio rastejando
lentamente na direção deles, meneando um pouco a cabeça feito um
filhote de cachorro que quer brincar mas não tem muita certeza se
você está bravo ou não com ele.

E então eles viram que grandes lágrimas escorriam nas bochechas
metálicas dele.

-- Qual é o problema? -- disse o Príncipe

-- Ninguém -- o dragão soluçou -- jamais me chamara de “querido” antes.

-- Não chore, querido dragão -- disse a Princesa. -- Vamos chamá-lo de
“querido” o quanto você quiser. Queremos amansá-lo.

-- Eu sou manso -- disse o dragão. -- Essa é a questão. É o que ninguém a
não ser vocês jamais descobriu. Sou tão manso que comeria de suas
mãos.

-- Comeria o que, dragão querido? -- disse a Princesa. -- Biscoitos,
talvez?

O dragão balançou devagar sua enorme cabeça.

-- Nada de biscoitos -- disse a Princesa carinhosamente. -- O que então,
querido dragão?

-- Sua gentileza me desdragona um bocado -- ele disse. -- Ninguém jamais
perguntou a nenhum de nós o que gostaríamos de comer. Sempre nos
oferecendo princesas, e aí as salvando, e nem uma vez “o que você
gostaria de beber para brindar à saúde do Rei?” Cruel e injusto, é o
que acho -- e começou a chorar de novo.

-- Mas o que você gostaria de beber para brindar à nossa saúde? -- disse
o Príncipe. -- Nós vamos nos casar hoje, não vamos, Princesa?

Ela disse que supunha que sim.

-- O que eu gostaria de beber à sua saúde? -- perguntou o dragão. -- Ah,
o senhor é decididamente um cavalheiro, é sim. Faço questão de
dizê-lo, sim senhor. Terei orgulho de beber à sua saúde e à da sua
bondosa dama só um pequeno golinho de -- a voz dele falhou. 

-- …e pensar que o senhor me pergunta assim todo amistoso -- continuou
ele. -- Sim, senhor, só um pequeno golinho de ga-ga-ga-ga-ga-gasolina,
isso que… faria bem a um dragão, senhor…

-- Tenho de monte no carro -- disse o Príncipe, e saiu descendo a
montanha feito um relâmpago. Era bom em julgar caráter, e sabia que
com aquele dragão a Princesa estaria segura.

-- Se me permite a ousadia -- disse o dragão -- enquanto o cavalheiro não
volta… talvez só para passar o tempo você poderia ter a gentileza de
me chamar de “querido” de novo, e se consentir em trocar um aperto de
garras com um pobre dragão velho que nunca foi inimigo de ninguém a
não ser dele mesmo… Seria o que faria do último dos dragões também o
mais orgulhoso dos dragões, desde o primeiro deles.

Ele estendeu uma enorme pata, e os grandes ganchos de aço que eram
suas garras se fecharam em volta da mão da Princesa tão suavemente
quanto as garras do urso do Himalaia em volta do pedaço de bolo que
você dá a ele através das grades do zoológico.

E então o Príncipe e a Princesa voltaram triunfantes ao palácio, com o
dragão seguindo-os como um cachorro de estimação. E durante todas as
festividades do casamento ninguém brindou à felicidade dos noivos com
mais sinceridade que o dragão de estimação da Princesa, o qual ela
logo batizara de Fido.

E quando o feliz casal estava instalado em seu próprio reino, Fido foi
até eles implorar que lhe permitissem ser útil.

-- Deve ter alguma coisinha que eu possa fazer -- ele disse, retinindo
suas asas e esticando suas garras. -- Minhas asas, garras e etc.
poderiam ter algum uso; isso sem nem falar em meu coração eternamente
agradecido.

De modo que o Príncipe providenciou que lhe fizessem uma sela ou um
assento especial para suas costas: enormemente comprido, era feito de
várias carrocerias de bonde soldadas juntas. 150 poltronas foram ali
instaladas, e o dragão, cujo maior prazer agora era dar prazer aos
outros, se deliciava em levar grupos de crianças para a praia. Voava
pelo ar com bastante facilidade carregando seus 150 passageirozinhos,
e ficava deitado na areia esperando pacientemente eles se disporem a
voltar. As crianças gostavam muito dele e costumavam-no chamá-lo de
“dragão querido”, uma palavra que nunca deixava de produzir lágrimas
de afeição e gratidão em seus olhos. E assim ele viveu, útil e
respeitado, até o outro dia mesmo -- quando aconteceu de alguém dizer,
ao alcance de seus ouvidos, que dragões estavam fora de moda, agora
que haviam aparecido tantas máquinas novas. Isso o incomodou tanto
que ele foi pedir ao Rei que o transformasse em alguma coisa menos
fora de moda, e o bondoso monarca no mesmo instante o transformou
numa invenção mecânica. O dragão, de fato, tornou-se o primeiro
avião.


%\chapter{Perseu e Andrômeda\subtitulo{Ovídio}}

Perseu pegou suas sandálias aladas e as pôs nos pés. Prendeu sua
espada curva à cintura, e com o movimento das asas em suas sandálias
alçou vôo e atravessou os céus. Era já o segundo dia em que estava
retornando de seu combate com a Medusa, trazendo como troféu a cabeça
daquela monstra de cabelos de serpente. Voou sobre inúmeros povos,
cujas terras se estendiam em todas as direções, até avistar as tribos
etíopes e o reino de Cepheus. Lá, a jovem e inocente Andromeda estava
a ponto de pagar injustamente pela insensata vaidade de sua mãe, a
rainha Cassiopéia, que se gabara de serem ela e sua filha muito mais
belas que as Nereidas, despertando a ira de Netuno, que enviara uma
inundação e um monstro marinho para devastar as costas do reino;
Cepheus consultara então o oráculo de Ammon, que lhe respondera que
seu reino só seria salvo se ele entregasse sua filha ao monstro
marinho.

Quando Perseu viu a princesa, com seus braços acorrentados à firme
rocha, teria achado que era uma estátua de mármore, não fosse pela
brisa que agitava os cabelos dela, e as cálidas lágrimas que
escorriam de seus olhos. Sem se dar conta, ele imediatamente se
apaixonou por ela. Encantado de ver tamanha beleza, deteve-se
maravilhado, e quase esqueceu de manter suas asas se movendo no ar.
Assim que pousou, disse:

— Você não devia estar assim acorrentada; os laços que lhe cabem são
aqueles que unem os corações daqueles que se amam! Diga-me seu nome,
eu lhe peço, e o nome de seu país, e porque está acorrentada.

A princípio, ela nada disse, pois sendo uma virgem, não ousava
dirigir-se a um homem. Teria escondido por pudor sua face com as
mãos, se elas não estivessem acorrentadas. O que ela podia fazer, ela
fez; e era deixar seus olhos se encherem ainda mais de lágrimas. Ante
a insistência de Perseu, que repetia suas perguntas, ficou com medo
que sua recusa em falar fosse entendida como admissão de culpa; então
ela lhe disse o nome do país, o dela, e também como sua mãe, uma bela
mulher, deixara sua beleza subir-lhe à cabeça.

Antes que ela terminasse, as águas se agitaram em ruidosos vagalhães,
e das profundezas do oceano veio o ameaçador monstro, tão enorme que
cobria toda a extensão das ondas. A jovem gritou. Seu desconsolado
pai estava bem perto, e sua mãe também. Os dois estavam tomados por
um profundo desespero, embora sua mãe ainda com mais razão. Nenhuma
ajuda podiam lhe oferecer, apenas as lágrimas e os lamentos que
cabiam nas circunstâncias. E assim ficaram, até que o recém-chegado
forasteiro, Perseu, se dirigisse a eles, dizendo:

— Haverá tempo de sobra para as lágrimas depois, mas para ajudá-la o
tempo é curto. Meu nome é Perseu, sou filho de Júpiter e Danae, a
qual era prisioneira numa torre quando Júpiter a fecundou sob a forma
de uma chuva de ouro. Sou aquele Perseu que venceu a Górgona de
cabelos de serpente, e que ousa viajar pelas brisas do ar com asas em
seus pés. Já seria o bastante para vocês me quererem como seu genro
se eu pedisse a mão de sua filha; mas se os deuses me ajudarem, vou
tentar acrescentar ainda outro serviço aos que já fiz. Quero esse
compromisso de vocês: que ela seja minha, se com a minha coragem eu
puder salvá-la.

Os pais dela concordaram — quem, de fato teria hesitado? Imploraram
pela ajuda dele, prometendo que além da filha deles, teria também o
reino como dote. 

Mas eis que, como um navio veloz cortando as ondas com sua proa aguda,
impulsionado pelos braços fortes da suada tripulação em seus remos, o
monstro avançou, rompendo as ondas com seu peito. Não estava mais
longe dos rochedos do que o quanto alcança o projétil de uma fronda
baleárica, quando de repente o herói, saltando da terra, lançou-se
muito alto nas nuvens. A sombra dele se projetou na superfície do
oceano, e o monstro a atacou com toda sua fúria. Então Perseu
investiu lá do alto; como age a águia de Júpiter ao ver do céu uma
serpente ao sol num deserto, e a agarra por trás, cravando suas
garras ávidas nas escamas do pescoço do réptil, para evitar o ataque
de suas cruéis presas, foi o que Perseu velozmente vindo do ar fez,
atacando o monstro pelas costas e, ao som de seus urros, enterrando
fundo no ombro direito dele sua espada curva. Atormentado por tão
profunda ferida, o monstro arqueou para o alto seu corpo, voltou a
afundá-lo nas águas, e se pôs a se virar de um lado para o outro como
um feroz porco selvagem cercado e aterrorizado por uma matilha de
cães ladrando. O herói, valendo-se de suas velozes asas, desviava-se
das terríveis mandíbulas do monstro, desferindo golpes com sua espada
curva sempre que tinha oportunidade; acertando ou as costas cobertas
de conchas ôcas da criatura, ou suas costelas, ou ainda o ponto em
que sua cauda se transformava no rabo de um peixe. Da sua boca o
monstro cuspia golfadas de água do mar tingidas com o vermelho do
sangue, e logo as asas de Perseu ficaram úmidas e pesadas com a
espuma. Sem se atrever a continuar a confiar em suas enxarcadas asas,
ele percebeu uma pedra cujo topo ficava acima da superfície quando as
águas estavam paradas, mas que o movimento das ondas cobria inteira.
Nela ele se apoiou, e segurando firme nas escarpas dela com a mão
esquerda, com a outra cravou sua espada duas, três, quatro vezes nos
flancos do monstro; tantas foram as vezes em que repetidamente o
atingiu que logo das costas do mar até a morada dos deuses no céu só
o que se ouvia eram aplausos e gritos de júbilo.

Cassiopéia e Cepheus ficaram felicíssimos, e acolheram Perseu como o
genro deles, dizendo que a casa deles lhe devia sua salvação e
continuidade. A jovem, causa e recompensa desse feito heróico, foi
libertada de suas correntes. O herói lavou suas mãos vitoriosas na
água do mar, e para que a áspera areia não prejudicasse a cabeça
cheia de serpentes da Medusa, antes de colocá-la no chão o forrou com
folhas e depois com algas-marinhas. As algas, colhidas frescas e
ainda vivas, mostraram-se sensíveis ao poder da monstruosa cabeça, e
se endureceram, adquirindo uma nova e estranha rigidez em suas folhas
e ramos. As ninfas do mar testaram então esse milagre, aplicando-o a
vários ramos, e ficaram encantadas ao ver que ele sempre se repetia;
e espalharam pelas ondas suas sementes, produzindo mais e mais dessa
nova substância. E até hoje o coral mantém essa mesma característica,
enrijecendo-se em contato com o ar; o que é uma planta debaixo da
água se torna rocha acima da superfície.

Perseu erigiu então três altares de turfa em honra a três deuses: o da
esquerda para Mercúrio, o da direita para Minerva, e no centro entre
os dois um para Júpiter. Para a deusa sacrificou uma vaca, para
Mercúrio um bezerro, e para o mais poderoso dos deuses um touro. E
sem mais demora pediu Andromeda como recompensa por sua grande
proeza, e a aceitou sem dote algum. Cupido e Hymen agitaram em frente
ao par as torchas nupciais; incenso em abundância tornou perfurmadas
as chamas, guirlandas foram penduradas no teto, e por toda a parte se
ouviram as liras e as flautas, e os cantos testemunhando e celebrando
a felicidade dos corações. Os grandes portões foram por inteiro
abertos, revelando o dourado luxo do palácio, e todos os nobres da
corte de Cepheus participaram do fausto banquete que foi servido.

\chapter{1. O príncipe e o dragão\subtitulo{Andrew Lang}}

Era uma vez um imperador que tinha três filhos. Eram bons rapazes, que
gostavam de caçar; e quase não havia dia em que algum dos três não
saísse para caçar.

Uma manhã o mais velho deles montou em seu cavalo e partiu para uma
floresta nas vizinhanças, onde havia toda espécie de animais
selvagens. Pouco depois de deixar o castelo, uma lebre saiu correndo
de uns arbustos, e cruzou a estrada. O jovem foi imediatamente atrás
dela, e a perseguiu por montanhas e vales, até a lebre se refugiar
num moinho que ficava do lado de um rio. O príncipe a seguiu, e
entrou no moinho; mas parou aterrorizado na porta pois, em vez da
lebre, o que encontrou foi um dragão, cuspindo fogo e fumaça. Ao ver
tão terrível monstro, o príncipe tentou fugir, mas uma língua feroz
enrolou-se em torno de sua cintura, e puxou-o para dentro da boca do
dragão, e ele nunca mais foi visto.

Uma semana se passou, e como o príncipe nunca mais que voltava, todo
mundo na cidade começou a ficar preocupado. Enfim, o irmão do meio
disse ao imperador que ele também gostaria de ir caçar, e que talvez
encontrasse alguma pista quanto ao desparecimento de seu irmão. Mas
mal os portões do castelo tinham se fechado atrás dele e a lebre saiu
correndo dos arbustos igual à outra vez, e fez o caçador perseguí-la
montanhas acima e vales abaixo, até chegarem ao moinho. Nele a lebre
entrou com o príncipe atrás dela e ...pronto!, em vez de lebre, lá
estava o dragão cuspindo fogo e fumaça; e lá veio a língua feroz que
se enrolou na cintura do príncipe e o carregou direto para a boca do
dragão, e ele nunca mais foi visto.

Dias e dias se passaram, e o imperador esperou e esperou por seus
filhos que nunca mais voltavam, sem conseguir pregar o olho nas
noites, pensando onde eles estariam e o que teria acontecido com
eles. O filho mais novo queria partir à procura de seus irmãos, mas o
imperador se recusava a ouví-lo, por medo de perdê-lo também. Mas o
príncipe pediu tanto para que ele o permitisse ir em busca deles, e
prometeu tanto que ia ser muito cauteloso e cuidadoso, que enfim o
imperador lhe deu permissão, e ordenou que o melhor cavalo de seus
estábulos fosse selado para ele.

Cheio de esperança o jovem príncipe partiu, mas assim que estava fora
da cidade a lebre saiu dos arbustos e foi perseguida por ele, até
chegarem ao moinho. Como antes, o animal entrou pela porta aberta,
mas dessa vez não foi seguido pelo príncipe. Mais sábio que seus
irmãos, o jovem voltou atrás, dizendo para si mesmo: “há muitas
outras lebres na floresta tão boas quanto essa que saiu dela; quando
eu as tiver caçado, volto para procurar esta.”

Por várias horas ele cavalgou por montanhas e vales, mas nada achu, e
por fim, cansado de procurar, voltou ao moinho. Lá ele encontrou uma
velha sentada, a qual ele cumprimentou cordialmente.

— Bom dia, mãezinha — disse, e a velha respondeu:

— Bom dia, filho.

— Diga-me, mãezinha — ele continuou — onde acho minha lebre?

— Filho — respondeu a velha — aquilo não era lebre, mas um dragão que
atraiu muitos homens atrás dele, e os comeu todos.

Ao ouvir essas palavras o príncipe sentiu um peso no coração, e
exclamou:

— Então meus irmãos devem ter vindo até aqui, e foram comidos pelo
dragão!

— Você adivinhou certo — respondeu a velha — e o melhor conselho que
posso lhe dar é que volte imediatamente para casa, antes que tenha o
mesmo destino. 

— Você não gostaria de vir comigo embora desse lugar hossível? — disse
o jovem.

— Também sou prisioneira dele — ela respondeu — e não posso me soltar
dessas correntes. 

— Então ouça — disse o príncipe. — Quando o dragão voltar, pergunte a
ele onde ele sempre vai quando sai daqui, e o que o faz tão forte; e
depois de conseguir extrair dele o segredo, me conte quando eu voltar
aqui.

Então o príncipe foi para casa, e a velha ficou no moinho, e assim que
o dragão voltou ela disse a ele:

— Onde você esteve dessa vez? Foi para longe que viajou?

— Sim, mãezinha, foi para longe que viajei — ele respondeu. Então a
velha começou a lisongeá-lo, e a elogiar a esperteza dele; e quando
achou que o deixara de bom humor, disse:

— Sempre me pergunto donde vem a sua força; gostaria muito que você me
contasse. Eu me inclinaria e beijaria o lugar só de puro amor! 

O dragão riu disso, e repondeu:

— Naquela pedra de lareira ali está o segredo da minha força.

Então a velha se apressou a beijar a pedra da lareira; e o dragão riu
ainda mais, e disse: 

— Sua criatura estúpida! Eu estava só brincando. Não é na pedra de
lareira, mas naquela grande árvore que está o segredo da minha força.


Então a velha se apressou a abraçar a árvore, e a beijou de coração. O
dragão deu uma enorme gargalhada ao ver o que ela estava fazendo.

— Velha tonta! — ele disse, assim que conseguiu falar — você realmente
acreditou que minha força vinha dessa árvore?

— Onde ela fica, então? — perguntou a velha, um tanto irritada, pois
não gostava que caçoassem dela. 

— Minha força — respondeu o dragão — fica longe daqui; tão longe que
você nunca a alcançará. Muito, muito longe daqui há um reino, e perto
de sua capital há um lago, e no lago há um dragão, e dentro do dragão
há um javali, e dentro do javali há um pombo, e dentro do pombo há um
pardal, e dentro do pardal fica minha força. 

E quando a velha ouviu isso, achou que não adiantava continuar a
elogiá-lo, pois nunca conseguiria tirar a força dele.

Na manhã seguinte, quando o dragão saíra do moinho, o príncipe voltou,
e a velha lhe contou tudo que a criatura tinha dito. Ele ouviu em
silêncio, e então voltou ao castelo, onde se vestiu como um pastor, e
pegando um cajado, saiu à procura de quem lhe empregasse como pastor
de ovelhas. 

Por algum tempo ele foi de aldeia em aldeia e de cidade em cidade, até
chegar à grande capital de um reino distante, cercada em três lados
por um enorme lago, que era o lago onde vivia o dragão. Como de
costume, ele foi parando todo mundo que encontrava nas ruas que
parecia estar precisando de um pastor, e pedindo para que o
empregassem, mas todos pareciam já ter seus pastores, ou não precisar
de nenhum. O príncipe estava começando a perder a esperança, quando
um homem que ouviu-o perguntar lhe disse que o melhor que ele tinha a
fazer era ir procurar o imperador, que estava precisando de alguém
para cuidar de seus rebanhos. 

— Você vaiu cuidar de minhas ovelhas? — disse o imperador, quando o
jovem se ajoelhou perante ele.

— De muito bom grado, sua majestade — respondeu o jovem, e ouviu
atentamente o que o imperador disse que era para ele fazer.

— Do lado de fora dos muros da cidade — disse o imperador — você vai
encontrar um grande lago, e nas margens dele ficam os prados mais
verdejantes de meu reino. Quando estiver levando as ovelhas para
pastar, elas vão correr direto para esses prados, e ninguém que foi
até lá jamais voltou vivo. Tenha cuidado, portanto, meu jovem, de não
deixar as ovelhas irem para onde elas quiserem, e trate de levá-las
para o lugar que achar melhor.

Com uma mesura o príncipe agradeceu ao imperador por seu aviso, e
prometeu fazer o melhor que podia para manter seguras as ovelhas
dele. Então deixou o palácio e foi até o mercado, onde comprou dois
galgos, um falcão, e uma flauta; depois disso, foi levar as ovelhas
para pastarem. No instante em que viram o lago, elas saíram trotando
o mais rápido que podiam na direção dos prados verdejantes em volta
dele. O príncipe não tentou impedí-las; ele apenas pôs seu falcão no
galho de uma árvore, suas flauta na relva, e mandou os galgos
sentarem quietos; então, arregaçando as mangas e enrolando as pernas
das calças, entrou na água gritando:

— Dragão! Dragão! Se você não é um covarde, venha lutar comigo!

E uma voz respondeu das profundezas do lago:

— Estou a sua espera, ó príncipe! 

E em seguida o dragão saiu da água, enorme e horrível de se ver. O
príncipe se atirou sobre ele e os dois se engalfinharam e lutaram até
o sol ficar alto, ao meio-dia. Então o dragão disse, sem fôlego:

— Ó príncipe, deixe-me mergulhar minha cabeça quente no lago, que eu o
arremesso até o topo do céu!

Mas o príncipe respondeu:

— Rá, rá! Meu caro dragão, não se gabe tão cedo! Pois se a filha do
imperador estivesse aqui, e me beijasse na testa, eu o jogaria ainda
mais alto! 

E subitamente o dragão o soltou e mergulhou de volta no lago.

Assim que anoiteceu, o príncipe apagou os sinais da luta, pôs seu
falcão no ombro, sua flauta debaixo do braço, e com seus galgos na
frente e o rebanho atrás voltou para a cidade. Ao passar pelas ruas,
as pessoas ficavam olhando-o surpresas, pois jamais antes havia
rebanho algum voltado do lago.

Na manhã seguinte ele se levantou cedo, e levou as ovelhas na direção
do lago. Desta vez, no entanto, o imperador enviou dois homens a
cavalo atrás dele, com ordens de vigiar o príncipe o dia inteiro.
Assim que viram as ovelhas correndo para os prados, subiram numa
colina íngreme, que dava vista para o lago. Quando o pastor chegou
lá, pôs como da outra vez sua flauta na relva, o falcão no galho de
uma árvore, e mandou os galgos sentarem quietos. Então arregaçou as
mangas e enrolou as calças, e entrou na água gritando:

— Dragão! Dragão! Se você não é um covarde, venha lutar comigo!

E o dragão respondeu: 

— Estou a sua espera, ó príncipe! 

E em seguida o dragão saiu da água, enorme e horrível de se ver. De
novo sos dois se atracaram e lutaram até o meio-dia, quando o sol
estava mais quente, e o dragão disse, sem fôlego:

— Ó príncipe, deixe-me mergulhar minha cabeça quente no lago, que eu o
arremesso até o topo do céu!

Mas o príncipe respondeu:

— Rá, rá! Meu caro dragão, não se gabe tão cedo! Pois se a filha do
imperador estivesse aqui, e me beijasse na testa, eu o jogaria ainda
mais alto! 

E subitamente o dragão o soltou e mergulhou de volta no lago.

Assim que anoiteceu, de novo o príncipe recolheu suas ovelhas, e
tocando sua flauta voltou com elas para a cidade. Assim que passou
pelos portões as pessoas saíram de suas casas para contemplá-lo
surpresas, pois nunca antes nenhum rebanho voltara do lago.

Enquanto isso os dois cavaleiros haviam retornado rapidamente, e
contaram ao imperador tudo o que viram e ouviram. O imperador ouviu
avidamente a história, e então chamou sua filha e a repetiu para ela:

— Amanhã — disse ao terminar — você acompanhará o pastor até o lago, e
então o beijará na testa como ele deseja.

Mas ao ouvir essas palavras a princesa explodiu em lágrimas, e disse
soluçando:

— O senhor quer mesmo que eu, sua única filha, vá até aquele lugar
horrível, donde muito provavelmente jamais voltarei?

— Nada tema, minha filha, tudo vai dar certo. Muitos pastores foram
até o lago e nenhum jamais retornou; mas este em dois dias lutou duas
vezes com o dragão e escapou sem nenhum ferimento. Então espero que
amanhã ele acabe matando de vez o dragão, e livrando este reino de um
monstro que matou tantos de seus mais corajosos homens.

Mal o sol começara a espiar por sobre as colinas na manhã seguinte, e
a princesa já estava ao lado do pastor, pronta para ir para o lago. O
pastor estava radiante de felicidade, mas a princesa apenas chorava
desconsoladamente.

— Enxugue suas lágrimas, eu lhe imploro — disse ele. — Basta você
fazer o que eu pedir, quando chegar a hora, e correr para beijar
minha testa, e nada haverá para temer. 

Muito contente o pastor foi tocando suas flauta enquanto conduzia seu
rebanho, só parando de vez em quando para dizer a moça que chorava ao
seu lado:

— Não chore tanto, Coração de Ouro; confie em mim e nada tema. 

E assim eles chegaram ao lago. 

— Dragão! Dragão! Se você não é um covarde, venha para termos uma
última luta!

E o dragão respondeu: 

— Estou a sua espera, ó príncipe! 

E em seguida o dragão saiu da água, enorme e horrível de se ver.
Rapidamente foi até a margem, e o príncipe arremeteu sobre ele, e os
dois se atracaram e lutaram até o meio-dia. E quando o sol estava
mais quente, o dragão gritou:

— Ó príncipe, deixe-me mergulhar minha cabeça quente no lago, que eu o
arremesso até o topo do céu!

Mas o príncipe respondeu:

— Rá, rá! Meu caro dragão, não se gabe tão cedo! Pois se a filha do
imperador estivesse aqui, e me beijasse na testa, eu o jogaria ainda
mais alto!

Mal ele disse isso e a princesa, que estava escutando, correu até ele
e o beijou na testa. Então o príncipe arremessou o dragão para o alto
até as nuvens, e quando ele caiu na terra de novo, se quebrou em mil
pedaços. Desses pedaços saltou um javali e fugiu correndo, mas o
príncipe pôs seus galgos no encalço dele, e eles pegaram o javali e o
dilaceraram em pedaços. Desses pedaços saltou uma lebre, e no mesmo
instante os galgos foram atrás dela, e a pegaram e a mataram; e da
lebre saiu um pombo. Imediatamente o príncipe soltou seu falcão, que
alçou vôo para o alto, e arremeteu sobre o pombo e o capturou para
seu amo. O príncipe cortou o pombo, e achou dentro dele o pardal,
como dissera a velha.

— Agora — exclamou o príncipe, segurando o pardal — você vai me dizer
onde posso encontrar meuis irmãos.

— Não me machuque — respondeu o pardal — e eu responderei de coração.
Atrás do castelo de seu pai há um moinho, e no moinho há três
delgados ramos. Corte esses ramos e bata em suas raizes com eles, e a
porta de ferro de um porão se abrirá. No porão você encontrará gente
o bastante, jovens e velhos, mulheres e crianças, para encher um
reino inteiro, e entre essa gente estarão seus irmãos.

A essa altura já anoitecia, então o príncipe se lavou no lago, pôs o
falcão no ombro e a flauta debaixo do braço, e com seus galgos a
frente e o rebanho atrás, marchou muito feliz para a cidade, a
princesa seguindo a todos, ainda tremendo de medo. E assim eles
passaram pelas ruas, atravessaram uma multidão surpresa, até chegarem
ao castelo.

Incógnito, o imperador havia seguido até o lago a cavalo, e se
escondera na colina, donde vira tudo o que tinha acontecido. Quanto
tudo acabou, e o poder do dragão tinha sido vencido para sempre, ele
voltou rapidamente para o castelo, e estava pronto para receber o
príncipe de braços abertos, e prometer a ele sua filha como esposa. O
casamento aconteceu com grande esplendor, e por toda uma semana a
cidade ficou cheia de lanternas coloridas, e mesas foram postas no
salão do castelo para todos que quisessem vir se servir. E quando a
festa terminou, o príncipe contou ao imperador quem ele realmente
era, e ao ouvir isso todos se rejubilaram ainda mais, e preparações
foram feitas para o príncipe e a princesa voltarem ao reino dele,
pois ele estava impaciente para libertar seus irmãos.

A primeira coisa que ele fez ao atingir sua terra natal foi correr até
o moinho, onde encontrou os três ramos como dissera o pardal. No
instante que ele bateu nas raízes uma porta de ferro se abriu, e do
porão saiu uma incontável multidão de homens e mulheres. Ele disse a
cada um que saía que estavam livres para ir para onde quisessem, e
ficou esperando junto à porta até seus irmão aparecerem. Eles ficaram
encantados de encontrá-lo, e de ouvir tudo que o príncipe fizera para
libertá-los daquele feitiço. E foram para casa com ele e o serviram
para o resto de suas vidas, pois disseram que só ele que se provara
bravo e leal merecia ser rei.

\chapter{3. Stan Bolovan\subtitulo{Andrew Lang}}

O que aconteceu uma vez, aconteceu mesmo; pois se não tivesse
acontecido essa história nunca teria sido contada.

Nos arredores de uma aldeia, bem onde os bois são deixados para pastar
e os porcos perambulam enfiando seus focinhos nas raízes das árvores,
havia uma casinha. 

Nela morava um homem com sua mulher, e a mulher passava o dia todo
triste.

— Cara esposa, o que há de errado para você ficar com sua cabeça
pendendo como uma rosa murcha? — perguntou o marido dela uma manhã. —
Você tem tudo o que quer; por que não pode viver contente como as
outras mulheres?

— Deixe-me em paz, e não tente descobrir a razão — ela respondeu,
explodindo em lágrimas; e o homem achou que não era uma boa hora para
ficar fazendo perguntas a ela, e saiu para trabalhar.

Ele não conseguia, entretanto, esquecer o assunto, e alguns dias
depois voltou a perguntar a razão da tristeza dela, mas recebeu
apenas a mesma resposta. Por fim ele não podia mais aguentar, e
tentou uma terceira vez, e então sua mulher voltou-se para ele e
respondeu:

— Por Deus! — ela exclamou — Por que você não pode deixar as coisas
como estão? Se eu lhe contasse, você iria ficar tão infeliz quanto
eu. Se ao menos você acreditasse que é melhor não saber de nada.

Mas nenhum homem jamais ficaria contente com uma resposta assim.
Quanto mais se implora para não perguntar, maior fica a curiosidade
de saber tudo.

— Bom, se você precisa tanto saber — disse enfim sua mulher — eu vou
dizer. Não temos sorte nessa casa, nenhuma sorte!

— Mas não é nossa vaca a que mais dá leite da aldeia? Não ficam nossas
árvores tão cheias de frutos quanto as colméias ficam de abelhas?
Alguém tem um milharal tão bom quanto o nosso? Olhe, você está
falando bobagem ao dizer isso.

— Sim, tudo o que você disse é verdade, mas não temos filhos.

Então Stan comprendeu, e quando um homem comprende e tem seus olhos
abertos não há mais o que fazer. Desse dia em diante a casinha
continha um homem infeliz além de uma mulher infeliz. E ver a
tristeza de seu marido deixou a mulher mais desconsolada que nunca.

E assim ficaram as coisas por um tempo.

Algumas semanas se passaram, e Stan resolveu consultar um sábio que
vivia a um dia de viagem de sua própria casa. O sábio estava sentado
em frente a porta de sua casa quando ele chegou lá, e Stan se
ajoelhou na frente dele.

— Dê-me filhos, senhor, dê-me filhos.

— Cuidado com o que você está pedindo — respondeu o sábio. — E se
filhos forem um fardo para você? Você é rico o suficiente para
alimentá-los e vestí-los? 

— Apenas faça com que eu os tenha, e eu me virarei de algum jeito! — e
o sábio lhe fez um sinal para ele partir.

Ele voltou para casa naquela noite cansado e sujo da viagem, mas com
esperança no coração. Quando estava se aproximando de sua casa, o som
de vozes chegou até seus ouvidos, e ele olhou donde vinham, e
percebeu o lugar todo cheio de crianças. Crianças no jardim, crianças
no quintal, crianças olhando de todas as janelas — pareceu ao homem
que todas as crianças do mundo haviam se juntado ali. E nenhuma era
maior que a outra, mas cada uma era menor que a outra, e cada uma era
mais barulhenta, mais petulante e mais atrevida que as outras, e Stan
parou ali e gelou de horror ao se dar conta que eram todas dele.

— Meu Deus! Quantas que há! Quantas! — murmurou para si mesmo.

— Ah, mas nem uma demais — sua mulher sorriu, aparecendo com mais uma
multidão de crianças nas barras da saia.

Mas mesmo ela descobriu que não era tão fácil cuidar de cem crianças,
e depois de alguns dias, quando elas tinham comido toda a comida que
havia na casa, começaram a gritar:

— Pai! Estou com fome! Estou com fome! 

E Stan coçou a cabeça e se perguntou o que ia fazer agora. Não que ele
achasse que havia crianças demais, pois sua vida parecia mais cheia
de alegria desde que elas apareceram, mas tinham chegado ao ponto em
que não sabia mais como alimentá-las. A vaca parara de dar leite, e
não estava ainda na época das árvores darem frutos.

— Sabe, minha velha — ele disse um dia a sua mulher — preciso ir
correr mundo e tentar trazer comida de algum jeito, embora não saiba
dizer donde vou tirá-la.

Para o homem com fome toda estrada é comprida, e ainda havia sempre a
lembrança de que tinha a fome de cem crianças para satisfazer, além
da dele mesmo.

Stan andou, andou e andou, até chegar ao fim do mundo, onde o que
existe se mistura com o que não existe, e lá ele viu, a uma pequena
distância, um ovil, com sete ovelhas dentro. Na sombra de umas
árvores estava o resto do rebanho.

Stan se escondeu, esperando que conseguiria fazer algumas escapulirem
com ele discretamente, para levá-las para alimentar sua família, mas
logo descobriu que isso não ia dar certo. Pois à meia-noite ouviu um
barulho, e era um dragão que viera voando, e levou embora um
carneiro, uma ovelha e um cordeiro, e três vacas que estavam ali por
perto. E além disso tirou também o leite de setenta e sete ovelhas,
que levou para sua velha mãe nele se banhar para recuperar sua
juventude. E isso acontecia toda noite.

O pastor se lamentava em vão: o dragão apenas ria, e Stan viu que
aquele não era um bom lugar para conseguir comida para sua família.

Mas apesar de entender que era quase impossível lutar contra um
monstro tão poderoso, a lembrança de suas crianças famintas em casa
se agarrava a ele feito um carrapicho, o qual não dá para se livrar,
e por fim ele disse ao pastor:

— O que você me daria se eu o livrasse do dragão?

— Um de cada três carneiros, uma de cada três ovelhas, um de cada três
cordeiros — respondeu o pastor.

— É uma barganha — disse Stan, embora naquele momento não soubesse
como, supondo que ele vencesse o dragão, iria conseguir levar um
rebanho tão grande para casa.

No entanto, esse problema podia ser resolvido depois. No momento, a
noite estava chegando, e ele precisava pensar como seria o melhor
jeito de lutar com o dragão.

Precisamente à meia-noite, Stan foi tomado por uma sensação horrível,
nova e estranha para ele — uma sensação que ele não encontrou
palavras para descrever a si mesmo, mas que quase o forçou a desistir
da batalha e pegar o caminho mais curto para casa. Ele se virou para
partir; mas aí lembrou das crianças, e se virou de novo.

— É você ou eu — disse Stan para si mesmo, e se posicionou junto ao
rebanho.

— Pare! — ele gritou de repente, quando o ar se encheu com o barulho
das asas do dragão descendo.

— Ora essa! — exclamou o dragão, olhando em volta. — Quem é você, e de
onde vem? 

— Eu sou Stan Bolovan, que come pedras toda noite, e durante o dia as
flores da montanha; e se você bulir com essas ovelhas eu vou entalhar
uma cruz em seu lombo.

Ao ouvir essas palavras o dargão parou bem quieto no meio da estrada,
pois sabia que havia encontrado um adversário a sua altura.

— Mas você vai ter antes que lutar comigo — ele disse com voz trêmula,
pois quando alguém o enfrentava de verdade ele não era nem um pouco
corajoso.

— Lutar com você? — retrucou Stan. — Ora, se posso matá-lo com um
sopro...

Então, pegando um grande queijo que estava a seus pés, acrescentou:

— Vá buscar uma pedra como essa no rio, para que não percamos tempo em
descobrir quem é o melhor.

O dragão fez o que Stan lhe pediu, e trouxe uma pedra do riacho.

— Você consegue tirar leite da sua pedra? — Stan perguntou.

O dragão catou sua pedra com uma mão, e espremeu-a até virar pó, mas
nenhum leite escorreu dela.

— Claro que não! — ele disse, com uma certa raiva.

— Bom, se você não consegue, eu consigo — respondeu Stan, e apertou o
queijo até leite escorrer entre seus dedos.

Quando o dragão viu aquilo, achou que já era mais que hora de voltar
para sua casa, mas Stan ficou no caminho dele.

— Ainda temos contas a acertar — ele disse — sobre o que você anda
fazendo por aqui.

E o pobre dragão estava com muito medo para se mexer, temendo que Stan
o matasse com um sopro e o enterrasse no meio das flores dos pastos
das montanhas.

— Escute — ele disse enfim. — Posso ver que você é uma pessoa muito
útil, e minha mãe está precisando de alguém como você. Digamos que
você preste serviços a ela por três dias, que duram tanto quanto um
de seus anos, e ela lhe pague com sete sacos cheio de ducados por
cada dia.

Três vezes sete sacos cheios de ducados! A oferta era muito tentadora,
e Stan não conseguiu resistir a ela. Não gastou palavras, apenas
acenou que concordava para o dragão, e os dois partiram pela estrada.

Foi uma jornada muito, muito longa, mas quando chegaram ao fim dela
encontraram a mãe do dragão, que era tão velha quanto o próprio
tempo, esperando por eles. Stan viu de longe os olhos dela brilhando
como lanternas, e quando entraram na casa viram uma enorme chaleira
no fogo, cheia de leite. Quando a velha mãe viu que seu filho voltara
de mãos vazias ficou muito brava, e fogo e chamas saíram de suas
narinas, mas antes de ela dizer qualquer coisa o dragão voltou-se
para Stan.

— Fique aqui — disse — e me espere; vou explicar as coisas para minha
mãe.

Stan já estava se arrependendo amargamente de ter vindo para tal
lugar, mas como já estava lá, não havia nada a fazer senão enfrentar
tudo calmamente, e não demonstrar que estava com medo.

— Olhe, mãe — disse o dragão assim que ficaram sozinhos. — Eu trouxe
esse homem para me livrar dele. Ele é um sujeito terrível, que come
rochas e pode tirar leite de pedras — e lhe contou o que acontecera
na noite passada.

— Ah, deixe-o comigo! — ela disse. — Nunca deixei um homem escapar por
entre meus dedos.

Então Stan teve que ficar prestando serviços à velha mãe.

No dia seguinte ela lhe disse que ele e seu filho deviam ver qual era
o mais forte, e catou uma enorme clava, envolta sete vezes com ferro.

O dragão pegou-a como se fosse uma pena, e depois de girá-la sobre sua
cabeça, atrirou-a a três milhas de distância, dizendo a Stan para
jogá-la mais longe se pudesse.

Andaram até o lugar onde caíra a clava. Stan se inclinou e
experimentou pegá-la; e então um grande medo lhe assomou, pois sabia
que ele e todas suas crianças juntas jamais conseguiriam levantar do
chão aquela clava. 

— O que você está fazendo? — perguntou o dragão.

— Eu estava pensando o quanto é bela essa clava, e me deu pena de que
seja ela a causar a sua morte.

— Minha morte? O que você quer dizer? — perguntou o dragão.

— Apenas que receio que, se atirá-la, você nunca mais verá outro
amanhecer. Você não faz idéia do quanto eu sou forte!

— Ah, não se preocupe, atire-a de uma vez.

— Se você quer isso mesmo, vamos festejar por três dias; ao menos
assim você terá três dias extras de vida.

Stan falou com tanta calma que dessa vez o dragão ficou com um pouco
de medo, embora não acreditasse que a coisa ia ser tão ruim quanto
Stan dissera.

Eles voltaram para a casa, pegaram toda a comida que acharam na
despensa da velha mãe, e voltaram para onde estava a clava. Então
Stan sentou-se no saco de mantimentos, e ficou tranquilamente
observando a lua que se punha.

— O que você está fazendo? — perguntou o dragão.

— Esperando a lua sair do meu caminho.

— Como assim? Não entendi.

— Você não está vendo que a lua está exatamente no meu caminho? Mas
claro, se você quiser, posso atirar a clava na lua.

Essas palavras deixaram o dragão inquieto pela segunda vez.

Ele tinha grande estima por aquela clava, que havia sido herança de
seu avô, e não tinha a menor vontade de que ela fosse parar na lua.

— Vou lhe dizer uma coisa — ele disse, depois de pensar um pouco. —
Não precisa atirar a clava. Eu a atiro uma segunda vez, e fica por
isso mesmo.

— Não, de jeito nenhum! — respondeu Stan. — Basta esperar a lua se
pôr.

Mas o dragão, temendo que Stan cumprisse suas ameaças, tentou toda
espécie de suborno para evitá-las, e no fim teve que prometer a Stan
sete sacos de ducados antes de enfim conseguir jogar de volta a clava
ele mesmo.

— Ah, nossa, ele é mesmo um homem forte — disse o dragão para sua mãe.
— Você acredita que foi a maior dificuldade evitar que ele atirasse a
clava na lua? 

Então a velha ficou inquieta também, só de pensar nisso! Atirar coisas
na lua não é brincadeira! Então não se falou mais na clava, e no diz
seguinte todos tinham outro assunto com que se preocupar.

— Vão buscar água! — disse a mãe, assim que amanheceu, e deu a eles
doze odres de pele de búfalo com a ordem de enchê-los até de noite.

Eles partiram na mesma hora para o riacho, e num piscar de olhos o
dragão enchera todos os doze, levou-os até a casa, e os trouxe de
volta para Stan. Stan estava cansado: mal conseguia levantar os odres
vazios, e teve um arrepio só de pensar o que iria acontecer qaundo
estivessem cheios. Mas a única coisa que fez foi tirar uma faca de
seu bolso e começar a cavar a terra perto do riacho.

— O que você está fazendo aí? Não vai carregar a água para casa? —
perguntou o dragão.

— O quê? Ora essa, isso é fácil demais! Eu vou é levar o riacho
inteiro!

Essas palavras fizeram o queixo do dragão cair. Era a última coisa que
passaria por sua cabeça, pois o riacho sempre estivera ali, desde os
tempos de seu avô.

— Vou lhe dizer uma coisa — disse. — Deixe que eu carrego os odres
para você.

— De jeito nenhum — respondeu Stan, continuando a cavar, e o dragão,
temendo que ele cumprisse sua ameaça, tentou toda espécie de suborno,
e no fim teve que prometer de novo sete sacos de ducados para fazer
Stan concordar em largar o riacho ali em paz e deixar o próprio
dragão levar a água para a casa.

No terceiro dia a velha mãe mandou Stan buscar lenha na floresta e,
como sempre, o dragão foi junto com ele.

Antes de você contar até três ele já tinha derrubado mais árvores que
Stan poderia ter cortado em toda a sua vida, e as arrumara
devidamente em fileiras. Quando o dragão tinha terminado, Stan
começou a olhar a sua volta e, escolhendo a maior das árvores, subiu
nela, onde cortou um longo cipó e com ele amarrou a ponta dela à
ponta da árvore seguinte ao lado. E assim ele fez com toda uma
fileira de árvores.

— O que você está fazendo aí? — perguntou o dragão.

— Basta você olhar para saber — repsondeu Stan, continuando calmamente
seu trabalho.

— Por que você está amarrando as árvores juntas?

— Para não ter trabalho à toa; quando eu arrancar uma, todas as outras
já virão junto. 

— Mas como você vai carregá-las para casa?

— Ora essa! Você ainda não entendeu que eu vou levar a floresta
inteira de volta comigo? — disse Stan, amarrando mais duas árvores
enquanto falava.

— Vou lhe dizer uma coisa — exclamou o dragão, tremendo de medo só de
pensar nisso — deixe que eu carrego a lenha para você, e eu lhe dou
sete vezes sete sacos cheios de ducados.

— Você é um bom sujeito, e eu concordo com sua proposta — Stan
respondeu, e o dragão carregou a lenha.

Então os três dias de serviço que era para ser contados como um ano já
haviam terminado, e a única coisa que preocupava Stan era: como levar
todos aqueles ducados de volta para casa?

Naquela noite o dragão e sua mãe tiveram uma longa conversa, mas Stan
ouviu-a inteirinha por um buraco no teto.

— Que infelicidade a nossa, mãe — disse o dragão — esse homem logo vai
nos dominar. Dê a ele o dinheiro, para podermos nos livrar dele.

Mas a velha mãe gostava de dinheiro, e a idéia não a agradava.

— Escute — ela disse — você precisa matá-lo esta noite.

— Tenho medo — disse ele.

— Não há nada a temer — retrucou a velha mãe. — Quando ele estiver
dormindo, pegue a clava e acerte-o na cabeça com ela. Vai ser fácil.

E teria sido mesmo, se Stan não tivesse ouvido tudo. E quando o dragão
e sua mãe apagaram as luzes, ele pegou a gamela dos porcos e a encheu
de terra, e a pôs em sua cama, cobrindo-a com suas roupas. Então ele
se escondeu sob a cama, e se pôs a roncar bem alto.

Logo o dragão entrou silenciosamente no quarto, e deu um tremendo
golpe no lugar onde a cabeça de Stan deveria estar. Stan gemeu bem
alto debaixo da cama, e o dragão saiu tão silenciosamente quanto
entrara. Assim que ele fechou a porta, Stan tirou dali a gamela dos
porcos, e deitou-se em seu lugar, depois de ter deixado tudo limpo e
arrumado; mas foi esperto o bastante para não pregar os olhos naquela
noite.

Na manhã seguinte ele veio para a sala quando o dragão e sua mãe
estavam tomando café da manhã.

— Bom dia — disse.

— Bom dia. Dormiu bem?

— Ah, muito bem, mas sonhei que uma pulga havia me mordido, e ainda a
estou sentindo.

O dragão e sua mãe se entreolharam.

— Você ouviu só? — ele sussurrou. — Ele falou de uma pulga. E eu
quebrei minha clava na cabeça dele.

Dessa vez a mãe ficou com tanto medo quanto seu filho. Não havia nada
a fazer com um homem como esse, e ela se apressou a encher os sacos
com os ducados, para se livrar de Stan o mais rápido possível. Mas
por sua parte Stan estava tremendo feito vara verde, pois não
conseguia levantar nem mesmo um saco do chão. Então ele ficou ali
parado olhando para eles.

— O que você está esperando aí? — perguntou o dragão.

— Ah, eu estava esperando aqui porque acabou de me ocorrer que
gostaria de ficar a serviço de vocês mais um ano. Tenho vergonha de
chegar em casa e verem que eu trouxe tão pouco. Tenho certeza que vão
dizer “olha só o Stan Bolovan, que em um ano acabou ficando tão fraco
quanto um dragão”.

Nesse momento uma exclamação de pasmo escapou tanto do dragão quanto
de sua mãe, que declarou que ela ia lhe dar sete vezes ou mesmo sete
vezes sete vezes o número de sacos se ele fosse embora.

— Vou lhe dizer uma coisa — enfim Stan disse. — Estou vendo que não
querem que eu fique, e não quero incomodá-los. Eu irei embora
imediatamente, mas sob a condição do dragão carregar para mim o
dinheiro até em casa; assim não passarei vergonha frente a meus
amigos.

Mal acabara de falar e o dragão já agarrara os sacos e os empilhara em
suas costas. Então ele e Stan partiram.

O caminho de volta, se na verdade não muito comprido, ainda assim
demorou um bocado para Stan, mas enfim ele ouviu as vozes de suas
crianças, e parou de repente. Não queria que o dragão ficasse sabendo
onde ele morava, temendo que algum dia ele viesse recuperar seu
tesouro. Não haveria nada que ele poderia dizer para se livrar do
monstro? De repente uma idéia lhe veio à cabeça, e ele se virou.

— Não sei bem o que fazer — disse. — Tenho cem filhos, e tenho medo
que eles possam machucá-lo, pois estão sempre prontos a entrar numa
briga. Em todo o caso, vou fazer todo o possível para protegê-lo.

Cem crianças! Isso não era brincadeira mesmo! O dragão até deixou cair
os sacos, tanto o seu terror, mas tratou de catá-los de novo. Foi
então que as crianças, que não tinham comido nada desde que seu pai
partira, vieram correndo na direção dele, brandindo facas na mão
direita e garfos na esquerda, e gritando “Papai! Dê-nos carne de
dragão! Queremos carne de dragão!”

Ao ver essa terrível cena o dragão não esperou nem mais um instante:
largou os sacos onde estava e saiu voando o mais rápido que podia, e
tão aterrorizado ficou com o destino do qual escapou por pouco que
desse dia em diante nunca mais teve coragem de mostrar a cara pelo
mundo de novo.

\chapter{Jörmungandr, a serpente de Midgard\subtitulo{Snorri Sturlson}}

O deus escandinavo Loki teve três filhos com a gigante Angboda:
Fenrir, Jörmungandr e Hel. Fenrir era um lobo terrível, que cresceu
até ter um tamanho gigantesco, e causava tanta destruição que os
outros deuses tiveram que prendê-lo, mas tão poderoso era que só
depois de muitas tentativas conseguiram; Hel foi por Odin encarregada
do reino subterrâneo dos mortos; e quanto a Jörmungandr, a serpente
de Midgard, era tão horrível que assim que Odin a viu atirou-a no
fundo do mar. 

No meio do oceano essa serpente cresceu tanto que acabou dando a volta
na terra toda, com sua boca mordendo sua cauda. E lá ficaria até o
Ragnarok, a batalha final ou o crepúsculo dos deuses, quando faria o
mar invadir a terra, ao avançar sobre ela com mosntruosa ira,
soprando seu veneno que se espalharia por todo o ar e toda a água,
até encontrar Thor que a enfrentaria num combate mortal para ambos;
pois após derrotá-la e matá-la, a apenas nove passos dali Thor cairia
morto, por causa do veneno de Jörmungandr.

Mas muito antes disso Thor a encontraria duas outras vezes. 

A primeira foi no castelo do gigante Utgard-Loki, em que Thor chegou
com seus companheiros Loki e Thjálfi. Ao vê-los entrar o gigante os
recebeu com o sorriso de desprezo, e disse:

— É esse bebezinho o poderoso Thor? Ainda assim, talvez você possa se
mostrar maior do que aparenta para mim; de que tipo de proezas você e
seus companheiros são capazes? Pois só é benvindo aqui quem é capaz
de feitos nos quais ultrapassem a maioria dos mortais. 

Quem respondeu primeiro foi o último a entrar, Loki:

— Sou capaz de algo que estou pronto a demonstrar: não há ninguém aqui
que consiga comer mais rápido do que eu. 

Utgard-Loki achou que seria um grande feito, e chamou um certo Logi
para competir com ele. Uma gamela cheia de carne foi posta entre os
dois, e cada um começou a comer o mais rápido que podia de cada lado,
até se encontrarem no meio dela. A essa altura Loki tinha comido toda
a carne e deixado apenas os ossos; mas Logi tinha comido não só a
carne como também os ossos e a própria gamela; e todos acharam que
Loki tinha perdido. 

Então Utgard-Loki perguntou de qual proeza o outro jovem seria capaz,
e Thjálfi

respondeu que ganharia na corrida de quem quer que ele chamasse.
Saíram todos para onde havia uma boa pista para correr, e um certo
Hugi foi convocado para competir com Thjálfi. Três vezes eles
correram, e nas três Thjálfi perdeu, em cada uma delas ficando ainda
mais para trás que da outra. 

Era então a vez de Thor, e ele disse que beberia mais que qualquer um.
Utgard-Loki mandou um servo trazer um chifre que ali era usado como
cálice, e disse:

— Quem bebe bem esvazia esse chifre de uma vez só; alguns até precisam
de duas, mas ninguém é tão fraco que precise de três.

Thor olhou o chifre, e nem lhe pareceu muito grande; embora um tanto
comprido. Mas ele estava com muita sede, e começou a beber, e foi
engolindo vigorosamente. Só que quando seu folego acabou, e ele olhou
o quanto já tinha bebido do chifre, pouco parecia ter diminuído do
que havia. Utgard-Loki disse:

— Esperava que Thor fosse capaz de beber mais; mas suponho que vá
precisar de só mais uma vez para terminar.

Thor nada disse, e se pôs a beber de novo; dessa vez, era ao menos bem
visível que o copo estava mais vazio, embora parecesse ter descido
menos que da outra. O gigante o provocou a uma terceira tentativa;
mas também nessa nem perto de começar a esvaziar o chifre Thor
chegou, e ele desistiu. Utgard-Loki desafiou-o então a outros feitos,
já que daquele ele não se mostrara à altura. Thor aceitou, e
Utgard-Loki disse:

— Os rapazes aqui gostam de fazer algo que nem é lá grande coisa:
erguer do chão o meu gato. Eu nem iria propor isso a você, se não
tivesse visto que você é bem menos capaz do que eu imaginava. 

Então pulou no chão do salão um gato cinzento, muito grande; e Thor
foi até ele e pôs as mãos sob a barriga dele para erguê-lo. Mas o
gato só foi arqueando o corpo enquanto Thor esticava os braços; e
quando Thor chegou ao máximo de altura que atingia, tudo o que
conseguiu foi levantar uma única pata do gato. 

— Foi como eu previa — disse Utgard-Loki. — O gato é muito grande,
enquanto Thor é pequeno demais, perto dos homens enormes que há aqui.


— Por mais que você me ache pequeno — Thor respondeu — estou agora
disposto a lutar com quem você quiser chamar, pois fiquei com raiva. 

Utgard-Loki, olhando em volta, disse:

— Não vejo aqui nenhum homem que não vá achar uma desonra lutar com
você. Mas posso chamar minha velha babá, Elli, e você lutará com ela
se quiser. Ela já venceu homens não menos fortes que você. 

No mesmo instante entrou no salão uma velha, de idade muito avançada.
Nem é preciso se estender muito sobre o que aconteceu; tudo o que
Thor tentava, ela reagia com igual disposição, até deixá-lo de
joelhos e Utgard-Loki interromper a luta. E já era tarde na noite, e
depois de comer e beber todos se recolheram.

Na manhã seguinte, Thor e seus companheiros estavam já de partida,
fora do castelo, quando Utgard-Loki veio ter com ele, e disse:

— Agora que você já está fora de meu castelo, e espero que nunca mais
volte, vou lhe contar a verdade. Eu sabia que você vinha, e de todo o
seu poder; para me defender, vali-me de ilusões. Loki comeu muito
rápido, mas Logi, com quem ele disputou, era ninguém menos que o
fogo, e por isso queimou ainda mais rápido até a gamela. Hugi, com
quem Thjálfi disputou a corrida, era ninguém menos que meu
pensamento, e não se poderia esperar que Thjálfi fosse mais rápido
que ele. Mais: quando você bebeu do chifre, e lhe pareceu baixar tão
devagar, foi para mim um milagre que eu não teria acreditado
possível. Pois, embora você não tenha percebido, a outra ponta
daquele chifre estava no mar; e agora, quando você fôr perto do mar,
poderá notar que o tanto que você bebeu o esvaziou, o que será para
sempre chamado de maré. E não menos surpreendente foi ver você
levantar uma pata do gato do chão, tanto que todos ficaram com medo;
pois o gato não era o que aparentava ser, mas a própria Serpente de
Midgard, que entre sua cabeça e sua cauda envolve a terra inteira.
Por fim, foi também uma grande milagre o tanto que você aguentou a
luta, e que tenha só ficado de joelhos; pois Elli, com quem você
lutou, é aquela que até agora ninguém venceu e jamais vencerá, e que
faz a todos caírem, a velhice. E agora parta, e será melhor que nunca
volte a me procurar, pois sempre defenderei meu castelo de forma
similar. 

Ao Thor ouvir essas palavras, imediatamente brandiu seu martelo; mas
quando ia lançá-lo viu que Utgard-Loki havia desaparecido, e também
seu castelo. Então Thor foi embora, mas decidiu que ao menos da
Serpente de Midgard ia procurar se vingar. 

E pouco depois, com esse intento, Thor partiu disfarçado como um jovem
rapaz. Ao anoitecer ele chegou na casa de um certo gigante, chamado
Hymir, e lá passou a noite como hóspede. Ao amanhecer Hymir levantou
e se vestiu, e se preparou para ir num barco a remo pescar no mar.
Thor despertou e imediatamente se aprontou, e pediu a Hymir para se
fazer ao mar remando junto com ele. Mas Hymir disse que Thor ia lhe
ser de pouca ajuda, sendo tão pequeno e tão jovem. 

— E você irá congelar, se eu ficar em alto mar tanto tempo quanto
pretendo. 

Mas Thor disse que ele seria capaz de remar para bem longe da terra,
pois não se podia saber ao certo quem ia ser o primeiro a pedir para
remar de volta. Thor tinha ficado com tanta raiva do gigante que por
pouco não usou seu martelo nele; mas ele se esforçou para ficar
calmo, pois pretendia usar sua força em outro assunto. Perguntou
então a Hymir o que levariam de isca, mas Hymir disse a ele que
arranjasse sua própria isca. Thor saiu então à procura, e viu um
rebanho de bois, que era de Hymir; ele pegou o maior boi e cortou-lhe
a cabeça, e a levou com ele para o mar. A essa altura Hymir já tinha
empurrado o barco para o mar. 

Thor entrou no barco e sentou na popa, pegou dois remos e começou a
remar; e Hymir viu que avançavam rápido por causa das remadas dele.
Hymir remava na proa, e logo já estavam bem longe no mar, e Hymir
disse que tinham chegado às àguas piscosas nas quais ele queria
ancorar e pescar linguado. Mas Thor disse que queria remar para ainda
mais longe, e deu um forte impulso; então Hymir disse que eles já
tinham ido tão longe que era perigoso ficar ali, por causa da
Serpente de Midgard. Thor disse que ainda ia remar um pouco, e foi o
que fez; mas Hymir já estava então morrendo de medo. Thor, assim que
largou os remos, preparou uma linha de pesca muito resistente, com um
anzol bem grande e tão resistente quanto ela. Então ele pôs a cabeça
do boi no anzol, e atirou-o na água, e ele foi direto para o fundo; e
é verdadeiro dizer que Thor enganou a serpente de Midgard tanto
quanto Utgard-Loki o tapeara antes, quando o fez erguê-la.

A Serpente de Midgard abocanhou a cabeça do boi, e o anzol entrou em
sua mandíbula; quando Jörmungandr percebeu, arremeteu para fugir com
tanta força que os dois punhos de Thor bateram na amurada. Então Thor
ficou com raiva, e invocando sua força divina, apoiou seus pés com
tanta força que eles atravessaram o casco do barco, e foram se
plantar no fundo do mar; ele então puxou a Serpente para a amurada. E
pode se dizer que ninguém viu nada de muito aterrador se não viu o
que se seguiu: como Thor dava olhares ferozes para a Serpente, e a
Serpente por sua vez olhava-a com muito ódio lá debaixo e cuspia
veneno. Daí, dizem, o gigante Hymir ficou pálido, e então amarelo, e
ficou morto de medo ao ver a Serpente, e como o mar invadia o barco.
No instante exato em que Thor agarrou seu martelo e levantou-o no ar,
o gigante achou sua peixeira e cortou a linha de Thor na amurada, e a
Serpente afundou no mar. Thor arremessou seu martelo atrás dela, e há
quem diga que esmagou a cabeça dela contra o fundo do mar, mas a
verdade é que a Serpente de Midgard continuou viva, se estendendo no
oceano. Mas Thor brandiu seu punho e esmurrou a orelha de Hymir, que
perdeu o equílibrio e caiu na água, e Thor viu as solas de seus pés
afundarem. E Thor voltou para a terra firme.

\chapter{Os salvadores da Pátria\subtitulo{Edith Nesbit}}

Tudo começou quando caiu um cisco no olho de Effie. Doía muito mesmo,
dava a sensação de ter uma fagulha incandescente no olho — só que
parecia ter pernas também, e asas feito uma mosca. Effie esfregou os
olhos e chorou — não um choro de verdade, mas o do tipo a que o olho
se entrega por conta própria, sem você precisar estar se sentindo
horrível por dentro — e então ela foi atrás do pai dela para que ele
tirasse o cisco de seu olho. O pai de Effie era médico, de modo que
ele sabia como tirar ciscos dos olhos — e ele o fez muito habilmente,
usando um pincel macio embebido em óleo de rícino.

Quando ele tinha tirado o cisco, disse:

— Isso é muito curioso.

Effie frequentemente tivera ciscos no olho antes, e seu pai sempre
pareceu achar normal — um tanto aborrecido e descuidado, talvez, mas
ainda assim normal. Ele nunca tinha achado curioso.

Effie ficou lá com seu lenço no olho, dizendo:

— Não acredito que enfim saiu. — As pessoas sempre dizem isso quando
tiram um cisco dos olhos.

— Ah sim, saiu — disse o doutor. — Está aqui, no pincel. E é muito
interessante.

Effie nunca ouvira seu pai falar isso sobre qualquer coisa em que ela
tivesse parte. Ela disse:

— O quê?

O doutor levou o pincel com muito cuidado para o outro lado da sala, e
pôs a ponta dele sob seu microscópio, e então ajustou os botões, e
espiou pela parte de cima do microscópio com um olho só.

— Minha nossa — ele disse. — Minha nossa! Quatro membros bem
desenvolvidos; um longo apêndice caudal; cinco dedos, de comprimento
desigual; quase igual a um dos Lacertidae, e no entanto há traços de
asas. 

A criatura retorceu-se um pouco no óleo de rícino, e ele continuou:

— Sim, uma asa similar à dos morcegos. Um novo espécime, sem a menor
dúvida. Effie, corra até o professor e peça para ele fazer a
gentileza de vir aqui por alguns minutos.

— Você podia me dar seis pence, papai — disse Effie — porque fui eu
que trouxe para você o novo espécime. Eu tomei bastante cuidado com
ele dentro de meu olho; e meu olho está mesmo doendo.

O doutor estava tão contente com o novo espécime que ele deu a Effie
um shilling; e logo o professor apareceu. Ele ficou para o almoço, e
ele e o doutor discutiram todo contentes a tarde inteira sobre o nome
e a família da coisa que saíra do olho de Effie.

Mas na hora do jantar outra coisa aconteceu. Harry, o irmão de Effie,
pescou alguma coisa dentro de seu chá, que ele a princípio achou que
era uma pequena centopéia. Ele estava a ponto de deixá-la cair no
chão, e acabar com a vida dela da maneira usual, quando ela se
sacudiu na colher — abriu duas asas molhadas, e se deixou cair na
toalha da mesa. E lá ficou, se esfregando com suas patas e esticando
as asas, e Harry disse:

— Ora, é uma salamandra minúscula!

O professor debruçou-se antes que o doutor pudesse dizer uma palavra.

— Dou-lhe meia coroa por ele, Harry, meu rapaz — disse, falando bem
rápido; e então catou-o cuidadosamente em seu lenço.

— É um novo espécime — disse. — E superior ao seu, Doutor.

Era um lagarto minúsculo, de um centímetro e meio de comprimento; com
escamas e asas.

De modo que agora tanto o doutor quanto o professor tinham o seu
espécime, e estavam ambos muito satisfeitos. Mas não demorou muito
para esses espécimes ficarem bem menos valiosos. Porque na manhã
seguinte, quando o engraxate [knife-boy?] estava engraxando as botas
do doutor, ele de repente largou a escova e a graxa, gritando que
tinha sido queimado.

E de dentro da bota saiu rastejando um lagarto do tamanho de um gato,
com asas grandes e brilhantes.

— Ora — disse Effie — eu sei o que é isso. É um dragão, igual ao que
São Jorge matou.

E Effie estava certa. Naquela tarde Towser foi mordido no jardim por
um dragão do tamanho de um coelho, que ele tentara caçar, e na manhã
seguinte todos os jornais só falavam dos “lagartos com asas” que
estavam aparecendo por todo o país. Os jornais não os chamavam de
dragões porque, claro, ninguém acredita em dragões hoje em dia — e de
jeito nenhum os jornais iam ser tolos a ponto de acreditar em contos
de fadas. A príncipio, havia apenas uns poucos, mas em uma ou duas
semanas o país estava simplesmente infestado de dragões de todos os
tamanhos, e era possível às vezes ver muitos deles no ar, feito um
enxame de abelhas. Eram todos iguais, exceto no tamanho. Eram verdes
com escamas, tinham quatro patas, uma cauda comprida, e grandes asas
parecidas com as dos morcegos, exceto que eram de um amarelo claro,
semi-transparente, como as caixas de câmbio das bicicletas [?].

Exalavam fogo e fumaça, como convém a todo dragão legítimo, mas ainda
assim os jornais continuaram fingindo que eram lagartos, até que o
editor do Standard foi pego por um deles e levado embora, e então o
resto do pessoal do jornal ficou sem quem lhes dizer no que deviam ou
não acreditar. De modo que, quando o maior elefante do zoológico foi
carregado embora por um dragão, os jornais desistiram de fingir, e
saíram com a manchete Alarmante Praga de Dragões na primeira página.

Você nem imagina o quanto era alarmante, e ao mesmo tempo o quanto era
irritante. Os dragões de tamanho grande eram certamente terríveis,
mas depois que se descobriu que eles iam para a cama cedo porque
tinham medo do ar frio da noite, bastava passar o dia inteiro dentro
de casa, para ficar a salvo deles. Mas os de tamanhos menores eram um
perfeito incômodo. Os do tamanho de centopéias ficavam caindo na
sopa, ou na manteiga. Os do tamanho de cachorros pulavam nas
banheiras, e o fogo e a fumaça dentro deles fazia com que a água fria
da torneira virasse vapor instantaneamente, de modo que pessoas
descuidadas podiam se escaldar seriamente. Os do tamanho de pombos
entravam nas cestas de costuras e nas gavetas e mordiam quem estava
com pressa de pegar uma agulha ou um lenço. Os do tamanho de uma
ovelha eram mais fáceis de se evitar, porque era possível vê-los
chegando; mas quando eles voavam pelas janelas e se aninhavam debaixo
das cobertas, e não se notava antes de entrar na cama, era sempre um
choque. Os desse tamanho não comiam gente, só alface, mas eles sempre
chamuscavam os lençóis e as fronhas horrivelmente. 

Claro, o Conselho do Condado e polícia fizeram tudo o que havia para
fazer; mas oferecer a mão da princesa a quem matasse o dragão não
adiantava. Esse jeito era muito bom nos velhos tempos, quando havia
apenas um dragão e uma princesa, mas agora havia bem mais dragões do
que princesas, mesmo a Família Real sendo bem grande. Além disso, ia
ser apenas um desperdício de princesas oferecê-las como recompensa a
quem matasse dragões, porque todo mundo matava tantos dragões quanto
podia, inteiramente por conta própria e sem pensar em recompensas,
apenas para tirar da frente aqueles bichos tão desagradáveis. O
Conselho do Condado encarregara-se de cremar todos os dragões
entregues entre as dez e as catorze horas, e carrinhos, carroças e
caminhões cheios de dragões mortos podiam ser vistos todos os dias da
semana fazendo uma longa fila na rua em que ficava o prédio do
Conselho. Meninos traziam carrinhos de mão cheios de dragões, e
crianças na volta da escola no fim da manhã paravam para deixar um ou
dois punhados de dragões que traziam em suas malas, ou no bolso,
embrulhados em seus lenços. E no entanto parecia continuar a haver
tantos dragões quanto antes. Então a polícia ergueu torres de pano e
madeira cobertas de cola. Quando os dragões ao voar batiam nessas
torres, ficavam grudados como moscas e vespas no papel mata-moscas da
cozinha; quando as torres ficavam todas cobertas de dragões, o
inspetor da polícia punha fogo nelas, queimando os dragões e o resto
junto.

E no entanto parecia haver ainda mais dragões que antes. As lojas
estavam cheias de veneno para dragão, e sabão anti-dragão, e cortinas
a prova de dragão para as janelas; e de fato, tudo o que era possível
fazer foi feito.

E no entanto parecia haver ainda mais dragões que antes.

Não era muito fácil descobrir o que envenava um dragão, porque eles
comiam as coisas mais variadas. Os maiores comiam elefantes, enquanto
havia elefantes, e depois passaram a comer cavalos e vacas. Um outro
tamanho não comia nada a não ser lírios do vale, e um terceiro
tamanho comia apenas primeiro ministros se os havia disponíveis, e em
não havendo, se alimentavam generosamente de meninos que trabalhavam
de uniforme como criados. Outro tamanho vivia de tijolos, e três
deles comeram dois terços da Enfermaria de South Lambeth numa tarde.

Mas o tamanho do qual Effie tinha mais medo era o dos tão grandes
quanto a sala de jantar; os desse tamanho comiam menininhas e
menininhos.

A princípio Effie e seu irmão ficaram muito satisfeitos com as
mudanças na vida deles. Era tão divertido ficar acordado a noite
inteira em vez de ir dormir, e brincar no jardim iluminado com luz
elétrica. E soava tão engraçado ouvir a mãe dizer, quando iam para a
cama:

— Boa noite, meus queridos, durmam bem o dia todo, e não levantem
muito cedo. Vocês não podem sair da cama até ficar bem escuro. Não
vão querer que os horríveis dragões os peguem.

Mas depois de um tempo cansaram de tudo aquilo. Queriam ver as flores
e as árvores crescendo no campo, e ver o sol brilhando do lado de
fora, em vez de através do vidro e da cortina a prova de dragão das
janelas. E queriam brincar na grama, o que não era permitido no
jardim iluminado com luz elétrica por causa da umidade do orvalho.

E eles queriam tanto sair lá fora, uma vez que fosse, na bela,
brilhante e perigosa luz do dia, que eles começaram a tentar achar
alguma razão para terem de sair. Só que não gostavam de desobedecer a
mãe deles.

Mas uma manhã a mãe deles estava ocupada preparando algum novo veneno
para dragão para pôr na adega, e o pai deles estava fazendo um
curativo na mão do engraxate [bootboy] que fora arranhado por um dos
dragões que gostava de comer primeiro ministros quando disponíveis,
de modo que ninguém lembrou de dizer às crianças “não levantem até
ficar escuro”.

— Vamos agora — disse Harry. — Não vai ser desobedecer. E eu sei
exatamente o que temos que fazer, só não sei como vamos fazer.

— O que temos que fazer? — disse Effie.

— Temos que acordar São Jorge, claro — disse Harry. — Ele era a única
pessoa da cidade que sabia como lidar com dragões; o pessoal dos
contos de fadas não conta. Mas São Jorge é uma pessoa de verdade, e
só está dormindo, à espera de ser acordado. Só que ninguém acredita
mais em São Jorge. Ouvi papai dizer isso.

— Nós acreditamos — disse Effie.

— Claro que sim. E você não percebe, Ef, qual a razão para eles não
conseguirem acordá-lo? Não dá para acordar alguém em quem não se
acredita, dá?

Effie disse que não, mas onde eles iam encontrar São Jorge?

— Precisamos ir procurá-lo — Harry disse decididamente. — Você vai
usar um vestido a prova de dragão, feito da mesma pano que as
cortinas. E eu vou passar no corpo o melhor veneno para dragão, e...

Effie apertou as mãos e pulou de alegria dizendo:

— Oh, Harry! Eu sei onde podemos achar São Jorge! Na Igreja de São
Jorge, claro.

— Hm — disse Harry, querendo que tivesse sido ele a pensar nisso. —
Você às vezes até que é inteligente, para uma menina.

Então na tarde seguinte, bem cedo, muito antes que os raios do
pôr-do-sol anunciassem a noite chegando, quando todo mundo iria
acordar para ir trabalhar, as duas crianças saíram da cama. Effie
enrolou em volta dela um xale de musselina a prova de dragões — não
havia tempo para fazer um vestido — e Harry fez dele uma meleca só
com veneno para dragão. Era garantidamente inofensivo para crianças e
inválidos, de modo que ele não precisava se preocupar.

Eles se deram as mãos e saíram para ir até a igreja de São Jorge. Como
você sabe, há muitas igrejas de São Jorge, mas por sorte eles viraram
a esquina que levava à certa, e lá se foram sob o sol brilhante,
sentindo-se muito corajosos e aventureiros.

Não havia ninguém nas ruas a não ser dragões, e a cidade estava
simplesmente infestada deles. Por sorte nenhum era do tamanho certo
para comer menininhos e menininhas, ou talvez esta história
terminasse aqui. Havia dragões na calçada, e dragões na rua, e
dragões tomando sol nas escadarias dos prédios públicos, e dragões
alisando as asas nos tetos. A cidade estava toda verde deles. Mesmo
quando as crianças saíram da cidade e seguiram pela estrada, elas
perceberam que os campos dos dois lados estavam mais verdes que o
normal, com todas aquelas escamas e caudas; e alguns dos menores
haviam feito ninhos de asbestos nas cercas-vivas de espiriteiro
florido.

Effie segurava a mão de seu irmão apertando muito, e quando um dragão
gordo bateu as asas perto de sua orelha ela deu um berro, fazendo com
que uma revoada de dragões verdes levantasse vôo do campo, se
espalhando pelo céu. As crianças podiam ouvir o ruído das asas deles
no ar.

— Oh, quero ir para casa — disse Effie.

— Não seja boba — disse Harry. — Com certeza você não esqueceu dos
Sete Campeões e suas princesas. As pessoas que vão ser os salvadores
da pátria nunca berram e dizem que querem ir para casa. 

— E nós ...somos? — Effie perguntou. — Salvadores, quero dizer.

— Você vai ver — disse o irmão dela, e eles seguiram em frente.

Quando chegaram à igreja de São Jorge encontraram a porta aberta, e
eles entraram; mas São Jorge não estava lá dentro, então eles saíram
para o cemitério do lado de fora da igreja, e logo acharam a grande
tumba de pedra de São Jorge, com a figura dele esculpida em mármore
do lado de fora, com sua armadura e capacete, e as mãos cruzadas
sobre o peito.

— Como vamos acordá-lo? — disseram. Então Harry falou com São Jorge,
mas ele não respondia; e aí ele tentou acordar o grande matador de
dragões chacoalhando seus ombros de mármore. Mas São Jorge nem notou.

Então Effie começou a chorar, e pôs os braços em volta do pescoço de
São Jorge o melhor que pôde no mármore, que ficava muito no caminho
nas costas; e ela beijou o rosto de mármore, e disse:

— Oh, caro, bom, gentil São Jorge, por favor acorde e nos ajude.

E com isso São Jorge abriu os olhos sonolentamente, se espreguiçou e
disse:

— Qual é o problema, menininha?

Então as crianças contaram tudo o que havia para contar; ele se virou
em seu mármore e apoiou-se num cotovelo para escutar. Mas quando
soube que havia tantos dragões balançou a cabeça.

— Assim não dá — ele disse — vão ser dragões demais para o velho
Jorge. Vocês deviam ter me acordado antes. Sempre fui a favor de
lutas justas: um homem, um dragão, era meu lema.

Bem naquele momento uma revoada de dragões passou por cima deles, e
São Jorge começou a desembainhar a espada.

Mas ele balançou a cabeça de novo e empurrou a espada de volta para o
lugar dela enquanto os dragões iam ficando pequenos ao se a
distanciarem.

— Não posso fazer nada — ele disse. — As coisas mudaram desde a minha
época. Santo André me contou. Ele foi acordado na greve dos
maquinistas, e veio conversar comigo. Disse que hoje em dia tudo se
faz com máquinas; deve haver algum maneira de dar um jeito nesses
dragões. Falando nisso, como tem estado o tempo ultimamente?

A pergunta soou tão descabida e indelicada que Harry se recusou a
responder, mas Effie disse pacientemente:

— Tem estado muito bom. Papai disse que é o verão mais quente que já
houve nesse país.

— Ah, foi o que imaginei — disse o campeão, pensativo. — Bom, a única
coisa que podia ajudar... Dragões não suportam o frio e a umidade,
essa é a única coisa. Se ao menos vocês conseguissem achar as
torneiras...

São Jorge estava começando a se ajeitar de novo em sua lápide de
pedra.

— Boa noite, sinto muito não poder ajudá-los — disse, bocejando por
trás de sua mão de mármore.

— Ah, mas você pode — exclamou Effie. — Diga: que torneiras?

— Ah, igual no banheiro — disse São Jorge, ainda mais sonolento. — E
tem um espelho, também: mostra o mundo todo e o que acontece nele.
São Dionísio quem me contou; disse que era uma coisa muito bonita.
Sinto muito não... Boa noite.

E ele voltou a seu mármore e num instante dormia profundamente.

— Nós nunca vamos achar as torneiras — disse Harry. — Escuta, não ia
ser terrível se o São Jorge acordasse justo quando houvesse um dragão
por perto, um do tamanho dos que comem campeões?

Effie tirou seu xale a prova de dragão.

— Não encontramos nenhum dos tamanho sala de jantar — ela disse. —
Acho que estamos seguros.

Então ela cobriu São Jorge com o pano, e Harry esfregou o máximo que
conseguiu de veneno para dragão na armadura de São Jorge, de modo a
deixá-lo bem seguro.

— Podemos nos esconder na igreja até escurecer — ele disse — e
então...

Mas naquele momento uma sombra escura se abateu sobre eles, e eles
viram que era um dragão exatamente do tamanho da sala de jantar em
casa.

E viram que tudo estava perdido. O dragão arremeteu para o solo e
pegou os dois com suas garras, Effie pelo cinto verde de seda dela, e
Harry pela pontinha da parte de trás de sua jaqueta de Eton [] — e
então, abrindo suas enormes asas amarelas, alçou vôo, fazendo um
barulhão igual a um vagão de terceira classe com o breque puxado.

— Oh, Harry — disse Effie — me pergunto quando ele vai nos comer!

O dragão estava voando por cima de florestas e campos com o lento
bater de suas enormes asas, atravessando um quarto de milha a cada
batida delas.

Harry e Effie podiam ver o campo lá embaixo, cercas e rios e igrejas e
casas de fazendas ficando para trás sob eles, muito mais rápido do
que passavam no mais rápido dos trens expressos.

E o dragão continuava a voar. As crianças viram outros dragões no ar
por onde eles passavam, mas o dragão que era do tamanho da sala de
jantar nunca parou para falar com nenhum deles, apenas continuou
voando adiante num ritmo constante.

— Ele sabe aonde está indo — disse Harry. — Ah, se a menos ele nos
largasse antes de chegar lá!

Mas o dragão segurava-os firme, e ele voou e voou e voou, até que
finalmente, quando as crianças estavam já bem tontas, aterrisou no
topo de uma montanha, com todas suas escamas retinindo. Ele então
deitou seu escamoso e verde corpo, ofegando, completamente sem fôlego
do tanto que voara. Mas suas garras estavam firmes no cinto de Effie
e na jaqueta de Harry.

Então Effie catou o canivete que Harry lhe dera de presente de
aniversário. Custara só seis pence, e ela já o tinha há um mês e até
agora tudo o que conseguira fazer com ele tinha sido apontar lápis de
ardósia, mas de algum jeito ela fez com que aquele canivete cortasse
o cinto dela, e ela se livrou dele, deixando o dragão apenas com uma
tira de seda verde em suas garras. Mas aquele canivete jamais
cortaria a jaqueta de Harry, e depois de tentar um bocado Effie viu
que não tinha jeito e desistiu. Mas com a ajuda dela Harry conseguiu
esgueirar-se para fora das mangas, de modo que o dragão ficou apenas
com uma jaqueta de Eton nas suas outras garras. Então as crianças
foram na ponta dos pés até uma rachadura nas pedras e entraram nela.
Era estreita demais para o dragão entrar também, e lá elas ficaram,
esperando para fazer caretas para o dragão quando ele tivesse
descansado o bastante para se sentar e começar a pensar em comer
eles. Ele ficou muito bravo, mesmo, quando eles fizeram caretas para
ele, e exalou fogo e fumaça, mas eles correram mais para dentro da
caverna para não serem alcançados, e ele acabou cansando e indo
embora.

Mas os dois estavam com medo de sair da caverna, então eles
continuaram para dentro, e a caverna foi se alargando e ficando
maior, e o chão dela era de areia macia, e quando eles chegaram ao
fim dela havia uma porta, na qual estava escrito: Sala Universal das
Torneiras. Privativo. Entrada Permitida a Ninguém.

Eles abriram a porta na mesma hora para espiar dentro, lembrando do
que São Jorge dissera.

— A nossa situação não pode ficar pior do que já está — disse Harry —
com um dragão esperando do lado de fora. Vamos entrar.

Entraram então decididamente na sala das torneiras, e fecharam a porta
atrás deles.

E agora estavam numa espécie de uma sala cavada na rocha sólida, e
havia torneiras ao longo de toda uma das paredes, e todas as
torneiras tinham rótulos feitos de plaquinhas de porcelana como os
das estâncias de águas mineirais. E como ambos eram capazes de ler
palavras de duas sílabas e às vezes até mesmo de três, eles
entenderam no mesmo instante que tinham achado o lugar de onde se
liga o tempo. Havia três torneiras grandes com os rótulos: “Sol”,
“Vento”, “Chuva”, “Neve”, “Granizo”, “Geada”; e várias menores,
dizendo: “Leve a Moderada”, “Chuvoso”, “Brisa do Sul”, “Bom para as
plantas crescerem”, “Patinação”, “Vento Sul”, “Vento Leste”, e por aí
afora. E a torneira grande com o rótulo “Sol” estava inteira aberta.
Não dava para ver nenhuma luz do sol — a caverna era iluminada por
uma clarabóia de vidro azul — de modo que eles acharam que a luz do
sol devia estar saindo por algum outro lugar, como acontece com a
torneira que lava as partes de baixo de certas pias de cozinha.

Então eles viram que do outro lado da sala havia apenas um grande
espelho, e olhando por ele podia se ver tudo que estava acontecendo
no mundo — e tudo de uma vez só, também, bem diferente da maioria dos
outros espelhos. Eles viram as carroças entregando dragões mortos no
prédio do Conselho do Condado, e viram São Jorge dormindo sob o
tecido à prova de dragão. E eles viram a mãe deles em casa chorando
porque seus filhos tinham saído na terrível e perigosa luz do dia, e
estava com medo de um dragão tê-las comido. E eles viram a Inglaterra
inteira, feito um grande mapa de quebra-cabeças, verde nas partes do
campo e marrom nas cidades, e preto nos lugares onde se faz carvão e
cerâmica e cutelaria e substâncias químicas. Cobrindo tudo, as partes
pretas, as verdes e as marrons, havia uma rede de dragões verdes. E
eles puderam ver que ainda era dia, e os dragões ainda não tinham ido
para a cama. Effie disse:

— Dragões não gostam do frio.

E ela tentou fechar o sol, mas a torneira estava com defeito, e era
essa a razão de estar fazendo tanto calor, e dos dragões terem sido
chocados a ponto de quebrarem seus ovos. Então eles deixaram a
torneira do sol em paz, e abriram a de neve e a deixaram inteira
aberta enquanto iam olhar no espelho. Lá eles viram os dragões
correndo para todos os lados como formigas fazem se você é cruel o
suficiente para derramar água num formigueiro, o que você nunca é,
claro. E cada vez caía mais neve.

Então Effie abriu inteira a torneira da chuva, e logo os dragões
estavam se mexendo menos, e aos poucos alguns deles foram ficando
completamente imóveis, e as crianças sabiam que a água tinha apagado
o fogo dentro deles e estavam mortos. Eles abriram então a de granizo
— só até a metade, de medo de quebrar as janelas das pessoas — e
depois de um tempo não havia mais dragões se movendo para ver.

As crianças souberam então que eram de fato os salvadores da pátria.

— Vão erguer um monumento para a gente — disse Harry — tão alto quanto
o do Almirante Nelson! Todos os dragões morreram.

— Espero que o que estava esperando pela gente do lado de fora esteja
morto! — disse Effie. — E quanto ao monumento, Harry, não sei não. O
que eles vão fazer com esse monte de dragões mortos? Iria levar anos
e anos para queimar todos, e nem dá para queimá-los agora que eles
estão todos ensopados. Gostaria que a chuva os levasse todos para o
mar.

Mas isso não aconteceu, e as crianças começaram a achar que não tinham
sido tão terrivelmente espertas assim, no final das contas.

— Para que será que essa coisa velha serve? — disse Harry. Tinha
achado uma torneira velha e enferrujada, que parecia não ser usada há
um monte de tempo. O rótulo de porcelana dela estava coberto de
poeira e teias de aranha. Quando Effie a limpou com a ponta de sua
saia (pois curiosamente ambas as crianças haviam saído sem seus
lenços) ela descobriu que estava escrito “Ralo”.

— Vamos abrí-la — ela disse. — Quem sabe leva os dragões.

A torneira estava emperrada de não ter sido usada por tanto tempo, mas
juntos eles conseguiram abrí-la, e correram para o espelho para ver o
que acontecera.

Um enorme e redondo buraco negro já havia se aberto bem no meio do
mapa da Inglaterra, e os lados do mapa se inclinaram, de modo que a
água da chuva corria toda para o buraco.

— Viva, viva, viva! — gritou Effie, e correu de volta para as
torneiras para abrir tudo o que parecia molhado, “Chuvoso”, “Bom para
as plantas crescerem”, e até “Vento Sul”, “Vento Sudeste”, pois
ouvira seu pai dizer que esses ventos traziam chuva.

E então a chuva caía torrencial em todo o país, e grandes vagas de
água corriam para o centro do mapa, e cataratas de água caíam no
grande buraco redondo no meio do mapa, e os dragões estavam sendo
levados para desaparecerem cano abaixo pelo ralo, em compactas massas
verdes e espalhados grumos verdes, dragões sozinhos e dragões às
dezenas, dos que carregavam elefantes aos que caíam no chá.

Logo não havia nenhum dragão sobrando. Então eles fecharam a torneira
“Ralo”, e fecharam até a metade a que estava marcada “Sol” — estava
quebrada, de modo que não puderam fechá-la de vez — e abriram “Leve a
moderada” e “Chuvoso”, e as duas torneiras emperraram, e não davam
mais para ser fechadas, o que explica o clima do país.

********

 Como eles voltaram para casa? Pela estrada de ferro de Snowdon,
claro.

E a pátria ficou agradecida? Bom, a pátria estava muito molhada. E
quando enfim a pátria ficou seca de novo estava mais interessda numa
nova invenção que usava eletricidade para assar muffins, e todos os
dragões já tinham sido quase esquecidos. Dragões não parecem assim
tão importantes depois de terem morrido e sumido todos e, você sabe,
nunca havia sido oferecida uma recompensa.

E o que disseram o pai e a mãe deles quando Effie e Harry chegaram em
casa?

É o tipo da pergunta boba que vocês crianças sempre fazem. No entanto,
só dessa vez não vou me importar em responder.

A mãe disse:

— Oh meus queridos, meus queridos, vocês estão salvos! Suas crianças
travessas, como puderam ser tão desobedientes? Já para a cama!

E o pai deles, o doutor, disse:

— Gostaria de ter sabido o que vocês iam fazer! Queria de ter
preservado um espécime. Joguei fora o que tirei do olho da Effie.
Pretendia pegar um espécime em melhores condições. Não previ essa tão
imediata extinção da espécie.

O professor nada disse, mas esfregou as mãos. Tinha guardado o seu
espécime — o do tamanho de uma pequena centopéia pelo qual dera meia
coroa a Harry — e o tem até hoje.

Você precisa conseguir que ele o mostre para você!

\chapter{2. O dragão\subtitulo{Giambattista Basile}}

Havia uma vez um Rei que exercia seu poder com tamanha tirania e
crueldade que, quando ele estava fora numa visita a um castelo longe
da cidade, seu trono foi usurpado por uma certa feiticeira. Então ele
consultou uma estátua de madeira que costumava dar respostas
oraculares, e ficou sabendo que recuperaria seus domínios quando a
feiticeira perdesse a visão. Mas vendo que a feiticeira, além de
muito bem protegida, conseguia saber só de olhar quem eram as pessoas
que ele mandava contra ela, e dava cabo delas, ele ficou bastante
desesperado, e por puro rancor, passou a matar toda mulher que se
aproximasse dele.

Depois de centenas terem tido o infortúnio de perderem suas vidas
assim, quis o acaso que chegasse até ele um moça chamada Porziella, a
mais bela criatura sobre a face da terra, e o Rei não conseguiu
evitar de se apaixonar por ela e fazer dela a sua esposa. Mas ele era
tão cruel e vingativo com as mulheres que, depois de um tempo, ia
matá-la igual ao resto; só que bem quando ele estava erguendo sua
adaga um pássaro deixou cair uma certa raiz no braço dele, provocando
tamanho tremor nele que a arma caiu de sua mão. Esse pássaro era uma
fada que, uns dias antes, adormecera num bosque, e estava para ser
atacada por um sátiro quando Porziella a acordou; e desde então a
fada a vinha seguindo, para poder retribuir essa gentileza.

Quando o Rei viu o que acontecera, achou que tinha sido a beleza do
rosto de Porziella que fizera seu braço falhar e enfeitiçara a adaga
para evitar que ele a cravasse nela como fizera com tantas outras.
Decidiu então não tentar uma segunda vez, e que ela morreria
encarcerada num sotão de seu palácio. Dito e feito: a infeliz
criatura foi presa entre quatro paredes, sem nada para comer ou
beber, e deixada lá para morrer aos poucos. 

O pássaro, vendo-a nessa situação terrível, consolou-a com palavras
gentis, dizendo para ela se animar e prometendo, como retribuição à
grande gentileza que ela fizera por ele, ajudá-la, mesmo se lhe
custasse a própria vida. Mas apesar dos muitos pedidos de Porziella,
o pássaro jamais lhe contava quem era, dizendo apenas que tinha uma
dívida com ela, e que não iria medir esforços para serví-la. E vendo
que a pobre moça estava faminta, saiu voando e retornou com uma faca
pontuda que pegara na despensa do rei, e disse a ela para fazer um
furo sobre a cozinha, pelo qual regularmente traria comida para
sustentá-la. Então Porziella foi furando até conseguir fazer uma
passagem para o pássaro, o qual esperou o cozinheiro ir buscar água
no poço e passou pelo buraco, pegando uma ave que estava cozinhando
no fogo e levando-a até Porziella; e então, para saciar a sede dela,
sem saber como carregar uma bebida, voou até a despensa, onde haviam
penduradas muitas uvas, pegou um cacho cheio e o levou para
Porziella; e isso ele fez regularmente por muitos dias.

Enquanto isso, Porziella deu a luz um menino saudável, do qual ela
cuidou e criou, com a constante ajuda do pássaro. E quando ele
cresceu, a fada (que era o pássaro) disse à mãe do menino para
aumentar o buraco, tirando quantas tábuas fossem necessárias do chão
para permitir que Miuccio (que era esse o nome do menino) passasse; e
depois de fazê-lo descer com a ajuda de umas cordas que trouxera,
colocar de volta no lugar as tábuas, para que ninguém pudesse ver
donde ele tinha vindo. Porziella fez o que o pássaro lhe disse; assim
que o cozinheiro saiu, fez seu filho descer, instruindo-o a nunca
dizer donde viera ou de quem era filho.

Quando o cozinheiro voltou e viu o belo menininho, perguntou a ele
quem era e donde viera; e o menino, lembrando da advertência de sua
mãe, disse que era um pobre menino abandonado que estava a procura de
um lar. Enquanto conversavam, o mordomo entrou, e vendo o vivo
menininho, achou que ele daria um pagem bonitinho para o Rei. Levou-o
então aos aposentos reais, e quando o Rei viu aquele menino tão
bonito que parecia uma verdadeira jóia, gostou muito dele, e o
aceitou a seu serviço como um pagem, e em seu coração como um filho,
e lhe ensinou tudo que um cavaleiro precisa saber, de modo que
Miuccio cresceu se tornando um dos melhores da corte, e o Rei o amava
muito. Por causa disso, a madrasta do Rei, que era a rainha, começou
a detestá-lo, e a ter uma profunda aversão a ele; e o ciúme e a
maldade dela cresceram na mesma medida em que o Rei mais favorecia
Miuccio; e ela decidiu passar sabão na escada do destino dele para
fazê-lo escorregar e levar um tombo. 

Foi assim que, numa noite em que ela e o Rei haviam afinado seus
instrumentos e tocavam música juntos, a Rainha disse ao Rei que
Miuccio se gabara de ser capaz de construir três castelos no ar.
Então na manhã seguinte, quando a lua, professora das sombras, dá
férias a seus alunos para que se instaure o festival do Sol, o Rei,
seja por ter ficado surpreso ou para contentar a Rainha, ordenou que
chamassem Miuccio, e mandou que ele construísse imediatamente três
castelos no ar como se dissera capaz, senão o faria dançar na forca.

Ao ouvir isso, Miuccio foi para seu quarto e começou a se lamentar
amargamente, vendo que o favor dos príncipes é de vidro, quebra fácil
e pouco dura. Enquanto estava assim infeliz, o pássaro surgiu, e
disse:

— Coragem, Miuccio, e nada tema tendo eu ao seu lado, pois sou capaz
de tirá-lo dessa fria. 

O pássaro então o instruiu a pegar cola e papelão e fazer três grandes
castelos; e chamando três grandes grifos [?], amarrou um castelo em
cada um deles, e lá saíram eles voando no ar. Miuccio chamou o rei,
que veio correndo com toda a sua corte ver aquilo; e ao perceber o
engenho de Miuccio sua afeição por ele ficou ainda maior, e o elogiou
extravagantemente, o que foi como neve para o cíume e fogo para a
raiva da Rainha, ao ver que seu plano tinha falhado. E ela, tanto
acordada quanto dormindo, não pensava em outra coisa a não ser como
iria conseguir se livrar daquele espinho em seus olhos. Até que
enfim, alguns dias depois, ela disse ao Rei:

— Já é chegada a hora de recuperarmos nossa grandeza e os prazeres de
nosso passado, pois Miuccio se ofereceu para cegar a feiticeira, e
fazer com que o reino volte a ser seu.

O Rei, sentindo cutucada aquela sua velha ferida, chamou Miuccio
naquele mesmo instante, e disse a ele:

— Estou muito surpreso com você, com a pouca consideração que mostra
pelo meu amor por você; pois, tendo o poder de restaurar-me no trono
que perdi, você se mantém assim insensível, em vez de se esforçar
para me libertar da miséria a que fui reduzido: de um reino a uma
floresta, de uma cidade a um pobre castelo, e de reinar sobre um
grande povo a ser servido por um punhado de lacaios magrelos.
Portanto, se você me quer bem, saía já para ir cegar a feiticeira que
tem a posse de minha propriedade, pois ao apagar as lanternas dela
você acenderá as luzes de minha honra, agora tão apagada e sombria.

Ao ouvir isso, Miuccio estava para responder que o Rei havia sido mal
informado e estava enganado, que ele não era nem um corvo para picar
olhos ou uma picareta para abrir buracos, mas o Rei disse:

— Nenhuma outra palavra; é o que quero, e que assim seja. Lembre-se:
em minha mente há uma balança com a recompensa num dos pratos, se
você fizer o que lhe pedi; e no outro a punição, se recusar a me
obedecer.

Miuccio, que nada podia contra um homem que não se deixaria demover,
foi para um canto lamentar-se, e o pássaro veio até ele e disse:

— Será possível, Miuccio, que toda hora você tem que estar se afogando
numa tempestade em copo de água? Mesmo se eu tivesse morrido você não
conseguiria fazer tanta onda. Você ainda não sabe que sua vida me
importa mais que a minha? Portanto não desanieme; venha comigo, que
eu vou ver o que posso fazer.

Ao dizer isso saiu voando, e pousou na floresta, onde começou a
chilrear e uma multidão de pássaros o cercou, aos quais contou a
história, e prometeu que aquele que se aventurasse a tirar a vista da
feiticeira teria uma salvaguarda contra as presas dos falcões e dos
milhafres; e um salvoconduto contra as armas, os arcos e as
armadilhas dos caçadores.

Havia entre eles uma andorinha cujo ninho ficava numa viga do palácio
real, que odiava a feiticeira porque, quando ela fazia seus maléficos
feitiços, mais de uma vez a obrigara a abandonar seu ninho por causa
das fumigações dela; por isso, em parte por vingança, e em parte para
ganhar a recompensa que o pássaro prometera, se ofereceu para fazer o
serviço. Saiu voando feito um raio para a cidade e, entrando no
palácio, encontrou a feiticeira deitada num sofá, sendo abanada por
duas donzelas. Então a andorinha se aproximou e, pousando bem em cima
da feiticeira, bicou seus olhos. E naquele momento a feiticeira, ao
ver a noite ao meio-dia, soube que àquelas alturas o reino estava
perdido; e aos berros, como uma alma condenada, abandonou o cetro e
fuigiu para se esconder numa certa caverna, onde ficou continuamente
batendo a cabeça na parede, e depois de passar um tempo assim seus
dias chegaram ao fim.

Quando a feiticeira partiu, os conselheiros mandaram embaixadores para
o Rei, pedindo a ele que voltasse, pois o cegamento da feiticeira lhe
possibilitara ver esse dia feliz. E ao mesmo tempo que eles, chegou
também Miuccio que, seguindo as instruções do pássaro, disse ao Rei:

— Servi vossa majestade da melhor forma que estava ao meu alcance; a
feiticeira está cega, o reino é seu. Portanto, se eu mereço uma
recompensa, a única coisa que quero é ser abandonado a minha própria
má sorte, sem ter de novo que me expor a esses perigos.

Mas o Rei o abraçou com grande afeto, pediu a ele que pusesse sua
boina e sentasse ao seu lado; e o quanto a Rainha ficou com raiva
disso tudo, só os céus sabem, pois pelo arco de muitas cores que
apareceu em sua face podia se prever o vento da tempestade que em seu
coração se armava contra o pobre Miuccio.

Não muito longe desse castelo vivia um feroz dragão, que havia nascido
na mesma hora que a Rainha; e tendo os astrólogos sido convocados por
seu pai para astrologizar esse evento, eles disseram que a filha dele
estaria segura enquanto o dragão estivesse seguro, e quando um
morresse, o outro necessariemente morreria também. Só uma coisa
traria a Rainha de volta a vida: untar suas têmporas, peito, narinas
e pulsos com o sangue do dragão.

Então a Rainha, sabendo da força e da fúria daquele animal, resolveu
mandar Miuccio para as garras dele, sem ter a menor dúvida de que o
monstro o comeria sem dificuldades, que ele seria como um moranguinho
na garganta de um urso. Ela virou-se para o Rei e disse: 

— Esse Miuccio é mesmo o tesouro de seu palácio, e você seria muito
ingrato se não o amasse, ainda mais que ele confessou a vontade que
tem de matar o dragão, o qual, apesar de ser meu irmão, é ainda assim
seu inimigo; e um só cabelo seu me importa mais que uma centena de
irmãos.

O Rei, que odiava mortalmente o dragão, e não tinha idéia de como
tirá-lo da sua frente, no mesmo instante chamou Miuccio, e disse a
ele:

— Eu sei que você pode fazer tudo aquilo que quer; assim, como você já
fez tanto, dê-me ainda um outro prazer, e então obterá de mim tudo o
que quiser. Vá neste mesmo instante matar o dragão, e me fará um
serviço singular, e eu o recompensarei muito bem. 

Ao ouvir isso Miuccio quase perdeu os sentidos, e assim que conseguiu
falar, disse ao Rei:

— Ai de mim, que dor de cabeça que essas contínuas provocações me dão!
Acaso minha vida é um tapete de pele de cabra que vossa majestade
pode pisar até gastar? Não se trata de uma pera descascada pronta a
cair na boca de alguém, mas de um dragão, que dilacera com suas
garras, despedaça com sua cabeça, esmaga com sua cauda, tritura com
seus dentes, envenena com seus olhos, e mata com seu sopro. Portanto
vossa majestade quer me mandar para a morte? É essa a sinecura que
ganho por ter lhe dado um reino? Quem é a alma perversa que lançou
esse dado à mesa? Que abominação ensina tais coisas e põe tais
palavras em sua real boca? 

Mas o Rei, que era mais firme que uma rocha em manter o que tinha
dito, bateu os pés no chão, e exclamou:

— Depois de tudo o que você já fez, vai me falhar justo agora? Nem
mais uma palavra; vá, livre meu reino dessa praga, ao menos que
prefira que eu o livre de sua vida.

Pobre Miuccio, que assim ia recebendo num minuto um favor, no próximo
uma ameaça, agora um afago na cabeça, e em seguida um chute, agora
uma palavra gentil, seguida de uma cruel! Ele refletiu sobre o quanto
era mutável a sorte na corte, e que preferia nunca ter conhecido o
Rei. E enquanto ele estava sentado num dos degraus, com a cabeça
entre os joelhos, lavando seus sapatos com suas lágrimas e aquecendo
o chão com seus suspiros, lá veio o pássaro com uma planta no bico, e
jogando-a para ele, disse:

— Levante-se, Miuccio, e crie coragem! Pois não é hora de brincar com
o resto de seus dias, mas de jogar gamão com a vida do dragão. Pegue
essa planta, e quando chegar à caverna daquele animal horrível,
jogue-a lá dentro, e no mesmo instante ele será acometido de tamanha
sonolência que logo adormecerá profundamente, e então, furando-o
bastante, enfiando nele uma boa faca, você logo dará um fim a ele. E
então volte, pois as coisas vão acabar melhores do que você imagina.

— Já chega! — exclamou Miuccio — Sei o que carrego sob meu cinto;
temos mais tempo que dinheiro, e quem tem tempo tem a vida. 

E dizendo isso ele se levantou, pôs uma faca em seu cinto e pegou a
planta, e foi até a caverna do dragão, que ficava sob uma montanha
tão enorme que as três montanhas que eram os degraus do gigante não
chegavam nem a sua metade. Ao chegar lá, jogou a planta dentro da
caverna, e no mesmo instante um sono profundo se apoderou do dragão,
e Miuccio começou a fazê-lo em pedaços. 

Bem no momento que ele estava ocupado com isso, a Rainha sentiu uma
dor aguda em seu coração, e se vendo em maus lençóis, percebeu o erro
que fizera, pois tinha comprado sua morte com dinheiro miúdo. Então
ela chamou seu enteado e disse o que os astrólogos haviam previsto —
como sua vida dependia da do dragão, e que ela temia que Miuccio o
matara, pois se sentia indo aos poucos. O rei respondeu:

— Se você sabia que a vida do dragão era o esteio da sua e a raiz de
seus dias, por que você me fez mandar Miuccio matá-lo? De quem é a
culpa? Você mesmo fez a maldade, e agora tem de pagar por ela; você
quebrou o vidro, agora pague a conta.

E a Rainha respondeu:

— Nunca achei que aquele rapazola tivesse a habilidade e a força de
derrotar um monstro que fez pouco de um exército, e esperava que ele
deixasse sua vida lá. Mas como eu errei, e o meu desígnio saiu de
rota, peço-lhe uma gentileza se você me ama. Quando eu estiver morta,
pegue uma esponja embebida no sangue do dragão e unte as extremidades
de meu corpo com ela antes de me enterrar. 

— Isso é bem pouco perto do amor que tenho por você — respondeu o Rei
— e se o sangue do dragão não fôr suficiente, acrescentarei do meu
para contentar-lhe. 

A Rainha ia lhe agradecer, mas sua respiração faltou ao abrir a boca
para falar; pois naquele momento Miuccio acabara de vez com a vida do
dragão. 

Assim que Miuccio voltou à presença do Rei com a notícia do que
fizera, o Rei mandou que ele voltasse lá buscar o sangue do dragão;
mas ficando curioso de ver a proeza que Miuccio realizara, o seguiu.
E quando Miuccio estava saindo dos portões do palácio, o pássaro o
encontrou, e disse:

— Onde você está indo? — E Miuccio respondeu:

— Estou indo aonde o Rei me mandou; ele me faz ir e voltar feito um
pêndulo; e nunca me deixa descansar uma hora.

— Para fazer o quê? — perguntou o pássaro.

— Buscar sangue do dragão — disse Miuccio.

— Ah, jovem infeliz! O sangue do dragão será sua perdição; pois esse
sangue fará reviver a raiz perversa de todos seus infortúnios. A
Rainha fica continuamente o expondo a perigos para que você perca a
vida; e o Rei, que deixa aquela criatura odiosa pôr as rédeas nele,
manda você, como um vagabundo, para pôr em risco a sua pessoa, que é
do mesmo sangue que corre nas veias dele. Mas o desgraçado não sabe
quem você é, embora a afeição inata que sente sente por você deveria
ter lhe mostrado o parentesco. Além do mais, os serviços que você
prestou ao Rei, e ele ter ganho um filho e herdeiro tão belo, deviam
obter a graça da infeliz Porziella, sua mãe, que há catorze anos está
enterrada viva num sotão, como um templo de beleza embutido num
quartinho.

Enquanto o pássaro dizia isso, o Rei, que ouvira cada palavra, avançou
para saber melhor a verdade da história; e descobrindo que Miuccio
era o filho dele e de Porziella, e que Porziella ainda estva viva no
sotão, no mesmo instante deu ordens para que ela fosse libertada e
trazida até ele. E quando a viu mais bela do que nunca, graças aos
cuidados que lhe dispensara o pássaro, a abraçou com enorme carinho,
e não se cansou de apertar em seus braços e junto a seu coração
primeiro a mãe e então o filho, pedindo perdão a Porziella pelo mau
tratamento que dera a ela, e ao filho pelos perigos a que o expusera.
Então ordenou que a vestissem nas mais ricas roupas, e a fez coroar
Rainha na frente de todos. E quando Rei soube que a preservação dela,
e seu filho ter escapado de todos os perigos eram inteiramente obra
do pássaro, que dera comida à primeira e conselhos ao segundo, ele
lhe ofereceu seu reino e sua vida. Mas o pássaro disse que não queria
nenhuma outra recompensa para seus serviços que ter Miuccio como seu
marido; e assim que disse isso transformou-se numa bela donzela e,
para a grande alegria do Rei e de Porziella, foi dada a Miuccio como
esposa. E então os recém-casados, para darem ainda maiores festas,
foram até seu próprio reino, onde eram ansiosamente esperados, todo
mundo atribuindo essa boa sorte a fada, pela gentileza que Porziella
lhe fizera; pois no fim das contas, “Uma boa ação nunca é em vão”.

\chapter{Dos mitológicos e cia:\break A história de Sigurd\subtitulo{Andrew
Lang}}

Havia uma vez um Rei no Norte que ganhara muitas guerras, só que agora
já estava velho. Mas ele se casou com uma outra mulher, e então um
certo Príncipe, que também queria casar com ela, veio atacá-lo com um
grande exército. O velho Rei lutou bravamente, mas sua espada acabou
se quebrando, ele se feriu e todos os seus homens fugiram. De noite,
depois que a batalha tinha terminado, sua jovem esposa veio
procurá-lo entre os mortos e feridos, e enfim o encontrou, e
perguntou se ele podia ser curado. Mas ele respondeu que não, sua
sorte acabara, sua espada quebrara, e ia morrer. E disse ainda que
ela ia ter um filho, e que esse filho seria um grande guerreiro, e
iria vingar-se do outro Rei seu inimigo. E pediu a ela que guardasse
os pedaços quebrados da espada, para que com eles uma nova espada
fosse feita para o seu filho; e essa nova arma deveria ser chamada de
Gram. 

Então ele morreu. E sua mulher chamou sua criada, e disse:

— Vamos trocar de roupa; e você será chamada pelo meu nome, e eu pelo
seu, caso os inimigos nos encontrem.

Foi o que fizeram, e se esconderam na floresta, mas então uns
forasteiros as encontraram e as levaram num navio para a Dinamarca. E
quando elas foram levadas até o Rei, este achou que a criada parecia
uma rainha, e a rainha uma criada. Então ele perguntou à rainha:

— Como você sabe no meio da noite que falta pouco para amanhecer? 

E ela disse:

— Eu sempre sei porque quando era mais jovem, costumava levantar para
acender os fogos, e ainda acordo na mesma hora.

“Estranho, uma rainha que acende os fogos”, o Rei pensou.

Então ele perguntou à Rainha que estava vestida de criada:

— Como você sabe no meio da noite que a aurora se aproxima? 

— Meu pai me deu um anel de ouro — ela disse — e sempre antes do
manhecer ele esfria em meu dedo.

— Rica essa casa em que as criadas usam ouro — disse o Rei. — Na
verdade você não é nenhuma criada, mas a filha de um rei. 

Então ele a tratou como convinha a uma rainha, e com o passar do tempo
ele teve um filho que chamou Sigurd, um menino belo e muito forte.
Ele tinha um tutor a acompanhá-lo, e um dia o tutor disse a ele para
ir pedir um cavalo para o Rei. 

— Escolha você mesmo o seu cavalo — disse o Rei; e Sigurd foi até a
floresta, onde encontrou um velho com uma barba branca, e disse a
ele:

— Venha! Ajude-me a escolher um cavalo.

O velho disse:

— Leve todos os cavalos para o rio, e escolha aquele que o atravessar
nadando. 

Então Sigurd levou-os até o rio, e só um deles o atravessou. Sigurd o
escolheu: seu nome era Grani, e vinha da linhagem de Sleipnir, e era
o melhor cavalo do mundo. Pois Sleipnir era o cavalo de Odin, o Deus
do Norte, e era tão rápido quanto o vento. 

Um dia ou dois depois disso, o tutor disse a Sigurd:

— Há um grande tesouro em ouro escondido não muito longe daqui, e
seria apropriado para você se o obtivesse. 

Mas Sigurd respondeu:

— Já ouvi falar desse tesouro, e sei que o dragão Fafnir o guarda, e
ele é tão enorme e terrível que nenhum homem ousa se aproximar dele.

— Ele não é maior que outros dragões — disse o tutor — e se você fosse
tão corajoso quanto seu pai não teria medo dele. 

— Não sou um covarde — disse Sigurd. — Por que você quer que eu lute
com esse dragão? 

Então seu tutor, que se chamava Regin, contou a ele que todo aquele
tesouro de ouro vermelho pertencera antes a seu pai. E seu pai teve
três filhos: o primeiro foi Fafnir, o dragão; o segundo foi Lontra,
que podia tomar a forma de uma lontra sempre que quisesse; e o último
foi ele, Regin, que era um grande ferreiro e fabricante de espadas. 

Havia então um anão chamado Andvari, que morava junto a uma piscina
sob uma cachoeira, e lá ele escondera um tesouro. E um dia Lontra
estava pescando ali, e tinha pego um salmão e o comido, e estava
dormindo numa pedra na forma de lontra. Alguém que passava por ali
jogou uma pedra na lontra e a matou; e tirou a pele dela, e a levou
para a casa do pai de Lontra. Então ele soube que seu filho estava
morto, e para punir a pessoa que o matara exigiu que a pele da lontra
fosse enchida com ouro, e coberta com ouro, ou senão teria que se ver
com ele. A pessoa que matara Lontra foi até a cachoeira e capturou o
anão que tinha o tesouro e o tirou dele. 

Só sobrara um anel, que o anão estava usando; mas até este foi tirado
dele. 

O pobre anão ficou muito bravo, e rogou uma praga: aquele ouro só iria
trazer má sorte para quem o possuísse, para sempre. 

Então a pele da lontra foi enchida com ouro e coberta com ouro
inteira, exceto por um pelo, e nele foi enfiado o último anel do
pobre anão. 

Mas não trouxe sorte para ninguém. Primeiro Fafnir, o dragão, matou
seu próprio pai; e então ele foi até o ouro e sobre ele se estendeu,
não deixando seu irmão ficar com nem um pouco; e nenhum homem ousava
se aproximar dele. 

Ao ouvir a história Sigurd disse a Regin:

— Faça-me uma boa espada para eu matar esse dragão. 

Regin fez uma espada para ele, mas ele a testou com um golpe num bloco
de ferro, e ela se quebrou. 

Outra espada foi feita, e Sigurd também a quebrou. 

Então Sigurd foi ter com sua mãe, e pediu os pedaços da espada de seu
pai, e os deu para Regin. E ele os forjou e martelou numa espada
nova, tão afiada que as bordas de sua lâmina pareciam incandescentes.

Sigurd experimentou essa espada num bloco de ferro, e ela não quebrou,
mas partiu ao meio o ferro. Então ele jogou um floco de lã no rio, e
quando ele flutuou até a lâmina cortou-se em dois. Sigurd disse que
aquela espada servia. Mas antes de ir lutar com o dragão ele liderou
um exército para lutar com os homens que haviam matado seu pai, e ele
executou o Rei deles, ficou com toda sua fortuna, e voltou para casa.


Depois de alguns dias, ele foi a cavalo com Regin até a charneca onde
o dragão costumava ficar. Então viu o rastro que o dragão deixava
quando ia num rochedo beber; e o rastro era como se um grande rio
tivesse passado e deixado um vale profundo. 

Então Sigurd foi até esse valo profundo, e cavou muitos buracos nele,
e num deles se escondeu com sua espada na mão. Lá ficou esperando, e
logo a terra começou a tremer com o peso do dragão rastejando até a
água. E uma nuvem de veneno era lançada à frente dele quando ele
bufava e urrava, de modo que seria morte certa ficar na frente dele. 

Mas Sigurd esperou até a metade do corpo do dragão ter rastejado sobre
o buraco, e então enfiou a espada Gram bem no coração dele. 

O dragão vergastou sua cauda em volta, despedaçando rochas e
derrubando árvores. E quando viu que ia morrer, disse:

— Quem quer que seja você que me matou, esse ouro será sua ruína, e a
ruína de todos aqueles que o possuírem. 

Sigurd disse:

— Eu não o tocaria se perdê-lo me fizesse nunca morrer. Mas todos os
homens morrem, e nenhum homem corajoso deixa a morte amedrontá-lo de
ter aquilo que deseja. Morra, Fafnir! 

E Fafnir morreu. 

E Sigurd passou a ser chamado de o Algoz de Fafnir, e Matador de
Dragões. 

Então Sigurd voltou, e encontrou com Regin, que disse a ele para assar
o coração de Fafnir e deixá-lo provar. 

Sigurd pôs o coração de Fafnir num espeto, e o pôs para assar. Mas
aconteceu dele tocá-lo por acidente, queimando o dedo. Ele pôs o dedo
na boca, e assim provou o coração de Fafnir. 

E imediatamente ele compreendeu a linguagem dos pássaros, e ouviu o
que diziam os picapaus:

— Eis Sigurd, assando o coração de Fafnir para outro, quando devia ser
ele mesmo a prová-lo e aprender toda a sabedoria. 

O pássaro ao lado disse:

— Eis Regin, pronto para trair Sigurd, que confia nele. 

O terceiro pássaro disse:

— Que ele corte a cabeça de Regin, e fique com todo o ouro para si
mesmo.

O quarto pássaro disse:

— Que ele faça isso, e então vá para Hindfell, o lugar onde Brynhild
dorme. 

Ao ouvir essas palavras, e como Regin tramava traí-lo, Sigurd cortou a
cabeça dele com um só golpe da espada Gram.

Então todos os pássaros cantaram, e era uma canção sobre uma bela
jovem que dormia num lugar cercado por fogo, à espera dele para
acordá-la. 

E Sigurd lembrou-se que havia uma história sobre uma bela dama
enfeitiçada num lugar muito distante dali. Ela estava sob um encanto,
que a fazia dormir num castelo cercado por chamas; lá ela dormiria
até que chegasse um cavaleiro capaz de atravessar o fogo para
acordá-la. Ele decidiu ir até lá, mas antes seguiu o horrível rastro
de Fafnir. E encontrou a caverna de Fafnir, que era bem profunda e
tinha portas de ferro, e estava cheia de braceletes, coroas e anéis
de ouro; e lá Sigurd achou também o Elmo do Pavor, que era de ouro e
deixava invisível quem o usava. Ele carregou tudo isso no bom cavalo
Grani, e então partiu para Hindfell ao sul. 

Já era de noite, e no cimo de uma colina Sigurd viu um fogo brilhando
vermelho no céu, e dentro das chamas um castelo, com um estandarte na
torre mais alta. Então ele arremeteu no fogo com seu cavalo Grani,
que o saltou com leveza, como se não passasse de um arbusto. E Sigurd
passou pelo portão do castelo, e viu alguém dormindo, usando uma
armadura. Ele tirou o elmo da pessoa adormecida, e para sua surpresa
viu que era uma dama das mais belas. Ela acordou e disse:

— Ah, é Sigurd, o filho de Sigmund, que quebrou a maldição, e veio
enfim me acordar?

A maldição tinha sido posta nela quando o espinho da árvore do sono
picou sua mão, tempos atrás, como uma punição por ter desagradado ao
deus Odin. E tempos atrás também ela tinha jurado nunca casar com um
homem que tivesse medo, e não ousasse atravessar a cerca de chamas.
Pois ela era uma guerreira, e ia para as batalhas armada como um
homem. Mas agora ela e Sigurd se apaixonaram, e prometeram ser fiéis
uma ao outro, e ele deu a ela um anel, que era o último anel do anão
Andvari. Então Sigurd partiu, e encontrou o castelo de um Rei que
tinha uma bela filha. O nome dela era Gudrun, e sua mãe era uma
feiticeira. Gudrun se apaixonou por Sigurd, mas ele só falava em
Brynhild, como ela era bela e o quanto ele a amava. Um dia então a
mãe feiticeira de Gudrun pôs sementes de papoula e outras drogas de
esquecimento numa taça mágica, e fez Sigurd beber à saúde dela, e no
mesmo instante ele esqueceu a pobre Brynhild, e se apaixonou por
Gudrun, e eles se casaram com grandes festejos.

A feiticeira, mãe de Gudrun, quis então que seu filho Gunnar casasse
com Brynhild, e disse a ele para ir com Sigurd fazer a corte a ela.
Eles partiram para o castelo do pai dela, pois Brynhild tinha saído
completamente da cabeça de Sigurd por causa da poção da feiticeira,
mas ela ainda lembrava dele e o amava. O pai de Brynhild disse a
Gunnar que ela só casaria com alguém capaz de atravessar o fogo em
frente a sua torre encantada, e para lá eles foram, e Gunnar tentou
passar pelas chamas, mas seu cavalo não quis enfrentá-las. Então
Gunnar tentou usar o cavalo de Sigurd, mas montado por ele Grani não
se movia. Daí Gunnar lembrou-se dos feitiços que sua mãe havia lhe
ensinado, e com eles fez Sigurd ficar exatamente igual a ele, e ele
exatamente igual a Sigurd. Então Sigurd, sob a forma de Gunnar,
montou em Grani, e Grani saltou a cerca de fogo, e ele foi até
Brynhild, mas ainda nada lembrava dela, por causa da poção de
esquecimento da feiticeira.

E Brynhild não teve alternativa a não ser prometer casar com ele, ser
a esposa de Gunnar, pois Sigurd estava sob a forma de Gunnar, e ela
prometera casar com quem quer que atravessasse o fogo. E ele deu a
ela um anel, e ela devolveu a ele o anel que ele tinha lhe dado antes
sob sua própria forma de Sigurd, aquele que era o último anel do
pobre anão Andvari. Ele voltou, e trocou de forma com Gunnar, e cada
um sendo ele mesmo de novo, voltaram ao castelo da Rainha feiticeira,
e Sigurd deu o anel do anão a sua esposa Gudrun. E Brynhild foi ter
com seu pai e disse que um príncipe chamado Gunnar havia atravessado
o fogo, e ela teria de casar com ele. 

— E no entanto eu achava — ela disse — que nenhum homem conseguiria
tal proeza a não ser Sigurd, o Algoz de Fafnir, que era o meu
verdadeiro amor. Mas ele esqueceu de mim, e tenho de manter minha
promessa. 

Então Gunnar e Brynhild se casaram, embora não tivesse sido Gunnar que
atravessara o fogo, mas Sigurd sob a forma dele. 

Quando acabou o casamento e todos os festejos, a mágica da poção da
feiticeira deixou a mente de Sigurd, e ele lembrou de tudo. Lembrou
que libertara Brynhild do encanto, e que era ela seu verdadeiro amor,
e como a esquecera e casara com outra mulher, e conquistara Brynhild
para ser a mulher de outro homem. 

Mas era corajoso, e nada disse aos outros, para não fazê-los
infelizes. Não conseguia escapar, entretanto, da maldição sobre todo
aquele que possuía o tesouro do anão Andavari, e seu anel de ouro
fatal. 

E logo a maldição voltou a se abater sobre todos eles. Pois um dia,
quando Brynhild e Gudrun estavam se banhando num rio, Brynhild foi
mais fundo nas águas, e disse que fizera isso para provar sua
superioridade sobre Gudrun; porque o marido dela, disse, tinha
atravessado o fogo que nenhum outro homem ousara enfrentar. 

Gudrun ficou muito brava, e disse que tinha sido Sigurd, e não Gunnar,
que atravessara o fogo, e recebera de volta de Brynhild aquele anel
fatal, o anel do anão Andvari. 

Então Brynhild viu o anel que Sigurd dera a Gudrun, e soube de tudo, e
ficou tão pálida como morta, e foi para casa. Durante aquela noite
ela ficou em silêncio. Na manhã seguinte, ela disse a Gunnar, o
marido dela, que ele era um covarde e um mentiroso, pois jamais
atravessara o fogo, mas mandara Sigurd no lugar dele, e fingira que
tinha sido ele. E disse que ele jamais voltaria a vê-la feliz em sua
casa, bebendo vinho, jogando xadrez, bordando com fio de ouro, ou
trocando palavras carinhosas. Então ela rasgou todos seus bordados e
chorou bem alto, para que todos na casa a ouvissem. Pois seu coração
se partira, e no mesmo instante também seu orgulho. Perdera seu
verdadeiro amor, Sigurd, o Algoz de Fafnir, e casara com um homem que
era um mentiroso. 

Sigurd então veio tentar consolá-la, mas ela não quis ouví-lo, e disse
que queria uma espada atravessando o coração dela. 

— Não terá que esperar muito — ele disse — até a espada atravessar o
meu coração, pois você não viverá muito depois que eu morrer. Mas,
Brynhild querida, console-se e siga vivendo, e ame Gunnar seu marido,
e eu lhe darei todo o ouro, o tesouro do dragão Fafnir. 

Brynhild disse:

— Tarde demais.

Sigurd se viu tomado de tamanha tristeza que seu coração tanto inchou
em seu peito que os anéis de sua malha de ferro se romperam.

Ele então se foi, e Brynhild decidiu matá-lo. Misturou veneno de
serpente com carne de lobo, e ofereceu-os num prato para o irmão mais
novo de seu marido; e quando ele comeu ficou enlouquecido, e foi até
o quarto de Sigurd enquanto ele dormia e o prendeu a cama
atravessando-o com uma espada. Mas Sigurd acordou, e agarrou a espada
Gram, e atirou-a no homem que fugia, partindo-o ao meio. E assim
morreu Sigurd, o Algoz de Fafnir, que nem dez homens seriam capazes
de matar num combate justo. Então Gudrun acordou e o viu morto, e
ficou aos prantos; Brynhild a ouviu e riu, mas o o bom cavalo Grani
se deitou e morreu de tanto pesar. E aí Brynhild se pôs a chorar
amarga e desesperadamente, até que seu coração se partiu de vez.
Então os outros puseram em Sigurd sua armadura dourada, e ergueram
uma fogueira a bordo de seu navio, e de noite nele deitaram os mortos
Sigurd e Brynhild, e o bom cavalo Grani, atearam fogo e o lançaram na
água. E o vento levou-o ardendo para o mar, as chamas brilhando na
escuridão. E Sygurd e Brynhild foram cremados juntos, e a maldição do
anão Andvari se cumpriu.

\chapter{O dragão relutante\subtitulo{Kenneth Grahame }}

Tempos atrás — pode ter sido centenas de anos atrás — num chalé a meio
caminho entre uma aldeia e as encostas dos Downs, no sul da
Inglaterra, vivia um pastor com sua mulher e seu filho. O pastor
passava os dias — e em certas épocas do ano as noites também — lá em
cima nas vastas encostas, tendo apenas o sol, as estrelas e as
ovelhas de companhia, bem longe da cordial tagarelice do mundo dos
homens e das mulheres. Mas seu filho, quando não estava ajudando seu
pai, e às vezes quando estava também, passava muito de seu tempo
imerso em grossos volumes que ele emprestava da gente culta afável e
dos vigários letrados da região em volta. E seus pais gostavam muito
dele, e também tinham bastante orgulho dele, embora não o dissessem
na frente dele, de modo que deixavam-no viver a vida dele e ler o
quanto quisesse; e em vez de frequentemente levar uns sopapos na
cabeça, como bem poderia ter acontecido com ele, era tratado mais ou
menos como um igual por seus pais, os quais achavam sensatamente que
era uma divisão de trabalho muito justa aquela em que eles forneciam
o conhecimento prático, e ele o que se encontrava nos livros. Sabiam
que o conhecimento dos livros não raro podia se revelar muito útil
numa emergência, apesar do que os seus vizinhos diziam. O que
sobretudo interessava o Menino era história natural e contos de
fadas, e ele ia lendo os dois assuntos conforme apareciam, de um
jeito meio ensanduichado, sem fazer quaisquer distinções; e realmente
esse modo dele de ler parece bastante sensato.

Um dia ao entardecer o pastor, que fazia algumas noites andava
incomodado e preocupado, fora de seu usual equilibrio mental, voltou
para casa tremendo todo e, sentando à mesa onde sua mulher e filho
estavam tranquilamente ocupados, ela com sua costura, ele com as
aventuras do Gigante sem Coração em seu Corpo, exclamou muito
agitado:

— Está tudo acabado para mim, Maria! Nunca mais de jeito nenhum eu vou
poder subir lá nas encostas, nem uma vez mais!

— Não fique assim — disse sua mulher, que era muito sensata. —
Primeiro conte-nos tudo o que aconteceu, o que foi que lhe causou
toda essa agitação, e então entre eu e o filho aqui, talvez a gente
consiga esclarecer o assunto.

— Começou algumas noites atrás — disse o pastor. — Sabe aquela caverna
que tem lá em cima; eu nunca gostei dela, de algum jeito, e as
ovelhas também não, e quando as ovelhas não gostam de alguma coisa em
geral há uma razão. Bom, já faz algum tempo que dela tem vindo uns
ruídos fracos, ruídos como suspiros profundos, com uns grunhidos no
meio; e às vezes um ronco, bem lá do fundo; ronco de verdade, mas de
algum jeito não um ronco honesto, como o meu e o seu de noite, você
sabe como é.

— Eu sei — observou o Menino, discretamente.

— Claro, fiquei com muito medo — o pastor continuou — e no entanto não
consegui ficar longe. Então hoje no fim da tarde, antes de vir
embora, fui de mansinho dar uma olhada ali pela caverna. E lá, meu
Deus!, lá estava ele, enfim o vi, tão bem quanto a vejo agora!

— Viu quem? — disse sua mulher, começando a se contagiar com o
aterrorizado nervosismo do marido.

— Ora, ele, estou lhe dizendo! — disse o pastor. — Ele estava parado
metade fora da caverna, e parecia estar apreciando o friozinho do fim
da tarde de um jeito meio poético. Era tão grande quanto quatro
cavalos de puxar carroça, e todo coberto de escamas brilhantes;
escamas azul escuro na parte de cima dele, passando para um verde
suave embaixo. Quando ele respirava, sobre suas narinas havia esse
tipo de ar tremido que se vê sobre as estradas num dia de sol quente
sem vento no verão. Ele estava com o queixo apoiado nas patas, e eu
diria que ele estava meditando sobre as coisas. Oh, sim, era um tipo
bastante pacífico de animal, sem ameaçar atacar ou aprontar ou fazer
qualquer coisa que não fosse certa ou direita. Admito tudo isso.
Ainda assim, o que posso fazer? Escamas, sabe, e garras, e com
certeza uma cauda, embora essa ponta dele eu não tenha visto; não
estou acostumado com essas coisas, e não me dou bem com elas, e isso
é um fato.

O Menino, o qual aparentemente ficara concentrado em seu livro durante
a récita de seu pai, fechou o volume, bocejou, cruzou as mãos atrás
da cabeça, e disse sonolento:

— Está tudo certo, pai. Não precisa se preocupar. É só um dragão.

— Só um dragão? — seu pai gritou. — O que você quer dizer, sentado aí,
você e seus dragões? Só um dragão, essa é boa! E o que você sabe
disso?

— Porque é, e porque eu sei — respondeu o Menino, calmamente. —
Escute, pai, você sabe que cada um de nós tem a sua
especialidade.Você sabe sobre ovelhas, e o tempo, e coisas assim; eu
sei sobre dragões. Eu sempre disse, você sabe, que aquela caverna lá
em cima era uma caverna de dragão. Eu sempre disse que alguma época
ela devia ter sido de um dragão, e deveria ser de um dragão agora, se
as regras valem alguma coisa. Bom, agora você me diz que ela tem um
dragão, e isso que é o certo. Eu fiquei bem menos surpreso agora do
que quando você me disse que não tinha um dragão nela. As regras
sempre dão certo se você espera com calma. Agora, por favor, deixe
que eu cuido disso. Irei dar um passeio por lá amanhã de manhã - não,
de manhã não dá, tenho uma pilha de coisas para fazer - bom, talvez
no fim da tarde, se eu tiver tempo, irei até lá ter uma conversa com
ele, e você verá que vai dar tudo certo. Só o que peço, por favor, é
que você não fique se preocupando por ali sem mim. Você não entende
nada deles, e eles são muito sensíveis, sabe?

— Ele está muito certo, pai — disse a mãe sensata. — Como ele disse,
dragões são especialidade dele e não nossa. Ele é ótimo nisso de
animais de livros, como todo mundo admite. E para falar a verdade, eu
mesma não fico nem um pouco contente, pensando no pobre daquele
animal sozinho lá em cima, sem um jantar quentinho ou alguém com quem
trocar idéias; e quem sabe poderemos fazer alguma coisa por ele; e se
ele não for respeitável nosso Menino vai logo descobrir. Ele tem um
certo jeito simpático de ser que faz todo mundo contar tudo para ele.

No dia seguinte, depois do jantar, o Menino subiu a estrada de
cascalho que leva até o topo dos Downs; e lá, sem dúvida, encontrou o
dragão, deitado preguiçosamente na relva na frente da caverna dele. A
vista dali é das mais magníficas. Para a direita e esquerda, léguas e
léguas das ondulantes encostas sem árvores dos Downs; à frente, o
vale, com as aglomerações das fazendas, e a trama de estradinhas
brancas passando por pomares e terras bem aradas e, bem longe no
horizonte, sinais das velhas cidades cinzentas. Uma brisa fresca
brincava com a superfície da relva e um pedaço prateado de uma enorme
lua estava aparecendo sobre distantes zimbros. Não era nenhuma
surpresa que o dragão parecesse estar num estado de espírito
tranquilo e satisfeito; e de fato, ao ir chegando perto o Menino
ouviu-o ronronando com uma feliz regularidade. “Bom, vivendo e
aprendendo!” ele disse para si mesmo. “Nenhum dos meus livros jamais
disse que dragões ronronavam!” 

— Olá, dragão — disse o Menino tranquilamente, quando chegou até onde
ele estava.

O dragão, ao ouvir alguém se aproximando, começara a fazer um esforço
para se levantar, por cortesia. Mas quando viu que era um Menino,
franziu as sobrancelhas com severidade.

— Você não me venha me bater — ele disse — ou jogar pedras, ou
espirrar água, ou qualquer coisa assim. Não vou tolerar, já vou logo
avisando.

— Não vou jogar nada em você — disse o Menino, aborrecido, largando-se
sentado na grama perto do bicho. — E não, pelo amor de Deus, fique me
dizendo “não isso” e “não aquilo”; eu já ouço tanto disso, e é
monótono, e enjoa. Eu só passei por aqui para perguntar como você vai
e esse tipo de coisas; mas se estou atrapalhando eu posso muito bem
ir embora. Tenho um monte de amigos, e nenhum deles pode dizer que eu
tenho o hábito de ficar insistindo quando não querem a minha
presença!

— Não, não, não vá embora bravo — o dragão apressou-se a dizer. — O
fato é que, eu estou muito bem mesmo aqui em cima; nunca sem alguma
ocupação, meu camarada, nunca sem alguma ocupação! Ainda assim, cá
entre nós, é um pouquinho chato às vezes.

O Menino mordeu um talo de grama e ficou chupando-o. 

— Pretende ficar muito tempo por aqui? — perguntou, polidamente. 

— No momento, não saberia dizer — respondeu o dragão. — Parece um
lugar bom o bastante; mas faz pouco tempo que estou aqui, e é preciso
ver mais e refletir e considerar bem antes de se decidir por ficar
num lugar. É uma coisa muito séria, escolher onde morar. Além disso -
e agora eu vou lhe contar uma coisa que você jamais teria adivinhado,
mesmo se tivesse tentado - o fato é que eu sou um vagabundo danado de
preguiçoso!

— Fico muito surpreso — disse o Menino, educadamente.

— É a triste verdade — o dragão continuou, se acomodando entre suas
patas e evidentemente encantado de ter enfim achado um ouvinte. — E
imagino que seja isso que me fez vir parar aqui, na verdade. Veja
você, todos os outros camaradas são tão ativos e sérios e todo esse
tipo de coisas, sempre sendo agressivos, e arranjando briga, e
varrendo as areias do deserto, e rondando as margens do mar, e
perseguindo cavaleiros por toda a parte, e devorando donzelas, e
aprontando em geral... já eu sempre gostei de fazer minhas refeições
na hora certa, e então deitar as costas em alguma rocha e tirar uma
sonequinha, e daí acordar e pensar como as coisas vão e como elas
sempre vão iguais, sabe? De modo que quando aconteceu eu fui pego de
surpresa.

— Quando o que aconteceu, posso saber? — perguntou o Menino.

— É precisamente isso que eu não sei — disse o dragão. — Suponho que a
terra espirrou, ou chacoalhou, ou o fundo caiu de algum lugar. De
qualquer modo houve um tremor, um estrondo e uma total barafunda, e
eu fui parar milhas debaixo da terra, e fiquei completamente preso.
Bom, graças a Deus, não preciso de muito, e de qualquer maneira tinha
paz e tranquilidade, e não ficava sempre sendo chamado para ir junto
fazer alguma coisa. E eu tenho uma mente tão ativa: sempre ocupada,
eu lhe asseguro! Mas o tempo foi passando, e minha vida começou a
ficar uma certa mesmice, e afinal achei que ia ser divertido abrir um
caminho para o andar de cima e ver o que os outros andavam fazendo.
Então eu cavei e escavei, e trabalhei assim e assado, e enfim
consegui sair através dessa caverna aqui. E eu gostei da região, e da
vista, e das pessoas - as poucas que eu vi - e no geral estou
inclinado a ficar por aqui.

— Com o que sua mente está sempre ocupada? — perguntou o Menino. —
Gostaria de saber. 

O dragão ficou ligeiramente vermelho e desviou os olhos. Enfim disse,
envergonhado:

— Você alguma vez, só de farra, já tentou fazer poesia... versos,
sabe?

— Claro que sim — disse o Menino. — Pilhas e pilhas. E algumas delas
são bem boas, tenho certeza, só que não há ninguém por aqui que ligue
para essas coisas. Minha mãe é muito simpática e etcétera, quando eu
as leio para ela, e também meu pai, quanto a isso. Mas de algum jeito
eles não...

— Exatamente — exclamou o dragão — exatamente o meu caso. Eles não...
e não há o que você possa fazer com isso. Agora, você eu vejo que tem
cultura, vi no mesmo instante, e eu gostaria de ouvir sua opinião
sincera sobre algumas coisinhas que eu fui pondo no papel, enquanto
estava por aqui. Fiquei terrivelmente satisfeito em conhecê-lo, e
espero que os outros vizinhos sejam igualmente agradáveis. Ontem à
noite mesmo esteve aqui um velho senhor muito simpático, mas ele
pareceu não querer incomodar.

— Era o meu pai — disse o Menino — e ele é um velho senhor simpático,
e qualquer dia eu posso apresentá-lo a ele se você quiser.

— Será que vocês dois não poderiam subir aqui amanhã para um jantar ou
algo assim? — perguntou o dragão ansiosamente. E acrescentou
polidamente — Claro, se vocês não tiverem nada melhor para fazer. 

— Muito obrigado mesmo — disse o Menino — mas não saímos para ir em
lugar nenhum sem minha mãe e, para falar a verdade, receio que ela
possa não aprová-lo muito. Veja você, não há como escapar da dura
realidade de que você é um dragão, há? E quando você fala de ficar
morando aqui, e os vizinhos, e por aí afora, não cosnigo evitar a
impressão de que você não percebe exatamente a sua situação. Você é
um inimigo da raça humana, afinal!

— Não tenho um inimigo no mundo — disse o dragão, alegremente. — Muito
preguiçoso para fazer algum, para começo de conversa. E se eu insisto
em ler minhas poesias para os outros, estou sempre pronto para ouvir
as deles!

— Ora essa! — exclamou o Menino. — Gostaria que você tentasse
realmente entender a sua situação. Quando as outras pessoas ficarem
sabendo sobre você, vão vir todas atrás de você com lanças e espadas
e todo tipo de coisas. Você terá que ser exterminado, de acordo com a
maneira deles de ver o mundo! Você é um flagelo, uma praga, um
monstro assassino!

— Não há uma palavra de verdade nisso tudo — disse o dragão,
balançando a cabeça solenemente. — Meu caráter provar-se-á íntegro
mesmo sob a mais estrita das investigações. E agora, eis um pequeno
sonetinho no qual eu estava trabalhando quando você apareceu...

— Ih, se você se recusa a ser sensato — exclamou o Menino,
levantando-se — eu vou embora para casa. Não, não posso dar só uma
olhadinha no soneto; minha mãe está me esperando. Vou vir vê-lo
amanhã, alguma hora, e você pelo amor de Deus vê se tenta entender
que é um flagelo pestilencial, ou você vai acabar entrando na pior
fria. Boa noite!

Foi fácil para o Menino deixar seus pais tranquilos em relação a seu
novo amigo. Eles sempre tinham confiado ao menino esse ramo, e
aceitaram o que ele disse sem um murmúrio. O pastor foi formalmente
apresentado, e muitos elogios e informações foram gentilmente
trocados. Sua mulher, no entanto, mesmo se dizendo disposta a fazer o
que estivesse a seu alcance — remendar coisas, pôr ordem na caverna,
ou cozinhar alguma coisinha quando o dragão ficara absorto em seus
sonetos e esquecera das refeições, como machos sempre fazem — não
pôde ser convencida a aceitá-lo completamente. O fato que ele era um
dragão e que “eles não sabiam quem ele era” parecia contar mais que
tudo para ela. No entanto, ela não fez objeção a que seu filhinho
passasse tranquilamente todo começo de noite com o dragão, desde que
voltasse para casa até às nove; e muitas noites agradáveis eles
tiveram, sentados na relva, enquanto o dragão contava histórias dos
velhos, bem velhos tempos, quando havia dragões de monte e o mundo
era um lugar mais animado que agora, e a vida cheia de frêmitos,
sobressaltos e surpresas.

Mas o que o Menino temia, todavia, não demorou a acontecer. Mesmo o
mais reservado e discreto dos dragões do mundo, se é tão grande
quanto quatro cavalos e coberto de escamas azuis, acaba não tendo
como escapar do conhecimento público. De modo que nas noites na
taverna da aldeia o fato que um dragão de verdade ficava cismando
numa caverna nos Downs era naturalmente um assunto das conversas.
Embora os aldeões estivessem extremamente com medo, estavam também
bastante orgulhosos. Era uma marca de distinção ter um dragão
próprio, acrescentava um charme a aldeia. Ainda assim, todos
concordavam que era o tipo da coisa que não se podia permitir que
continuasse. O terrível monstro devia ser exterminado, o campo devia
ficar livre daquela praga, daquele terror, daquele flagelo
destruidor. O fato que nem mesmo um único poleiro de galinhas havia
sido prejudicado desde que o dragão chegara não era levado em conta,
nem se permitia que tivesse algo a ver com o assunto. Ele era um
dragão, e não podia negá-lo, e se preferia não se comportar como um
isso era lá problema dele. Mas apesar de muita valente falação nenhum
herói aparecia disposto a pegar espada e lança para libertar a aldeia
de seu padecimento e adquirir fama imortal; e a inflamada discussão
de todas as noites acabava em nada. Enquanto isso o dragão, um boêmio
feliz, refastelava-se na relva, apreciava os pôres- do-sol, contava
anedotas antediluvianas para o Menino, e polia seus velhos versos ou
cogitava novos.

Um dia o Menino, ao entrar na aldeia, descobriu que estava tudo com
uma aparência de festa, embora nenhuma constasse do calendário.
Tapetes e panos de cores alegres estavam pendurados nas janelas, os
sinos da igreja repicavam ruidosamente, a rua estreita estava cheia
de flores, e a população inteira estava se empurrando dos dois lados
dela, tagarelando, empurrando, e mandando uns aos outros ficarem mais
para trás. O Menino viu um amigo da mesma idade dele e o chamou. 

— O que está acontecendo? — gritou. — São atores, ou ursos, ou o
circo, ou o quê?

— Está tudo bem — seu amigo gritou de volta. — Ele está vindo.

— Quem está vindo? — quis saber o Menino, sendo empurrado na multidão.

— Ora, quem; São Jorge, claro — respondeu seu amigo. — Ele ouviu falar
do dragão, e está vindo com o propósito de matar o mortífero monstro,
e nos libertar de seu horrivel jugo. Nossa, vai ser uma luta
daquelas!

Aquela era uma novidade e tanto! O Menino achou que devia se
certificar ele mesmo, e se insinuou pelo meio das pernas dos mais
velhos, xingando-os o tempo todo pelo seu mal-educado hábito de ficar
empurrando. Quando conseguiu chegar na fila da frente, ficou
esperando ansiosamente a chegada.

Logo veio da ponta mais distante da multidão o som das ovações. Em
seguida, o barulho compassado dos cascos do cavalo de batalha
chegando fez seu coração bater mais rápido, e logo ele se descobriu
gritando junto com o resto quando, em meio aos brados de boas-vindas,
os gritinhos agudos das mulheres, o erguimento de bebês e a agitação
de lenços, São Jorge avançou lentamente pela rua. O coração do Menino
parou quieto e ele respirava aos soluços, a beleza e a graça do herói
iam muito além de qualquer coisa que ele já tinha visto. Sua armadura
era incrustada em ouro, seu elmo pendia da sela, e seus densos
cabelos louros emolduravam um rosto de uma delicadeza inexprimível,
até que se via a firmeza dos olhos. Ele puxou as rédeas em frente à
estalagem e os aldeões se aglomeraram em volta com saudações e
agradecimentos e enumerações loquazes de injúrias, injustiças e
opressões. O Menino ouviu a voz séria e gentil do Santo, assegurando
a todos que tudo iria ficar bem agora, e que ele ia lutar por eles,
reparar as injustiças e livrá-los de seu inimigo; então ele desceu do
cavalo e entrou pela porta, e a multidão se amontoou atrás dele. Mas
o Menino saiu em disparada para o morro o mais rápido que suas pernas
conseguiam.

— Está tudo acabado, dragão! — ele gritou assim que viu o bicho. — Ele
está vindo! Ele já está aqui! Você vai ter que criar vergonha e enfim
fazer alguma coisa!

O dragão estava lambendo suas escamas e esfregando-as com um pedaço de
flanela que a mãe do Menino lhe emprestara, até ficar brilhante feito
uma grande turquesa.

— Não seja violento, menino — ele disse sem se virar. — Sente e
recupere o fôlego, e tente lembrar que o sujeito rege o predicado, e
então talvez você poderá me fazer a gentileza de dizer quem está
vindo.

— Está certo, fique frio — disse o Menino. — Só quero ver você
conseguir ficar na metade dessa tranquilidade depois que eu contar a
novidade. É só o São Jorge quem está vindo, não passa disso; ele
chegou na aldeia faz uma meia hora. Claro que você pode dar uma surra
nele, um sujeito grandão como você! Mas achei que eu devia
prevení-lo, porque com certeza ele vai vir para cá logo cedo, e ele
tem a lança mais comprida e mais mal encarada que você jamais viu!

E o Menino ficou de pé e se pôs a pular para lá e para cá de puro
prazer com a perespectiva da batalha.

— Essa não — gemeu o dragão. — Isso é muito desagradável. Eu não quero
vê-lo, e isso não se discute. Não tenho o menor interesse em conhecer
o sujeito. Tenho certeza que ele não é boa gente. Você precisa ir lá
dizer a ele para ir embora imediatamente, por favor. Diga a ele que
pode escrever se quiser, mas eu não vou recebê-lo. Não estou para
ninguém, no momento. 

— Ah, dragão, dragão — disse o menino, implorando — não seja insensato
e teimoso. Você tem que lutar com ele alguma hora, você sabe, porque
ele é o São Jorge e você é o dragão. Melhor se livrar disso logo, e
então você pode continuar com seus sonetos. E você precisava levar um
pouco em consideração os outros também. Se por aqui anda meio chato
para você, imagine só o quanto tem sido chato para mim!

— Meu caro homenzinho — disse o dragão solenemente — faça o favor de
entender, de uma vez por todas, que eu não posso lutar e eu não vou
lutar. Eu jamais lutei em toda a minha vida, e não vou começar agora,
só para você ter um espetáculo de feira. Nos velhos tempos eu sempre
deixava os outros, os que eram sérios, se encarregarem das lutas
todas, e sem dúvida essa é a razão de eu ter o prazer de estar aqui
agora.

— Mas se você não lutar ele vai cortar fora sua cabeça! — o Menino
disse quase num soluço, desconsolado de perder tanto a luta quanto
seu amigo.

— Ah, acho que não — disse o dragão, de seu jeito indolente. — Você
vai ser capaz de dar um jeito. Tenho inteira confiança em você, você
é tão bom para cuidar das coisas. Seja um bom sujeito, desça até lá,
e resolva tudo. Deixo inteiramente por sua conta.

O Menino percorreu todo o caminho de volta à aldeia num estado de
grande desânimo. Antes de mais nada, não ia haver combate algum;
depois, seu bom e honrado amigo o dragão não se mostrou nem um pouco
heróico como ele teria gostado; e por fim, fosse o dragão no fundo um
herói ou não, não fazia a menor diferença, pois sem a menor dúvida
São Jorge ia cortar fora a cabeça dele. “Dar um jeito, sei”, ele
disse amargamente para si mesmo. “O dragão trata a coisa toda como se
tivesse sido convidado para um chá e uma partida de croquê.”

Os aldeões estavam indo para suas casa quando ele passou pela rua,
todos muito animados, e jubilantemente discutindo a esplêndida luta
que estava para acontecer. O Menino seguiu até a estalagem, e foi até
o salão principal, onde São Jorge estava agora sentado sozinho,
considerando suas chances na luta, e as tristes histórias de saques e
violências tão recentemente despejadas em seus ouvidos solidários.

— Posso entrar, São Jorge? — o Menino perguntou polidamente, da porta
onde parara. — Eu gostaria de conversar com o senhor sobre esse
assunto do dragão, se o senhor não estiver cansado dele a essa
altura.

— Sim, pode entrar, Menino — disse o Santo gentilmente. — Outra
história de maldade e infortúnio, receio. Sua boa mãe ou seu honrado
pai, foi algo assim, de que o tirano o privou? Ou de alguma terna
irmãzinha, ou irmão caçula? Bom, logo você será vingado.

— Nada assim — disse o Menino. — Há algum mal-entendido em algum
lugar, e eu queria esclarecê-lo. O fato é que trata-se de um bom
dragão.

— Exatamente — disse São Jorge, sorrindo satisfeito. — Entendo
perfeitamente. Um bom dragão. Creia-me, eu nem por um momento lamento
que ele seja um adversário à altura da minha espada, em vez de um
espécime fraco da sua tribo nociva.

— Mas ele não é uma tribo nociva — disse o Menino todo aflito. —
Caramba, como ficam estúpidas as pessoas quando metem uma idéia na
cabeça! O que eu estou dizendo é que ele é um bom dragão, e meu
amigo, e conta as histórias mais bonitas que você já ouviu, todas
sobre os velhos tempos e quando ele era pequeno. E ele é muito gentil
com a minha mãe, e ela faria qualquer coisa por ele. E meu pai gosta
dele também, embora não tenha lá muito interesse em arte e poesia, e
sempre durma quando o dragão começa a falar sobre estilo. Mas o fato
é que ninguém consegue deixar de gostar dele assim que o conhece. Ele
é tão cativante e confiánte, tão simples quanto uma criança!

— Sente-se, e aproxime sua cadeira — disse São Jorge. — Gosto de
alguém que defende seus amigos, e tenho certeza que o dragão tem seus
pontos positivos, se tem um amigo feito você. Mas essa não é a
questão. Durante toda esta noite eu fiquei ouvindo, com inexprimível
angústia e pesar, histórias de assassinatos, roubos, e demais
maldades; talvez um pouco exageradas demais, nem sempre inteiramente
convincentes, mas no geral constituindo uma lista de crimes dos mais
sérios. A história nos ensina que o pior dos malfeitores não raro se
mostra todo virtuoso em seu lar; e eu receio que seu culto amigo,
apesar das qualidades pelas quais ele mereceu (e devidamente) sua
afeição, tem que ser exterminado o quanto antes.

— Ah, o senhor engoliu todas as lorotas que esse fulanos ficaram lhe
contando — disse o Menino, sem paciência. — Ora, os aldeões daqui são
os maiores contadores de lorota de toda a região. É um fato sabido e
notório. O senhor é um forasteiro por aqui, senão já teria ouvido
falar. Tudo o que eles querem é uma luta. Eles são capazes de
qualquer coisa para conseguir lutas; é do que eles mais gostam.
Cachorros, touros, dragões; qualquer coisa, desde que seja uma luta.
Pois se nesse momento mesmo há um pobre de um inocente texugo no
estábulo aqui atrás; eles iam se divertir com ele hoje, mas
resolveram guardá-lo para depois que o seu caso terminar. E não tenho
a menor dúvida que eles ficaram lhe dizendo que grande herói o senhor
é, e como tem tudo para vencer, em defesa do que é certo e justo, e
por aí afora; mas posso lhe assegurar, acabo de vir da rua, e eles
estavam apostando seis contra quatro que o dragão ganha!

— Seis contra quatro no dragão — São Jorge murmurou tristemente,
apoiando o rosto na mão. — É um mundo perverso, este, e às vezes eu
me ponho a achar que toda a maldade dele não fica inteiramente
engarrafada dentro dos dragões. Ainda assim, será que esse ardiloso
animal não o iludiu quanto ao verdadeiro caráter dele, para que a sua
boa impressão dele servisse de cobertura às malvadas façanhas dele?
Não, não poderia até mesmo haver, neste preciso instante, alguma
desafortunada princesa aprisionada na lugúbre caverna dele?

Mas São Jorge se arrependeu do que disse assim que acabou de falar, ao
ver o quanto o Menino ficara genuinamente incomodado.

— Posso lhe assegurar, São Jorge — ele disse — que não há nada nem
sequer parecido na caverna dele. O dragão é um cavalheiro de verdade,
cada centímetro dele, e posso dizer que ninguém ficaria mais chocado
ou magoado do que ele, se ouvisse o senhor falando... falando desse
jeito em assuntos quanto aos quais ele tem os mais elevados
princípios!

— Bom, talvez eu tenha sido exageradamente crédulo — disse São Jorge.
— Talvez eu tenha me enganado quanto ao animal. Mas o que podemos
fazer? Aqui estamos, o dragão e eu, quase frente a frente, ambos
supostamente ávidos pelo sangue um do outro. Não consigo ver uma
saída, minimamente. O que você sugere? Você não tem como dar um
jeito, de algum modo?

— Foi exatamente o que o dragão disse — retrucou o Menino, bastante
exasperado. — Sério, isso de vocês deixarem para eu resolver tudo...
Bom, suponho que não dá para convencê-lo a ir embora discretamente,
dá?

— Impossível, receio — disse o Santo. — Completamente contra as
regras. Você sabe disso tanto quanto eu. 

— Bom, então, escute aqui — disse o Menino. — Ainda está cedo, será
que não daria para o senhor vir comigo encontrar com o dragão, e
tentar resolver com ele? Não é muito longe, e qualquer amigo meu será
muito bem recebido. 

— Bem, é inteiramente contra as regras — disse São Jorge, se
levantando — mas realmente parece ser a coisa mais sensata a fazer.
Você está se dando a um trabalhão por conta de seu amigo — ele
acrescentou afavelmente, ao passarem juntos pela porta. — Mas
anime-se! Talvez afinal não precise haver luta alguma.

— É, mas eu espero que haja sim, no entanto — disse o Menino,
nostalgicamente. 


\bigskip

— Trouxe um amigo que quer conhecê-lo, dragão — disse o Menino, um
tanto alto.

O dragão acordou num sobressalto.

— Eu estava só, hm, meditando sobre as coisas — disse, do seu jeito
simples. — Muito prazer em ser apresentado ao senhor. O tempo está
excelente, não?

— Esse é São Jorge — disse o Menino, secamente. — São Jorge, deixe-me
apresentá-lo ao dragão. Subimos até aqui para resolver discretamente
as coisas, dragão, e agora pelo amor de deus veja se consegue ter um
pouco de simples bom senso, para ver se chegamos a alguma solução
prática e eficiente, porque já estou enjoado de opiniões e teorias
sobre a vida e tendências pessoais, e todo esse tipo de coisas.
Talvez eu deva acrescentar que minha mãe está me esperando.

— Fico muito feliz em conhecê-lo, São Jorge — começou o dragão um
tanto nervoso — porque ouvi dizer que é um grande viajante, e eu
sempre fui mais do tipo caseiro. Mas posso lhe mostrar muitas
relíquias, muitos aspectos interessantes de nossa região, se fôr
ficar por algum tempo...

— Eu acho — disse São Jorge, de seu jeito franco e simpático — que o
melhor a fazer é seguirmos o conselho de nosso jovem amigo, e tentar
chegar a algum entendimento, um acordo sério, sobre esse nosso
pequeno problema. Agora, você não acha que o plano mais simples,
afinal, seria apenas lutarmos, de acordo com as regras, e que vença o
melhor? Estão apostando em você, devo lhe dizer, lá na aldeia, mas eu
não ligo para isso.

— Oh, sim, dragão, lute — disse o Menino. — Iria nos poupar tantos
aborrecimentos!

— Meu jovem amigo, você cale a boca — disse o dragão, severamente. E
prosseguiu:— Creia-me, São Jorge, não há ninguém no mundo que eu
gostaria de contentar mais do que este jovem cavalheiro aqui. Mas a
coisa toda é bobagem, e convencionalismo, e burrice popular. Não há
absolutamente nada pelo que lutar, do começo ao fim. E de qualquer
forma eu não vou, e isso encerra o assunto.

— Mas supondo que eu o obrigue a lutar —— disse São Jorge, um tanto
exasperado.

— Não há como — disse o dragão, triunfantemente. — Eu simplesmente
entraria na minha caverna e voltaria por um tempo ao buraco donde
vim. E você logo enjoaria de ficar sentado do lado de fora esperando
eu sair para lutar. E assim que tivesse ido embora, daí eu sairia
todo contente, pois para falar a verdade, gosto desse lugar, e
pretendo ficar aqui!

São Jorge observou por um instante a bela paisagem em torno deles.

— Mas esse seria um lugar maravilhoso para uma luta — ele recomeçou,
persuasivo. — Essas encostas do Downs de arena, eu em minha armadura
dourada fazendo contraste com suas escamas azuis! Pense só que bela
pintura daria!

— Agora você está tentando me pegar através de minhas sensibilidades
artísticas — disse o dragão. — Mas não vai funcionar. Não que não
daria uma pintura muito bela, como você disse — acrescentou, cedendo
um pouco.

— Parece que estamos chegando mais perto de tratar de negócios — o
Menino opinou. — Você precisa entender, dragão, que vai ser
necessário haver uma luta de algum tipo, porque você não vai querer
se enfiar de novo naquele velho buraco sujo e lá ficar até sabe-se lá
quando.

— Podia ser arranjada — disse São Jorge, pensativo. — Eu tenho que
enfiar minha lança em algum lugar, claro, mas não precisa ser um
lugar que doa muito. Tem tanto de você que com certeza deve haver
alguns pedaços de sobra em alguma parte. Aqui, por exemplo, bem atrás
da coxa. Não há de doer muito, só aí!

— Assim você me faz cócegas, Jorge — disse o dragão,
envergonhadamente. — Não, esse lugar não vai dar de jeito nenhum.
Mesmo se não doesse, e tenho certeza que vai, e muito, iria me fazer
rir, e ia estragar tudo.

— Vamos tentar outro lugar, então — disse São Jorge, pacientemente. —
Debaixo do pescoço, por exemplo; todas essas dobras de pele grossa,
se eu enfiar minha lança aí você nem vai notar...

— Sim, mas você tem certeza que consegue enfiá-la bem no lugar certo?
— o dragão perguntou, ansiosamente.

— Claro que tenho — disse São Jorge, com confiança. — Deixe comigo! 

— É exatamente porque eu tenho que deixar com você que estou
perguntando — disse o dragão, com uma certa irritação. — Sem dúvida
você vai lamentar profundamente qualquer erro que fizer no calor da
hora; mas não vai lamentar nem a metade do que eu vou! No entanto,
suponho que é preciso confiar nos outros, ao longo da vida, e seu
plano parece, no geral, o melhor que há.

— Escute aqui, dragão — interrompeu o Menino, um pouco ciumento por
seu amigo, que parecia estar ficando com a pior parte da barganha. —
Eu não estou entendendo direito como você entra nisso! Vai haver uma
luta, aparentemente, e você vai levar uma surra; o que eu quero saber
é: o que você vai ganhar com isso?

— São Jorge, — disse o dragão — conte a ele, por favor, o que irá
acontecer depois de eu ter sido vencido no mortal combate.

— Bom, de acordo com as regras suponho que eu vou levá-lo em triunfo
até a praça do mercado ou o que servir como tal — disse São Jorge.

— Precisamente — disse o dragão. — E então?

— E então haverá ovações e discursos e o resto — continuou São Jorge.
— E eu explicarei que você foi convertido, e viu o erro em sua
conduta, e assim por diante.

— Certo — disse o dragão. — E aí?

— Ah, e aí... — disse São Jorge — Ora, suponho que vai haver o usual
banquete.

— Exatamente — disse o dragão. — E é aí que eu entro. Olhe aqui —
continuou, dirigindo-se ao Menino. — Eu morro de tédio aqui em cima,
e ninguém realmente gosta de mim. Vou ser apresentado à Sociedade, eu
vou, graças a ajuda gentil de nosso amigo aqui, que está se dando a
tanto trabalho por mim; e você vai ver que tenho todas as qualidades
para que gostem de mim as pessoas que realmente contam! Bom, agora
que está tudo resolvido, e se vocês não se importarem, sou um sujeito
de modos antiquados, não gostaria de mandá-los embora, mas...

— Lembre-se, você vai ter que fazer direito a sua parte na luta,
dragão! — disse São Jorge, ao perceber a insinuação e se levantando
para partir. — Quero dizer dar investidas, exalar fogo, e por aí
afora!

— Sei investir muito bem — respondeu o dragão, todo confiante. — Já
quanto a exalar fogo, é surpreendente como fácil se perde a prática;
mas farei o melhor que puder. Boa noite!

Já haviam descido o morro e estavam quase de volta na aldeia quando
São Jorge parou de repente.

— Sabia que tinha esquecido de alguma coisa! — disse. — Precisa haver
uma princesa. Aterrorizada e acorrentada a uma rocha, essa coisa
toda. Menino, será que você não podia arranjar uma Princesa?

O menino estava no meio de um bocejo tremendo.

— Estou morto de cansaço — ele reclamou — e não posso arranjar uma
Princesa, ou qualquer outra coisa, a essa hora da noite. E minha mãe
está me esperando, e faça o favor de parar de ficar me pedindo para
arranjar coisas até amanhã!

Na manhã seguinte as pessoas começaram a afluir para o morro bem cedo,
com suas roupas de domingo e carregando cestas com garrafas
aparecendo, todos querendo assegurar bons lugares para o combate. Não
era exatamente uma questão simples, porque era bem possível, claro,
que o dragão ganhasse, e nesse caso nem mesmo os que tinham apostado
seu dinheiro nele achavam que poderiam esperar que ele tratasse os
seus apoiadores de maneira muito diferente do que o resto. Os lugares
eram escolhidos, portanto, com muito critério e com a garantia de uma
fuga rápida em caso de emergência; e a fila da frente era composta na
maioria de meninos que haviam escapado ao controle dos pais e agora
se espalhavam e rolavam na relva, ignorando as estridentes ameaças e
avisos a eles disparados por suas ansiosas mães lá atrás.

O Menino tinha garantido um bom lugar na frente, bastante perto da
caverna, e estava tão ansioso quanto um diretor de palco na estréia.
Seria possível contar com o dragão? Ele podia mudar de idéia e
arruinar completamente a performance; ou então, percebendo que a
coisa fora planejada muito às pressas, sem nem mesmo um ensaio, podia
estar muito nervoso para aparecer. O Menino olhou atentamente a
caverna, mas não havia nela sinal de vida ou ocupação. Teria o dragão
escapado no meio da noite?

As partes mais altas do morro estavam agora cobertas de espectadores,
e logo um som de aplausos e os acenos de lenços indicaram que havia
algo a vista que o Menino, do tanto que estava para cima no lado do
dragão, não podia ainda ver. Um minuto mais e as plumas vermelhas de
São Jorge apareceram no topo do morro, quando o Santo foi chegando
lentamente à grande parte plana que ia até a soturna entrada da
caverna. Muito galante e belo ele estava, em seu alto cavalo de
batalha, sua armadura dourada refletindo o sol, sua enorme lança
ereta, a pequena flâmula branca, com uma cruz carmesim, tremulando na
ponta dela. Ele puxou as rédeas e ficou imóvel. As filas de
espectadores recuaram um pouco, nervosamente; e mesmo os meninos na
frente pararam de puxar o cabelo e se cutucar uns aos outros, e se
inclinaram para a frente, na expectativa.

— Agora, dragão — murmurou o menino impaciente, sem conseguir ficar
parado no lugar. Ele não precisava ter ficado preocupado, se
soubesse. As possibilidades dramáticas da coisa haviam interessado
imensamente o dragão, e ele estava acordado desde de madrugada,
preparando-se para sua primeira aparição em público com inteira
dedicação, como se os anos tivessem voltado para trás e ele ainda
fosse um pequeno dragãozinho brincando de santos-e-dragões no chão da
caverna de sua mãe com suas irmãs; uma brincadeira em que o dragão
sempre ganhava.

Um surdo múrmurio, e um intermitente resfolegar, fez-se ouvir então;
aumentando para se tornar um ensurdecedor rugido que pareceu encobrir
todo o platô. Em seguida uma nuvem de fumaça encobriu a entrada da
caverna, e do meio dela o dragão em pessoa, brilhante, azul-marinho,
magnífico, avançou solene e esplêndido; e todo mundo fez “oooh!” como
se tivessem visto um potente fogo de artifício! Suas escamas
resplandeciam, sua longa e espinhosa cauda serpenteava no chão, suas
garras arrancavam tufos de relva e os lançavam por cima de suas
costas, e fumaça e fogo eram incessantemente expelidos de suas iradas
narinas. 

— Oh, muito bem, dragão — o Menino gritou, entusiasmado. E para si
mesmo acrescentou: “não achava que ele tivesse tanto jeito para a
coisa”.

São Jorge abaixou sua lança, inclinou a cabeça, cravou os calcanhares
no cavalo e investiu tronitruante pela relva. O dragão atacou com um
rugido e um guincho — uma enorme e azul mistura serpenteante e
resfolegante de escamas, garras, espinhos e fogo.

— Errou! — gritou a multidão. Houve um instante de emaranhamento de
armadura dourada com escamas azul-turquesa e cauda espinhuda, e então
o grande cavalo, puxando por seu lado, carregou o Santo, sua lança
balançando no ar, quase até a entrada da caverna.

O dragão sentou e rosnou malevolamente, enquanto São Jorge com
dificuldade manobrava seu cavalo de volta à posição.

“Fim do primeiro round!”, pensou o Menino. “Como eles se saíram bem!
Mas espero que o Santo não se entusiasme demais. No dragão dá para
confiar. Que excelente ator que ele é!”

São Jorge enfim conseguiu fazer que seu cavalo ficasse firme, e
enquanto enxugava a testa deu uma olhada em volta. Vendo o Menino,
sorriu e fez um gesto afirmativo com a cabeça, e mostrou três dedos
por um instante.

“Parece que está tudo planejado”, disse o Menino para si mesmo. “O
terceiro round vai ser o final, evidentemente. Gostaria que durasse
um pouco mais. E o que diabos aquele tonto do dragão está aprontando
agora?”

O dragão estava aproveitando o intervalo para para fazer uma exibição
de sua investida para a multidão. A investida dele, é necessário
explicar, consistia de correr num amplo círculo, mandando vagas e
ondas de movimento por toda a extensão de sua espinha, de suas
orelhas pontudas até o último espinho na ponta de sua comprida cauda.
Quando se é recoberto de escamas azuis, o efeito é particularmente
notável; e o Menino lembrou o desejo recentemente expresso do dragão
de se tornar um sucesso na sociedade. 

São Jorge então juntou suas rédeas e começou a se mover para a frente,
abaixando a ponta de sua lança e se firmando na sela.

— Segundo round! — todo mundo gritou, entusiasmadamente; e o dragão
parou sua exibição, sentou-se e começou a pular de um lado para outro
com enormes saltos desajeitados, bradando feito um pele-vermelha.
Isso naturalmente desconcertou o cavalo, e ele deu uma brusca
guinada, o Santo só não caindo por ter se agarrado a crina; e quando
eles passaram a toda por ele, o dragão deu uma safada mordida na
cauda do cavalo que fez o pobre animal desembestar ensandecido
encosta abaixo, de modo que as palavras do Santo, com um dos pés fora
do estribo, por sorte ficaram inaudíveis para a platéia.

O segundo round produziu audível prova de um sentimento favorável ao
dragão. Os espectadores não tardaram em apreciar um combatente que
mantinha tão bem sua posição e que claramente queria uma exibição
limpa; e muitos comentários encorajadores chegaram aos ouvidos de
nosso amigo enquanto ele se pavoneava para lá e para cá, o peito
empinado e a cauda no ar, apreciando enormemente de sua nova
popularidade.

São Jorge tinha apeado e estava apertando as cilhas, e dizendo ao seu
cavalo, com um fluxo bastante oriental de metáforas, exatamente o que
pensava dele, sua família, e sua conduta na presente situação; então
o Menino foi até o lado do Santo, e segurou a lança para ele.

— Está sendo uma ótima luta, São Jorge! — disse com um suspiro. — Será
que não daria para fazê-la durar mais? 

— Bem, acho melhor não — respondeu o Santo. — O fato é que seu velho
amigo simplório está ficando metido, agora que começaram a
aplaudí-lo, e ele bem pode esquecer o combinado e se meter à besta, e
aí não há como saber onde isso vai dar. Eu vou terminar com ele neste
round.

Ele subiu na sela e pegou a lança que o Menino lhe deu.

— Mas não fique com medo — ele acrescentou gentilmente. — Marquei o
lugar certo, e ele com certeza vai me ajudar o melhor que puder,
porque sabe que é sua única chance de ser convidado ao banquete.

São Jorge então encurtou sua lança, trazendo a sua empunhadura bem
debaixo do braço; e, em vez de galopar como antes, foi trotando em
direção dragão, que se agachou com a aproximação dele, sacudindo sua
cauda até fazê-la estalar no ar como um chicote. O Santo foi virando
ao chegar perto dele e circundou-o cautelosamente, mantendo os olhos
fixos no lugar marcado; enquanto o dragão, adotando tática similar,
moveu-se com cuidado no mesmo círculo, ocasionalmente fintando com a
cabeça. Os dois ficaram assim medindo o adversário, enquanto os
espectadores acompanhavam com a respiração em suspenso.

Embora o round tenha durado alguns minutos, o fim foi tão rápido que
tudo que o Menino viu foi um movimento rápido como um relâmpago do
braço do Santo, e então um rodamoinho e uma confusão de espinhas,
garras, cauda, e tufos de grama arrancados. A poeira se assentou, os
espectadores correram para lá dando vivas e aplaudindo, e o Menino
conseguiu ver que o dragão estava caído, preso ao solo pela lança, e
São Jorge apeara e estava com um pé sobre ele.

Tudo parecia tão genuíno que o Menino correu esbaforido para lá,
esperando que o bom e velho dragão não estivesse ferido de verdade.
Quando ele se aproximava, o dragão ergueu uma de suas enormes
pálpebras, piscou solenemente, e fechou-a de novo. Ele estava
firmemente preso a terra pela lança, mas o Santo a acertara no lugar
combinado, onde havia sobra, e não parecia nem fazer cócegas.

— Não vai cortar a cabeça, Santo? — perguntou um fulano da multidão
que aplaudia. Tinha apostado no dragão, e naturalmente estava um
pouco ressentido.

— Bom, acho que hoje não — respondeu São Jorge, todo simpático. —
Veja, isso pode ser feito qualquer dia. Não há a menor pressa. Acho
que é melhor voltarmos a aldeia antes, para os comes e bebes, e então
eu vou ter uma conversa séria com ele, e vocês verão que ele vai se
tornar um dragão muito diferente.

Com as palavras mágicas comes e bebes, a multidão formou uma procissão
e aguardou silenciosamente o sinal de partida. A hora de falar,
aplaudir e apostar já tinha passado, chegava a hora de agir. São
Jorge, puxando a lança com as duas mãos, soltou o dragão, que se
levantou e se sacuidiu e passou seus olhos por suas escamas, espinhos
e o demais, para ver se estava tudo em ordem. Então o Santo montou em
seu cavalo e tomou a dianteira da procissão, com o dragão seguindo
humildemente ao lado do menino, e com os sedentos espectadores
mantendo um respeitoso intervalo atrás dele.

Houve momentosos eventos quando todos chegaram a aldeia e se
organizaram em frente à estalagem. Depois dos comes e bebes São Jorge
fez um discurso, no qual ele informou à platéia que havia removido o
terrível flagelo, com muito trabalho e inconveniência de sua parte, e
que agora não era mais para eles ficarem resmungando e imaginando que
tinham infortúnios, porque não tinham. E que eles não deviam gostar
tanto assim de lutas, porque da próxima vez podiam bem ter que
lutarem eles mesmos, que de jeito nenhum ia ser a mesma coisa. E que
havia um certo texugo no estábulo da estalagem que deveria ser
imediatamente libertado, e que ele pessoalmente ia garantir que isso
fosse feito. Então ele disse que o dragão andara refletindo sobre as
coisas, e vira que há dois lados em toda questão, e não ia mais fazer
isso, e se eles fossem bons talvez ele ficasse morando ali. De modo
que eles deviam ficar amigos, não serem preconceituosos, e ficar por
aí achando que sabiam tudo o que há para saber, porque não sabiam,
nem de longe. E ele advertiu-os quanto ao pecado de fantasiar, e
inventar histórias e fazer os outros acreditarem só por serem
plausíveis e cheias de detalhes. Então ele sentou-se de novo, em meio
a muitos vivas arrependidos, e o dragão cutucou o Menino e cochichou
que nem ele mesmo teria dito melhor. Então todo mundo foi embora se
arrumar para o banquete.

Banquetes são sempre coisas agradáveis, consistindo em sua maior parte
em comer e beber; mas a coisa especialmente boa de um banquete é que
ele acontece no fim de alguma coisa, e não é preciso mais se
preocupar, e o dia seguinte parece ficar muito longe. São Jorge
estava feliz porque tinha havido uma luta e ele não precisara matar
ninguém; porque ele realmente não gostava de matar, embora em geral
tivesse que fazê-lo. O dragão estava feliz porque tinha havido uma
luta, e além de não ter se machucado nem um pouco nela ganhara
popularidade e uma posição garantida na sociedade. O Menino estava
feliz porque tinha havido uma luta e apesar disso seus dois amigos
estavam no maior dos entendimentos. E todos os outros estavam felizes
porque tinha havido uma luta e — bem, eles não precsiavam de nenhuma
outra razão para estarem felizes. O dragão esforçou-se para dizer a
coisa certa para todo mundo, e provou-se a alma da festa; enquanto o
Santo e o Menino ficavam olhando, com a impressão de estarem
assistindo a uma festa em que a honra e a glória eram inteiramente do
dragão. Mas eles não se incomodaram, sendo boa gente, e o dragão não
tinha ficado orgulhoso ou ingrato. Ao contrário, a cada dez minutos
ele se inclinava na direção do Menino e dizia, comovido:

— Olhe, você vai me levar para casa, não vai? — E o Menino sempre
fazia que sim, embora tivesse prometido a sua mãe não voltar muito
tarde.

Por fim o banquete terminou, os convidados foram embora com muitos
boas noites e congratulações e convites, e o dragão, que tinha ficado
para se despedir até o último deles, emergiu na rua seguido pelo
Menino, enxugou a testa, suspirou, sentou na calçada e olhou para as
estrelas.

— Foi uma noite excelente! — murmurou. — Excelentes estrelas!
Excelente lugarzinho! Acho que vou ficar aqui mesmo. Não estou com a
menor vontade de subir aquele maldito morro. Menino prometeu me levar
para casa. Melhor Menino fazer isso! Responsabilidade não é minha.
Responsabilidade é do Menino!

E seu queixo afundou no peito largo dele e ele pegou tranquilamente no
sono.

— Ah, levanta, dragão — gritou o Menino, desconsolado. — Vocês sabe
que minha mãe está me esperando, e eu estou tão cansado, e você me
fez prometer que ia levá-lo para casa, e eu não sabia o que você
queria dizer, senão não teria prometido! 

E o menino sentou na calçada ao lado do dragão dormindo e começou a
chorar.

A porta atrás deles se abriu, um recorte de luz iluminou a rua, e São
Jorge, que havia saído para dar uma caminhada no ar frio da noite,
viu as duas figuras sentadas ali, o enorme e imóvel dragão e o
menininho chorando.

— Qual é o problema, Menino? — ele perguntou gentilmente, do lado
dele.

— Ah, é esse grandesíssimo desse porco dorminhoco desse dragão! —
soluçou o menino. — Primeiro me fez prometer levá-lo para casa, aí
disse que era melhor eu fazer isso mesmo, e então se põe a dormir!
Mais fácil levar um monte de feno para casa! E estou tão cansado, e
minha mãe...

E ele começou a chorar de novo.

— Não fique assim — disse São Jorge. — Eu vou ajudá-lo, e nós dois
vamos levá-lo para casa. Acorda, dragão! — ele disse incisivamente,
chacoalhando o dragão pelo ombro.

O dragão abriu os olhos todo sonolento.

— Que noite, Jorge! — murmurou — Que...

— Escute aqui, dragão — disse o Santo, firmemente. — Aqui está esse
camaradinha esperando para levá-lo para casa, e você sabe muito bem
que ele já devia estar na cama há duas horas atrás, e o que a mãe
dele vai dizer eu nem imagino, e qualquer menos um porco egoísta
teria feito ele ir para cama faz tempo...

— E ele vai para a cama! — exclamou o dragão, levantando-se. —
Pobrezinho, imagine só, ainda acordado a essa hora! É uma vergonha, é
isso o que é, e eu não acho, São Jorge, que você teve muita
consideração... Mas vamos embora imediatamente, e chega de discussões
ou papo furado. Dê-me sua mão, Menino; e obrigado, Jorge, um braço
amigo morro acima é só o que eu queria!

E lá se foram os três abraçados morro acima, o Santo, o dragão, e o
Menino. As luzes na aldeia começaram a se apagar; mas haviam as
estrelas, e uma lua no fim da noite, enquanto eles subiam os Downs
juntos. E, quando eles viraram a última esquina e desapareceram de
vista, pedaços de uma velha canção chegaram trazidos pela brisa da
noite. Não tenho certeza quem estava cantando, mas eu acho que era o
dragão!

\chapter{Os quatro irmãos habilidosos\subtitulo{Jacob \& Wilhelm Grimm }}

Era uma vez um homem pobre que tinha quatro filhos. Quando cresceram,
disse a eles:

— Meus queridos filhos, vocês precisam agora ir correr mundo, pois
nada tenho para lhes oferecer. Viajem até outros países. Aprendam um
ofício e tentem fazer o melhor que puderem. 

Os quatro irmãos se prepararam então para a jornada, despediram-se do
pai, e partiram juntos. Depois de viajarem por algum tempo, chegaram
a uma encruzilhada, donde partiam quatro estradas diferentes. O filho
mais velho disse:

— Vamos nos separar aqui, mas vamos combinar de nos encontrarmos aqui
daqui a quatro anos. Durante esse tempo, cada um tentará a sua sorte.

Então cada um partiu em seu caminho, e o mais velho encontrou um
homem, que lhe perguntou onde ele estava indo e quais eram seus
planos.

— Quero aprender um ofício — ele respondeu.

— Então venha comigo — disse o homem — e torne-se um ladrão.

— Não — o filho mais velho respondeu. — Não é um ofício considerado
honesto hoje em dia, e nele geralmente se termina pendurado na forca.


— Ah — disse o homem — você não precisa ter medo da forca. Vou lhe
ensinar como obter o que de outro jeito não se consegue, e vou lhe
mostrar como fazê-lo sem ser pego. 

Então o filho mais velho se deixou convencer, e o homem lhe ensinou
como ser um ladrão habilidoso. O jovem ficou tão bom que nada que ele
quisesse ter ficava a salvo. 

O segundo irmão encontrou um homem que lhe fez a mesma pergunta sobre
o que ele queria aprender no mundo. 

— Eu não sei — ele respondeu. 

— Então venha comigo e torne-se um astrônomo. Você não vai achar nada
melhor que isso, pois nunca mais algo ficará oculto de sua vista.

O segundo filho gostou da idéia, e se tornou um astrônomo tão
habilidoso que quando acabou seus estudos e ia partir, seu mestre lhe
deu um telescópio e disse:

— Com isto você poderá ver tudo o que acontece na terra e no céu, e
nada ficará oculto para você.

O terceiro irmão tornou-se aprendiz junto a um caçador, e aprendeu
tudo o que é preciso saber sobre a caça, e revelou-se um caçador de
primeira. Seu mestre lhe deu um rifle de presente quando ele ia
partir, dizendo:

— Você não vai errar nada que mirar com esse rifle. Sempre acertará.

O irmão mais novo também encontrou um homem que lhe perguntou quais
eram seus planos.

— Você gostaria de ser um alfaiate?

— Acho que não — disse o jovem. — Nunca tive a menor vontade de ficar
sentado de manhã até a noite usando uma agulha ou passando o ferro
para lá e para cá.

— Ora, deixe disso — respondeu o homem. — Você não sabe do que está
falando. Comigo você aprenderá um tipo totalmente diferente da
alfaiataria, que é respeitável e decente e poderá mesmo lhe trazer
grande honra.

O jovem se deixou convencer. Seguiu com o homem e dele aprendeu todos
os princípios daquele ofício, e quando ia partir o homem lhe deu uma
agulha, dizendo:

— Com esta agulha você será capaz de costurar tudo o que encontrar,
seja algo mole como um ovo ou duro como aço. E você será capaz de
fazer de qualquer coisa uma peça inteira sem que nenhuma costura
apareça.

Quando os quatro anos combinados se passaram, os quatro irmãos se
encontraram na encruzilhada. Abraçaram-se e beijaram-se, e voltaram
para a casa do pai deles.

— Ora — disse o pai encantado — vejam só o que os bons ventos me
trouxeram de volta!

Eles contaram ao pai o que tinha acontecido com eles, e como cada um
tinha aprendido um ofício. Estavam sentados em frente à casa sob uma
grande árvore, e o pai esntão disse:

— Bom, agora vou pôr a prova o que vocês são capazes de fazer. — Olhou
para cima e disse para o segundo filho:— Há um ninho de tentilhão
entre dois galhos no alto desta árvore. Diga-me quanto ovos há nele.

O astrônomo pegou seu telescópio, apontou-o para o topo da árvore, e
disse:

— Cinco.

Então o pai disse para o filho mais velho:

— Vá catar os ovos sem incomodar o pássaro que os está chocando.

O ladrão habilidoso trepou na árvore e catou os cinco ovos sob o
pássaro, que continuou ali chocando sem nada perceber. Trouxe os ovos
para o pai, que os pegou e os pôs um em cada canto da mesa, e o
quinto no meio, e disse para o caçador:

— Quero que você parta os cinco ovos pela metade com um só tiro.

O caçador mirou e atirou nos ovos exatamente como seu pai pedira. De
fato, acertou os cinco com um único tiro, e certamente devia ter um
pouco dessa pólvora que dobra esquinas. 

— Agora é a sua vez — disse o pai para o quarto filho. — Quero que
você costure de novo os ovos, e também os filhotes dentro deles, de
modo a consertar qulquer estrago que o tiro tenha feito a eles.

O alafaite pegou sua agulha e costurou como seu pai pedira. Quando
terminou, coube ao ladrão pôr de volta no ninho os ovos sob o pássaro
sem que este nada notasse. O pássaro chocou-os até os filhotes saírem
da casca, e quando eles apareceram tinham listas vermelhas em volta
do pescoço, que era aonde o alfaite os havia costurado.

— Sim — disse o velho a seus filhos — todo o elogio que eu puder lhes
fazer é pouco. Vocês usaram bem o seu tempo e aprenderam algo útil.
Não sei dizer qual merece os maiores elogios, mas se vocês logo
tiverem uma oportunidade para usar suas habilidades, descobriremos
qual é o melhor.

Pouco depois houve uma grande comoção no país; a filha do rei havia
sido raptada por um dragão. O rei se preocupava dia e noite com isso,
e ele mandou proclamar que aquele que resgatasse sua filha poderia se
casar com ela. Os quatro irmãos discutiram juntos a situação.

— Essa pode ser a chance de mostrarmos do que somos capazes — e
decidiram partir juntos para libertar a filha do rei.

— Vou enocontrá-la sem demora — disse o astrônomo, e olhou através de
seu telescópio. — Já a estou vendo. Está sentada num rochedo no meio
do mar muito longe daqui, e junto a ela está o dragão, vigiando-a. 

Então o astrônomo foi até o rei e requisitou um navio para ele e seus
irmãos. Nele navegaram até chegarem ao rochedo. A filha do rei estava
sentada lá, mas o dragão estava dormindo com a cabeça no colo dela.

— Não posso arriscar um tiro — disse o caçador. — Poderia também
acertar a bela moça. 

— Bom, então vou usar minha habilidade — disse o ladrão, e se
esgueirou até onde a jovem estava e roubou-a do dragão tão
silenciosamente e agilmente que o monstro nada notou e continuou
roncando.Os irmãos se apressaram a partir felizes com ela, e
embarcaram e se fizeram ao mar. Mas ao acordar, o dragão descobriu
que a filha do rei havia sumido e, bufando furiosamente, voou atrás
deles. Quando estava pairando sobre o barco, pronto para investir
nele, o caçador mirou com seu rifle e acertou-o no coração. O monstro
caiu morto, mas era tão imenso e pesado que despedaçou o barco
inteiro em sua queda. Por sorte, aqueles que estavam a bordo
conseguiram agarrar algumas tábuas e com elas flutuar no mar. De novo
estavam numa situação de extremo perigo, mas o alfaiate, sempre muito
alerta, pegou sua agulha maravilhosa e costurou juntas as tábuas com
alguns grandes pontos. Então sentou-se nas tábuas e juntou todos os
pedaços do navio, que então costurou de uma maneira tão habilidosa
que em pouco tempo o barco estava inteiro de novo e eles puderam
navegar em segurança para casa.

Quando o rei viu de volta a sua filha, ficou muito contente, e disse
para os quatro irmãos:

— Um de vocês casará com a princesa, mas vocês precisam decidir entre
vocês qual vai ser.

Uma séria dicussão começou então entre os irmãos, pois todos tinham o
que reivindicar. O astrônomo disse:

— Se eu não tivesse visto a filha do rei, todas as habilidades de
vocês teriam sido inúteis. É por isso que ela deve ser minha esposa.

O ladrão disse:

— Você a ter visto de nada adiantaria se eu não a tivesse roubado do
dragão. É por isso que ela deve ser minha esposa.

O caçador disse:

— O monstro os teria feito em pedaços junto com a filha do rei se
minha bala não o tivesse acertado. É por isso que ela deve ser minha
esposa.

O alfaiate disse:

— E se eu não tivesse remendado o barco com minha habilidade, todos
vocês teriam se afogado horrívelmente. É por isso que ela deve ser
minha esposa.

Então o rei anunciou a decisão a que chegara.

— Cada um de vocês tem um motivo justo para casar com minha filha, mas
como é impossível ela casar com os quatro casarem, nenhum ficará com
ela. Mas eu os recompensarei dando meio reino para cada um.

Os irmãos ficaram satisfeitos com essa decisão e disseram:

— É melhor dessa maneira do que ficarmos uns contra os outros. 

Então cada um recebeu meio reino, e viveram felizes com o pai deles
por todo o tempo que Deus lhes concedeu.

\chapter{São Jorge e o dragão\subtitulo{Jacobus de Voragine/William Caxton}}

São Jorge era um cavaleiro que nascera na Capadócia. Um dia ele foi
para a província da Líbia, onde havia uma cidade chamada Sylene. E
perto dessa cidade havia uma lagoa feito um mar, e nela um Dragão que
ameaçava envenenar todo o país. Um dia o povo se juntou para ir
matá-lo, mas ao vê-lo se apavoraram e fugiram. E quando chegou a
noite o dragão ia envenenar o povo com o ar que expirava, de modo que
eles decidiram dar todo dia duas ovelhas para alimentá-lo, para que
assim ele não fizesse mal ao povo. 

Quando as ovelhas acabaram chegou a vez dos homens, e por um decreto
na cidade se estabeleceu que seriam entregues primeiro as crianças e
os jovens, e quem se recusasse a entregar seus filhos iria em lugar
deles. E assim foi que muitas crianças e jovens foram entregues ao
Dragão, tantas que acabou chegando a vez da filha do Rei. O Rei
entristecido disse a seu povo:

— Pelo amor dos deuses, levem ouro e prata e tudo o que tenho, mas
deixem eu ficar com minha filha. 

Eles responderam:

— Como assim, foi vossa majestade quem fez a lei e mandou cumprir, de
tal modo que agora todos nossos filhos estão mortos, e agora quer
fazer diferente? Sua filha deverá ser entregue, ou então levaremos
vossa majestade. 

Quando o Rei viu que não havia nada a fazer começou a chorar, e disse
para a filha dele que agora jamais iria ver as bodas dela. Então ele
voltou ao povo e pediu oito dias de prorrogação, e o povo lhe
concedeu. E quando os oito dias se passaram voltaram a ele dizendo
que já era a hora, pois ele podia ver que a cidade estava perecendo.

Então o Rei fez vestirem sua filha como se ela fosse se casar, e a
abraçou e beijou, e a abençoou, e a conduziu para o lugar em que
estava o Dragão. 

Quando ela estava lá, São Jorge passou por ali, e vendo a dama
perguntou o que ela fazia ali, e ela disse:

— Siga em seu caminho, belo jovem, para que não morra você também.

Então ele disse:

— Diga-me o que a aflige e porque você chora, e nada tema.

Quando ela viu que ele tanto queria saber, ela disse que havia sido
deixada para o Dragão. São Jorge disse:

— Bela jovem, nada tema, pois eu vou ajudá-la em nome de Jesus Cristo.


— Pelos deuses — ela disse — siga seu caminho, não fique aqui comigo,
pois você pode não conseguir me salvar.

Enquanto eles falavam o Dragão apareceu, e veio correndo até eles, e
São Jorge montou em seu cavalo, e desembainhou a espada e fez o sinal
da cruz, e cavalgou bravamente em direção ao Dragão, e o acertou com
sua lança e o feriu gravemente, jogando-o por terra. Então ele disse
para a jovem:

— Pegue seu cinto e amarre-o em volta do pescoço do Dargão, e não
tenha medo.Quando ela fez isso o Dragão se pôs a seguí-la como se
fosse um animal domesticado e inofensivo. Ela levou-o até a cidade, e
o povo fugiu para as montanhas e os vales, dizendo que iam ser todos
mortos. Então São Jorge disse: 

— Nada temam, creiam em Jesus Cristo, e não tardem a serem batizados,
que eu matarei o Dragão. 

Então o Rei e todo seu povo foi batizado, e São Jorge matou o Dragão
cortando-lhe a cabeça, e ordenou que ele fosse jogado no campo, e foi
preciso três carros de boi para tirá-lo da cidade. 

Haviam sido batizados quinze mil homens, sem contar mulheres e
crianças, e o Rei mandou erguer uma igreja para Nossa Senhora e São
Jorge, onde ainda hoje brota uma fonte de água da vida que cura os
doentes que dela bebem.

Depois disso o Rei ofereceu a São Jorge tanto dinheiro quanto tinha,
mas ele recusou tudo e pediu que fosse dado aos pobres em nome de
Deus, e exigiu do Rei quatro coisas, a saber, que ele se encarregasse
das igrejas, que ele honrasse os padres, que ouvisse diligentemente
seus sermões, e que ele tivesse piedade dos pobres; e depois de
beijar o Rei ele partiu.


\chapter{Tio James, ou o dragão púrpura\subtitulo{Edith Nesbit}}

A princesa e o menino do jardineiro estavam brincando no quintal.

— O que você vai fazer quando crescer, Princesa? — perguntou o menino
do jardineiro.

— Eu gostaria de casar com você, Tom — disse a princesa. — Você se
incomodaria?

— Não — disse o menino do jardineiro. — Não me incomodaria muito. Eu
casarei com você se você quiser; se eu tiver tempo.

Pois o menino do jardineiro pretendia, assim que crescesse, ser um
general e um poeta e o primeiro ministro e um almirante e um
engenheiro civil. Enquanto isso, era o primeiro em todas as matérias
da escola, e o primeiríssimo em geografia.

Quanto à Princesa Mary Ann, ela era uma menininha muito boa, e todos a
amavam. Era sempre gentil e educada, mesmo com seu tio James ou
outras pessoas de quem ela não gostava muito; e embora não fosse
muito esperta, para uma princesa, sempre tentava fazer suas lições.
Mesmo quando você sabe perfeitamente bem que não consegue fazer suas
lições, não custa tentar, e às vezes por algum feliz acaso você
realmente consegue fazê-las. E ela tinha de verdade um bom coração:
sempre tratava bem seus animais de estimação. Ela nunca batia em seu
hipopótamo quando ele quebrava as bonecas dela dando cambalhotas, e
nunca esquecia de dar comida a seus rinocerontes no pequeno viveiro
deles no quintal. O elefante dela a adorava, e às vezes Mary Ann
deixava sua babá bastante brava quando escondia a adorável coisinha
debaixo das cobertas e o deixava dormir com sua longa tromba apoiada
carinhosamente no pescoço dela, e sua cabecinha aninhada junto à real
orelha direita. 

Quando a princesa se comportava a semana inteira — pois, como todas as
boas crianças vivas de verdade, ela às vezes era travessa, mas nunca
má — a babá a deixava convidar seus amiguinhos para virem
quarta-feira de manhã e passarem o dia todo, pois era a quarta-feira
o fim de semana naquele país. E então, de tarde, quando todos os
pequenos duques e duquesas e marqueses e condessas tinham acabado de
comer seu pudim de arroz e lavado as mãos e os rostos depois, a babá
perguntava, como se já não soubesse:

— E agora, meus amores, o que vocês querem fazer esta tarde? — e a
resposta era sempre a mesma:

— Ah, vamos no Jardim Zoológico dar um passeio no porquinho-da-índia e
dar comida para os coelhos e ouvir o arganaz [dormouse?] dormindo.

Então [] iam todos para o Jardim Zoológico, onde vinte deles ao mesmo
tempo podiam ser levados para passear no porquinho-da-índia, e onde
mesmo os pequenininhos podiam dar comida aos enormes coelhos se
houvesse por perto algum adulto gentil o suficiente para levantá-los.

Sempre havia alguma pessoa assim, porque na Rotundia todo mundo era
gentil — exceto uma pessoa.

Agora que você já leu até aqui você deve saber, claro, que o reino da
Rotundia era um lugar notável; e se você for uma criança atenta — e é
claro que você é — nem precisa que eu lhe conte o que era a coisa
mais notável da Rotundia. Mas no caso de você não ser uma criança
atenta — e é claro que é possível que você não seja — vou lhe dizer
de uma vez por todas o que era essa coisa notável. Todos os animais
eram do tamanho errado! E eis como eles ficaram assim.

Há muito, muito, muito tempo atrás, quando nosso mundo era só terra
solta e ar e fogo e agua misturadas de qualquer jeito feito um pudim,
e girava loucamente tentando fazer com que as diferentes coisas
fossem parar em seus devidos lugares, um pedaço redondo de terra
escapou e saiu girando por conta própria pela água, bem quando ela
estava começando a tentar se espalhar toda lisa num oceano decente. E
quando o grande pedaço redondo de terra saiu voando, girando e
girando tanto quanto podia, encontrou com um pedaço comprido de rocha
sólida que escapara de outra parte da mistura pudinhenta, e essa
rocha era tão dura, e estava indo tão rápido, que a ponta dela
atravessou o pedaço redondo de terra e saiu do outro lado, de forma
que as duas juntas ficaram parecendo um pião
muito-muito-muito-grande-demais.

Receio que tudo isso seja muito chato, mas você sabe que geografia
nunca é muito emocionante, e afinal, eu preciso lhe dar um pouco de
informação mesmo num conto de fadas — como os caroços [] na geléia.

Bom, quando a rocha pontuda acertou o pedaço redondo de terra o choque
foi tão grande que fez as duas juntas saírem girando pelo ar — que
estava justo naquele momento indo para o seu devido lugar, como todo
o resto das coisas — só que, quis o acaso, elas esqueceram para que
lado estavam indo, e se puseram a girar no sentido errado. Então, o
Centro da Gravidade — um enorme gigante que estava administrando esse
negócio todo — despertou no meio da terra e começou a resmungar.

— Depressa — ele disse. — Venham para baixo e fiquem parados,
entenderam?

De modo que a rocha e o pedaço redondo de terra caíram no mar, e a
ponta da rocha acertou um buraco em que cabia direitinho no fundo
rochoso do mar, e lá girou para o lado errado mais sete vezes e então
parou quieto. E aquele pedaço redondo de terra tornou-se, depois de
milhões de anos, o reino da Rotundia.

Aqui termina a aula de geografia. Agora, um pouco de história natural,
para não ficarmos com a impressão de que estamos perdendo tempo. É
claro que a consequência da ilha ter girado para o lado errado foi
que quando os animais começaram a crescer na ilha, todos eles
cresceram nos tamanhos errados. O porquinho-da-índia, você já sabe,
era tão grande quanto nossos elefantes, e o elefante — um bichinho
dos mais adoráveis — era do tamanho daqueles cachorrinhos minúsculos
e absurdos que as damas às vezes carregam no colo. Os coelhos eram
mais ou menos do tamanho de nossos rinocerontes, e em tudo quanto era
parte selvagem da ilha eles tinham feito tocas tão grandes quanto
túneis de trem. O arganaz [?dormouse] era, claro, a maior de todas as
criaturas. Não dá nem para começar a dizer o quanto ele era grande.
Mesmo se você pensar em elefantes não vai adiantar nada. Por sorte só
havia um, e estava sempre dormindo. Caso contrário, não acho que os
rotundianos teriam conseguido viver com ele. Mas o fato é que
construíram para ele uma casa, e ele economizava a despesa de ter um
banda de metais, porque não daria para ouvir banda alguma quando o
arganaz [dormouse] falava dormindo.

Os homens, mulheres e crianças dessa ilha maravilhosa eram do tamanho
certo, porque seus ancestrais tinham chegado com o Conquistador muito
depois da ilha ter parado e os animais crescido nela. 

Agora terminou a aula de história natural, e se você estava prestando
atenção, você sabe mais sobre a Rotundia que qualquer um lá, exceto
três pessoas: o Lorde Diretor da Escola, o tio da princesa — que era
um mago, e sabia de tudo sem ter que aprender — e Tom, o filho do
jardineiro.

Tom aprendera mais que qualquer um na escola porque ele queria ganhar
um prêmio. O prêmio oferecido pelo Lorde Diretor da Escola era uma
História da Rotundia, numa belíssima encadernação, com o brasão real
na contracapa. Mas depois do dia em que a princesa disse que
pretendia casar com Tom, o menino do jardineiro pensou melhor, e
decidiu que o melhor prêmio do mundo era a princesa, e esse era o
prêmio que Tom pretendia ganhar; e quando você é filho de jardineiro
e decidiu casar com uma princesa, você logo descobre que quanto mais
aprender na escola melhor. 

A princesa sempre brincava com Tom nos dias que os pequenos duques e
marquesas não vinham para o chá — e quando ele disse a ela que tinha
quase certeza de que ia ganhar o primeiro prêmio, ela bateu palmas e
disse:

— Caro Tom, meu bom e sabido Tom, você merece todos os prêmios. E eu
vou lhe dar meu elefante de estimação, e você pode ficar com ele até
nos casarmos.

O elefante de estimação se chamava Fido, e o filho do jardineiro o
levou no bolso de seu casaco. Era o mais adorável elefantinho que
você já viu — com cerca de dezoito centímetros de comprimento. Mas
ele era muito, muito sábio — não poderia ser mais sábio se tivesse um
quilômetro de altura. Ele se instalou confortavelmente no bolso de
Tom, e quando Tom punha a mão, Fido enrolava sua trombinha em torno
dos dedos dele com uma terna confiança que logo fez o menino se
afeiçoar muito ao seu novo bichinho. E com isso tudo, o elefante, a
afeto da princesa, e a certeza de que no dia seguinte ia ganhar a
História da Rotundia, numa belíssima encadernação, com o brasão real
na contracapa, Tom mal conseguiu pregar o olho. E, além do que, o
cachorro ficou latindo insuportavelmente. Só havia um cachorro na
Rotundia — o reino não podia arcar com mais do que um: era um
chiuaua, do tipo que em outras partes do mundo mede apenas vinte e um
centímetros do fim de seu delicado focinho a ponta de seu adorável
rabinho; mas na Rotundia ele era maior do que eu possa realmente
esperar que você acredite. E quando ele latia, seu latido era tão
enorme que preenchia toda a noite e não deixava lugar para sono,
sonho, conversa fiada, ou o que quer que fosse. Ele nunca latia para
coisas que aconteciam na ilha — ele tinha uma visão de mundo mais
ampla do que isso; mas quando navios passavam às tontas na escuridão,
tropeçando nas rochas da ponta da ilha, ele latia uma ou duas vezes,
só para os navios saberem que não podiam vir brincar por ali quando
bem entendessem.

Mas nessa noite em particular ele latiu, latiu e latiu — e a princesa
disse:

— Minha nossa, que coisa, gostaria que ele parasse, estou com tanto
sono. 

E Tom disse para si mesmo:

— Por que será que ele está latindo tanto? Assim que ficar claro vou
lá ver.

Então quando começou a amanhacer num bonito rosa e amarelo, Tom
levantou e saiu. E o tempo todo o chiuaua latia, de modo que as casas
tremiam, e as telhas do palácio faziam um barulho feito o das latas
de leite numa carroça em que o cavalo é brincalhão.

“Vou até o pilar”, pensou Tom, enquanto atravessava a cidade. O pilar
era, claro, o topo do pedaço de rocha que atravessara a Rotundia
milhões de anos antes, e fizera-a girar no sentido errado. Era bem no
meio da ilha, e sua ponta era tão alta que quando se estava no topo
dele dava para ver bem mais longe em volta do que do chão.

Quando Tom saiu da cidade e seguiu pelas colinas, pensou como era
agradável ver os coelhos no orvalho da luminosa manhã saltitando com
suas crias na entrada de suas tocas. Ele não chegou muito perto dos
coelhos, claro, porque quando um coelho daquele tamanhão brinca nem
sempre olha para onde está indo, e poderia facilmente esmagar Tom com
sua pata, e então ficar muito consternado depois. E Tom era um bom
menino, e não gostaria de deixar nem mesmo um coelho triste.
Centopéias em nosso país frequentemente saem do seu caminho quando
acham que você vai pisar nelas. Elas também tem bom coração e não
iriam gostar de vê-lo consternado depois.

E assim Tom foi indo, olhando os coelhos e vendo a manhã ficar mais e
mais vermelha e dourada. E o chiuaua latia o tempo todo, fazendo os
sinos da igreja tilintarem, e a chaminé da fábrica de maçãs balançar
de novo.

Mas quando Tom chegou ao pilar, viu que não ia precisar ir até o topo
dele para descobrir o que fazia o cachorro ficar latindo.

Porque ali, perto do pilar, estava um dragão púrpura muito grande.
Suas asas eram como velhos guarda-chuvas púrpuras que tinham tomado
chuva demais, e sua cabeça era enorme e lisa, como a parte de cima de
um cogumelo púrpura, e sua cauda, que era púrpura também, era muito,
muito, muito longa, e fina e dura, como a correia do chicote de uma
carruagem.

Ele estava lambendo uma das suas asas de guarda-chuva púrpura, e de
vez em quando gemia e apoiava sua cabeça contra o pilar como se
estivesse se sentindo fraco. Tom na hora descobriu o que acontecera.
Uma revoada de dragões púrpuras deve ter passado pela ilha durante a
noite, e aquele ali deve ter batido e quebrado sua asa no pilar.

Todo mundo é bom com todo mundo na Rotundia, e Tom não estava com medo
do dragão, embora nunca tivesse falado com um antes. Frequentemente
ficava observando-os voando por cima do mar, mas jamais imaginara que
iria conhecer um pessoalmente.

De modo que ele então disse:

— Tenho a impressão de que você não está se sentindo muito bem.

O dragão balançou sua enorme cabeça púrpura. Não podia falar, mas como
todos os outros animais, entendia bem o bastante quando queria.

— Posso lhe oferecer alguma coisa? — Tom perguntou, muito polido.

O dragão abriu seus olhos púrpuras e sorriu com um ar de interrogação.

— Um ou dois bolinhos, que tal? — disse Tom, querendo convencê-lo. —
Há uma bela árvore de bolinhos bem perto daqui.

O dragão abriu sua grande boca púrpura e lambeu seus lábios púrpuras,
então Tom saiu correndo e chacoalhou a árvore de bolinhos, e logo
voltou com um punhado de bolinhos de groselha fresquinhos, e no
caminho catou também uns do tipo de Bath, que cresciam nuns arbustos
perto do pilar.

Porque, claro, outra consequência da ilha ter girado para o lado
errado era que todas as coisas que nós temos que fazer — bolos,
tortas, biscoitos — davam em árvores ou arbustos, mas na Rotundia é
preciso fazer as couve-flores, os repolhos, as cenouras, as maças e
as cebolas, do mesmo jeito que nossas cozinheiras fazem pudins ou
tortinhas. 

Tom deu todos os bolinhos ao dragão, dizendo:

— Tome, tente comer um pouco. Você logo vai se sentir melhor.

O dragão comeu os bolinhos, inclinou a cabeça em agradecimento um
tanto desajeitadamente, e se pôs a lamber sua asa de novo. Então Tom
deixou-o lá e voltou para a cidade com a notícia e todo mundo ficou
tão entusiasmado com um dragão de verdade vivo estando na ilha — algo
que nunca acontecera antes — que todos foram lá vê-lo, em vez de ir
na premiação, e o Lorde Diretor da Escola foi junto com o resto.
Acontece que ele estava com o prêmio de Tom, a História da Rotundia,
no bolso — aquela encadernada em couro, com o brasão real na capa — e
o livro caiu do bolso, e o dragão o comeu, de modo que afinal Tom
nunca ganhou o premio. Mas o dragão, que o ganhou, não gostou dele.

— Talvez seja melhor assim — disse Tom. — Eu podia também não ter
gostado do prêmio, se o tivesse ganho.

Aquele dia era uma quarta-feira, de modo que quando perguntaram aos
amigos da princesa o que eles queriam fazer, todos os pequenos
duques, marqueses e condes disseram:

— Vamos lá ver o dragão.

Mas as pequenas duquesas, marquesas e condessas disseram que estavam
com medo.

Então a princesa Mary Ann elevou a voz com realeza, e disse:

— Não sejam bobas, porque só em contos de fadas e histórias da
Inglaterra e coisas assim que as pessoas são más e querem fazer mal
uma a outra. Na Rotundia todo mundo é bom, e ninguém tem nada a
temer, a menos que tenha aprontado, e daí é para o seu próprio bem.
Vamos todos lá ver o dragão. Podíamos levar para ele uns [acid
drops].

E lá se foram eles. Todas as crianças com títulos de nobreza tiveram a
sua vez de dar [acid drops] ao dragão, e ele pareceu contente e
lisonjeado, e abanou sua cauda púrpura o mais que dava sem ficar
inconveniente, pois era uma cauda muito, muito comprida. Mas quando
chegou a vez da princesa de dar o [acid drop] para o dragão, ele
abriu um sorriso muito largo, e abanou a cauda até o último
centímetro dela, como se dissesse: “Oh, que princesinha simpática,
gentil e bonita”. Mas bem no fundo de seu malvado coração o que ele
estava dizendo era “Oh, que princesinha simpática, gorda e bonita,
queria mesmo era comer você em vez desses bobos [acid drops]”. Mas
claro que ninguém o ouviu, exceto o tio da princesa, e ele era um
bruxfeiticeiro, e acostumado a ouvir atrás de portas. Era parte do
seu ofício.

Agora, você há de lembrar que eu lhe contei que havia só uma pessoa má
na Rotundia, e não dá mais para eu esconder que esse Malvado Completo
era o tio James, o tio da princesa. Magos são sempre maus, como você
sabe pelos seus livros de contos de fadas, e alguns tios são maus,
como você pode ver em Babes in the Wood, ou em Norfolk Tragedy, e
pelo menos um James era mau, como você aprendeu na história da
Inglaterra. E se alguém é um mago, e é também um tio, e ainda se
chama James, nada de bom pode se esperar dele. Ele é um Malvado
Completo Triplo — e não vai fazer nada que preste. 

Fazia tempo que o tio James queria se livrar da princesa e ficar com
todo o reino para ele. Ele não gostava de muitas coisas —
praticamente a única coisa que importava para ele era um bom reino —
mas ele nunca soubera bem como conseguir o que queria, porque todo
mundo é tão bom na Rotundia que feitiços malvados não funcionavam, só
escorriam desses impecáveis ilhéus como água das costas de um pato.
Agora, no entanto, Tio James achou que poderia ser a sua chance,
porque sabia que agora haviam duas pessoas malvadas na ilha que iriam
se apoiar mutuamente: ele próprio e o dragão. Ele nada disse, mas
trocou um sugestivo olhar com o dragão, e todo mundo foi para casa
porque era hora do chá. E ninguém vira o sugestivo olhar a não ser
Tom.

Tom foi para casa, e contou tudo para seu elefante. A inteligente
criaturinha ouviu atentamente, e então pulou do joelho de Tom para a
mesa, onde havia um calendário de mesa que a princesa dera a Tom de
presente de natal. Com sua minúscula tromba o elefante apontou para
uma data — quinze de agosto, o aniversário da princesa, e olhou
ansiosamente para seu dono.

— O que foi, Fido, meu bom elefantinho? — disse Tom, e o sagaz animal
repetiu o gesto anterior. Então Tom entendeu.

— Ah, alguma coisa vai acontecer no aniversário dela? Tudo bem.
Ficarei atento. — E ele ficou.

No começo o povo da Rotundia estava bastante satisfeito com o dragão,
que morava ali no pilar e se alimentava das árvores de bolinhos, mas
logo ele começou a dar seus passeios. Ele se escondia nas tocas
feitas pelos grandes coelhos; e excursionistas, fazendo caminhadas
nas colinas, viam sua longa e firme cauda, que parecia com um
chicote, sinuosamente sumindo fora de vista dentro de uma toca, e
antes que eles tivessem tempo de dizer “lá vai ele”, sua horrível
cabeça púrpura aparecia saindo de outro buraco de coelho — talvez bem
atrás deles — ou ria baixinho para si mesmo bem detrás das orelhas
deles. E a risada do dragão não era nada agradável. Essa espécie de
esconde-esconde no começo divertia as pessoas, mas aos poucos começou
a dar nos nervos delas: e se você não sabe o que isso é, peça a sua
mãe para lhe contar da próxima vez que você estiver brincando de
cabra-cega quando ela estiver com dor de cabeça. Então o dragão
passou a ter o hábito de estalar sua cauda, como se faz com chicotes,
e isso também dava nos nervos das pessoas. E aí também certas coisas
começaram a sumir. E você sabe o quanto isso é desagradável, mesmo
numa escola particular, e num reino público é muito pior, claro. As
coisas que sumiam não eram muito importantes a príncipio — alguns
elefantinhos, um hipopótamo ou dois, e umas poucas girafas, e coisas
assim. Então um dia o coelho preferido da princesa, chamado
Frederick, desapareceu, e então veio a terrível manhã em que o
chiuaua tinha sumido. Ele ficara latindo o tempo todo desde que o
dragão aparecera na ilha, e as pessoas tinham ficado acostumadas ao
barulho. De modo que quando os latidos subitamente pararam, todo
mundo acordou, e foi ver o que era. E o cachorrinho chiuaua se fôra!

Mandaram um menino ir acordar o exército, para que ele fosse procurar
o cachorro. Mas o exército tinha sumido também! E então as pessoas
começaram a ficar com medo. Nesse momento, Tio James saiu na sacada
do palácio, e fez um discurso ao povo. Ele disse:

— Amigos, meus concidadãos, não posso esconder mais nem de mim nem de
vocês que essa dragão púrpura é um pobre exilado sem um tostão, um
forasteiro indefeso em meio a nós e, além de um dragão, ele é um
problema sem fim.

O povo pensou na cauda do dragão e aplaudiu.

O tio James continuou:

— Algo aconteceu com um gentil e inofensivo membro de nossa
comunidade. Não sabemos o que aconteceu. 

Todo mundo pensou no coelho chamado Frederick e gemeu. 

— As defesas de nossa pátria foram engolidas.

Todo mundo pensou no pobre exército.

— Só há uma coisa a fazer. — Tio James estava se deixando inflamar
pelo assunto. — Poderemos algum dia nos perdoar se, por termos
negligenciado uma simples precaução, perdermos mais coelhos? Ou
mesmo, quem sabe, nossa marinha, nossa polícia, nosso corpo de
bombeiros? Pois eu os advirto que o dragão púrpura nada respeitará,
nem o que há de mais sagrado.

Todo mundo pensou em si próprio — e quis saber:

— Qual é a simples precaução?

Então tio James disse:

— Amanhã é o aniversário do dragão. Ele está acostumado a ganhar um
presente em seu aniversário. Se ele ganhar um bom presente ficará
ansioso de ir mostrá-lo para seus amigos, e ele vai voar até eles e
nunca mais voltará. 

A multidão aplaudiu entusiasmada, e a princesa bateu palmas em sua
sacada.

— O presente que o dragão espera — disse o tio James, todo animado — é
um tanto caro. Mas, quando se dá um presente, não pode haver
má-vontade, especialmente quando é para uma visita. O que o dragão
quer é uma princesa. Só temos uma princesa, é verdade; mas que fique
longe de nós mostrar um caráter mesquinho num momento como este. E é
de nehuma valia um presente que nada custa a quem o dá. A disposição
de vocês em ficar sem sua princesa apenas mostrará o quão generosos
vocês são.

A multidão começou a chorar, pois todos amavam a princesa, embora
tivessem compreendido bem que seu maior dever agora era serem
generosos e dar ao dragão o que ele queria.

A princesa começou a chorar, porque não queria ser o presente de
aniversário de ninguém — e muito menos de um dragão púrpura. E Tom
começou a chorar porque ficou com muita raiva. 

Ele foi direto para casa e contou para seu elefantinho; e o
elefantinho o reconfortou e o animou tanto que logo os dois estavam
muito concentrados num pião que o elefante estava fazendo girar com
sua trombinha.

Na manhã seguinte, Tom foi logo cedo ao palácio. Deu uma olhada pelas
colinas — quase não havia mais coelhos brincando por lá agora — e
então colheu rosas brancas e foi jogá-las na janela da princesa até
ela acordar e aparecer.

— Suba e venha me beijar — ela disse.

Então Tom subiu na roseira branca e beijou a princesa na janela,
dizendo:

— Muitas felicidades pelo dia de hoje.

Mary Ann começou a chorar, e disse:

— Oh, Tom, como você pôde dizer isso? Quando você sabe muito bem
que...

— Ih, não chore — Tom disse. — Ora, Mary Ann, o que você acha que eu
vou estar fazendo enquanto o dragão estiver recebendo seu presente de
aniversário? Não chore, minha cara Mary Ann! Fido e eu já arranjamos
tudo. Você só precisa fazer o que eu lhe disser.

— Só isso? — disse a princesa. — Ah, isso é fácil, já fiz tanto.

Então Tom lhe disse o que era para ela fazer. E ela beijou-o várias
vezes.

— Ah, meu caro, bom e esperto Tom! — ela disse. — Que bom que eu lhe
dei o Fido. Vocês dois me salvaram. Adoro vocês.

Na manhã seguinte tio James pôs seu melhor paletó e seu melhor chapéu,
e o manto bordado com cobras douradas — era um mago, dado portanto a
um gosto extravagante em roupas — e chamou uma carruagem para levar a
princesa.

— Venha, meu pequeno presente de aniversário — ele disse ternamente. —
O dragão vai ficar tão contente. E fico feliz em ver que você não
está chorando. Você sabe, minha filha, nunca se é pequena demais para
aprender a pensar na felicidade dos outros antes da sua própria. Eu
não ia gostar que minha sobrinha fosse egoísta, ou quisesse negar um
prazer banal a um pobre dragão doente, longe de casa e de seus
amigos.

A princesa disse que ia tentar não ser egoísta.

A carruagem estava chegando ao pilar, e lá estava o dragão, sua
horrível cabeça púrpura brilhando no sol, e sua horrível boca púrpura
meio aberta.

Tio James disse:

— Bom dia, prezado senhor. Trouxemos um presentinho para o seu
aniversário. Não queríamos deixar que essa data passasse sem uma
homenagem, em especial por ser o aniversário de um forasteiro em
nosso meio. Temos poucas posses, mas grandes corações. Só temos uma
princesa, mas é com generosidade que a damos; não é, minha pequena?

A princesa disse que supunha que sim, e o dragão chegou um pouco mais
perto. 

De repente uma voz gritou:

— Corra! — e era o Tom, e ele tinha trazido do zoológico o
porquinho-da-índia e duas lebres; e disse:— Só para ficar mais justo.


Tio James ficou furioso.

— O que o senhor pretende — berrou — se intrometendo numa cerimônia do
estado com seus coelhos ordinários e o resto? Vá embora, menininho
travesso, vá brincar em outro lugar com eles.

Mas enquanto ele falava os dois coelhos tinham chegado um de cada lado
dele, seus corpos se avantajando enormes, e então o espremeram entre
eles dois de modo a afundá-lo no pelo grosso deles, e ele quase ficou
sufocado. A princesa, enquanto isso, tinha corrido para o outro lado
do pilar e estava espiando detrás dele o que estava acontecendo. Uma
multidão havia seguido a carruagem desde a cidade; e agora tinham
chegado ao cenário da “cerimônia do estado”, e todos gritaram:

— Não é justo! Queremos jogo limpo! Não podemos voltar atrás em nossa
palavra desse jeito. Dar uma coisa e depois tirá-la? Ora, isso não se
faz. Deixem o pobre exilado estrangeiro ganhar seu presente de
aniversário.

E eles tentaram pegar Tom, mas o porquinho-da-índia ficou no caminho.

— Sim — Tom gritou — Jogo limpo é jóia. E o seu pobre exilado vai ter
a princesa, se conseguir pegá-la. Agora, Mary Ann!

Mary Ann olhou em volta do grande pilar e gritou para o dragão:

— Buu! Você não me pega! — e começou a correr o mais rápido que podia,
e o dragão correu atrás dela. Quando a princesa tinha corrido meio
quilômetro ela parou, deu a volta numa árvore, e correu de volta para
o pilar e em torno dele, com o dragão atrás dela. Ele era tão
comprido que não conseguia virar tão rápido quanto ela. E a princesa
deu voltas e mais voltas correndo em torno do pilar. A primeira volta
foi bem longe do pilar, e as seguintes foram ficando mais e mais
perto, com o dragão atrás dela o tempo todo; e ele estava tão ocupado
com tentar pegá-la que nem sequer notou que Tom tinha prendido na
rocha do pilar a ponta da sua comprida e firme cauda que parecia um
chicote, de modo que quanto mais o dragão corria em torno do pilar,
mais voltas ele fazia sua cauda dar no pilar. Era exatamente igual
enrolar a corda num pião, só que o eixo do pião era o pilar e a corda
era a cauda do dragão. E o mago estava preso entre as lebres, e não
podia ver nada a não ser escuridão, ou fazer nada a não ser sufocar.

Quando o dragão estava enrolado no pilar o máximo que dava para ele
ficar — feito uma linha num carretel — a princesa parou de correr, e
embora não tivesse lhe sobrado muito fôlego, conseguiu dizer:

— Rá! Quem ganhou agora?

Isso irritou tanto o dragão que ele resolveu usar toda a sua força, e
abriu suas enormes asas púrpuras, e tentou voar para pegá-la. Claro
que ao fazer isso ele puxou sua cauda, e a puxou com muito força, com
tanta força que a cauda ia ter que vir, e o pilar ia ter que virar
junto com a cauda, e a ilha ia ter que virar junto com o pilar, e num
instante a cauda estava solta, e a ilha girando em torno de si mesma
exatamente igual a um pião. Ela girava tão rápido que todo mundo se
jogou no chão e se segurou firme a si mesmo, porque sentiram que
alguma coisa ia acontecer. Todo mundo menos o mago, que estava
sufocando entre as lebres, e não sentia nada a não ser pelo e raiva.

E alguma coisa realmente aconteceu. O dragão tinha feito o reino da
Rotundia girar no sentido que devia ter ido no começo do mundo, e
enquanto ele girava, todos os animais começaram a mudar de tamanho.
Os porquinhos-da-índia ficaram pequenos, e os elefantes grandes, e os
homens e as mulheres e as crianças teriam mudado de tamanho também,
se não tivessem tido o bom senso de se segurar a si mesmos, o mais
firme que podiam, com as duas mãos; algo que, claro, não se poderia
esperar que os animais soubessem como fazer. E o melhor de tudo foi
que quando os animais pequenos ficaram grandes e os animais grandes
ficaram pequenos, o dragão também ficou pequeno, e caiu aos pés da
princesa: uma pequena, rastejante e púrpura lagartixa com asas. 

— Que coisinha mais engraçadinha — disse a princesa, ao vê-lo. — Vou
ficar com ela de presente de aniversário. 

Mas enquanto todo mundo estava deitado no chão, se segurando firme a
si mesmos, Tio James, o mago, jamais pensou em se segurar — tudo que
ele estava pensando era em como castigar lebres e filhos de
jardineiros; de modo que quando os grandes animais ficaram pequenos,
ele ficou pequeno como eles, e o pequeno dragão púrpura, quando caiu
aos pés da princesa, viu ali um maguinho muito pequeno chamado tio
James. E o dragão ficou com ele porque queria um presente de
aniversário.

De modo que então todos os animais tinham novos tamanhos — e no começo
parecia muito estranho para todo mundo os elefantes enormes e
pesadões, e uma [dormouse] minúscula, mas logo se acostumaram, e
pensam nisso tão pouco quanto nós.

Tudo isso aconteceu há vários anos atrás, e outro dia eu vi na Gazeta
da Rotúndia uma notícia sobre o casamento da princesa com Lord Thomas
Jardineiro, S.K.D., e eu sei que ela não teria casado com ninguém a
não ser o Tom, de modo que suponho que o fizeram um lorde para o
casamento — e S.K.D. quer dizer, naturalmente, Sagaz Conquistador do
Dragão. Se você acha que está errado é só por não saber como é a
ortografia na Rotundia. O jornal dizia que entre os belos presentes
do noivo para a noiva estava um enorme elefante, no qual o casal fez
o desfile do casamento. Só pode ser o Fido. Você lembra que o Tom
prometeu devolvê-lo a princesa quando casassem. A Gazeta da Rotundia
chamou o casal de “o par feliz”. Brilhante o jornal chamá-los assim —
é uma expressão tão inédita e bonita, e eu acho que é mais verdadeira
que muitas das coisas que se vê nos jornais.

Porque, veja, a princesa e o filho do jardineiro gostavam tanto um do
outro que não havia como não ficarem felizes — e além disso, eles tem
um elefante só deles para levá-los para passear. Se isso não é o
suficiente para fazer alguém feliz, não sei o que é. Se bem que, é
claro, conheço gente que só poderia ser feliz se tivesse uma baleia
para navegar nela, e talvez nem mesmo assim. Mas é gente gananciosa e
glutona, do tipo que é bem capaz de se servir quatro vezes de pudim,
o que nem Tom nem Mary Ann jamais fizeram.

\chapter{4. A rã benfazeja\subtitulo{Marie-Catherine Le Jumel de Barneville de La Motte, Comtesse d´Aulnoy}}

Era uma vez um rei que estava em guerra com seus vizinhos fazia tempo.
Depois de muitas batalhas, os inimigos sitiaram sua capital; ele
temia por sua rainha, que estava grávida, e pediu a ela que se
retirasse para um castelo que mandara fortificar, no qual só havia
estado uma vez. A rainha se valeu de súplicas e de lágrimas para
convencê-lo a deixá-la ficar com ele; queria compartilhar com ele o
seu destino, e foi aos prantos que embarcou na carruagem que a
levaria embora. O rei mandou alguns de seus guardas para
acompanhá-la, e prometeu que assim que pudesse iria secretamente
visitá-la; mas era apenas algo para lhe dar esperança, pois o castelo
ficava muito longe, era cercado por uma densa floresta, e a menos que
se conhecesse o caminho muito bem, não era fácil chegar nele.

A rainha partiu, muito triste de deixar seu marido em meio aos perigos
da guerra. A jornada foi feita aos poucos, para que ela não ficasse
doente pela fadiga de uma viagem tão longa; e enfim ela chegou ao
castelo, muito inquieta e infeliz. Depois de descansar o bastante,
ela quis passear nos arredores, e nada encontrou que pudesse
entretê-la: olhava para todos os lados, e tudo o que via eram lugares
desertos que lhe davam mais tristeza que prazer. Contemplava a
paisagem desolada, e às vezes se dizia:

— Nem se compara o lugar em que estou aos que estive em toda a minha
vida! Se eu ficar aqui muito tempo, vou acabar morrendo. Com quem
conversar num lugar tão solitário? Com quem acalmar minhas
inquietudes? O que eu fiz ao rei para ele me exilar assim? Parece que
ele queria que eu sentisse toda a amargura de sua falta, me relegando
a um castelo tão desagradável.

Era assim que ela se lamentava; e embora ele escrevesse a ela todos os
dias, e lhe desse notícias bem favoráveis sobre o andamento da
guerra, ela ficava cada vez mais aflita. Resolveu então voltar a onde
estava o rei; mas como os oficiais que ele tinha mandado com ela
tinham sido instruídos a não levá-la de volta enquanto não chegassem
ordens expressas para isso, ela nada disse de suas intenções, e fez
com que lhe arranjassem um pequeno coche, dizendo que gostaria de
sair para caçar de vez em quando. Ela mesmo conduziria os cavalos,
acompanhada de seus cães; e assim dona de sua própria carruagem,
poderia partir quando quisesse. Só restava uma dificuldade: ela não
conhecia os caminhos da floresta; mas ela se convenceu que os deuses
a levariam a um porto seguro; e depois de fazer alguns pequenos
sacrifícios a eles, convocou todo mundo para partir numa grande
caçada; ela iria em seu coche, e cada um iria em diferentes direções,
para não deixar aos animais selvagens nenhuma chance de escaparem. E
assim todos se dividiram; a jovem rainha, que acreditava que logo ia
rever seu marido, se vestira em grande estilo; seu chapéu tinha
plumas de diferentes cores, seu vestido era coberto de pedras
preciosas, e a beleza dela, que era incomum, a fazia parecer uma
segunda Diana.

Quando estavam todos já completamente envolvidos pelos prazeres da
caça, ela fez seus cavalos, usando a voz e o chicote, saírem
galopando a toda velocidade, e acabou perdendo o controle deles; tão
rápido ia a carruagem que parecia carregada pelos ventos, tão veloz
que era difícil acompanhá-la com a vista. A pobre rainha então se
arrependeu, tarde demais, de sua temeridade.

— O que eu pretendia? — disse ela — Por que fui resolver conduzir
sozinha cavalos tão selvagens e tão pouco dóceis? Ai de mim! O que
irá me acontecer? Ah, se o rei souber o perigo a que me expus, o que
será dele, que me ama tanto que me quis longe de sua capital para me
deixar em segurança? Eis como respondi ao seu carinhoso cuidado, e o
filho amado que carrego em meu ventre vai ser tanto quanto eu vítima
de minha imprudência!

Suas dolorosas lamentações ecoavam pelos ares; ela invocou os deuses,
ela pediu socorro às fadas, mas os deuses e as fadas a haviam
abandonado. A carruagem descontrolada acabou virando, e ela não teve
a presença de espírito de saltar dela, e ficou com o pé preso entre o
estribo e a roda; e não é difícil imaginar que só um milagre poderia
salvá-la, depois de um acidente tão terrível. 

Ela ficou caída por terra, ao pé de uma árvore, sem sentidos e com o
rosto coberto de sangue; e por muito tempo permaneceu nesse estado.
Quando enfim reabriu os olhos, viu perto dela uma mulher de uma
altura gigantesca, vestida apenas com uma pele de leão, com os
breaços e pernas nus, os cabelos presos por uma pele seca de
serpente, cuja cabeça ficava pendurada em suas costas; tinha uma
clava de pedra numa mão, e nela se apoiava como uma bengala, e na
outra mão um aljava cheia de flechas.Uma figura tão extraordinária
fez a rainha achar que estava morta, pois não podia crer que depois
de um acidente tão terrível pudesse ainda estar viva, e disse para si
mesma:

— Não fico nem um pouco surpresa, que se tenha tanto medo de morrer; o
que se vê no outro mundo é bem terrível.

A gigante, que a escutara, não pode evitar rir da impressão dela de
estar morta.

— Reanime-se — disse ela — saiba que ainda está entre os vivos. Mas
seu destino nem por isso será menos triste. Sou a Fada Leoa, e moro
aqui perto; você vai ter que vir passar o resto da sua vida comigo. 

A rainha a olhou tristemente, e disse:

— Se a senhora quiser, madame Leoa, me levar até meu castelo, e dizer
ao rei o que quer de resgate, tenho certeza que pelo tanto que ele me
ama não recusará nem mesmo a metade de seu reino.

— Não — ela respondeu. — Já sou rica o bastante, e estou entediada de
viver sozinha há tanto tempo; você parece animada, talvez me divirta.

E ao terminar de dizer essas palavras, transformou-se numa leoa, e
carregou a rainha em suas costas, levando-a ao fundo de sua terrível
caverna. Lá, ela a curou passando uma poção em seus ferimentos.

Qual não foi a surpresa e o desconsolo da rainha, ao se ver naquele
pavoroso lugar! Descia-se até ele por dez mil degraus, que levavam
quase que ao centro da terra; não havia nenhuma outra luz que a de
grandes lanternas que se refletiam num lago de mercúrio. Estava
repleto de monstros, cujas diferentes feições aterrorizariam uma
rainha menos tímida; corujas e gralhas, alguns corvos e outras aves
de mau augúrio se ouviam; à distância era possível ver uma montanha
donde descia um riacho de águas morosas, feito das lágrimas que
apaixonados infelizes verteram. As árvores jamais tinham folhas e
frutos, e a terra era coberta de espinheiros e urtigas. A comida era
apropriada a um lugar tão maldito: raízes secas, castanhas e frutos
secos era tudo o que havia para saciar a fome dos desafortunados que
caíam nas mãos da Fada Leoa.

Assim que a rainha estava em condições de trabalhar, a Fada lhe disse
que ela poderia fazer uma cabana para ela, porque ia ficar o resto da
vida ali. Ao ouvir tais palavras a rainha não conseguiu segurar as
lágrimas.

— Oh! O que eu lhe fiz — exclamou ela — para ficar presa aqui? Se o
fim de minha vida, que pressinto próximo, lhe der algum prazer,
mate-me logo, que é tudo que ouso esperar de sua compaixão. Mas não
me condene a passar uma vida longa e deplorável longe de meu marido. 

A Leoa zombou da dor dela, e disse que a aconselhava a enxugar suas
lágrimas, e tratar de tentar agradá-la; pois se ela não se
comportasse bem, faria dela a pessoa mais infeliz do mundo.

— Que posso então fazer — replicou a rainha — para comover seu
coração?

— Eu adoro — disse ela — tortas de mosca; quero que você encontre um
jeito de conseguir moscas o suficiente para me fazer uma bem grande e
muito gostosa.

— Mas — disse a rainha — não vejo nenhuma mosca por aqui. E se
houvesse, não é claro o bastante para capturá-las; e se eu as
capturasse, jamais fiz torta alguma; de modo que você me deu uma
ordem que não posso cumprir.

— Não importa — disse a impiedosa Leoa. — O que eu quero é isso.

A rainha nada disse; pensou que, apesar da Fada cruel, só tinha uma
vida a perder, e na situação em que estava, o que podia temer? Em vez
então de ir procurar moscas, ela se sentou sob um teixo, e começou
seus tristes lamentos.

— Qual não será sua dor, meu amado esposo, quando vier me procurar e
não me encontrar! Achará que morri ou lhe fui infiel, e prefiro que
você chore a perda de minha vida que a de meu amor. Talvez encontrem
na floresta a carruagem despedaçada, e todos os enfeites que eu pus
para agradar a você; ao ver isso, você não terá mais como duvidar de
minha morte, e como posso saber se não irá conceder a alguma outra a
parte que você me havia dado de seu coração? Mas ao menos eu não
ficarei sabendo, pois nunca mais retornarei ao mundo.

Ela teria continuado assim por muito tempo, se não tivesse ouvido
acima dela o sombrio grasnado de um corvo. Ela ergueu os olhos, e com
o pouco de luz que havia, viu de fato um grande corvo com uma rã no
bico, pronto a devorá-la. 

— Embora nada aqui me reconforte — ela disse — nem por isso posso
deixar de salvar uma pobre rã, que está numa situação tão terrível
quanto a minha. 

Ela pegou o primeiro galho que encontrou, e bateu com ele no corvo. A
rã caiu, ficou por um tempo aturdida, mas logo recobrou seus sentidos
de rã.

— Bela rainha — ela disse — você é a única pessoa bondosa que
encontrei nesse lugar, desde que a curiosidade me trouxe aqui. 

— Por qual maravilha você é capaz de falar, rãzinha — disse a rainha —
e quem são as outras pessoas que você encontrou aqui? Pois até agora
não vi ninguém.

— Todos os monstros do qual o lago está repleto — respondeu a rã —
estiveram antes no mundo; uns no trono, outros com a confiança de
seus soberanos, e mesmo as amantes de alguns reis, que muito sangue
fizeram ser derramado; são aquelas que você vê metarmofoseadas em
sanguessugas. O destino os envia aqui por algum tempo, sem que nenhum
que aqui vem volte melhor ou se emende. 

— Que muitos malvados juntos — disse a rainha — não se ajudem a se
emendar, compreendo bem; mas quanto a você, comadre Rã, o que faz
aqui?

— A curiosidade me fez vir — ela respondeu. — Sou meio-fada, meu poder
é limitado para algumas coisas, mas sem limites em outras; se a Fada
Leoa me descobrisse em seus domínios, me mataria. 

— Como é possível, então, se fada ou meio-fada, um corvo estava pronto
a comê-la? 

— É simples de entender — disse a Rã. — Quando estou com meu capuzinho
de rosas na cabeça, tenho meus poderes e nada temo; mas infelizmente
deixei-o de lado no charco, e então esse corvo maldito me atacou;
confesso que se não fosse por você, não existiria mais; e como eu lhe
devo a vida, se eu puder fazer qualquer coisa para ajudar a sua, pode
pedir.

— Ai de mim, cara Rã! — disse a rainha. — A Fada malvada que me fez
prisioneira quer que eu lhe faça uma torta de moscas; não há nenhuma
aqui, e se houvesse, a luz não seria suficiente para capturá-las,e eu
corro o risco de morrer nas mãos dela. 

— Deixe comigo — disse a Rã. — Vou providenciá-las.

Ela em seguida se cobriu de açúcar, e chamou seis mil outras rãs suas
amigas para fazer o mesmo. Foram em seguida para um lugar cheio de
moscas; a fada malvada as tinha em cativeiro, só para poder usá-las
para atormentar certos desafortunados. Assim que sentiram o cheiro do
açúcar, escaparam apenas para se grudarem no açucar; e as seis mil
rãs voltaram a galope para onde estava a rainha. Jamais se fizera
tamanha captura de moscas, e jamais uma torta melhor que aquela que a
rainha fez para a Fada Leoa; ela ficou muito surpresa ao recebê-la,
não compreendendo como a rainha conseguira capturá-las.

A rainha então cortou alguns ciprestes para começar a construir sua
casinha, pois já tinha ficado exposta demais as intempéries do ar,
que lá era venenoso. A Rã veio oferecer generosamente seus serviços,
e chefiando todas as outras que a haviam ajudado com as moscas, logo
construíram uma pequena cabana, a mais bonita do mundo. Mas mal a
rainha se deitou nela e todos os monstros do lago, ciumentos do que
ela fizera, vieram atormentá-la com a pior algazarra jamais ouvida.
Ela se levantou toda assustada, e fugiu; era o que queriam os
monstros. Um dragão, outrora tirano de um dos reinos mais belos do
universo, tomou posse da casinha.

A pobre rainha quis se lamentar; mas todos zombaram dela; os monstros
a vaiaram, e a Fada Leoa disse que, se tivesse que ouvir mais
lamentações, ela iria apanhar. Só lhe restou se calar e recorrer à
Rã, que era a melhor pessoa do mundo. As duas choraram juntas, pois
com seu capuzinho de rosas a Rã era capaz de chorar e rir como
qualquer pessoa. E então a Rã disse:

— Minha amizade por você é tão grande que mal posso esperar para lhe
ajudar a ter uma outra cabana.

E no mesmo instante se pôs a cortar madeira, e tão expediente foi que
o pequeno palácio rústico da rainha ficou pronto a tempo dela já
passar aquela mesma noite nele. A Rã, atenciosa a tudo que era
necessário à rainha, fez para ela uma cama perfumada com tomilho
selvagem. 

Assim que a fada malvada ficou sabendo que a rainha não mais dormia no
chão, mandou-a chamar, e lhe perguntou:

— Quem são os homens ou deuses que a protegem? Essa terra, sempre
sofrendo com chuvas de enxofre e fogo, jamais produziu uma folha que
fosse, e apesar disso ouvi dizer que ervas perfumadas crescem a sua
volta!

— Ignoro a causa — respondeu a rainha. — Se há alguma razão, só pode
ser à criança que carrego, que quem sabe será menos infeliz que eu. 

— Tenho vontade — disse a Fada — de ter um bouquê com as flores mais
raras; veja se sua sorte consegue providenciá-las. Porque senão,
providenciarei muitos golpes para você, e nisso sou maravilhosa. 

A rainha se pôs a chorar; as ameaças a assustaram, e a impossibilidade
de conseguir as flores deixou-a desesperada. Ela voltou para sua
casinha; e a Rã veio ter com ela.

— Por que você está tão triste? — disse.

— Ai, cara comadre, quem não estaria? A Fada quer um bouquê das mais
belas flores; onde vou achá-las? Você sabe como são as poucas que
nascem aqui; e no entanto minha vida corre perigo, se eu não a
contentar. 

— Amável princesa — disse a Rã — é preciso tratar de tirá-la dessa
situação. Há aqui um morcego, que é uma das poucas criaturas daqui
com quem me dou. É uma boa pessoa, e é mais rápido que eu. Darei a
ele meu capuz de rosas, e com a ajuda dele, ele conseguirá trazer as
flores. 

A rainha fez a ela uma mesura em agradecimento, pois não havia meio de
abraçar a pequena Rã. 

Sem demora a Rã foi atrás do morcego, e poucas horas depois ele
voltou, trazendo sob as asas flores admiráveis. A rainha as levou
para a Fada, que ficou ainda mais surpresa que da outra vez, não
fazendo idéia por qual milagre a rainha se saíra tão bem.

Mas o que a rainha mais queria, e pensava sem cessar, era em algum
meio de escapar dali. Ela comunicou essa sua vontade a Rã, que disse:

— Rainha, permita-me antes de mais nada consultar o meu capuz, e então
agiremos seguindo seus conselhos.

Ela o pegou, o pôs sobre palha, e na frente dele, fez um fogo onde
jogou zimbro, alcaparras e ervilhas; e coaxou cinco vezes. A
cerimônia terminada, pôs na cabeça o capuz, e começou a falar como um
oráculo:

— O destino, mestre de tudo, a proíbe de sair deste lugar; você terá
uma princesa mais bela que a deusa do amor; qaunto ao resto, não
adianta se preocupar, só o tempo trará resposta.

A rainha baixou os olhos, e verteu algumas lágrimas, mas decidiu crer
em sua amiga.

— Ao menos — ela disse — não me abandone; acompanhe meu parto, já que
ele terá de ser aqui.

A rã a consolou o melhor que pôde.

Mas já é tempo de falar do rei. Enquanto seus inimigos sitiavam sua
capital, ele não podia enviar mensageiros buscarem a rainha; mas
depois de várias incursões de seu exército, obrigou os inimigos a se
retirarem, e a felicidade dessa ocasião ele sentiu bem menos para ele
mesmo do que para sua querida rainha, pois agora ele poderia ir
buscá-la sem medo. Ele não sabia do desastre, pois nenhum de seus
oficiais teve a coragem de ir avisá-lo. Haviam encontrado a carruagem
da rainha despedaçada, os cavalos fugidos, e espalhado pelo chão todo
o uniforme de amazona que ela pusera para ir encontrá-lo.

Como não duvidavam da morte dela, e acreditavam que havia sido
devorada, resolveram dizer ao rei que ela morrera subitamente. Essa
notícia funesta fez o rei achar que morreria também de dor; cabelos
arrancados, lágrimas vertidas, gritos comoventes, soluços, suspiros,
e outros direitos dos viúvos, nada foi poupado na ocasião.

Depois de passar vários dias sem ver ninguém, e sem querer que o
vissem, ele voltou para sua grande cidade, carregando o peso de seu
enorme luto, que lhe caía melhor no coração do que nas roupas. Todos
os embaixadores dos reinos vizinhos vieram lhe dar os pêsames; e
depois das cerimônias que são inesparáveis dessa espécie de
catástrofe, ele se dedicou à paz de seus súditos, livrando-os da
guerra e fomentando o comércio. 

A rainha nada soube disso; a hora de dar a luz chegou, e foi muito
feliz; os céus lhe deram uma pequena princesa, tão bela quanto a Rã
previra; elas a chamaram Moufette, e a rainha obteve permisão da Fada
Leoa para a criar, com grande dificuldade; pois a Fada teve muito
vontade de devorá-la, tanto era feroz e bárbara.

Moufette, uma pequena maravilha, logo fez seis meses; e a rainha,
olhando-a com uma ternura misturada com piedade, dizia sem parar:

— Ah! Se o rei seu pai a visse, minha pobre pequena, que alegria ele
iria ter, como você lhe seria querida! Mas talvez, nesse mesmo
momento, ela já tenha começado a me esquecer; ele nos crê para sempre
sepultadas no horror da morte; talvez uma outra já ocupe o meu lugar
em seu coração.

Esses pensamentos tristes lhe custavam muitas lágrimas; e a Rã, que a
amava de verdade, disse um dia, ao vê-la chorar assim:

— Se você quiser, irei procurar seu marido; a viagem é longa e eu
caminho lentamente; mas enfim, mais cedo ou mais tarde, espero chegar
lá.

Essa proposta foi recebida com imensa alegria pela rainha; ela juntou
suas mãos e as mãozinhas de Moufette, para mostrar à Rã o quanto
ficaria agradecida por ela fazer tal viagem; e assegurou-lhe que o
rei também lhe ficaria muito grato. 

— Mas — continuou — qual será a utilidade dele saber que estou nesse
lúgubre lugar? Será impossível ele me tirar daqui.

— Deixemos isso aos deuses — disse a Rã — e por nossa parte façamos o
que só depende de nós.

Logo elas se despediram; a rainha escreveu com seu próprio sangue num
lenço, pois não tinha nem tinta, nem papel; pedia ao rei para
acreditar em tudo que a virtuosa Rã dissesse, pois era ela a
encarregada de lhe levar as notícias.

Ela demorou um ano e quatro dias só para subir os dez mil degraus que
separavam aquele lúgubre lugar, onde deixara a rainha, do mundo. Um
outro ano ela demorou para preparar sua comitiva, pois não queria se
apresentar numa grande corte como uma feiosa rã dos pântanos.
Providenciou uma liteira grande o bastante para caber dois ovos, toda
revestida de casco de tartaruga, e puxada por jovens lagartos;
cinquenta pererecas eram suas damas de honra; ratos-da-água, vestidos
de pajem, abriam o cortejo, seguidos dos caracóis a quem ela confiara
sua guarda; enfim, nada de mais lindo jamais se vira, coroado pelo
capuzinho de roses, bem abertas e frescas. Era um pouco vaidosa,
tanto que até pôs um pouco de rouge; houve até quem dissesse que ela
exagerara na maquiagem, como costumam fazer as damas desse país, mas
era intriga de suas inimigas.

A viagem demorou sete anos, durante os quais a pobre rainha sofreu
indescritivelmente; se não tivesse a bela Moufette para
reconfortá-la, teria morrido cem vezes. Era só essa maravilhosa
criaturinha abrir a boca e dizer uma única palavra para deixar sua
mãe completamente encantada; e até mesmo a Fada Leoa ela cativou com
seu charme. Tanto que, após seis anos, ela quis que elas a
acompanhassem nas caçadas, desde que o que elas caçassem ficassem com
ela. 

Foi uma alegria para pobre rainha rever o sol! Ela ficara tão
desacostumada a ele que teve medo de ficar cega. Quanto a Moufette,
era tão esperta que aos cinco ou seis anos nada escapava de suas
flechas; e assim mãe e filha amansaram um pouco a ferocidade da Fada.

A Rã enquanto isso caminhava por montes e vales, de dia e de noite, e
enfim chegou perto da capital onde morava o rei. Ficou muito surpresa
de ver por toda parte gente dançando e festejando; todos riam e
cantavam, e quanto mais se aproximava da capital, maiores eram a
alegria e o júbilo. Sua comitiva dos pântanos surprendia todo mundo,
e todos a seguiam; a multidão atrás dela era tão grande ao entrar na
vila que teve dificuldade em chegar ao palácio. E era lá que a
magnificência chegava ao auge. O rei, viúvo há nove anos, tinha enfim
resolvido atender às súplicas de seus súditos; ia se casar de novo,
com uma princesa na verdade não tão bela quanto sua primeira esposa,
mas da qual ele gostava bastante. 

A boa Rã desceu de sua liteira, e entrou no palácio, seguida de toda
sua comitiva. Ela nem teve que pedir uma audiência; o monarca, sua
noiva e todos os príncipes estavam muito curiosos de saber o que
fizera vir ao palácio tão singular comitiva.

— Majestade — ela disse — não sei se a notícia que trago lhe dará
alegria ou tristeza; as bodas que está por celebrar me convencem de
sua infidelidade à rainha.

— A lembrança dela me é muito querida — disse o rei (vertendo algumas
lágrimas que não conseguiu segurar). — Mas é preciso que você saiba,
gentil rã, que os reis nem sempre fazem o que querem. Faz nove anos
que meus súditos me pressionam para que eu case de novo; devo a eles
herdeiros. Foi assim que conheci essa princesa que me pareceu muito
encantadora.

— Eu o aconselho a não se casar com ela, pois a poligamia é crime; a
rainha não está morta. Eis aqui uma carta escrita com o próprio
sangue dela, e que ela me encarregou de lhe entregar-lhe. Vossa
majestade tem agora uma princesa, Moufette, que é mais bela que todos
os céus juntos.

O rei pegou o lenço onde a rainha havia escrito algumas palavras, o
beijou, o enxarcou com suas lágrimas, e o mostrou a todos, dizendo
que reconhecia a letra de sua esposa, e fez mil perguntas à Rã, que
as respondeu com muito espírito e vivacidade. A princesa noiva, e os
embaixadores, encarregados de acompanhar a celebração de seu
casamento, fizeram cara feia.

— Como vossa majestade — disse o mais importante deles — com base
apenas nas palavras de um sapinho como esse, pode romper um
compromisso tão solene? Essa escória dos pântanos tem a insolência de
vir mentir em vossa corte, e tem o prazer de se ver sendo ouvida? 

— Senhor embaixador — retrucou a Rã — saiba que eue não sou escória
nenhuma dos pântanos, e já que é preciso mostrar minha ciência,
avante, fadas, apareçam.

Todas as pererecas, ratos, caracóis e lagartos apareceram com ela a
frente; mas não mais tinham a aparência desses animaizinhos; eram
agora altos e de porte majestoso, com as faces bonitas, os olhos mais
brilhantes que as estrelas, cada um portando uma coroa de pedras
preciosas na cabeça, e sobre os ombros mantos reais de veludo muito
compridos, cujas pontas eram levadas por anões. Ao mesmo tempo, eis
que trumpetes, tímpanos, oboés e tambores se fizeram soar nos ares,
melífluos e guerreiros, e todas as fadas começaram um balé dançado
com tanta leveza que a cada reviravolta se elevavam até os arcos do
salão. O rei e sua futura esposa ficaram um mais surpreso que o
outro, ainda mais quando aqueles honrados bailarinos se
metamorfosearam em flores, nem por isso menos dançantes do que quando
tinham pernas e pés; jasmins, junquilhos, violetas, cravos e
angélicas bailavam animadamente, em movimentos que encantavam tanto a
vista quanto o olfato.

Um instante depois, as flores desapareceram, e em seus lugares várias
fontes surgiram; elevaram-se rapidamente, e a água caiu de volta num
largo canal que se formou ao longo do castelo; estava repleto de
pequenas galeras coloridas e douradas, tão bonitas e galantes que a
princesa insistiu que seus embaixadores a acompanhassem num passeio
numa delas. E eles consentiram, achando que tudo aquilo não passava
de um entrenimento que se concluiria com a feliz celebração das
bodas. 

Assim que eles embarcaram, a galera, o rio e todas as fontes
desapareceram, e as rãs voltaram a ser rãs. O rei perguntou onde
estava sua princesa; a Rã retrucou:

— Vossa majestade não deve ter outra que a rainha sua esposa; se elas
não fossem minhas amigas, não teria me intrometido em seu casamento;
mas ela tem tanto mérito, e sua filha Moufette é tão adorável, que
vossa majestade não devia perder nem mais um momento e partir para
libertá-las. 

— Eu lhe digo, senhora Rã, que se eu não achasse que minha mulher está
morta, faria qualquer coisa no mundo para revê-la.

— Depois de todas as maravilhas que fiz aqui — disse ela — me parece
que vossa majestade já devia estar convencida do que lhe digo; deixe
seu reino em boas mãos, e não demore a partir. Eis um anel que lhe
dará os meios de ver a rainha, e de falar com a Fada Leoa, embora ela
seja a criatura mais terrível que há no mundo.

O rei não vendo mais a princesa que lhe estava destinada, sentiu que
sua paixão por ela rápido se arrefecia, ao mesmo tempo que a que
tinha pela rainha ganhava novas forças.

Ele partiu sem querer a companhia de ninguém, e deu presentes
consideráveis à Rã. 

— Não perca jamais a coragem — ela disse. — São terríveis as
dificuldades que terá de enfrentar; mas espero que conseguirá o que
deseja. 

O rei, reconfortado por essas promessas, não quis nenhum outro guia
que o anel para ir procurar sua rainha. 

Enquanto isso, Moufette crescia, e sua beleza se aperfeiçoava tanto
que todos os monstros do lago de mercúrio se apaixonaram por ela; não
era incomum dragões de uma aparência lamentável virem se estender
junto a ela. Embora ela os tivesse sempre visto, seus belos olhos
jamais se acostumava a eles, e ela fugia e se escondia entre os
braços de sua mãe. 

— Ainda ficaremos muito tempo aqui? — dizia. — Nossos infortúnios
nunca mais vão acabar? 

A rainha lhe dava grandes esperanças para reconfortá-la, mas no fundo
ela mesma não tinha nenhuma; a ausência da Rã, seu profundo silêncio,
tanto tempo tendo se passado sem ter notícia nenhuma do rei; tudo
isso a afligia demais. 

A Fada Leoa acabara se acostumando a levá-las caçar; ela era gulosa, e
adorava as presas que elas conseguiam matar para ela. Como
recompensa, dava apenas os pés ou a cabeça; mas mesmo assim já era
muito poderem rever a luz do dia. A Fada assumia a figura de uma
leoa; e a rainha e sua filha se sentavam em suas costas, e assim
percorriam as florestas.

O rei, conduzido pelo anel, chegou à floresta; e as viu passar como um
raio. Não foi visto, e quis seguí-las, mas elas estavam tão rápido
que logo desapareceram de vista. Apesar dos contínuos tormentos da
rainha, sua beleza em nada se alterara; ela pareceu ao rei mais
adorável que nunca. Toda sua paixão se reacendeu; e não tendo a menor
dúvida que a jovem princesa que estava com ela era sua querida
Moufette, decidiu que mesmo se tivesse percer mil vezes, jamais
abandonaria a missão de salvá-las.

O prestativo anel o conduziu ao lúgubre lugar em que a rainha ficara
por tantos anos; ele não ficou pouco surpreso de ter que descer tão
fundo na terra, mas o que viu lá ainda o deixou mais atônito. A Fada
Leoa, que tudo sabia, não tinha dúvida do dia e da hora em que ele ia
chegar; e estava decidida a combatê-lo com todo o seu poder.

Ela construiu no meio do lago de mercúrio uma palácio de cristal, que
vagava ao sabor das ondas; nele, encarcerou a pobre rainha e sua
filha, e em seguida advertiu aos monstros que estavam todos
apaixonados por Moufette:

— Vocês perderão essa bela princesa, se não se aliarem a mim em
defendê-la de um cavaleiro que está vindo raptá-la. 

Os monstros prometeram nada negligenciar do que podiam fazer; cercaram
o palácio de cristal; os mias leves se puseram sobre o teto e nas
paredes; os outros nas portas, e o resto no lago. 

O rei, aconselhado pelo fiel anel, foi primeiro até a caverna da Fada;
ela o esperava na forma de leoa. Assim que ele apareceu, saltou na
direção dele; ele desembainhou a espada com uma coragem que ela não
previra; e como ela esticara a pata para derrubá-lo, ele a cortou bem
na articulação. Ela deu um grito enorme, e caiu; ele se aproximou e
pôs o pé na garganta dela, jurando matá-la; e apesar da terrível
fúria dele, ela não se deixou intimidar.

— O que você quer de mim? — perguntou.

— Quero puní-la por ter raptado minha mulher — respondeu ele
bravamente — e quero que você a devolva para mim, ou a estrangularei
agora mesmo!

— Dê uma olhada no lago — ela disse — e veja se ela não está em meu
poder.

O rei olhou na direção que ela indicou, e viu a rainha e sua filha no
palácio de cristal, vagando à deriva no lago de mercúrio. 

Ele achou que ia morrer de alegria e de dor: chamou-as com toda força,
e elas o ouviram; mas como alcançá-las? Enquanto ele pensava nisso, a
Fada Leoa desapareceu.

O rei corria ao longo das margens do lago; quando estava de um lado
pronto a alcançar o palácio transparente, ele se afastava numa
velocidade tremenda; e suas esperanças davam assim toda vez em nada.
A rainha, que temia que ele acabasse desistindo, gritava a ele para
não perder a coragem, que a Fada Leoa queria apenas cansá-lo, mas que
um verdadeiro amor supera todas as dificuldades. E de lá de dentro
ela e Moufette estendiam-lhe as mãos, em atitude de súplica. Ao ver
isso, o rei se sentiu tomado por novas forças; elevou a voz, e jurou
pelo Styx e pelo Acheron que preferia passar o resto de sua vida
naquele lugar infeliz a partir sem elas. 

Era preciso que ele tivesse grande perseverança; passou por maus
bocados. A terra, cheia de espinhos, era sua cama; tudo o que comia
era frutos selvagens mais amargos que fel, e toda hora tinha que
entrar em combate com os monstros do lago. Um marido que se comporta
assim para recuperar sua mulher é sem dúvida do tempo das fadas, e
como ele agiu diz bem a época de meu conto. 

Três anos se passaram sem que o rei fizesse algum progresso; estava já
quase desesperado; cem vezes tomou a resolução de se jogar no lago; e
o teria feito, se visse nesse último gesto algum remédio contra o
sofrimento da rainha e de sua filha. E estava ele a correr como
sempre, uma hora para um lado do lago, outra hora para outro, quando
um pavoroso dragão o chamou, e disse:

— Se você jurar pela sua coroa, seu cetro, e seu manto real, por sua
esposa e sua filha, que me dará um certa guloseima que muito aprecio,
e que eu pedirei quando tiver vontade, vou levá-lo sobre minhas asas,
e apesar de todos os monstros do lago que guardam o castelo de
cristal, prometo que conseguiremos tirar de lá a rainha e a princesa
Moufette. 

— Ah! Caro dragão — exclamou o rei — eu juro, a você e a toda a sua
espécie dragoniana, que eu lhe darei o que quiser comer até se
saciar, e serei para sempre seu servo e criado.

— Não prometa nada — retrucou o dragão — se não for capaz de manter
sua palavra; pois tamanhos tormentos lhe ocorerrão que você se
lembrará pelo resto de sua vida. 

O rei reiterou sua promessa; morria de impaciência de libertar sua
amada rainha. Montou então sobre as costas do dragão, como teria
feito se fosse o mais belo cavalo do mundo; e quando os monstros
vinham para impedir a passagem deles, acabavam brigando entre eles
mesmos, e só se ouvia o silvo agudo das serpentes, e só se via fogo,
e enxofre e salitre se espalhavam por todos os lados. Enfim chegaram
ao castelo de cristal, e tiveram que renovar seus esforços: os
morcegos, abutres e corvos todos o impediam de entrar, mas o dragão
com suas garras, seus dentes e sua cauda, despedaçava os mais
ousados. A rainha por seu lado, vendo aquele enorme combate, quebrou
as paredes de cristal com chutes; e dos pedaços se serviu como aramas
para ajudar seu esposo; eles enfim obtiveram a vitória, e se
reuniram, e o encantamento se esvaiu com o trovão de um raio caindo
no lago, que o deixou silencioso. 

 O dragão desaparecera como todo o resto, e sem que o rei pudesse
descobrir como, havia sido transportado de volta à sua capital, e
junto com ele estavam a rainha e Moufette, num salão magnífico,
sentados a uma mesa deliciosamente bem servida. Jamais se vira um
espanto semelhante ao deles, nem um alegria maior. Todos os súditos
acorreram para ver sua rainha e a jovem princesa, ambas, por algum
prodígio, tão soberbamente vestidas que eram ofuscantes as pedras
preciosas de seus trajes.

Não é difícil imaginar todos os festejos a que se dedicou essa corte
rejubilante. Havia mascaradas, bailados, gincanas e torneios, que
atraíram a presença dos príncipes mais importantes do mundo; e os
belos olhos de Moufette se detiam em todos. Entre os mais bonitos e
distintos, era o príncipe Moufy quem mais se destacava; todos o
elogiavam e admiravam, e a jovem Moufette, que até então só vivera
com as serpentes e dragões do lago, logo se rendeu a seus encantos; e
nenhum dia se passava sem que ele não fizesse a ela novos galanteios,
pois a amava apaixonadamente; e em audiência para declarar suas
intenções, informou ao rei e à rainha que seu principado era de uma
beleza e tamanho a merecerem particular atenção. O rei disse que
cabia a Moufette escolher o marido dela, e que ele não faria nenhuma
pressão, que atenderia à vontade dela, e que era o único meio de
serem todos felizes. O príncipe ficou encantado com essa resposta,
pois sabia das vezes que se encontraram que ela não lhe era
indiferente; e quando ele se declarou a ela, ela disse que se não
fosse ele o seu marido, não ia querer nenhum outro. Moufy, jubilante
de alegria, se ajoelhou aos pés dela, e da maneira mais terna, pediu
que ela jamais se esquecesse da palavra que lhe dera. 

Ele foi direto aos aposentos do rei e da rainha, e lhes contou dos
progressos de seu amor por Moufette, e suplicou que não fizessem sua
felicidade demorar. Eles deram seu consetimento com prazer; o
príncipe tinha tantas qualidades que parecia ser o único digno a ter
a mão da maravilhosa Moufette. Eles então ficaram noivos, antes que o
príncipe retornasse à sua terra natal, onde providenciaria os
preparativos para o casamento; mas jamais teria partido sem se
assegurar que seria feliz ao retornar. A princesa não conseguiu se
despedir dele sem verter muitas lágrimas, pois sombrios
pressentimentos a afligiam; e a rainha, vendo o sofrimento do
príncipe, lhe deu um retrato de sua filha, e pediu a ele, pelo amor
deles todos, que não se preocupasse com preparativos muito
magníficos, para demorar o menos possível para voltar. Ele respondeu:

— Jamais terei tanto prazer em lhe obedecer como nesta ocasião; meu
coração está muito envolvido para que seja negligente com aquilo que
me fará feliz. 

Ele partiu em seguida; e a princesa Moufette ficou aguardando seu
retorno ocupando-se com a música e os instrumentos que aprendera a
tocar naqueles últimos meses, e nos quais se saía maravilhosamente
bem. Um dia que ela estava tocando no quarto da rianha, o rei entrou
com o rosto coberto de lágrimas, e abraçou sua filha.

— Ah, minha filha! — exclamou. — Ah, pai desafortunado que sou! Ah,
que rei infeliz! 

Ele não conseguiu dizer mais nada: os soluços entrecortaram a sua voz.
A rainha e a princesa atônitas lhe perguntaram qual era o problema
dele; e enfim ele disse que acabara de chegar um gigante de uma
altura desmesurada, que se dizia embaixador do dragão do lago, o qual
exigia o cumprimento da promessa que o rei tinha feito em troca de
sua ajuda para combater e vencer os monstros, e queria que ele
entregasse a princesa Moufette, a qual pretendia comer numa torta. 

A rainha, ao ouvir essa triste notícia, se desfez num pranto, e
abraçou sua filha. 

— Prefiro perder a vida — ela disse — a me resolver a entregar minha
filha a esse monstro! Que ele tome nosso reino e tudo o que
possuímos! Pai desnaturado, como pode concordar com tamanha barbárie?
O quê, minha filha numa torta?!? Não quero nem pensar nisso;
tragam-me esse embaixador bárbaro; talvez minha aflição o comova. 

O rei nada respondeu; foi ter com o gigante, e em seguida o levou até
a rainha, que se atirou a seus pés, ela e sua filha implorando que
tivesse piedade delas, e convencesse o dragão a ficar com tudo que
eles tinham, e salvasse a vida de Moufette. Mas ele respondeu que não
dependia dele, que o dragão era muito teimoso e guloso; que quando
ele metia na cabeça de comer qualquer coisa, nem todos os deuses
juntos conseguiriam lhe tirar a vontade; e que lhes aconselhava a
obedecê-lo de bom grado, porque infortúnios ainda piores poderiam
lhes acontecer. Ao ouvir essas palavras a rainha desmaiou, e a
princesa teria feito o mesmo, se não tivesse que amparar sua mãe.

Mal essa triste notícia se espalhou no palácio e toda cidade já sabia,
e só se ouvia choro e lamentação, pois Moufette era adorada. O rei
não conseguia se resolver a entregar sua filha; e o gigante, que já
esperara vários dias, começava a perder a paciência, e a proferir
ameaças terríveis. Enquanto isso a rainha e o rei se diziam:

— O que de pior ainda pode nos acontecer? Nem mesmo se o dragão do
lago vier nos devorar a todos ficaremos pior, pois estaremos perdidos
se entregarmos nossa Moufette. 

Foi então que o gigante teve notícias de seu senhor: se a princesa
aceitasse se casar um sobrinho dele, o dragão consentiria em deixá-la
viver; e que, de resto, esse sobrinho era bonito e distinto, e um
príncipe, e a princesa poderia viver muito contente com ele. 

Essa proposta abrandou um pouco o sofrimento das majestades; a rainha
foi falar com a princesa, mas descobriu que esse casamento era para
ela uma possibilidade mais remota que a morte. 

— Não serei de forma alguma capaz — disse ela — de conservar minha
vida sendo infiel; vocês me prometeram ao príncipe Moufy, e eu não
serei de nenhum outro. Deixem-me morrer; o fim de minha vida trará
paz à de vocês. 

O rei veio, e disse a sua filha tudo que a ternura mais intensa pode
suscitar; mas ela se manteve firme em sua resolução. Em conclusão,
foi resolvido que ela seria conduzida ao topo de uma montanha onde o
dragão do lago viria buscá-la. 

Fizeram-se os preparativos para este triste sacrifício: nem mesmo os
de Iphigenia e os de Psiquê foram tão lugubres; só se via roupas de
luto, e rostos pálidos e consternados. Quatrocentas jovens da maior
qualidade se vestiram de túnicas brancas, e se coroaram com ramos de
cipreste para acompanhá-la; carregaram-na numa liteira aberta de
veludo negro, para que todos vissem essa maravilha dos deuses; seus
cabelos estavam soltos sobre os ombros, e ela usava um coroa feita de
jasmins, com alguns cravos no meio. Só o que parecia comovê-la era
dor do rei e da rainha, que a seguiam na mais profunda tristeza; o
gigante, armado dos pés à cabeça, ia do lado da liteira onde estava a
princesa; e como a olhava com um ar guloso, devia ter a sua parte no
repasto assegurada; ressoavam no ar os suspiros e os soluços, e o
caminho estava inundado de lágrimas. 

— Ah, Rã! — exclamou a rainha — você me abandonou! Ai, por que você me
ajudou naquele lugar sombrio, se agora nada faz? Preferia ter morrido
lá! Não posso ver todas minhas esperanças perdidas! Não posso ver
Moufette sendo devorada! 

Enquanto ela assim se remoía, o cortejo continuava a avançar, embora
lentamente; e enfim chegou ao topo da montanha fatal. Lá, o choro e
as lamentações se redobraram com tal intensidade que jamais se vira
cena tão triste; e o gigante exortou todos a fazerem suas despedidas
e se retirarem. Era só o que havia a fazer, pois naquele tempo se era
bastante simplório, e não se procurava soluções para nada.

O rei e a rainha se afastaram e subiram outra montanha com toda a
corte, pois de lá poderiam ver o que aconteceria com a princesa. E
logo eles puderam perceber no céu um dragão que tinha meia-légua de
comprimento, e que mal conseguia voar, mesmo tendo seis asas, tanto
era o peso de seu corpo, todo coberto de grossas escamas azuis, e de
espinhos envenenados; sua cauda dava cinquenta voltas e meia, cada
uma de suas garras era do tamanho de um moinho de vento, e se via em
sua boca escancarada três fileiras de dentes tão grandes quanto os de
um elefante. 

Mas enquanto ele aos poucos se aproximava, a leal Rã, montada num
gavião, voava rapidamente até o príncipe Moufy. Ela estava com seu
capuz de rosas; e embora ele estivesse trancado em seu gabinete, ela
entrou sem precisar de chave. 

— O que está fazendo aqui, enamorado infeliz? — ela disse a ele. —
Fica sonhando com as belezas de Moufette, quando ela está exposta à
pior das catástrofes? Eis uma folha de roseira; soprando nela, farei
para você um cavalo singular, você vai ver. 

Imediatamente um cavalo verde apareceu; tinha doze patas e três
cabeças; uma lançava fogo, a outra bombas, e a terceira balas de
canhão. Ela lhe deu então uma espada que media dezoito varas [1,10m],
mas era mais leve que uma pena; e o revestiu de um único diamante, o
qual o envolveu como uma roupa e, embora fosse mais duro que rocha,
era tão maleável que em nada o incomodava. 

— Parta — ela disse — corra, voe em defesa daquela que você ama; o
cavalo verde que eu lhe dei o levará até onde ela está; quando a
tiver salvo, diga a ela a parte que tive. 

— Generosa fada — exclamou o príncipe — não posso no momento
demonstrar toda minha gratidão; mas me declaro para sempre seu servo
mais fiel.

Ele montou no cavalo de três cabeças, que imediamente se pôs a galopas
com suas doze patas, e era mais rápido que três dos melhores cavalos
juntos, de modo que ele não tardou a chegar no topo da montanha, onde
viu sua amada princesa completamente sozinha, e o pavoroso dragão se
aproximando lentamente. O cavalo se pôs a lançar chamas, bombas e
balas de canhão, o que foi não pouca surpresa para o monstro: recebeu
vinte golpes desses projéteis em sua garganta, descamando-a um pouco;
e as bombas perfuraram um de seus olhos. Ele ficou furioso, e quis
atacar o príncipe; mas a espada de dezoito varas era tão boa de
manejar que o príncipe fazia o que queria com ela, enfiando aqui e
ali, ou se servindo dela como um chicote. As garras do dragão teriam
sido fatais para o príncipe, não fosse a roupa de diamante
impenetrável. 

Moufette o havia reconhecido de longe, pois o diamante que o cobria
era muito brilhante e transparente, e a apreensão em que ela ficou
foi a mais profunda de que é capaz uma jovem apaixonada; mas o rei e
a rainha começaram a sentir em seus corações alguns raios de
esperança, pois era muito extraordinária a visão de um cavalo de três
cabeças e doze patas que lançava fogo e balas, e de um príncipe
dentro de um diamante armado com uma formidável espada chegando no
momento mais necessário, e combatendo com tanto valor. O rei pôs seu
chapéu na bengala, e rainha seu lenço na ponta de um bastão, para
acenar ao príncipe, e o encorajar. Todo o séquito deles fez a mesma
coisa. Na verdade, ele nem precisava, só o seu coração e o perigo em
que estava sua amada bastavam para lhe dar toda coragem do mundo. 

Como ele teve de se esforçar! A terra estava coberta com os espinhos,
garras, chifres, asas e escamas do dragão; seu sangue escorria de mil
lugares; era todo azul, e como o do cavalo era verde, produziu-se um
matiz singular no solo. O príncipe caiu cinco vezes, e se levantou
todas elas, tratando de montar de novo em seu cavalo, e então era
fogo e disparos de canhões como jamais se vira igual. Enfim o dragão
perdeu suas forças e caiu, e o príncipe lhe deu um golpe no ventre
que lhe fez uma tremenda ferida; mas — o que não vai ser fácil de
acreditar e no entanto é tão verdadeiro quanto o resto desse conto —
dessa ferida o que saiu foi o príncipe mais bonito e mais encantador
que jamais se vira, vestido de veludo azul com bordados de ouro e
pérolas, e com um elmo a moda grega com plumas brancas. Ele correu
com os braços abertos para abraçar o príncipe Moufy. 

— Meu generoso libertador, o que eu não lhe devo! — exclamou ele —
Você acaba de me libertar da mais terrível prisão que algum soberano
já se viu encarcerado! Fui a ela condenado pela Fada Leoa, e fazia
dezesseis anos que eu padecia assim! E seu poder era tal que, mesmo
contra a minha própria vontade, ela me forçava a devorar aquela bela
princesa; leve-me até ela, para que eu possa explicar meu infortúnio.


O príncipe Moufy, surpreso e encantado com uma aventura assim tão
surpreendente, quis se mostrar não menos gentil que aquele príncipe,
e os dois se apressaram a se reunir a bela Moufette que, de sua
parte, agradecia mil vezes aos deuses por uma felicidade tão
inesperada. O rei, a rainha e toda a corte já tinham se juntado a
ela, e todo mundo falava ao mesmo tempo, ninguém se entendia, e se
chorava quase tanto de alegria quanto se havia chorado de tristeza. E
enfim, para não faltar nada à festa, a bondosa Rã apareceu dos ares,
montada num gavião com guisos nos pés. Ao ouvirem o tilintar, todos
olharam para cima, e viram brilhar o capuz de rosas como um sol, e a
Rã tão bela quanto a aurora. A rainha avançou até ela, e a pegou numa
de suas patinhas; no mesmo instante a sábia Rã se metamorfoseou e
apareceu como uma grande rainha, com o rosto mais atraente do mundo. 

— Eu vim — ela exclamou — para coroar a fidelidade da princesa
Moufette, que preferiu arriscar sua vida a renegar seu amor; um
exemplo raro no século em que estamos, mas que será ainda mais raro
nos séculos por vir.

E em seguida ela pegou duas coroas de mirtilos e as pôs sobre as
cabeças dos dois apixonados que se amavam, e dando três toques com
sua varinha, fez com que todos os ossos do dragão se erguessem para
formar um arco do triunfo, em memória da grande aventura que acabara
de acontecer. 

E então toda esse belo e numeroso bando se encaminhou para a cidade,
cantando e celebrando e se rejubilando, com tanta alegria quanto fora
a tristeza com o sacrifício da princesa. Esperou-se apenas o dia
seguinte para as bodas; e é fácil imaginar a felicidade em que se
deram.

\chapter{5. O dragão e sua avó\subtitulo{Jacob \& Wilhelm Grimm}}

Houve uma vez uma grande guerra, e o rei, que tinha muitos soldados,
pagava-os tão mal que eles não conseguiam viver de seus soldos. Três
de seus soldados então se reuniram, e decidiram desertar. Um deles
disse aos outros:

— Se nos pegarem, vão nos enforcar no cadafalso. O que vocês acham que
devemos fazer?

— Vocês estão vendo aquele campo de trigo ali? Se nos escondermos
nele, ninguém vai nos achar. O exército não pode entrar nele, e
amanhã ele vai partir.

Então eles se esconderam no campo de trigo, mas o exército não partiu;
ficou onde estava, bem perto deles. Os três soldados ficaram ali por
dois dias e duas noites, e já estavam particamente morrendo de fome;
mas arriscar-se a sair era morrer na certa. Enfim, disseram:

— De que adiantou desertar? Vamos ter uma morte miserável aqui!

Foi então que um dragão flamejante apareceu voando pelo ar. Mergulhou
no campo, e perguntou porque eles estavam escondidos ali.

— Somos três soldados — eles responderam — e desertamos porque éramos
muito mal pagos. Agora morreremos de fome se ficarmos aqui, e seremos
dependurados no cadafalso se saírmos daqui.

— Se vocês me servirem por sete anos — disse o dragão — eu os levo por
cima do exército, e não serão pegos.

— Como não temos outra escolha — eles disseram — vamos aceitar sua
oferta.

Então o dragão pegou-os em suas garras, e levou-os pelo céu sobre o
exército, e os pôs no chão de novo bem longe dali.

O dragão deu a eles um pequeno chicote e disse:

— Basta estalar no ar esse chicote, e muito dinheiro aparecerá na
frente de vocês. Vocês então poderão viver como grandes senhores,
manter cavalos, e passear por aí em carruagens. Mas depois de sete
anos vocês serão meus.

Então ele abriu um livro na frente deles, e fez cada um dos três
assinar.

— Eu então proporei uma adivinha — ele disse — e se vocês a
decifrarem, estarão livres e fora do meu poder.

O dragão saiu então voando, e eles partiram com o chicotinho. Tiveram
todo o dinheiro que queriam, se vestiram muito bem, e conseguiram um
lugar no mundo. Onde quer que fossem viviam com esplendor e muita
diversão, andavam sempre a cavalo ou em carruagens, comeram e
beberam, mas nada de mal fizeram.

O tempo passou muito rápido, e quando os sete anos estavam chegando ao
fim dois deles ficaram terrivelmente ansiosos e com medo, mas o
terceiro não se preocupou nem um pouco, dizendo:

— Não tenham medo, irmãos, não nasci ontem; vou decifrar a charada.

Eles foram para um campo, e se sentaram, e os dois ficaram
cabisbaixos. Uma velha que passava por ali perguntou porque eles
estavam tão tristes.

— Ai de nós! O que lhe adianta saber? Você não poderá nos ajudar.

— Nunca se sabe — ela respondeu. — Apenas me contem qual é o problema.

Então eles contaram a ela como tinham se tornado servos do dragão por
sete longos anos, e como ele lhes dera dinheiro em abundância, feito
amoras; mas como tinham assinado seus nomes eles eram dele, a menos
que no final dos sete anos decifrassem uma charada. A velha disse:

— Se vocês quiserem alguma ajuda, um de vocês precisa entrar na
floresta, e lá encontrar uma construção de pedras caindo aos pedaços
que parece uma pequena casa. Ele deve entrar nela, e lá achará ajuda.

Os dois melancólicos pensaram “Isso não vai nos salvar!”, e ficaram
aonde estavam. Mas o terceiro e mais animado deles se levantou e foi
para a floresta até achar a cabana de pedras. Na cabana havia uma
mulher muito velha, que era a avó do Dragão. Ela perguntou como ele
tinha chegado ali, e o que viera fazer ali. Ele contou tudo o que
acontecera, e como ela gostou dele, ficou com pena dele, e disse que
ia ajudá-lo.

Ela levantou uma enorme pedra que cobria a entrada do porão, dizendo:

— Esconda-se aí; dá para ouvir tudo o que se diz na sala. Basta ficar
quieto e não se mexer. Quando Dragão vier, eu vou perguntar qual é a
charada, pois ele me conta tudo; preste bem atenção no que ele
disser.

À meia-noite o Dragão chegou voando e pediu seu jantar. Sua avó pôs a
mesa, e serviu comida e bebida até ele ficar satisfeito, e eles
comeram e beberam juntos. Então durante a conversa ela perguntou como
tinha sido o dia dele, e quantas almas havia conquistado.

— Não tive muita sorte hoje — ele disse — mas pelo menos três soldados
já tenho garantidos.

— É mesmo? Três soldados! — ela disse. — Mas soldados são valentes. E
se eles conseguirem escapar?

— Eles são meus — respondeu o Dragão com desdém. — A charada que vou
lhes propor eles serão incapazes de decifrar.

— Que espécie de charada? — ela perguntou.

— Vou lhe dizer o seguinte. No fundo do Mar do Norte jaz um macaco
morto; será o assado deles; e uma costela de uma baleia, que será a
colher de prata deles; e um casco oco de um cavalo morto, que será o
cálice de vinho. 

Quando o Dragão foi para a cama, sua velha avó puxou a pedra e deixou
o soldado sair.

— Prestou bastante atenção em tudo?

— Sim — ele respondeu. — Sei o suficiente, e poderei me virar
esplendidamente.

Então ele saiu secretamente por uma outra janela, e apressou-se a
voltar até onde estavam seus camaradas. Contou-lhes como o Dragão
havia sido tapeado pela avó dele, e como ouvira de sua própria boca a
resposta da charada.

Todos ficaram então de excelente humor, pegaram o chicote, e estalaram
montes de dinheiro brotando do chão. Quando os sete anos acabaram, o
Dragão veio com seu livro e, apontando as assinaturas deles, disse:

— Vou levá-los para o inferno comigo, onde vocês terão uma refeição.
Se vocês me disserem qual será o assado, estarão livres e poderão
ficar com o chicote.

O primeiro soldado respondeu:

— No fundo do Mar do Norte jaz um macaco morto; esse vai ser o assado.

O Dragão ficou muito aborrecido, e resmungou bastante, e perguntou ao
segundo soldado:

— Mas qual será a colher de vocês?

— A costela de uma baleia será nossa colher de prata.

O Dragão fez uma careta, e grunhiu de novo três vezes:

— Hum, hum, hum — e perguntou ao terceiro soldado:— E você sabe qual
será o cálice de vinho de vocês?

— O casco de um velho cavalo será nosso cálice de vinho.

Então o Dragão saiu voando dando um grito agudo, e não mais tinha
poder sobre eles. E os três soldados pegaram o chicotinho,
chicotearam todo o dinheiro que queriam, e viveram felizes até o fim
de suas vidas.

\chapter{Os domesticadores de dragão\subtitulo{Edith Nesbit}}

Havia uma vez um castelo muito, muito velho — tão velho que suas
muralhas, torres, torreões, arcos e portões tinham desmoronado em
ruínas, e de todo seu esplendor antigo só restavam dois quartinhos; e
era ali que o ferreiro John instalara sua forja. Ele era muito pobre
para morar numa casa adequada, e ninguém cobrava aluguel pelos dois
quartos na ruína, porque todos os senhores do castelo tinham se ido e
morrido havia muitos anos. Então era ali que John usava seus foles e
martelava seu ferro e fazia todo serviço que aparecesse. O que não
era muito, pois a maior parte ia para o prefeito da cidade, que
também era ferreiro, mas com um negócio em grande escala, com sua
enorme forja de frente para a praça da cidade, doze aprendizes, todos
martelando como se fosse um ninho de picapaus, doze artífices [] para
mandar nos aprendizes, uma forja de primeira, um martelo automático e
foles elétricos, e tudo muito bonito e bem instalado. Então é claro
que o pessoal da cidade, quando precisava ferrar um cavalo ou
consertar um eixo, ia na forja do prefeito. O ferreiro John se virava
o melhor que podia, com um ou outro serviço que conseguia de
viajantes e forasteiros que não sabiam que a forja do prefeito era
muito superior. Os dois quartos eram quentes e protegidos das
intempéries, mas não muito grandes; então o ferreiro se acostumara a
guardar seu ferro, sua solda, o pouco de carvão que tinha e o resto
de suas tralhas no calabouço sob o castelo. Era de fato um excelente
calabouço, com um belo teto em abóbadas e enormes anéis de ferro que
ficavam presos nas paredes, muito convenientes para acorrentar
prisioneiros, e num canto um lance de escadas de largos degraus
descendo ninguém sabia para onde. Mesmo os senhores do castelo nos
bons velhos tempos nunca souberam onde aqueles degraus iam dar, mas
volta e meia eles chutavam algum prisioneiro degraus abaixo, de um
jeito despreocupado e esperançoso, e efetivamente, os prisioneiros
jamais voltavam. O ferreiro nunca ousara ir mais longe que o sétimo
degrau, e nem eu — de modo que sei tanto quanto ele o que havia lá
embaixo.

O ferreiro John tinha uma mulher e um filho ainda bebê. Quando a
mulher não estava cuidando da casa ela ficava embalando o bebê e
chorando, lembrando os dias felizes em que vivia com seu pai, que
tinha dezessete vacas e vivia muito bem no campo, e de como John
vinha fazer-lhe a corte nas noites de verão, todo elegante e vistoso,
com flores na lapela. E agora o cabelo de John estava ficando
grisalho, e quase nunca havia o suficiente para comer.

Quanto ao bebê, chorava bastante em diferentes horários; mas de noite,
quando sua mãe ia para a cama, ele sempre começava a chorar, como se
fosse a coisa certa a fazer naquela hora, de modo que ela mal
conseguia dormir. Isso a deixava muito cansada.

O bebê podia compensar suas noites ruins durante o dia, se quisesse,
mas sua pobre mãe não. Então, sempre quando ela não tinha o que fazer
ela ficava sentada chorando, porque estava esgotada de tanto
trabalhar e se preocupar.

Uma noite o ferreiro estava ocupado em sua forja. Estava fazendo uma
ferradura para o bode de uma senhora muito rica, que queria ver se o
bode ia gostar de usar ferradura, e também se a ferradura ia custar
cinco ou seis pences antes de encomendar o jogo todo. Esse era o
único serviço que John conseguira naquela semana. E enquanto ele
trabalhava sua mulher embalava o bebê que, por milagre, não estava
chorando.

Então, por cima do barulho do fole e do martelo no ferro, veio um
outro som. O ferreiro e sua mulher se entreolharam.

— Eu não ouvi nada — disse ele.

— Nem eu — disse ela. 

Mas o barulho ficou mais alto — e os dois estavam tão ansiosos para
não ouví-lo que ele martelou a ferradura do bode com mais força do
que jamais martelara na vida, e ela começou a cantar para o bebê,
algo que ela não tinha ânimo para fazer havia semanas.

Mas mesmo com o fole, o martelo e a cantoria o barulho vinha cada vez
mais alto, e quanto mais eles tentavam não ouví-lo mais o ouviam. Era
como o som de uma enorme criatura ronronando, ronronando, ronronando
— e a razão pela qual eles não queriam acreditar nele era que vinha
do grande calabouço lá embaixo, onde ficava o ferro velho, a lenha e
o pouco de carvão, e os degraus que desciam para a escuridão e iam
dar ninguém sabia aonde.

— Não pode ser nada no calabouço — disse o ferreiro, enxugando o
rosto. — Ora, vou ter que ir lá pegar mais carvão daqui a pouco.

— Claro que não tem nada lá. Como poderia ter? — disse a mulher dele.
E eles se esforçaram tanto para acreditar que não podia haver nada lá
que logo estavam quase acreditando. 

Então o ferreiro pegou sua pá com uma mão e seu martelo de rebitar com
a outra, e pendurou a velha lanterna no dedinho e foi lá embaixo
pegar o carvão.

— Não estou levando o martelo porque acho que tem alguma coisa lá, —
ele disse — mas porque ele é útil para quebrar os pedaços maiores de
carvão.

— Entendo perfeitamente — disse sua mulher, que havia trazido o carvão
em seu avental naquela mesma tarde, e sabia que não passava de pó de
carvão.

Então ele desceu a escada em caracol para o calabouço e parou no fim
dos degraus, segurando a lanterna acima de sua cabeça só para ver que
o calabouço realmente estava vazio, como sempre. Metade dele estava
vazia como sempre, exceto pelo ferro velho e o resto das tralhas, e a
lenha e o carvão. Mas o outro lado não estava vazio. Estava bem
cheio, e do que ele estava bem cheio era de dragão. 

— Deve ter subido por aqueles degraus horríveis vindo sabe-se lá de
onde — disse o ferreiro para si mesmo, tremendo todo, enquanto
tentava subir de volta pela escada em caracol sem ser visto.

Mas o dragão foi mais rápido que ele — avançou uma enorme garra e
pegou-o pela perna, e ao se mover fez um barulho feito o de um molho
de chaves gigante, ou feito o daquelas lâminas de ferro que se usam
para fazer trovões em pantominas. 

— Não, não vai não — disse o dragão numa voz cheia de pequenas
explosões, feito uma bombinha que falhou.

— Pobre de mim — disse o coitado do John, tremendo mais que nunca na
garra do dragão. — Belo fim para um respeitável ferreiro!

O dragão pareceu ficar muitíssimo impressionado com esse comentário.

— Você se importaria em repetir isso? — disse, bastante polido.

Então John repetiu, bem claramente:

— Belo-fim-para-um-respeitável-ferreiro.

— Não sabia — disse o dragão. — Ora, quem diria! Você é exatamente a
pessoa que eu queria.

— Foi o que eu entendi do que você disse antes — disse John, seus
dentes batendo.

— Oh, não, não quis dizer o que você quis dizer — disse o dragão — mas
que eu precisaria que você fizesse um serviço para mim. Numa das
minhas asas os rebites se soltaram um pouco acima da junta. Você
poderia arrumar isso?

— É possível, senhor — disse John, polidamente, porque deve-se sempre
ser polido com um cliente em potencial, mesmo se este fôr um dragão.

— Um mestre artesão... Você é um mestre, não é? Então, vai enxergar no
ato qual é o problema — o dragão continuou. — Dê a volta e venha
sentir as minhas placas, sim?

John timidamente deu a volta quando o dragão soltou-o de sua garra; e
efetivamente, a asa do dragão estava solta, e várias das placas perto
da junta precisavam ser rebitadas. 

O dragão parecia ser quase inteiro feito de uma armadura de ferro — de
um tom fulvo, vermelho-ferrugem na cor; da umidade, sem dúvida — e
por baixo parecia ser coberto de algo peludo. 

O ferreiro dentro de John despertou, e ele se sentiu mais à vontade. 

— Um ou dois rebites vão fazer bem ao senhor, com certeza — ele disse.
— Na verdade, o senhor precisa é de muitos rebites.

— Bom, ao trabalho, então — disse o dragão. — Você conserta minha asa
e aí eu saio para comer toda a cidade, e se você tiver feito um
serviço realmente bom, eu o deixo para ser comido por último. Que
tal?

— Senhor, eu não quero ser o último a ser comido — disse John.

— Por isso não, você então será o primeiro a ser comido — disse o
dragão.

— Eu também não quero isso, senhor — disse John.

— Vá se danar então, seu homem tolo — disse o dragão. — Tão tolo que
nem sabe o que quer. Vamos, comece a trabalhar.

— Eu não gostei desse serviço, senhor, — disse John — essa é a
verdade. Sei como é fácil acidentes acontecerem. Está tudo muito bem,
tudo muito certo: “por favor me rebite, e eu o comerei por último”; e
daí você começa o serviço e dá um cutucão ou um beliscão debaixo dos
rebites e pronto, é só fogo e fumaça, e nenhuma desculpa vai
adiantar. 

— Eu dou minha palavra de honra de dragão — disse o próprio.

— Eu sei que o senhor não vai fazer de propósito — disse John — mas
qualquer cavalheiro é capaz de dar um pulo e uma fungada ao ser
cutucado, e uma fungada sua basta para acabar comigo. Agora, se o
senhor me deixasse prendê-lo?

— Seria tão pouco digno — objetou o dragão.

— Sempre se prende um cavalo — disse John — e ele é o “nobre animal”. 

— Sim, muito bem — disse o dragão — mas como vou saber se você vai me
soltar depois de me rebitar? Dê-me algo como garantia. O que tem mais
valor para você? 

— Meu martelo — disse John. — Um ferreiro sem martelo não é nada. 

— Mas você vai precisar dele para me rebitar. Você precisa pensar em
outra coisa, e rápido, senão eu vou comê-lo primeiro. 

Nesse momento o bebê no quarto lá em cima começou a berrar. A mãe dele
tinha ficado tão quieta que ele achou que ela tinha ido para a cama,
e que era hora de começar. 

— O que que é isso? — disse o dragão, com um sobressalto em que todas
as placas de seu corpo fizeram barulho. 

— É só o bebê — disse John.

— E o que é isso? — perguntou o dragão. — Algo que tem valor para
você?

— Bem, sim, senhor, bastante — disse o fereiro.

— Então traga-o aqui — disse o dragão. — Vou ficar com ele até que
você termine de me rebitar, e assim você poderá me amarrar. 

— Certo, senhor — disse John — mas devo avisá-lo. Bebês são veneno
para dragões, então não vou enganá-lo. Tudo bem tocá-lo, mas não vá
pô-lo na boca. Detestaria ver um cavalheiro tão distinto como o
senhor passando mal.

O dragão ronronou com esse elogio e disse:

— Está certo, tomarei cuidado. Agora vá buscar a coisa, o que quer que
seja.

Então John correu degraus acima o mais rápido que pôde, porque sabia
que se o dragão ficasse impaciente antes de estar preso, bem podia
derrubar o teto do calabouço numa só levantada de suas costas,
matando todos nas ruínas. Sua mulher estava dormindo, apesar do choro
do bebê; e John o pegou, levou-o para baixo e colocou-o entre as
garras dianteiras do dragão.

— Basta o senhor ronronar para ele, — disse — e ele vai ficar como um
anjo.

O dragão começou a ronronar, o que agradou tanto ao bebê que ele parou
de chorar.

Então John remexeu na pilha de ferro velho até achar nela uma grossas
correntes e uma enorme coleira feita nos tempos em que os homens
cantavam ao trabalhar e o faziam de todo o coração, de modo que as
coisas que faziam eram resistentes o bastante para aguentar o peso de
mil anos; e perto disso, um dragão não é nada. 

John prendeu o dragão com a coleira e as correntes, e quando tinha
fechado bem os cadeados nelas, começou seu trabalho vendo quantos
rebites iam ser necessários.

— Seis, oito, dez... vinte, quarenta — disse ele. — Não tenho nem
metade dos rebites necessários na minha oficina. Se o senhor me der
licença, vou dar um pulo numa outra forja para pegar algumas dúzias.
Não vou demorar nem um minuto.

E lá se foi ele, deixando o bebê entre as garras dianteiras do dragão,
rindo e dando gritinhos de prazer com o enorme ronronar dele.

John correu o mais rápido que pôde para a cidade, e procurou o
prefeito e as demais autoridades municipais. 

— Há um dragão em meu calabouço — disse. — Eu o acorrentei. Agora
preciso da ajuda dos senhores para tirar meu bebê dele. 

E ele contou a eles toda a história.

Mas acontece que todos tinham compromissos naquela noite; então eles
elogiaram a esperteza dele, e disseram-se muito satisfeitos em deixar
o assunto nas mãos dele.

— Mas e o meu bebê? — disse John.

— Ah, sim — disse o prefeito. — Se alguma coisa acontecer, você sempre
poderá lembrar que seu bebê morreu por uma causa nobre. 

Então John voltou para casa, e contou parte da história para sua
mulher. 

— Você deu o bebê para o dragão! — ela exclamou. — Ah, seu pai
desnaturado!

— Ssh — disse John, e contou mais um pouco. E aí disse:— Eu vou
descer. Depois que eu estiver lá embaixo você vem, e se você manter a
cabeça no lugar o menino vai ficar bem.

E para baixo foi o ferreiro, e lá estava o dragão ronronando com toda
a força para manter o bebê quieto.

— Vê se você se apressa — ele disse. — Não vou conseguir ficar fazendo
todo esse barulho a noite inteira.

— Senhor, eu sinto muito — disse o ferreiro — mas todas as oficinas
estão fechadas. O serviço vai ter que ficar para amanhã. E não se
esqueça que o senhor prometeu se encarregar do bebê. Receio que
talvez ache um pouco cansativo. Boa noite.

O dragão ronronou até ficar completamente sem fôlego — e aí ele teve
que parar, e assim que tudo ficou em silêncio o bebê achou que todo
mundo já tinha ido dormir, então já estava na hora de começar. E
começou.

— Oh não — disse o dragão. — Isso é terrível. 

Ele tentou afagar o bebê com sua garra, mas isso só o fez chorar mais
que nunca.

— E eu estou tão cansado — disse o dragão. — Queria tanto poder ter
uma boa noite de sono.

O bebê continuou berrando.

— Eu nunca mais vou ter paz depois disso — disse o dragão. — É o
bastante para arruinar os nervos de qualquer um. Quieto, assim,
quietinho, isso.

E ele tentou fazer o bebê parar de chorar como se fosse um filhote de
dragão. Mas quando ele começou a cantar “Dorme meu dragãozinho” o
bebê berrou mais e mais e mais.

— Não consigo fazê-lo parar — o dragão disse; e de repente ele viu uma
mulher sentada na escada.

— Ei, escute aqui — ele disse — você entende alguma coisa de bebês?

— Sim, um pouco — disse a mãe.

— Então gostaria que você pegasse este, para eu poder dormir um pouco
— disse o dragão, bocejando. — Você pode trazê-lo de volta amanhã
antes do ferreiro chegar.

Então a mãe pegou o bebê e o levou para cima e contou para o marido
dela, e eles foram felizes para a cama, pois tinham pego o dragão e
salvo o bebê.

No dia seguinte John desceu ao calabouço e explicou direitinho para o
dragão em que pé as coisas estavam, e então pegou um portão de fero
com uma grade e o instalou no fim da escada, e o dragão miou
furiosamente por dias e dias, mas quando descobriu que não adiantava
nada ficou quieto.

John foi então falar com o prefeito, e disse:

— Eu prendi o dragão e salvei a cidade.

— Nobre preservador da cidade — exclamou o prefeito — faremos uma
subscrição pública em seu nome e o coroaremos em público com uma
coroa de louros.

O prefeito então contribuiu com cinco libras, e cada uma das demais
autoridades municipais deu três, e outras pessoas deram guinéus e
coroas, e enquanto a arrecadação estava sendo feita o prefeito
encomendou, pagando do próprio bolso, três poemas ao poeta municipal
para celebrarem a ocasião. Os poemas foram muito mais admirados,
especialmente pelo prefeito e pelas demais autoridades municipais.

O primeiro poema abordava a nobre conduta do prefeito ao conseguir que
o dragão fosse preso. O segundo descrevia a esplêndida ajuda que lhe
prestaram as demais autoridades municipais. E o terceiro expressava o
orgulho e a alegria do poeta em ter a honra de louvar tais feitos,
perto dos quais as ações de São Jorge ficam parecendo perfeitamente
banais para quem tem o coração sincero e a mente bem-equilibrada.

Quando a subscrição terminou, mil libras haviam sido arrecadadas, e
formou-se um comitê para decidir o que fazer com elas. Um terço foi
para pagar um banquete para o prefeito e as demais autoridades
municipais; outro terço foi gasto na compra de uma corrente de ouro
com um dragão para o prefeito, e com medalhas de ouro com dragões
para as demais autoridades municipais; e o que sobrou serviu para
pagar as despesas do comitê.

De modo que para o ferreiro não ficou nada, exceto a coroa de louros e
saber que quem realmente salvara a cidade tinha sido ele. Mas depois
disso as coisas melhoraram um pouco para o ferreiro. Para começar, o
bebê já não chorava mais tanto quanto antes. Aí, a senhora rica que
tinha o bode ficou tão comovida com a nobre ação de John que
encomendou um jogo completo de ferraduras por 2 shillings e 4 pence,
e até pagou 2 shillings e 6 pence em grato reconhecimento ao espiríto
público da conduta do ferreiro. E então turistas passaram a vir em
excursões de muito longe, e pagavam dois pence cada para descer a
escada em caracol e espiar pela grade de ferro o dragão enferrujado
no calabouço — e para cada grupo eram mais três pence se quisessem
vê-lo sob a luz de um fogo de artifício, o qual sendo de duração
extremamente curta dava um lucro líquido de dois pence e meio toda
vez. E a mulher do ferreiro começou a servir chá completo por nove
pence cada, e com tudo isso as coisas foram melhorando a cada semana.

O bebê — batizado de John como seu pai, e apelidado de Johnnie — foi
crescendo. Ele ficou muito amigo de Tina, a filha do funileiro que
morava quase ali na frente. Ela era uma menina adorável de olhos
azuis e rabo-de-cavalo louro, e estava enjoada de ouvir a história de
como Johnnie, quando bebê, ficara aos cuidados de um dragão de
verdade.

As duas crianças costumavam ir juntas espiar pela grade de ferro o
dragão, e às vezes o ouviam miar miseravelmente. E eles acendiam um
fogo de artifício de meio pence para olhá-lo. E eles foram ficando
maiores e mais sabidos.

Até que um dia o prefeito e as demais autoridades municipais, que
tinham ido caçar lebres em seus mantos dourados, voltaram correndo e
gritando para a cidade, com a notícia de que um gigante manco,
corcunda e tão grande quanto uma igreja estava vindo pelos pântanos
para a cidade. 

— Estamos perdidos. — disse o prefeito. — Darei mil libras para quem
quer que consiga manter o gigante longe da cidade. Sei o que ele
come, pelos seus dentes.

Ninguém parecia saber o que fazer. Mas Johnnie e Tina estavam ouvindo,
e eles se entreolharam, e saíram correndo o mais rápido que podiam. 

Passaram correndo pela forja, e desceram correndo os degraus até o
calabouço, e bateram na porta de ferro. 

— Quem é? — perguntou o dragão.

— Somos só nós — disseram as crianças.

E o dragão estava tão cheio de ter passado dez anos sozinho que disse:

— Entrem, meus queridos.

— Você não vai machucar a gente, ou cuspir fogo na gente, ou algo
assim? — Tina perguntou. E o dragão respondeu:

— Por nada neste mundo.

Então eles entraram e falaram com ele, e lhe disseram como o tempo
estava lá fora, e o que havia nos jornais, e por fim Johnnie disse:

— Tem um gigante manco na cidade. Ele veio atrás de você.

— É mesmo? — disse o dragão, mostrando os dentes. — Ah, se ao menos eu
não estivesse preso!

— Se nós o soltarmos você poderia fugir antes que ele conseguisse
pegá-lo.

— É, eu poderia fugir — respondeu o dragão. — Mas talvez não fugisse.

— O quê, você não iria lutar com ele, iria? — disse Tina.

— Não — disse o dragão. — Sou da paz, sou sim. Vocês me soltem, que
vocês verão. 

Então as crianças tiraram as correntes e a coleira do dragão, e ele
derrubou uma das paredes do calabouço e saiu; só deu uma parada na
porta da forja para o ferreiro rebitar sua asa.

Ele encontrou o gigante manco no portão da cidade, e o gigante
martelou o dragão com sua clava como se estivesse martelando ferro
fundido, e o dragão se comportou como tal, todo fogo e fumaça. Era um
espetáculo aterrador, e as pessoas ficaram olhando de longe, caindo
sentadas com o choque de cada golpe, mas sempre se levantando para
ver o resto.

Por fim o dragão venceu, e o gigante escapuliu pelos pântanos, e o
dragão, que ficara muito cansado, foi para casa dormir, anunciando
sua intenção de comer a cidade inteira na manhã seguinte. Ele voltou
para o seu velho calabouço porque era um forasteiro na cidade, e não
sabia de nenhuma outra acomodação respeitável. Então Tina e Johnnie
foram falar com o prefeito e as demais autoridades municipais, e
disseram:

— O caso do gigante está encerrado. Por favor, dêem-nos a recompensa
de mil libras.

Mas o prefeito disse:

— Não, não, meu menino. Não foram vocês que encerraram o caso do
gigante, foi o dragão. Suponho que vocês o acorrentaram de novo?
Quando ele vier buscar a recompensa ele a terá. 

— Ele ainda não está acorrentado — Johnnie disse. — Devo mandá-lo
buscar a recompensa?

Mas o prefeito disse que ele não precisava se incomodar; e então
ofereceu mil libras a quem quer que acorrentasse de novo o dragão.

— Eu não confio no senhor — disse Johnnie. — Sei como o senhor tratou
meu pai quando ele acorrentou o dragão.

Só que as pessoas que estavam escutando atrás da porta interromperam,
e disseram que se Johnnie conseguisse prender de novo o dragão eles
iriam depor o prefeito e deixar Johnnie ser prefeito no lugar dele.
Pois andavam insatisfeitos com o prefeito já há algum tempo, e uma
mudança não seria nada má. Então Johnnie disse:

— Feito — e saiu de mãos dadas com Tina. Eles chamaram todos os
amiguinhos deles, e disseram:

— Vocês nos ajudam a salvar a cidade?

E todas as crianças disseram:

— Sim, claro que ajudamos. Vai ser divertido!

— Bom, então — disse Tina — vocês tem que trazer suas tigelas de pão
com leite para a forja amanhã na hora do café da manhã.

— E se eu virar prefeito — disse Johnnie — darei um banquete, e
convidarei todos vocês. E do começo ao fim só serão servidos doces.

Todas as crianças prometeram, e na manhã seguinte Tina e Johnnie
fizeram uma enorme banheira rolar escada abaixo para o calabouço.

— Que barulho é esse? — perguntou o dragão?

— E só a respiração de um gigante enorme — disse Tina. — Ele já foi
embora.

Então, quando todas as crianças da cidade trouxeram suas tigelas de
pão com leite, Tina derramou-as na banheira, e quando ela estava
cheia Tina bateu na porta com a grade de ferro e disse:

— Podemos entrar?

— Oh, sim — disse o dragão. — Está muito chato aqui.

Eles entraram, e com a ajuda de nove outras crianças carregaram a
banheira e a puseram perto do dragão. Então todas as outras crianças
foram embora, e Tina e Johnnie se sentaram e começaram a chorar.

— O que é isso? — perguntou o dragão. — E qual é o problema?

— Isso é pão com leite — disse Johnnie. — É nosso café da manhã, todo
ele.

— Bom — disse o dragão — não vejo porque vocês precisam de café da
manhã. Eu vou comer todo mundo na cidade assim que acabar de
descansar.

— Caro senhor dragão — disse Tina — eu gostaria que você não nos
comesse. Você gostaria de ser comido?

— De jeito nenhum — o dragão confessou — mas ninguém vai me comer.

— Não sei não — disse Johnnie. — Tem esse gigante...

— Eu sei. Eu briguei com ele, e dei uma surra nele.

— É, mas apareceu um outro. Aquele com que você brigou era só o
filhinho dele. Esse é o dobro daquele.

— É sete vezes o tamanho daquele — disse Tina.

— Não, nove vezes — disse Johnnie. — Ele é maior que o campanário da
igreja.

— Ora essa — disse o dragão. — Por essa eu não esperava.

— E o prefeito contou a ele onde poderia encontrá-lo — Tina continuou
— e ele estará vindo para comê-lo assim que terminar de afiar seu
facão. O prefeito disse a ele que você era um dragão selvagem, mas
ele não se importou. Disse que só comia dragões selvagens; e com
molho de pão.

— Isso é muito chato — disse o dragão. — E suponho que essa meleca
gosmenta na banheira seja o molho de pão?

As crianças disseram que era. E acrescentaram:

— Claro, molho de pão só acompanha dragões selvagens. Dragões
domesticados são servidos com molho de maçã e recheio de cebola. Que
pena que você não é um; aí o gigante não ia nem querer saber de você
— eles disseram. — Adeus, pobre dragão, nunca mais o veremos, e agora
você vai ficar sabendo como é ser comido.

E começaram a chorar de novo.

— Sei, mas... escutem aqui — o dragão disse. — Vocês não podiam fingir
que eu era um dragão domesticado? Digam ao gigante que eu sou só um
pobre dragãozinho tímido que vocês tem como animal de estimação.

— Ele nunca vai acreditar — disse Johnnie. — Se você fosse nosso
dragão domesticado nós o manteríamos preso, sabe. Não iríamos nos
arriscar a perder um animalzinho de estimação tão querido e
bonitinho.

Então o dragão implorou a eles que o prendessem no mesmo instante, e
eles o fizeram; com as correntes e a coleira feitas anos atrás, nos
tempos em que os homens cantavam ao trabalhar, e resistentes o
bastante para aguentar qualquer coisa.

E as crianças foram contar para o povo o que tinham feito, e Johnnie
foi feito prefeito, e deu um glorioso banquete exatamente como
dissera que ia fazer: com nada a não ser doces. Começou com balas de
geléia e sonhos, e prosseguiu com laranjas, maria-moles, cocadas,
folhados de geléia, tortas de framboesa, sorvetes, e suspiros, e
terminou com bombas de chocolate, pão de mel e [acid drops].

Tudo isso estava muito bom para Johnnie e Tina; mas se você é uma
criança de bom coração, talvez tenha ficado com pena do pobre dragão,
enganado e tapeado, e acorrentado num calabouço sem graça sem nada
para fazer a não ser ficar pensando o tempo todo nas falsidades
chocantes que Johnnie lhe dissera.

Quando ele se deu conta de como tinha sido tapeado, o pobre dragão
prisioneiro começou a chorar, e suas enormes lágrimas caíram sobre
suas placas enferrujadas. E ele começou a se sentir fraco, como às
vezes acontece com as pessoas que ficaram chorando, especialmente se
não tiverem comido nada durante dez anos.

Então a pobre criatura enxugou os olhos e olhou em volta, e viu ali a
banheira cheia de pão com leite. Aí, pensou: “se gigantes gostam
dessa coisa mole e branca, talvez eu goste também”, e experimentou um
pouco; e gostou tanto que comeu tudo.

E quando veio a próxima leva de turistas, e Johnnie acendeu o fogo de
artifício, o dragão disse timidamente:

— Desculpe incomodá-lo, mas você poderia me trazer mais pão com leite?

Então Johnnie providenciou gente para percorrer todo dia a cidade para
recolher o pão com leite das crianças para o dragão. As crianças eram
alimentadas às custas da cidade — e com o que quisessem; de modo que
elas só comiam tortas e bolinhos e doces, e disseram que o pobre
dragão podia se servir à vontade de todo o pão com leite deles.

Quando Johnnie já tinha sido prefeito durante dez anos, ele se casou
com Tina, e na manhã de seu casamento eles foram visitar o dragão.
Ele tinha se tornado bastante manso, e suas placas enferrujadas
tinham caído em vários lugares, e debaixo era macio e peludo de
afagar. Então eles o afagaram. E ele disse:

— Não sei como eu alguma vez pude gostar de comer qualquer outra coisa
que não fosse pão com leite. Agora sou um dragão domesticado, não
sou?

 E quando eles disseram que sim, o dragão falou:

— E já que fiquei domesticado, vocês podiam me soltar, não?

Algumas pessoas receariam em confiar nele, mas Johnnie e Tina estavam
tão contentes no dia de seu casamento que jamais acreditariam que
alguem fosse capaz de fazer algo ruim. Então eles soltaram as
correntes, e o dragão disse:

— Com licença um momento, há duas ou três coisinhas que eu gostaria de
ir buscar — e ele foi até aqueles degraus misteriosos e desceu por
eles, desparecendo dentro da escuridão. E quando ele se movia, mais e
mais das suas placas enferrujadas caíam.

Poucos minutos depois eles ouviram o barulho dele subindo os degraus.
Tinha trazido alguma coisa na boca — era um saco de ouro.

— Para mim não serve para nada — ele disse. — Talvez vocês achem útil.

E eles lhe agradeceram muito gentilmente.

— Tem mais donde esse veio — disse ele, e trouxe mais, e mais, e mais,
até eles falarem para ele parar. Eles agora eram ricos, e seus pais e
suas mães também. De fato, todo mundo ficou rico, nunca mais houve
gente pobre na cidade. E todos ficaram ricos sem trabalhar, o que não
é nada certo; mas o dragão nunca tinha ido na escola, como você, de
modo que ele não podia saber.

Quando o dragão saiu do calabouço, seguindo Johnnie e Tina para o azul
e dourado brilhante que estava o dia do casamento deles, ele piscou
os olhos como um gato faz com a luz do sol, e se sacudiu, e o resto
das suas placas caiu, e as asas junto com elas, e ele ficou sendo só
um gato de um tamanho muito, muito enorme. E daquele dia em diante
ele foi ficando mais e mais peludo, e ele foi o começo de todos os
gatos que existem. Nada do que era dragão nele ficou, a não ser as
garras, que todos os gatos ainda tem, como você pode facilmente
verificar.

E eu espero que você tenha percebido o quanto é importante alimentar
seu gato com pão com leite. Se ele ficar sem nada para comer a não
ser ratos e passarinhos, ele bem pode se tornar maior e mais feroz, e
mais escamoso e mais caudoso, e arranjar asas e virar o começo de
todos os dragões. E daí vai haver todo aquele aborrecimento de novo.


\chapter{O último dragão\subtitulo{Edith Nesbit}}

Claro que você sabe que houve uma época em que dragões eram tão comuns
quanto ônibus são hoje, e quase tão perigosos. Mas como se esperava
de todo príncipe que tivera uma boa educação que matasse um dragão e
salvasse uma princesa, começou a haver cada vez menos dragões, até
ficar frequentemente bem difícil para uma princesa achar um dragão do
qual ser salva. E por fim não havia mais dragões na França, nem mais
dragões na Alemanha, ou na Espanha, ou na Itália, ou na Rússia.
Alguns sobraram na China, e ainda estão lá, mas são frios e de
bronze, e nunca houve nenhum, claro, na América. Mas o último dragão
de verdade vivo que sobrou estava na Inglaterra, e claro que isso foi
há muito tempo atrás, antes do que se chama História da Inglaterra
ter começado. Esse dragão vivia na Cornualha em cavernas enormes no
meio dos rochedos, e era um excelente e enorme dragão, com nada menos
que vinte e um metros da ponta de seu temível focinho ao fim de sua
terrível cauda. Exalava fogo e fumaça, e fazia barulho ao andar, pois
suas escamas eram feitas de ferro. Suas asas eram iguais a meios
guarda-chuvas — ou como asas de morcego, só que milhares de vezes
maiores. Todo mundo tinha muito medo dele, e com razão.

Acontece que o rei da Cornualha tinha uma filha, e claro que quando
ela fizesse dezesseis anos teria que ir enfrentar o dragão. Essas
histórias sempre se contam ao anoitecer nos quartos das crianças das
famílias reais, de modo que a Princesa sabia o que lhe esperava. O
dragão não iria comê-la, claro — porque o príncipe ia chegar para
salvá-la. Mas a Princesa não conseguia deixar de pensar que seria
muito mais agradável nada ter a ver com o dragão, de jeito nenhum —
nem mesmo ser salva dele.

— Todos os príncipes que eu conheço são uns menininhos tão
completamente bobos — ela disse a seu pai. — Por que eu tenho de ser
salva por um príncipe?

— É como sempre se faz, minha querida — disse o Rei, tirando a coroa e
pondo-a na grama, pois estavam sozinhos no jardim, e mesmo reis
precisam relaxar às vezes.

— Papai querido, — disse a Princesa, quando terminou de fazer uma
coroa de margaridas e a pôs na cabeça do Rei, onde devia estar a
outra. — Papai querido, não podíamos amarrar um dos principezinhos
bobos para o dragão ir comer, e então seria eu a matar o dragão e
salvar o Príncipe? Eu esgrimo muito melhor que qualquer um dos
príncipes que conhecemos. 

— Que idéia inapropriada para uma dama! — disse o Rei, e pôs de volta
a coroa, ao ver o Primeiro Ministro vindo com uma cesta de Decretos
novinhos para ele assinar. — Tire essa idéia da cabeça, minha filha.
Eu salvei sua mãe de um dragão, e você não pretende se achar melhor
que ela, espero?

— Mas esse é o último dragão. É diferente de todos os outros.

— Como assim? — perguntou o Rei.

— Porque ele é o último — disse a Princesa, e foi para sua aula de
esgrima, a qual ela levava muito a sério. Ela levava a sério todas as
aulas dela; porque não conseguia desistir da idéia de lutar com o
dragão. Ela as levou tão a sério que se tornou a princesa mais forte,
corajosa, hábil e sensata da Europa. A mais bonita e a mais simpática
ela sempre fora. 

E os dias e os anos se passaram, até que enfim chegou a véspera do dia
em que a Princesa era para ser salva do dragão. O príncipe
encarregado dessa valorosa façanha era um príncipe pálido, de olhos
grandes e com a cabeça cheia de matemática e filosofia, mas que
infelizmente negligenciara suas aulas de esgrima. Ele ia passar a
noite no palácio, e houve um banquete.

Depois do jantar a Princesa mandou seu papagaio de estimação para o
príncipe com um bilhete. Dizia: 

“Por favor, Príncipe, venha até o terraço. Quero falar com você sem
ninguém ouvindo. — A Princesa.”

E lá foi ele, claro — e viu de longe o vestido prateado dela brilhando
entre as sombras das árvores, como água sob a luz das estrelas. E
quando chegou bem perto, disse:

— Princesa, a seu serviço — e dobrou seu joelho coberto de
tecido-bordado-em-ouro e pôs a mão em seu coração coberto de
tecido-bordado-em-ouro.

— Você acha — disse a Princesa sinceramente — que você vai ser capaz
de matar o dragão?

— Matarei o dragão — disse o Príncipe firmemente — ou sucumbirei
tentando. 

— Você sucumbir não vai servir para nada — disse a Princesa. 

— É o mínimo que posso fazer — disse o Príncipe.

— O meu medo é que acabe sendo o máximo que você vai poder fazer —
disse a Princesa.

— É a única coisa que posso fazer — disse ele — a menos que eu mate o
dragão.

— Por que você deveria fazer alguma coisa por mim é o que eu não
entendo — ela disse.

— Mas eu quero — ele disse. — Você precisa saber que eu a amo mais do
que qualquer outra coisa no mundo. 

Ao dizer isso ele pareceu tão gentil que a Princesa começou a gostar
um pouquinho dele.

— Escute aqui, — ela disse — ninguém mais vai sair amanhã. Você sabe
que eles me amarram numa rocha, me deixam lá; e então todo mundo sai
em disparada para casa e fecha as venezianas e as deixa fechadas até
você entrar na cidade triunfante, gritando que matou o dragão, e eu
vou atrás de você no cavalo chorando de alegria.

— Ouvi dizer que é assim que se faz — ele disse.

— Bom, você me ama o suficiente para vir muito rápido e me soltar, e
então lutamos juntos contra o dragão?

— Não seria seguro para você.

— Muito mais seguro para nós dois se eu estiver solta com uma espada
na mão do que amarrada e indefesa. Concorda, vai.

Ele não podia negar nada a ela. Então ele concordou. E no dia seguinte
tudo aconteceu como ela dissera. 

Quando ele cortou as cordas que a amarravam à rocha, os dois ficaram
na montanha deserta olhando um para o outro.

— Tenho a impressão — o Príncipe disse — que essa cerimônia podia ter
sido arranjada sem o dragão.

— Sim, — disse a Princesa — mas como ela foi arranjada com o dragão...

— É uma pena matar o dragão, o último do mundo — disse o Príncipe.

— Bom, então, não vamos — disse a Princesa. — Vamos ensiná-lo a em vez
de comer princesas, comer na nossa mão. Dizem que tudo pode ser
amansado através da bondade.

— Amansá-lo assim implica em dar-lhe algo para comer — disse o
Príncipe. — Você tem alguma coisa para comer?

Ela não tinha, mas o príncipe admitiu que tinha alguns biscoitos. 

— O café da manhã foi tão cedo — ele disse — que eu achei que você ia
ficar com fome depois da luta.

— Bem pensado — disse a princesa, e eles pegaram um biscoito com cada
mão. E olharam aqui e olharam acolá, mas nada de dragão à vista.

— Mas eis o rastro dele — disse o Príncipe, apontando onde a rocha
estava marcada e arranhada de modo a fazer uma pista até a entrada de
uma caverna escura. Era como as marcas de uma carroça numa estrada de
Sussex, misturadas com as pegadas de uma gaivota na areia da praia. —
Veja, aqui é ele arrastando sua cauda de metal e essas são as marcas
de suas garras de aço.

— Melhor não pensar quão duras são a cauda e as garras dele — disse a
Princesa — senão vou começar a ficar com medo, e você sabe que não dá
para amansar nada, não importa a sua bondade, se você estiver com
medo. Vamos. É agora ou nunca.

Ela pegou o Príncipe pela mão e os dois correram pela pista que levava
à entrada escura da caverna. Mas não correram até dentro dela. Era de
fato muito escura demais.

Então eles ficaram do lado de fora, e o Príncipe gritou: 

— Ó de casa! Ei, dragão! Tem alguém em casa?

E da caverna eles ouviram uma voz respondendo e um monte de estrondos
e estalos. Soava como se um cotonifício estivesse se espreguiçando e
se levantando de seu sono.

O Príncipe e a Princesa tremeram, mas fiacaram ali firmes.

— Dragão, Dragão! — disse a Princesa — Por favor saia e fale com a
gente. Trouxemos um presente para você.

— Ah, sei; conheço os presentes de vocês — grunhiu o dragão com uma
voz trovejante. — Uma dessas preciosas princesas, eu suponho? E eu
tenho que sair e lutar por ela. Bom, vou logo lhes dizendo, eu não
pretendo fazer isso. A um combate justo eu não diria não, uma luta
justa e sem favorecimentos; mas essas lutas arranjadas que você tem
de perder, não. É o que eu lhes digo. Se eu quisesse uma princesa, eu
iria atrás de uma, quando me desse vontade; mas eu não quero. O que
vocês acham que eu iria fazer com uma, se eu pegasse uma?

— Comê-la, ...não? — a Princesa disse com um ligeiro tremor na voz. 

— Eca! Comer uma gororoba dessas? — disse o dragão rudemente. — Eu nem
tocaria na horrorosa coisa. 

A voz da Princesa ficou mais firme.

— Você gosta de biscoitos? — ela perguntou.

— Não — grunhiu o dragão.

— Nem mesmo aqueles caros com cobertura de chocolate?

— Não — grunhiu o dragão.

— Mas então do que você gosta? — perguntou o Príncipe. 

— Vão embora e não me encham mais — grunhiu o dragão, e eles puderam
ouvir ele voltando para dentro, e os rangidos e estalos dele virando
ecoaram pela caverna como o barulho dos martelos a vapor [?] do
arsenal de Woolwich.

O Príncipe e a Princesa olharam um para o outro. O que eles iam fazer?
Claro que não ia adiantar voltar para casa e dizer ao Rei que o
dragão não queria princesas — porque Sua Majestade era muito
antiquada e jamais iria acreditar que um dragão moderno pudesse de
alguma forma ser diferente de um dragão à moda antiga. Eles não
podiam entrar na caverna e matar o dragão. De fato, a menos que ele
atacasse a Princesa não ia ser nada justo matá-lo.

— De alguma coisa ele deve gostar — a Princesa sussurrou, e chamou-o
de novo com uma voz tão doce quanto mel e cana-de-açúcar:— Dragão!
Dragãzinho querido!

— O quê? — berrou o dragão. — Repita isso!

E eles oviram o dragão vindo na direção deles no escuro da caverna. A
princesa teve um arrepio, e disse num fiapo de voz:

— Dragão! Dragãozinho querido!

E então o dragão saiu. O Príncipe puxou sua espada e a Princesa a dela
— a bonita espada com cabo de prata que o Príncipe trouxera em seu
automóvel. Mas eles não atacaram; recuaram lentamente enquanto o
dragão saía, com todo seu imenso e escamoso tamanho, e se estendia
sobre a rocha, suas enormes asas meio abertas e os reflexos prateados
dele resplandescendo feio diamantes no sol. Por fim não tinham mais
para onde recuar — o rochedo escuro atrás dele barrava a passagem — e
com as costas contra a pedra eles ficaram com as espadas em punho,
esperando.

O dragão foi chegando cada vez mais perto, e agora eles podiam ver que
ele não estava exalando fogo e fumaça como esperavam; veio rastejando
lentamente na direção deles, meneando um pouco a cabeça feito um
filhote de cachorro que quer brincar mas não tem muita certeza se
você não está bravo com ele.

E então eles viram que grandes lágrimas escorriam nas bochechas
metálicas dele.

— Qual é o problema? — disse o Príncipe

— Ninguém — o dragão soluçou — jamais me chamara de “querido” antes.

— Não chore, querido dragão — disse a Princesa. — Vamos chamá-lo de
“querido” o quanto você quiser. Queremos amansá-lo.

— Eu sou manso — disse o dragão. — Essa é a questão. É o que ninguém a
não ser vocês jamais descobriu. Sou tão manso que comeria de suas
mãos.

— Comeria o que, dragão querido? — disse a princesa. — Biscoitos,
talvez?

O dragão balançou devagar sua enorme cabeça.

— Nada de biscoitos — disse a princesa carinhosamente. — O que então,
querido dragão?

— Sua gentileza me desdragona um bocado — ele disse. — Ninguém jamais
perguntou a nenhum de nós o que gostaríamos de comer. Sempre nos
oferecendo princesas, e aí as salvando, e nem uma vez “o que você
gostaria de beber para brindar à saúde do Rei?” Cruel e injusto, é o
que acho — e começou a chorar de novo.

— Mas o que você gostaria de beber para brindar à nossa saúde? — disse
o Príncipe. — Nós vamos nos casar hoje, não vamos, Princesa?

Ela disse que supunha que sim.

— O que eu gostaria de beber à sua saúde? — perguntou o dragão. — Ah,
o senhor é decididamente um cavalheiro, é sim. Faço questão de
dizê-lo, sim senhor. Terei orgulho de beber a sua saúde e a da sua
bondosa dama só um pequeno golinho de — a voz dele falhou. 

— ...e pensar que o senhor me pergunta assim todo amistoso — continuou
ele. — Sim, senhor, só um pequeno golinho de ga-ga-ga-ga-ga-gasolina,
isso que... faria bem a um dragão, senhor...

— Tenho de monte no carro — disse o Príncipe, e saiu descendo a
montanha feito um relâmpago. Era bom em julgar caráter, e sabia que
com aquele dragão a Princesa estaria segura.

— Se me permite a ousadia — disse o dragão — enquanto o cavalheiro não
volta, ...talvez só para passar o tempo você poderia ter a gentileza
de me chamar de “querido” de novo, e se consentir em trocar um aperto
de garras com um pobre dragão velho que nunca foi inimigo de ninguém
a não ser dele mesmo... Seria o que faria do último dos dragões
também o mais orgulhoso dos dragões, desde o primeiro deles.

Ele estendeu uma enorme pata, e os grandes ganchos de aço que eram
suas garras se fecharam em volta da mão da Princesa tão suavemente
quanto as garras do urso do Himalaia em volta do pedaço de bolo que
você dá a ele através das grades do zoológico.

E então o Príncipe e a Princesa voltaram triunfantes ao palácio, com o
dragão seguindo-os como um cachorro de estimação. E durante todas as
festividades do casamento ninguém brindou à felicidade dos noivos com
mais sinceridade que o dragão de estimação da Princesa, o qual ela
logo batizara de Fido.

E quando o feliz casal estava instalado em seu próprio reino, Fido foi
até eles implorar que lhe permitissem ser útil.

— Deve ter alguma coisinha que eu possa fazer — ele disse, retinindo
suas asas e esticando suas garras. — Minhas asas e garras e etcétera
poderiam ter algum uso; isso sem nem falar em meu coração eternamente
agradecido.

De modo que o Príncipe providenciou que lhe fizessem uma sela ou um
assento especial para suas costas: enormemente comprido, era feito de
várias carrocerias de bonde soldadas juntas. Cento e cinquenta
poltronas foram ali instaladas, e o dragão, cujo maior prazer agora
era dar prazer aos outros, se deliciava em levar grupos de crianças
para a praia. Voava pelo ar com bastante facilidade carregando seus
cento e cinquenta passageirozinhos, e ficava deitado na areia
esperando pacientemente eles estarem dispostos para voltar. As
crianças gostavam muito dele e costumavam-no chamá-lo de dragão
querido, uma palavra que nunca deixava de produzir lágrimas de
afeição e gratidão em seus olhos. E assim ele viveu, útil e
respeitado, até o outro dia mesmo — quando aconteceu de alguém dizer,
ao alcance de seus ouvidos, que dragões estavam fora de moda, agora
que haviam aparecido tantas máquinas novas. Isso o incomodou tanto
que ele foi pedir ao rei que o transformasse em alguma coisa menos
fora de moda, e o bondoso monarca no mesmo instante o transformou
numa invenção mecânica. O dragão, de fato, tornou-se o primeiro
avião.



\part[Sobre as histórias e os autores]{Sobre as histórias\\ e os autores}

\chapter*{\ }
\markboth{sobre as histórias e os autores}{marcos maffei}

\section{“Perseu e Andrômeda”}

{\centering
Ovídio (43 a.C.--17 d.C.) --- \textit{Metamorphoses},\\ Livro \textsc{iv}, 663-764 (2--8 d.C.)
\par\smallskip}

Públio Ovídio Naso nasceu em Sulmo, próximo a Roma. Seu pai, que queria
que ele seguisse uma carreira política, enviou"-o a Roma para estudar
com os melhores mestres de retórica, mas Ovídio, seguindo sua inclinação “natural”,  tornou"-se poeta. 
Sua primeira obra publicada foi \textit{Amores}, uma série de poemas curtos. 
Em seguida escreveu \textit{Heroínas}, uma série
de cartas de mulheres célebres da mitologia, como Penélope,
Ariadne e Helena, a seus amados ausentes. Depois, escreve \textit{Arte
do amor}, um manual de sedução e um retrato da Roma da época, e
\textit{Remédios do amor}, em que fingia se retratar do que
dissera na obra anterior. Escreveu ainda uma tragédia, \textit{Medeia}, que se
perdeu; mas a obra"-prima que lhe traria fama imortal (como aliás ele
mesmo profetiza no fim do livro) seria \textit{Metamorfoses}. Um enorme poema
dividido em quinze partes e quase doze mil versos, em que reconta uma infinidade
de histórias, sobretudo da mitologia grega, tendo como elemento
unificador a transformação: do caos em harmonia, de animais em
pedras, de homens e mulheres em animais ou estrelas. Seu talento 
poético e narrativo fez desse livro uma das versões
mais fascinantes e lidas dessas histórias, mesmo que a ideia de
metamorfose em muitos momentos sirva só como um pretexto para que ele
vá emendando os episódios que quer contar uns nos outros e tenha de
recorrer a uma variedade de artifícios, dos mais elegantes aos mais
artificiais, para justificar sua inclusão --- um deles é o de se demorar
num detalhe irrelevante para o que está sendo contado, só por conter
alguma metamorfose, como é o caso do coral no trecho aqui apresentado,
embora seja justo admitir que Perseu estava voltando de uma aventura
que, de fato, envolvia portentosas transformações: dos cabelos da
Medusa em cobras e de quem a via em pedra. No ano 8, quando esse
poema estava quase pronto e \textit{Fastos}, em que descrevia os
festivais romanos de cada mês e os mitos que os fundavam, pela
metade, Ovídio foi condenado ao exílio de Roma pelo imperador Augusto. Não se conhece
exatamente a razão da desgraça do poeta, que passaria no exílio a maior parte dos
últimos nove anos de sua vida. Mas
sua influência, e em especial das \textit{Metamorfoses}, seria enorme em toda
a literatura ocidental; do século \textsc{xii} ao \textsc{xvii} não há quase nenhum
grande autor que a ele não se refira ou nele não tenha se inspirado. 

\chapter*{\ }

\section[“Stan Bolovan” (de Rumanischen Märchen)]{\vspace*{-.4em}“Stan Bolovan”\break (de Rumanischen Märchen)}
\enlargethispage{.8em}

{\centering
Andrew Lang (1844-1912) --- \textit{The Crimson Fairy Book}\\ (O livro carmim dos contos de fadas) (1903); 
\textit{The Violet Fairy Book} (O livro lilás dos contos de fadas) (1901)
\par\smallskip}

O escritor, crítico, antropólogo e folclorista Andrew Lang nasceu na
Escócia e, quando criança, além de ouvir avidamente todas as
histórias e lendas da região, “lia tudo quanto era conto de fada que
encontrava; conhecia bem todas as fadas de \textit{Sonho de uma noite de
verão}, todos os fantasmas de Walter Scott e detestava máquinas de
qualquer espécie”. Odiou ter que aprender grego na escola, até
descobrir Homero, que seria uma de suas paixões pelo resto da vida:
traduções da \textit{Odisseia} e da \textit{Ilíada} estão entre seus primeiros
trabalhos publicados e, em 1890, colaboraria com H.~Rider Haggard (o
autor das \textit{Minas do rei Salomão}) em um romance sobre as últimas
aventuras de Odisseu, em busca de Helena depois da morte de Penélope
e Telêmaco. Em vez de uma carreira acadêmica em Oxford, optou por
escrever para \mbox{jornais} e revistas em Londres, tornando"-se um influente
crítico literário. O interesse que sempre teve por lendas e contos do
folclore o direcionou para a antropologia, em que foi um dos
pioneiros nos estudos de mitologia comparada, publicando duas obras
importantes sobre o assunto, \textit{Custom and Myth} (Costume e mito, 1884) e \textit{Myth, Ritual
and Religion} (Mito, ritual e religião, 1887). E resultou também na série de livros pela qual
ficou mais conhecido, as doze coletâneas de contos de fada que
iniciaria em 1889 com \textit{The Blue Fairy Book} (O livro azul dos contos de fadas). 
Com 37 histórias, o livro incluía os contos mais famosos de Perrault, Aulnoy, dos Grimm e das
\textit{Mil e uma noites}, mas também outros menos conhecidos do folclore inglês,
escocês e escandinavo. Não havia originalmente a intenção de fazer
uma série, mas o sucesso foi tão grande que em 1890 aparecia \textit{The Red
Fairy Book} (O livro vermelho dos contos de fadas), com mais histórias das mesmas fontes e também do folclore
russo, e em 1892 \textit{The Green Fairy Book} (O livro verde dos contos de fadas). 
Seriam mais nove, terminando em 1910 com \textit{The Lilac Fairy Book} (O livro
lilás dos contos de fadas), e além dos autores conhecidos e do folclore
europeu, acabariam incluindo histórias tradicionais de todos
os continentes. Lang atuava principalmente como editor dos livros,
selecionando as histórias e encomendando as traduções ou adaptações
para outros, em especial sua mulher Leonore, que fez a maior parte delas.
Lang escreveu e publicou também alguns contos de fadas próprios, e
editou mais treze coletâneas de histórias: de aventura, de animais,
de heróis, de fantasmas e de mistério, a partir da História
propriamente dita, da mitologia grega e das \textit{Mil e uma noites}; a última
delas foi \textit{The Strange Story Book} (O livro das histórias estranhas), 
que sairia um ano após sua morte. 

\chapter*{\ }

\section{“São Jorge e o dragão”}

{\centering
Jacobus de Voragine (1228/9--1298) --- \textit{Legenda Aurea} (Lenda dourada) (c. 1253--70)
\par\smallskip}

Com o cristianismo, matar dragões tornou"-se um serviço com frequência
atribuído a santos; pelo menos uma dúzia deles, em algum
momento de sua vida, teria enfrentado e vencido tais encarnações do
mal. O mais popular é certamente São Jorge; seu combate com o dragão
disseminou"-se em lendas por toda a Europa e foi tema de inúmeros
pintores desde a Idade Média. Do verdadeiro Jorge, pouco se sabe:
teria sido um mártir cristão no Oriente Médio e sua suposta
sepultura se encontra em Lydda, na Palestina. Do século \textsc{vi} em diante,
lendas dos feitos heroicos desse santo guerreiro começaram a se
espalhar pela Europa, ganhando especial ímpeto com as Cruzadas, das
quais era tido como um inspirador; em torno dos séculos \textsc{xiii} e \textsc{xiv},
não se sabe exatamente por quê, tornou"-se o santo padroeiro da
Inglaterra. Sua sepultura fica perto do local onde teria ocorrido o
mitológico combate entre Perseu e o monstro marinho; supõe"-se que
dessa proximidade teria surgido a lenda de São Jorge e o dragão, já
que há similaridades entre as duas histórias. Essa e outras lendas
sobre ele encontram"-se na \textit{Legenda Aurea}, escrita em latim entre 1253
e 1270 pelo arcebispo de Gênova, Jacobus de Voragine (ou Jacopo de
Varazze), uma coletânea de narrativas das vidas de 175 santos, de
relatos de eventos das vidas de Cristo e da Virgem Maria, e de
informações sobre as datas do calendário litúrgico. Imensamente
popular na Idade Média, a obra foi traduzida (e gradualmente
ampliada) para todas as línguas europeias. Foi um dos primeiros livros
a serem publicados em inglês, por William Caxton, em 1483 (\textit{The Golden
Legend}; essa tradução inglesa de Caxton foi a utilizada aqui). Entre
outras histórias famosas de santos e dragões, Jacopus conta também a
de Santa Margarida, padroeira dos partos. Aprisionada por se recusar
a casar com o governador de Antioquia, em sua cela aparece um dragão,
que ela domina com o sinal da cruz. Em outra versão que Jacopus
descarta como apócrifa, ela é engolida pelo dragão e o poder da cruz
faz com que ela saia da barriga dele. Quanto a São Jorge, outro livro
que popularizou sua lenda foi \textit{The Seven Champions of Christendom} (Os
sete campeões da cristandade, 1596--97, de Richard Johnson), 
mencionado na história de Edith Nesbit “Os salvadores da pátria”. 

\chapter*{\ }

\section{“A história de Sigurd”} 

{\centering
Anônimo islandês do século \textsc{xiii} --- \textit{Saga dos volsungos}
\par\smallskip}

Saga, em islandês, significa simplesmente história em prosa e várias
foram escritas no chamado período clássico da literatura islandesa,
os séculos \textsc{xii} e \textsc{xiii}, tratando da vida e dos feitos de reis e
famílias que de fato existiram, com variável fidelidade histórica. Ou
então, de heróis lendários, de procedência tanto escandinava quanto
germânica, as \textit{Fornaldar Sögur} (sagas, ou histórias, da Antiguidade). A
\textit{Saga dos volsungos} é a mais conhecida
delas. Escrita por um autor anônimo na segunda metade do século \textsc{xii},
conta lendas sobre heróis que aparecem também na \textit{Edda
Poética}, o manuscrito da mesma época que é uma compilação da \mbox{poesia}
islandesa. Começa com os ancestrais de Sigurd, relata a
história deste e de Brynhild, e então o destino de sua mulher Gudrun
depois de sua morte. Também da mesma época é a versão germânica
destas lendas, o \textit{Nibelungenlied} (Canção dos nibelungos), mais centrado na figura de Siegfried
(Sigurd) e já mais próximo do romance de cavalaria medieval cristão,
deixando de lado muitos de seus elementos pagãos e fantásticos. Foi
utilizada aqui a adaptação de Andrew Lang (uma das poucas que fez
pessoalmente) da tradução em inglês de William
Morris e Eirikr Magnusson para \textit{O livro vermelho dos 
contos de fadas}, publicado em 1890.

\chapter*{\ }

\section[“Os salvadores da pátria”; “O último dragão”]{“Os salvadores da pátria”;\break “O último dragão”} 

{\centering
Edith Nesbit (1858-1924) --- \textit{The Book of Dragons}\\ (O livro dos dragões) (1900)
\par\smallskip}

Edith Nesbit passou os primeiros nove anos de sua vida no Colégio
Agrícola de Kennington (então um subúrbio rural de Londres), do
qual seu pai era diretor, cargo assumido por sua mãe depois da
morte dele, quando Edith tinha três anos. Em 1867, sua mãe decidiu ir
para a França, em busca de um clima mais apropriado para a saúde de sua filha
mais velha, Mary. Seguiu"-se um período nômade, que incluiu escolas
internas que Edith detestou (de uma delas teria tentado fugir), mas
também uma casa enorme e deliciosa na Bretanha. Foi para morar em uma casa
parecida no campo que a família voltou para a Inglaterra, quando ela
tinha treze anos. As explorações e aventuras dela e de seus irmãos
nos arredores de ambas inspirariam mais tarde as das crianças de seus
livros. Três anos depois, quando sua mãe não tinha mais condições de manter
a casa, mudaram"-se para Londres. Edith publicou um poema e logo
começou a produção contínua de ficção popular para revistas, o que
garantiria também por muito tempo o sustento de sua família após seu
casamento em 1880. Seu marido, Hubert Bland, depois de uma
malsucedida tentativa de ter um negócio próprio, devotaria todas
as suas energias à Fabian Society, sociedade socialista que fundaram 
e de que fizeram parte George Bernard
Shaw e H.G.~Wells, entre outros. A casa de Nesbit tornou"-se um
animado ponto de encontro de intelectuais. Edith ignorava as convenções 
comportamentais de sua época: fumava, recusava"-se a seguir a desconfortável moda feminina e
usava o cabelo sempre curto. Produziu todo
tipo de ficção para garantir a sobrevivência, mas foi só em 1898 que
experimentou escrever para crianças. Com \textit{The Story of the Treasure
Seekers} (História dos caçadores de tesouro), 
as aventuras de cinco irmãos tentando restaurar a fortuna da
família Bastable, convincentemente narrada por um deles, Nesbit
iniciaria sua obra verdadeiramente original, que a faria famosa e a
tornaria uma das mais importantes e influentes autoras inglesas de
literatura para crianças. Uma série de histórias de dragões — entre
elas as presentes neste livro —  seriam reunidas em \textit{O livro dos dragões},
publicado em 1900; no ano seguinte sairia \textit{The Wouldbegoods},
continuação das aventuras da família Bastable; e em 1902 \textit{Five
Children and it} (Cinco crianças e aquilo), o livro que consolidaria esse gênero tão
reproduzido na literatura infantil, em que o cotidiano contemporâneo das
crianças se combina com elementos fantásticos (no caso, cinco irmãos
de férias numa casa de campo, e Psammead, um duende da areia que
realiza desejos). Com ele, \textit{The Phoenix and the Carpet} (A fênix e o tapete, 1904) e \textit{The
Story of the Amulet} (A história do amuleto, 1905), um dos primeiros livros infantis a
incluírem viagens no tempo, formariam uma trilogia; e além de um
terceiro com os Bastable, Nesbit escreveu mais sete livros, e
publicou também algumas coletâneas de histórias curtas. 

\chapter*{\ }

\section{“O dragão relutante”} 

{\centering
Kenneth Grahame (1859-1932) --- \textit{Dream Days}\\ (Dias de sonho) (1898)
\par\smallskip}

Kenneth Grahame nasceu em Edimburgo, na Escócia; antes de fazer cinco
anos, sua mãe morreu e ele e seus irmãos ficaram aos cuidados de uma
avó fria e distante e de um pai incompetente, que abandonaria
definitivamente a família em 1867. Sem condições financeiras para ir
à universidade de Oxford (seu sonho era tornar"-se escritor e
seguir uma carreira acadêmica), em 1879 começou a trabalhar no Banco
da Inglaterra, onde ficou até 1908.
Manteve, todavia, inclinações e amizades literárias e, em
1887, começou a publicar em revistas. Em 1893, publicou uma
coletânea de seus ensaios com o título \textit{Pagan Papers} (Papéis pagãos), e dois anos
depois \textit{The Golden Age} (A era dourada), uma narrativa em primeira pessoa sobre a
infância de cinco irmãos numa casa de campo, que fez um enorme
sucesso. Apesar de ter sido escrito para adultos, o livro capta com
precisão o universo e as preocupações das crianças e se tornaria uma influência significativa
na literatura infantil da época. Em 1898, Grahame publicou uma
continuação, \textit{Dias de sonho}; nela, aparece a história “O dragão
relutante”, contada para as cinco crianças. Mesmo com o enorme sucesso 
dos dois livros, ele só voltaria a escrever oito anos
depois, a partir das histórias que começara a contar para seu filho,
que acabariam se tornando \textit{The Wind in the Willows} (O vento nos
salgueiros), um clássico da literatura infantil inglesa. Publicado em
1908, este foi seu último livro.



\end{document}
