
\chapter{Sobre as histórias e os autores}

\section{“Perseu e Andrômeda”}

Ovídio (43 a.C.--17 d.C.) -- \textit{Metamorphoses}, Livro IV, 663-764 (2-8 dC)

Publius Ovidius Naso nasceu em Sulmo, a 150 km de Roma. Seu pai queria
que ele seguisse uma carreira pública, e enviou-o a Roma para estudar
com os melhores mestres de retórica; mas ele preferiu ser poeta. Sua
primeira obra publicada é Amores, uma série de poemas curtos sobre o
assunto do título; em seguida escreveu Heroides (Heroínas), uma série
de cartas de mulheres célebres da mitologia, tais como Penélope,
Ariadne e Helena, a seus amados ausentes; e então Ars Amatoria (Arte
do amor), um manual de sedução e um retrato da Roma da época, e
Remedia Amoris (Remédios do amor), em que fingia se retratar do que
dissera na obra anterior. Escreveu ainda uma tragédia, Medea, que se
perdeu; mas a obra-prima que lhe traria fama imortal (como aliás ele
mesmo profetiza no fim do livro) seria Metamorphoses. Um enorme poema
em quinze livros e quase 12 mil versos, em que reconta uma infinidade
de histórias sobretudo da mitologia grega, tendo como elemento
unificador a transformação: do caos em harmonia, de animais em
pedras, de homens e mulheres em animais ou estrelas. Seu talento,
tanto poético quanto como narrador, fez desse livro uma das versões
mais fascinantes e lidas dessas histórias, mesmo que a idéia de
metamorfose em muitos momentos sirva só como um pretexto para que ele
vá emendando uns nos outros os episódios que quer contar, e tenha de
recorrer a uma variedade de artifícios, dos mais elegantes aos mais
forçados, para justificar sua inclusão (um deles é o de se demorar
num detalhe irrelevante para o que está sendo contado, só por conter
alguma metamorfose, como é caso do coral no trecho aqui apresentado,
embora seja justo admitir que Perseu estava voltando de uma aventura
que de fato envolvia portentosas transformações: dos cabelos da
Medusa em cobras, e de quem a via em pedra). No ano 8, quando esse
poema estava quase pronto, e Fasti (Calendário — em que descrevia os
festivais romanos de cada mês e os mitos que os fundavam) pela
metade, Ovídio foi exilado de Roma por Augusto, não se sabe
exatamente por quê; e o poeta passaria no exílio, a maior parte do
tempo nostálgico e amargo, os nove anos restantes de sua vida. Mas
sua influência, e em especial das Metamorphoses, seria enorme em toda
a literatura ocidental; do século XII ao XVII não há quase nenhum
grande autor que a ele não se refira ou nele não tenha se inspirado. 


\section{“Stan Bolovan” (de Rumanischen Märchen)}

Andrew Lang (1844-1912) -- \textit{The Crimson Fairy Book} (data?); 
\textit{The Violet Fairy Book}  (data?)

O escritor, crítico, antropólogo e folclorista Andrew Lang nasceu na
Escócia, e quando criança, além de ouvir avidamente todas as
histórias e lendas da região, “lia tudo quanto era conto de fada que
encontrava; conhecia bem todas as fadas de Sonho de uma noite de
verão e todos os fantasmas de Walter Scott, e detestava máquinas de
qualquer espécie”. Odiou ter que aprender grego na escola, até
descobrir Homero, que seria uma de suas paixões pelo resto da vida:
traduções da Odisséia e da Ilíada estão entre seus primeiros
trabalhos publicados, e em 1890 colaboraria com H. Rider Haggard (o
autor das Minas do Rei Salomão) num romance sobre as últimas
aventuras de Odisseus, em busca de Helena depois da morte de Penélope
e Telêmaco. Em vez de uma carreira acadêmica em Oxford, optou por
escrever para jornais e revistas em Londres, tornando-se um influente
crítico literário. O interesse que sempre teve por lendas e contos do
folclore o direcionou para a antropologia, em que foi um dos
pioneiros nos estudos de mitologia comparada, publicando duas obras
importantes sobre o assunto, Custom and Myth (1884) e Myth, Ritual
and Religion (1887). E resultou também na série de livros pela qual
ficou mais conhecido, as doze coletâneas de contos de fada que
iniciaria em 1889 com The Blue Fairy Book. Com 37 histórias, o livro
incluía os contos mais famosos de Perrault, Aulnoy, dos Grimm e das
1001 noites, mas também outros menos conhecidos do folclore inglês,
escocês e escandinavo. Não havia originalmente a intenção de fazer
uma série, mas o sucesso foi tão grande que em 1890 aparecia The Red
Fairy Book, com mais histórias das mesmas fontes e também do folclore
russo, e em 1892 The Green Fairy Book. Seriam mais nove, terminando
em 1910 com The Lilac Fairy Book, e além dos autores conhecidos e do
folclore europeu, acabariam incluindo histórias tradicionais de todos
os continentes. Lang atuava principalmente como editor dos livros,
selecionando as histórias e encomendando as traduções ou adaptações
para outros, em especial sua mulher Leonore, que fez a maioria delas.
Lang escreveu e publicou também alguns contos de fadas próprios, e
editou mais treze coletâneas de histórias: de aventura, de animais,
de heróis, de fantasmas e de mistério, a partir da História
propriamente dita, da mitologia grega e das 1001 noites; a última
delas foi The Strange Story Book, que sairia um ano após sua morte. 


\section{“São Jorge e o dragão”}

Jacobus de Voragine (1228/9-1298) -- \textit{Legenda Aurea} (c. 1253-70)

Com o cristianismo, matar dragões tornou-se um serviço com freqüência
atribuído a santos; diz-se de pelo menos uma dúzia deles que em algum
momento de sua vida teriam enfrentado e vencido tais encarnações do
mal. O mais popular é certamente São Jorge; seu combate com o dragão
disseminou-se em lendas por toda a Europa, e foi tema de inúmeros
pintores desde a Idade Média. Do verdadeiro Jorge, pouco se sabe:
teria sido um mártir cristão no Oriente Médio, e sua suposta
sepultura se encontra em Lydda, na Palestina. Do século VI em diante
lendas dos feitos heróicos desse santo guerreiro começaram a se
espalhar pela Europa, ganhando especial ímpeto com as Cruzadas, das
quais era tido como um inspirador; em torno dos séculos XIII e XIV,
não se sabe exatamente por quê, tornou-se o santo padroeiro da
Inglaterra. Sua sepultura fica perto do local onde teria ocorrido o
mitológico combate entre Perseu e o monstro marinho; supõe-se que
dessa proximidade teria surgido a lenda de São Jorge e o dragão, já
que há similaridades entre as duas histórias. Essa e outras lendas
sobre ele encontram-se na Legenda Aurea, escrita em latim entre 1253
e 1270 pelo arcebispo de Genova, Jacobus de Voragine (ou Jacopo de
Varazze), um coletânea de narrativas das vidas de 175 santos, de
relatos de eventos das vidas de Cristo e da Virgem Maria, e de
informações sobre as datas do calendário litúrgico. Imensamente
popular na Idade Média, a obra foi traduzida (e gradualmente
aumentada) em todas as línguas európeias. Foi um dos primeiros livros
a serem publicados em inglês, por William Caxton, em 1483 (The Golden
Legend; essa tradução inglesa de Caxton foi a utilizada aqui). Entre
outras histórias famosas de santos e dragões, Jacopus conta também a
de Santa Margarida, padroeira dos partos: aprisionada por se recusar
a casar com o governador de Antióquia, em sua cela aparece um dragão,
que ela domina com o sinal da cruz; ou, em outra versão que Jacopus
descarta como apócrifa, ela é engolida pelo dragão, e o poder da cruz
faz com que ela abra a barriga dele. Quanto a São Jorge, outro livro
que popularizou sua lenda foi The Seven Champions of Christendom (Os
sete campeões da cristandade;1596-7, de Richard Johnson), o qual é
mencionado na história de Edith Nesbit “Os salvadores da pátria”. 


\section{“A história de Sigurd”} 

anônimo islandês do século XIII -- \textit{Volsunga Saga}

Saga, em islandês, significa simplesmente história, em prosa, e várias
foram escritas no chamado período clássico da literatura islandesa,
os séculos XII e XIII, tratando da vida e dos feitos de reis e
famílias que de fato existiram, com variável fidelidade histórica; ou
então, de heróis lendários, de procedência tanto escandinava quanto
germânica, as Fornaldar Sögur (sagas ou histórias da antiguidade). A
Volsunga Saga (a história dos Volsungs) é a melhor e mais conhecida
delas. Escrita por um autor anônimo na segunda metade do século XII,
conta em prosa lendas sobre heróis que aparecem também na Edda
Poética, o manuscrito da mesma época que é uma compilação da poesia
islandesa de até então; começa com os ancestrais de Sigurd, relata a
história deste e de Brynhild, e então o destino de sua mulher Gudrun
depois de sua morte. Também da mesma época é a versão germânica
destas lendas, o Nibelungenlied, mais centrado na figura de Siegfried
(Sigurd) e já mais próximo do romance de cavalaria medieval cristão,
deixando de lado muitos de seus elementos pagãos e fantásticos. Foi
utilizada aqui a adaptação de Andrew Lang (uma das poucas que fez
pessoalmente; veja na página 157) da tradução em inglês de William
Morris e Eirikr Magnusson para The Red Fairy Book, publicado em 1890.

\section{“Os salvadores da pátria”; “O último dragão”} 

Edith Nesbit (1858-1924) -- \textit{The Book of Dragons} (1900)}

Edith Nesbit passou os primeiros nove anos de sua vida no Colégio
Agricultural de Kennington (então um subúrbio rural de Londres), do
qual seu pai era diretor, e do qual sua mãe se encarregou depois da
morte dele, quando Edith tinha três anos. Em 1867 sua mãe decidiu ir
para a França, atrás de um clima melhor para a saúde de sua filha
mais velha, Mary; seguiu-se um período nômade, que incluiu escolas
internas que Edith detestou (de uma delas teria tentado fugir) mas
também uma casa enorme e deliciosa na Bretanha. Foi para uma casa
similar no campo que a família voltou para a Inglaterra, quando ela
tinha treze anos; as explorações e aventuras dela e de seus irmãos
nos arredores de ambas inspirariam mais tarde as das crianças de seus
livros. Três anos depois, sua mãe não mais tendo condições de manter
a casa, mudaram-se para Londres. Edith publicou um poema, e logo
começou a produção contínua de ficção popular para revistas, o que
garantiria também por muito tempo o sustento de sua família após seu
casamento em 1880. Seu marido, Hubert Bland, depois de uma
mal-sucedida tentativa de ter um negócio próprio, devotaria todas
suas energias à Fabian Society, uma sociedade para promover o
socialismo, da qual ambos foram fundadores, junto com George Bernard
Shaw e H. G. Wells, entre outros. A casa de Nesbit tornou-se um
animado ponto de encontro de intelectuais. Além de suas convicções
socialistas, também em seus hábitos Edith ignorava as convenções da
época: fumava, recusava-se a seguir a desconfortável moda feminina e
usava o cabelo sempre curto. Apesar de até então ter produzido todo
tipo de ficção para garantir a sobrevivência, foi só em 1898 que
experimentou escrever para crianças. Com The Story of the Treasure
Seekers, as aventuras de cinco irmãos tentando restaurar a fortuna da
família Bastable, convincentemente narrada por um deles, Nesbit
iniciaria sua obra verdadeiramente original, que a faria famosa e a
tornaria uma das mais importantes e influentes autoras inglesas de
literatura para crianças. Uma série de histórias de dragões — entre
elas as aqui apresentadas —  seriam reunidas em The Book of Dragons,
publicado em 1900; no ano seguinte sairia The Wouldbegoods,
continuação das aventuras da família Bastable; e em 1902, Five
Children and it, o livro que consolidaria esse gênero desde então tão
comum na literatura infantil, em que o cotidiano comtemporâneo das
crianças se combina com elementos fantásticos (no caso, cinco irmãos
de férias numa casa de campo, e Psammead, um duende da areia que
concede desejos). Com ele, The Phoenix and the Carpet (1904) e The
Story of the Amulet (1905), um dos primeiros livros infantis a
incluírem viagens no tempo, formariam uma trilogia; e além de um
terceiro com os Bastable, Nesbit escreveu mais sete livros, e
publicou também algumas coletâneas de histórias curtas. 


\section{“O dragão relutante”} 

Kenneth Grahame (1859-1932) -- \textit{Dream Days} (1898)

Kenneth Grahame nasceu em Edimburgo, na Escócia; antes de fazer cinco
anos, sua mãe morreu, e ele e seus irmãos ficaram aos cuidados de uma
avó fria e distante e um pai incompetente, que abandonaria
definitivamente a família em 1867. Sem condições financeiras para ir
à universidade de Oxford (seu sonho era tornar-se escritor e/ou
seguir uma carreira acadêmica), em 1879 começou a trabalhar no Banco
da Inglaterra, onde ficou até 1908, tendo sido várias vezes
promovido. Manteve, todavia, inclinações e amizades literárias, e em
1887 começou a contribuir para revistas. Em 1893 publicou uma
coletânea de seus ensaios com o título Pagan Papers, e dois anos
depois The Golden Age, uma narrativa em primeira pessoa sobre a
infância de cinco irmãos numa casa de campo, que fez um enorme
sucesso. Apesar de ter sido escrito para adultos, e seu humor
resultar do distanciamento irônico do narrador, o livro capta com
precisão o universo e as preocupações das crianças, e em especial sua
incompreensão por parte dos adultos, o que fez com que fosse muito
lido também por crianças e se tornasse uma influência significativa
na literatura infantil da época. Em 1898, Grahame publicou uma
continuação, Dream Days; nela, aparece a história “O dragão
relutante”, contada para as cinco crianças. Mesmo os dois livros
tendo feito um enorme sucesso, ele só voltaria a escrever oito anos
depois, a partir das histórias que começara a contar para seu filho,
que acabariam se tornando The Wind in the Willows (O vento nos
salgueiros), um clássico da literatura infantil inglesa. Publicado em
1908, este foi seu último livro.

