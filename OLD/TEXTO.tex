\pagestyle{myheadings}
\markboth{Crônicas}{Braulio Tavares}
\oneside %%%%%%%%%%%%%%%%%%%%%%%%%%%%%%%%%%%%%%%%%%%%%%%%%%%%%%%%%%%%%%%
\paginabranca

\chapter{Introdução}

{\itshape
Em março de 2003, a convite de Rômulo Azevedo e Luís Carlos de
Sousa, iniciei a publicação de uma coluna diária no Jornal
da Paraíba.  O jornal me deu completa liberdade para tratar
de qualquer assunto, ou mesmo de nenhum assunto, e impôs como única
condição o fato de que cada artigo não poderia ter menos de 2.900 nem
menos de 3.000 caracteres mais espaços.

Desde então, publiquei mais de 1.600 artigos, dos quais os meus
editores na Hedra escolheram 30 para compor este volume.  Os textos
não foram modificados.  Saem aqui exatamente como saíram no jornal,
com mínimas correções de digitação e redação.  Achei conveniente
colocar no final do volume um conjunto de notas e glossário para
explicar alguns regionalismos ou termos técnicos, situar alguns
artigos no contexto do momento em que foram escritos, e sugerir
alguns caminhos para que o leitor interessado possa explorar alguns
temas.}

\chapter{O que é o Tempo?}

Santo Agostinho, interrogado sobre o que era o Tempo, disse algo como
(cito de memória): “Na verdade, ao que parece existe apenas o
Presente. Porque o Passado não passa da lembrança presente de um fato
passado; e o Futuro não passa da visualização presente de uma
possibilidade futura”.  Os termos empregados pelo santo-filósofo não
foram bem estes, mas acho que a intenção é esta, e se ofendi ao
filósofo espero que pelo menos o santo me perdoe.

Há um outro raciocínio de Santo Agostinho que, modéstia à parte,
também me ocorreu certa vez. Como definir o Presente? Digamos que é o
dia de hoje. Mas se estou às 3 horas da tarde, como posso dizer que
este dia inteiro é o Presente? Porque 15 horas já transcorreram desde
a meia-noite passada, e isto então é Passado; e ainda faltam cerca de
9 horas para chegar o dia seguinte, e é óbvio que estas horas estão
situadas no Futuro. Se considero como Presente “a presente hora”, o
mesmo raciocínio se aplica aos minutos em que ela se divide, e nessa
pisada seríamos forçados a definir o Presente como um instante
infinitesimal e indivisível de tempo, um micro-instante de Presente
puro, sem nenhum restinho de Passado e nenhuma beirinha de Futuro. 

Não é esta, contudo, a nossa experiência intuitiva do mundo.
Experimentamos continuamente o Passado e o Futuro através dos
estímulos sensoriais que recebemos. Mesmo ao dizer “está chovendo”,
não queremos nos referir à chuva que caiu naquele microssegundo
específico. Este presente contínuo de “está chovendo” é contaminado
pelo Passado e pelo Futuro. Queremos dizer que choveu até este
segundo, e que tudo indica que no próximo segundo a chuva continuará
a cair. 

O Presente é uma reação mental provocado pelo impacto do mundo sobre
nossos sentidos. No mundo físico existe uma “seta” implacável, que
geralmente lemos de duas maneiras: a) nós vimos do Passado e estamos
indo na direção do Futuro, ou, b) os acontecimentos estão vindo do
Futuro, passam por nós, e desaparecem no Passado.  O Presente é uma
criação nossa, uma leitura nossa do que os nossos sentidos percebem. 
Nossa mente é um torvelinho onde se misturam as coisas que acabaram
de acontecer, e as que achamos que irão acontecer em seguida; é uma
salada contínua de passado e futuro, de memórias e expectativas, e a
essa salada chamamos Presente.

O Passado e o Futuro são o mundo. O Presente sou eu. O Presente inclui
minha experiência factual de Passado, e minha experiência virtual de
Futuro. E proponho (se não para o leitor, pelo menos para mim mesmo,
que ando precisado) a seguinte definição de Tempo: O Passado é o
tempo que nunca me pertenceu, é tudo aquilo que ocorreu antes do meu
nascimento. O Presente é tudo que começou a ocorrer desde então, é
todo o período a que me refiro quando digo: “No meu tempo…”  E o
Futuro é o tempo que nunca me pertencerá, é tudo que irá ocorrer após
o instante da minha morte.

\chapter{O tamanho do Tempo}

Para organizar o mundo, inventamos uma ficção poderosíssima (mas
ficção, mesmo assim): a de que o Tempo é algo linear, constante,
matematicamente divisível. Relógios e calendários servem para
domesticar nossa mente nesse sentido, fazendo-nos crer que é próprio
da natureza do Tempo ser medido, e de que ele é de fato composto por
tijolos grandes, que se dividem em tijolos menores, que por sua vez
se subdividem em tijolinhos minúsculos, e assim por diante.  É um
artifício intelectual; útil, mas limitado. Os conceitos de ano, mês,
dia, segundo, etc. não são conceitos arbitrários, porque se referem a
aspectos físicos que de fato existem; mas não são suficientes para
descrever nossa relação com o Tempo.

Existe o Tempo do corpo e o Tempo da mente. Relógios e calendários são
nossa maneira de fazer a mente pensar de acordo com o Tempo do corpo,
o Tempo do universo físico, onde a Terra, o Sol, as Estrelas e a Lua,
que se movem em ciclos regulares, nos dão um ponto de partida, pontos
de referência comuns a todas as pessoas.  Isto, contudo, é o Tempo
externo, o Tempo do mundo, que pode ser medido com uma régua. O Tempo
da mente, o Tempo da nossa memória emocional e intuitiva, teria que
ser medido com um pedaço de elástico, porque ele se estica ou se
encolhe de acordo com a maior ou menor energia psíquica que aplicamos
nele. Daí os “relógios moles” de Salvador Dali, uma das grandes
sacadas da Arte para exprimir essa flexibilidade de nossa percepção.

Jorge Luís Borges admirava-se de ser capaz de recordar com nitidez um
fato ocorrido há cinqüenta anos e não poder adivinhar o que sucederia
no dia seguinte, “que estava muito mais próximo”.  Este pequeno
paradoxo nos mostra a inutilidade de tentarmos sempre espacializar o
Tempo, vê-lo como uma régua, uma escala linear onde nossa consciência
se move sempre na mesma velocidade, na mesma direção, e na mesma
ordem: 1, 2, 3, 4, 5…  Nosso Tempo mental é o contrário. Se estamos
conversando com um amigo de infância, podemos passar cinco minutos
conversando em tempo real, depois nos calamos por dez segundos
enquanto avaliamos “de fora” um período de anos de nossa vida, e em
seguida passamos cinco minutos rememorando momentos que, somados,
ocupariam meses de tempo cronológico.

O Tempo mental se parece com esses websites onde imagens aumentam de
tamanho ou se transformam em outras no momento em que passamos o
mouse sobre elas.  Quando alguma coisa nos evoca um fato
emocionalmente relevante, ele cresce, invade nossa mente por inteiro,
distorce o tempo real que nosso corpo está experimentando. A evocação
de um minuto traumático ou de intenso prazer pode se reiterar durante
um dia inteiro, como um disco enganchado.  Em vez de vermos o Tempo
como uma linha reta com o Presente situado entre o Passado e o
Futuro, poderíamos visualizá-lo como um cacho de bolhas de espuma em
ordem não-cronológica, cada uma delas se expandindo ao ser tocada
pela nossa consciência.

\chapter{As águas do tempo}

Quando as águas do tempo vão passando, vêm trazendo mil ciscos na
corrente. São as pequenas sujeiras desta vida, as palavras amargas
que se escutam, os gestos brutos, as incompreensões por desdém ou por
insensibilidade, as descortesias dos desconhecidos. Vêm as sujeiras
maiores também, que são as ruindades que nos fazem, as sacanagens
planejadas à nossa revelia, o escárnio gratuito de quem não ousaria
agir assim na nossa frente, as provocações dos vizinhos, as armações
dos antagonistas no trabalho, dos falsos amigos.

Não tem força no mundo que consiga nos livrar dos esgotos da maldade.
Tudo é despejado através de nós: vasculho, cisco, gravetos, bagaços
de laranja, massa amorfa de plásticos e papéis, restos de comida,
tufos de mato, lama do chão. O ser humano despeja esses resíduos
metafóricos à sua volta, voluntariamente ou não, pelo simples fato de
existir, de entrar em atrito com os outros, de ter que atravessar
todos os dias esta selva sem lei onde no primeiro contato com alguém
nunca dá para se saber direito quem é fera ou quem está ferido.

Essa enxurrada passa através de nossa alma como se passasse por uma
peneira. Nossa alma é uma peneira com aquela treliça metálica cheia
de quadradinhos, através da qual a enxurrada escoa, e onde os
detritos maiores são retidos. Aqui vai ficando um cisco, ali fica um
caco de vidro, mais adiante um fósforo, uma tampa de garrafa, uns
fios soltos de piaçava, uma meia furada, lata de cerveja amassada,
embalagem vazia… A água passa e essas coisas maiores vão ficando
presas na treliça, vão se instalando ali, e a lama que continua a
passar acaba servindo como uma espécie de argamassa que recobre esses
resíduos, fixando-os, deixando-os presos ali com uma firmeza de quem
se prepara para esperar que até as pirâmides se desmanchem em pó.

E o quê que uma pessoa faz, quando se vê assim, com a alma feito um
filtro de sujeiras, todo contaminado pela poluição da vida? Bem, os
outros eu não sei. Mas quando começa a doer, quando começa a
incomodar muito (quando meu time perde, por exemplo), eu costumo
deitar assim de noite e deixar que o silêncio e a escuridão passem
através de mim, como se fosse uma água lenta mas forte, uma água que
tudo arrasta. E aí eu abro (como? não sei, só sei que é assim) os
espaços da treliça, que em vez de estreitinhos como papel
milimetrados vão se alargando, ficam como grades de
palavras-cruzadas, depois ficam como tabuleiro de xadrez… E quanto
mais eles se alargam mais água vai passando, e no passar elas
desalojam a sujeira acumulada, que vai se desprendendo, se
esfarelando, vai sendo lavada, vai sendo levada embora. Eu fecho os
olhos e deixo essas águas rolarem. São as mesmas águas do tempo que
trouxeram aquilo para dentro de mim, mas como não param de passar,
basta alargar os quadrados da peneira, e tudo que veio vai embora,
vai embora, vai embora, e me deixa enxagüado pelas águas sem fim da
noite e do silêncio.

\chapter{Eu era feliz e não sabia}

Quando eu era pequeno, Ataulfo Alves era um dos compositores mais
conhecidos no país, algo como Martinho da Vila hoje em dia. Suas
músicas tocavam o tempo todo, todo mundo queria gravá-las. Uma das
mais conhecidas é a canção nostálgica em que ele relembra sua cidade
natal, Miraí (\textsc{mg}), com versos simples e emotivos: “Eu daria tudo que
tivesse pra voltar ao tempo de criança… Eu não sei por quê que a
gente cresce, se não sai da mente essa lembrança”.  Ele recorda o
ambiente da cidadezinha, as pessoas que sumiram no tempo: “Que
saudade da professorinha que me ensinou o b-a-ba… Onde andará
Mariazinha? Meu primeiro amor, onde andará?” E termina: “Eu igual a
toda meninada, tanta travessura que eu fazia! Jogo de botão sobre a
calçada… Eu era feliz e não sabia!”

Este verso final incorporou-se à nossa linguagem cotidiana, virou uma
parte do falar brasileiro, e não sei de honra maior para um verso
escrito por um indivíduo. Dizemos isto a propósito de tudo, a
propósito de qualquer situação passada que na hora não parecia grande
coisa mas que, quando a vemos em retrospecto, a gente sente uma falta
danada. Certa vez, quando participei da criação de motes para o
Congresso de Violeiros de Campina Grande, propus o mote: “A gente só
é feliz / quando não sabe que é”. Era no tempo em que o Congresso
lotava o Ginásio da \textsc{aabb} com milhares de estudantes. Éramos felizes,
e não sabíamos.

Ataulfo parece sugerir que existe um certo conflito entre a felicidade
e a consciência desta felicidade. Quando estamos totalmente
absorvidos por êxtases ou epifanias, não sobra muito tempo para
botarmos as mãos nos bolsos e pensarmos, “puxa vida, que momento
legal este!”  Parece sugerir também que esse tipo de felicidade só
existe na infância, naquele momento em que já somos grandes o
bastante para fruir com intensidade as coisas boas da vida
(Mariazinha, o jogo de botão, etc.), mas não sabemos ainda das
desilusões e dos sofrimentos que nos aguardam mais adiante.

Basta pegarmos a máquina-do-tempo, no entanto, para percebermos que
não é bem assim. Lá está Ataulfinho, de calção, sentado na calçada,
jogando botão em cima de uma tábua e vendo Mariazinha pular corda ali
perto: a saia subindo e descendo… Esta é a imagem que lhe ficará na
memória meio século depois, mas ao retornar àquele instante
específico ele sente virem à tona uma horda de coisas ruins que já
esquecera. A prova de Geografia amanhã, para a qual não estudou nada.
A ameaça feita por Zezim da esquina de dar-lhe uns cascudos por causa
de uma guerra aérea de corujas. O pai, que vive adoentado, gemendo,
recusando-se a tomar remédio e dizendo que “não é nada”. Os dez
tostões a mais que pegou do troco da bodega e que a mãe anda
procurando em altas vozes por dentro de casa. Felicidade? Claro. A
felicidade é a memória passada a limpo, expurgada dos “quatrocentos
golpes” que nos ferem e nos magoam a cada dia. Só se é feliz hoje
muito tempo depois.

\chapter{A arte de olhar diferente}

O professor tinha uma montanha de provas para corrigir e resolveu dar
à turma de garotos de dez anos alguma coisa que os mantivesse
ocupados. Passou o seguinte dever: “Somar todos os números de 1 a
100: 1 + 2 + 3 + 4 + 5…” Todos começaram a fazer os cálculos, de
testa franzida. Ele nem tinha terminado de corrigir a primeira prova
quando um dos garotos aproximou-se e colocou sobre a mesa sua lousa
com a resposta. (É, naquele tempo os estudantes escreviam em lousas,
pequenos quadros-negros portáteis). O professor olhou, o queixo caiu:
5.050. A resposta (que ele já conhecia) estava certa. Ele desconfiou
de algum truque - mas qual? Pediu explicações.

O garoto explicou. “Eu achei que, como era para somar todos, a ordem
não iria fazer muita diferença. Em vez de ir somando pela ordem,
somei o primeiro com o último, ou seja, 1 + 100. Deu 101. Aí, somei o
segundo com o penúltimo: 2 + 99 = 101. Depois somei o terceiro com o
antepenúltimo: 3 + 98 = 101. Ora, o resultado tinha que ser sempre
esse, porque os números somados, de um lado, era sempre um-a-mais, e
do outro era sempre um-a-menos. E isso evidentemente ia continuar até
chegar no meio da lista: 50 + 51 = 101. Isso queria dizer que eram 50
somas de dois em dois, todas dando 101. Ora, 50 vezes 101 é 5.050, tá
ligado?”

O ano era 1787, o país era a Alemanha, e o menino se chamava Carl
Friedrich Gauss, que veio a ser conhecido na Europa como “o Príncipe
dos Matemáticos”. Morreu aos 78 anos, e quarenta anos após sua morte
seus diários matemáticos ainda eram publicados, trazendo surpresas e
mais surpresas, descobertas e mais descobertas. O episódio acima foi
narrado por Paul Karlson em “A Magia dos Números”, que li na
adolescência. Eric Temple Bell, em “Men of Mathematics”, dá uma outra
versão, inclusive dizendo que o problema era bem mais difícil do que
esse. Mas não vem ao caso agora. O que foi, de fato, que o menino
fêz?

O menino tinha um caminho aberto à sua frente: era só somar os
números, do primeiro ao último. Ele, no entanto, preferiu procurar
outra maneira. Talvez por achar (como nós, os preguiçosos do mundo,
sempre achamos) que “deve ter uma maneira menos trabalhosa de fazer
isso”. Tinha. E o que o levou a somar o primeiro número com o último?
Eu diria que foi um pouco de espírito lúdico, aquele espírito de
“vamos entrar nesse beco só pra ver onde vai dar”. Mas indica outra
coisa. Enquanto os outros alunos, de nariz enfiado na lousa, só
enxergavam o 1, depois o 2, depois o 3, e assim por diante, Gauss deu
dez passos atrás e contemplou mentalmente uma longa linha de números,
parecendo uma fita métrica esticada. E ele estendeu mentalmente os
braços, pegou com a mão esquerda o 1 e com a mão direita o 100, e
tentou somá-los. O resto é consequência. Às vezes é bom afastar o
nariz do problema, e vê-lo de corpo inteiro. Às vezes o próprio
formato do problema já sugere a solução.

\chapter{A flor do coco}

Minha mãe vivia fazendo bolos, tapiocas, cocadas, um monte de quitutes
caseiros que requeriam coco. E toda vez que ela pegava um coco para
partir perguntava aos filhos que estivessem por perto: “Vai querer a
água ou a flor?” Eram duas opções irresistíveis, profundo dilema
filosófico, daqueles de travar a placa-mãe de qualquer filho. Partido
o coco, a água era recolhida num caneco e entregue a um, enquanto
outro recebia a “flor”, ou seja, a primeira raspagem da carne branca
e úmida que o coco guarda em seu interior. Coco ralado já é uma coisa
gostosa; avaliem a primeira raspa, a raspa daquela superfície
molhada, macia, ainda guardando a leve carnosidade que tem a polpa do
coco verde. Depois de raspada a flor, o resto do coco, conquanto
saboroso, não tinha o mesmo frescor, não trazia a mesma brisa ao
paladar.

Chamem-me pseudo-intelectual, se quiserem, mas acho que com os livros
se dá algo parecido. Quando descobrimos no balcão ou na prateleira um
livro que nos atrai e o compramos, tudo nele ainda tem o sabor de
novo. E nada se compara àquele primeiro contato quando, na
tranquilidade do gabinete de leitura, abrimos o pacote e podemos por
fim examiná-lo devagar, folheá-lo, conhecê-lo aos poucos. Examinamos
a capa, lemos o texto de contracapa, as orelhas; vamos ao índice,
vamos ao índice remissivo quando o há, corremos o polegar pelas
folhas, admiramos as ilustrações, lemos um pedacinho aqui, outro ali,
saboreamos o prefácio…

E aí ocorre algo curioso. No dia seguinte, quando pegamos o livro de
novo, é como se um pequeno encanto já tivesse se desvanecido. O livro
não tem mais aquele frescor, aquele gosto de coisa nova. Para todos
os efeitos, não o lemos ainda, mas por outro lado é como se ele já
tivesse perdido a novidade. Porque o que ele nos deu, naquela
primeira noite de contato, foi a sua flor-do-coco, foi a superfície
intacta e virgem de coisa nova, desconhecida, repleta de infinitas
possibilidades. Depois daquela manuseada inicial, depois daquelas
primeiras folheadas, o livro perdeu o seu verniz de Desconhecido e de
Mistério. Fazia parte do mundo e seus mistérios; agora faz parte de
nós mesmos e de nosso bocejante repertório de coisas já conhecidas.

Chamem-me moralista, mas palpita-me que é isto que ocorre também com o
Cavalheiro Casanova, com Don Juan e com os demais grandes
conquistadores da História. O que eles buscam não é uma mulher, é o
verniz de Desconhecido, de Novidade e de Mistério que qualquer mulher
traz num primeiro contato; é aquela sensação de frescor de um sabor
jamais provado antes, de um sabor que tivesse estado se guardando a
vida inteira para ser desfrutado pelo paladar do conquistador.
Experimentada a flor, os 99\% restantes do coco tornam-se (para eles)
redundantes e supérfluos. O conquistador é um vampiro que não se
alimenta de sangue, mas de ineditismo. Sua vida é uma busca
incessante de novos amores, não por serem amores, mas por serem
novos.

\chapter{As casas bem-assombradas}

Passei a vida vendo filmes de terror e lendo histórias de terror; e um
clichê que parece estar presente no mundo inteiro é o da Casa
Mal-Assombrada. Este tema envolve duas premissas: 1) de que após a
morte as almas das pessoas continuam existindo e manifestando-se no
mundo material; 2) de que essa existência pós-morte é sempre de
sofrimento, e as manifestações são sempre de violência, agressão,
lamentação, etc. A justificativa científica é de que quando alguém
morre em sofrimento, seja por violência física ou por infelicidade
psicológica, sua alma fica presa àquele local, podendo ser ativada
pela presença de outras pessoas.

Gostaria de informar aos caros leitores que isto é apenas parte da
verdade. Existem também as Casas Bem-Assombradas, frequentadas pelos
espíritos das pessoas que viveram em paz ali e que de vez em quando
ali retornam, não para expiar uma culpa ou reconstituir um suplício,
mas para fazer uma visita saudosa àquele lugar onde foram tão
felizes. Meus pais já moraram numa casa (alugada) onde de vez em
quando éramos visitados por uma velhinha simpática, que ficou muito
amiga de minha mãe, e uma vez por mês aparecia nos fins de tarde para
tomar um café e conversar no terraço. Ela nunca disse que já tinha
morado ali, mas eu, com o olho sherlockiano das crianças, percebia
que ela sempre arranjava um pretexto para ir à cozinha, ir ao
banheiro, caminhar pelo quintal, dar uma espiada saudosa nos quartos.
Ela enganou a todos, menos a mim.

Como são espíritos equilibrados, felizes, de-bem-com-a-morte, esses
indivíduos fazem o que podem para não nos assustar. Poderiam
materializar-se de repente na sala, ou no jardim; mas preferem dobrar
a esquina, vir andando pela calçada, tocar a campainha… A felicidade
que experimentaram ali é um ímã tão poderoso quanto o sofrimento dos
fantasmas tradicionais. Grande parte dos lugares assombrados em nosso
mundo são aqueles onde as pessoas passaram momentos felizes: cinemas,
clubes, salões de dança, bares, estádios de futebol. Neste último
caso, vi há algum tempo no jornal uma foto da torcida do Treze no
Amigão, onde reconheci um guri de seus 13 anos que frequentava muito
as cadeiras do Presidente Vargas quando eu era garoto, até a camisa
era a mesma.

Observe bem, caro leitor, quando fôr ao cinema. No meio daquela
platéia geralmente jovem, buliçosa, aqui e acolá veremos uns sujeitos
discretos que chegam cedo, compram seu ingresso e vão direto para as
poltronas da frente; às vezes são casais que vêm de mãos dadas, como
quem mantém vivo um ritual antigo. Já me perguntei por que não vejo
pessoas conhecidas entre eles, mas imagino que quando se materializam
a sala de projeção é o que menos importa (pode ser em qualquer
cidade, qualquer país): o que importa para eles é o filme em cartaz.
Quando dizem “vou ao Cinema”, é a este cinema com C maiúsculo que
vêm, àquele onde pagamos para ver os fantasmas de gente que não está
mais aqui.

\chapter{Ato de misericórdia}

Venho andando pela rua e um cara me aborda. Diz que é do Paraná. Está
desempregado, foi expulso da pensão por falta de pagamento, não dorme
há três dias, me pergunta o que fazer. É um cara de seus quarenta e
poucos anos, meio careca, bem vestido (ou pelo menos com uma roupa
equivalente à minha), fala com correção. Diz que é professor mas está
procurando emprego de garçom. Eu, que já fui professor, meto a mão no
bolso e entrego dez reais ao sujeito com votos de boa sorte.

Quando giro nos calcanhares e vou embora, vem a raiva. Sou mesmo um
otário! O primeiro malandro que me pára na rua leva dez reais com
meia-conversa! Deixei de comprar um livro de Rubem Fonseca agora
mesmo, na Feirinha do Livro, que era dez reais. Aí vem o conversador
e leva “dezinho” assim, sem mais nem menos. Espanto para longe essa
idéia, mas quando abro a janela da mente para que vá embora, uma
outra se insinua, solerte. Acabo de cometer um terrível erro. O cara
é sequestrador. Quando nos despedimos, não apertamos as mãos, eu não
disse meu nome, e ele o dele? Claro que me deu um nome falso; eu,
idiota como sou, poderia ter dito Felisberto ou Venceslau, mas não,
dei de graça a informação que ele vai usar para, seguindo-me até o
prédio em que moro, convencer o porteiro a deixá-lo entrar dizendo
que é meu amigo. (E de quebra pedindo mais 10 reais ao coitado do
João). Quando eu abrir a porta (quem disse que eu olho pelo olho
mágico?), ele vai me render com um 38, pedir 1 milhão de dólares, um
colete à prova de balas e um helicóptero.

Tudo isso dura o tempo que eu levo para passar na primeira banca;
basta ver uma capa de revista ou manchete de jornal que minha cabeça
(que tem a péssima mania de pensar o tempo todo) mude de assunto e
passe a pesar os prós-e-contras da Alca ou a teoria aerodinâmica de
Luana Piovani. Mas de noite, diante do teclado, pensando qual foi o
fato principal do dia, me vem este à memória. Admito que sou mesmo
besta. Qualquer papo me comove. Nunca fui pobre, nunca passei
necessidade, mas já me vi muitas vezes em cidade estranha, sem ter
onde dormir, sem conhecer ninguém, com dinheiro contado no bolso.
Aqui mesmo no Rio, no meu tempo de estudante, já dormi mais de uma
vez sentado num banco da Rodoviária, por ser o lugar que eu mais
conhecia na cidade inteira. Por que não ajudarei o cara?

Ninguém é tão bonzinho que não seja interesseiro. Cada vez que nossa
mão direita pratica uma boa ação, nosso olho esquerdo espreita o dia
futuro em que receberemos de volta, com juros, essa poupança
esperançosa. O cantador João Furiba conta em suas memórias como certa
noite em Petrópolis, sem dinheiro no bolso e debaixo de chuva, viu um
carro parar e o motorista oferecer-lhe uma carona. Era um sujeito a
quem ele tinha emprestado um dinheiro muitos anos antes, e que surgiu
naquela noite fria, como quem sai de dentro da cartola do Destino.
Então pronto. Se aconteceu com João Furiba, por que não aconteceria
comigo?

\chapter{A Livraria Pedrosa}

A literatura fantástica e de ficção científica usa de modo recorrente
o conceito de Portal (em inglês, “gateway”): uma fenda ou atalho no
espaço-tempo, um limite que, uma vez cruzado, transporta o indivíduo
a outro ponto do Universo, por mais remoto que seja. Uma espécie de
cabine telefônica: o sujeito entra nela em Londres, aperta um botão,
e ao sair está na Lua ou em outro sistema solar.  Havia um desses
portais na Campina Grande onde cresci. Algumas horas passadas lá
dentro equivaliam a meses ou anos passados não apenas em outros
pontos do espaço, mas em outros períodos do tempo, fosse o Brasil
colonial, a Inglaterra vitoriana ou o antigo Egito.

Agachado junto às estantes e aos balcões da Livraria, sob o olhar
sempre vigilante e sempre condescendente de Seu Pedrosa, desenvolvi
desde menino a nobre arte de ler um livro por fora, quando não
podemos comprá-lo: ler a contracapa, a orelha, o índice, o prefácio,
as legendas das ilustrações. Não aconselho esse método aos
intelectuais sérios, mas recomendo-o vivamente aos meninos de dez
anos cuja curiosidade pelo mundo está na proporção inversa da mesada
que recebem. Foi ali que desenvolvi o hábito de, indo a uma livraria,
passar o pente fino. Parede por parede, estante por estante, lombada
por lombada. Em meia hora leio o equivalente a um livro inteiro; e
então pego um volume previamente escolhido e levo-o ao caixa, para
dar ao livreiro um mínimo de compensação.

Não era a única boa livraria daquela Campina Grande. A Livraria Nova,
em frente ao Alfredo Dantas, me proporcionou muitas descobertas e
revelações; na Livraria Universal, na frente da galeria do Palomo,
comprei meus primeiros livros de cinema; a lojinha das Edições de
Ouro, ao lado do Capitólio, era uma pequena gruta de Aladim; e foi no
sebo de Câmara, perto da Varig, que descobri o “Kaos” de Jorge
Mautner e minha primeira antologia de Drummond. Mas a Pedrosa era a
soma disto elevada ao quadrado. Quando fui a Lisboa receber um prêmio
de ficção científica, tive que explicar aos amigos portugueses que
não era carioca, apesar de morar no Rio, e que conhecera a ficção
científica comprando, numa livraria do interior da Paraíba, os livros
portugueses da “Colecção Argonauta”.

O tempo passa, tão devagar quanto os cabelos pretos. Quando cruzo
aquela esquina já não vejo a Livraria, mas ainda escuto a voz de meu
pai: “Me pega na Pedrosa às duas, pra gente descer de táxi.” Descobri
que as livrarias passam, mas já tinha descoberto antes que os livros
ficam; e não será por saber disto que alguns homens se animam a criar
livrarias?  O correr da vida faz com que se fechem alguns dos Portais
que nos transportavam a outros mundos, mas é da natureza destes
portais fazer com que a gente aprenda a passar sozinho para o outro
lado. Ainda tenho livros onde continua pregado aquele selinho amarelo
dizendo: “Faça do Livro o seu melhor Amigo”. O que teria sido de mim
sem esta frase?

\chapter{Eu vou estar enviando}

É a mais recente praga que se alastra pela língua portuguesa do
Brasil, e seu principal grupo de risco são os executivos empresariais
e suas secretárias. “Pois não, senhor… Eu vou estar enviando o seu
contrato amanhã cedo…” Esta construção tão desajeitada é uma tradução
aproximada do inglês “I shall (ou “I will”) be sending you the
contract tomorrow morning…”  Se você quiser que a língua inglesa
desmorone como as Muralhas de Jericó é só retirar-lhe os verbos
auxiliares (e os pronomes também, aliás), e ninguém conjuga mais
nada. Português não é assim. Se você disser: “Enviarei seu contrato
amanhã cedo”, todo mundo entende. Mas não, o pessoal acha que é um
defeito usar uma palavra só quando podem-se amontoar três ou quatro,
e em razão disto eles vão estar amontoando esses monstrengos até o
dia do Juízo Final.

O linguajar “burocratês” tem uma longa folha-corrida de delitos
cometidos contra a beleza, a funcionalidade ou a simplicidade do
idioma; às vezes, contra os três numa tacada só.  A razão principal
disto é a tentativa de parecer chique. Se vocês prestarem atenção,
verão que no linguajar dos escritórios existe uma separação tipo
“casa grande \& senzala” entre verbos sinônimos; sempre existe um
verbo “chique” para substituir um sinônimo “vulgar”. Por exemplo: em
burocratês as pessoas não esperam: elas aguardam. Ninguém manda uma
carta: envia, ou remete. Na burocracia, ninguém pede: solicita. Em
caso de dúvida, não se deve perguntar, e sim indagar.  Os chefes não
mandam: eles determinam. E assim por diante.

Faço esta crítica porque acho que a razão dela está muito próxima à de
uma outra batalha que se trava por aí: a do uso indiscriminado de
aportuguesamentos de palavras em inglês. Eu não tenho preconceitos
nacionalistas, como aliás deve ser óbvio para quem lê esta coluna. 
Acho normal dizer saite, draive, acessar, deletar, deu um bug. Ainda
não acertei a dizer mause em vez de “mouse” mas um dia eu chego lá.
Qual é o problema, então? O problema é que estes termos não foram
criados por submissão colonizada à língua do imperialismo, mas por
simplicidade, atalho, encurtamento de caminhos para a expressão. Não
é o caso das expressões no parágrafo acima, em cujo uso eu detecto
uma angústia freudiana de parecer chique, de se distanciar da classe
social imediatamente abaixo.

O problema não é aportuguesar palavras em inglês, é escrever em
português como se fosse um inglês mal traduzido. Sou um leitor de
histórias em quadrinhos, mas a qualidade da tradução dos álbuns é
constrangedora. Prefiro pagar o triplo e ler no original, porque pelo
menos vou ter uma idéia do que os personagens estão dizendo.  Quando
deformamos, diluímos e sub-aproveitamos o português, aí sim, estamos
abrindo caminho para que outras línguas o suplantem, porque tudo que
dizemos nessas outras línguas parece fazer sentido, e a nossa própria
língua parecerá sempre uma tradução mal-feita.

\chapter{O tímido}

Li uma vez, há muitos anos, um comentário de alguém que dizia: “Os
verdadeiros heróis da Humanidade são os tímidos. Napoleão invadindo a
Rússia ou Colombo descobrindo a América não estavam fazendo mais do
que sua obrigação, seu trabalho rotineiro. Mas um rapaz tímido que
atravessa um salão e tira uma moça pra dançar, esse sim, está movendo
um Himalaia”. Li isso aos quinze anos, e podem falar mal dos textos
de auto-ajuda, mas sem o apoio moral desta singela citação eu talvez
não tivesse movido alguns dos meus himalaias pessoais.

O tímido é um sujeito que todo dia passa por dez Gênesis e onze
Apocalipses. O mundo acaba e recomeça a toda hora, e quem decide uma
coisa ou outra é o modo como as outras pessoas o tratam. A descrição
mais simples de um indivíduo tímido é: “um indivíduo que hesita, ou
que deixa de agir, porque teme que sua ação o coloque numa posição de
perigo, ou de embaraço, ou de envergonhamento público”. Woody Allen
disse uma vez que existem sujeitos capazes de fazer um papel ridículo
até mesmo quando estão sozinhos, sem ninguém olhando. É uma boa pista
para identificar a raiz desse medo pânico que o tímido tem do “ato de
agir”. Como dizia Sartre, “o inferno são os outros”. O tímido é
basicamente um narcisista, e os outros são o espelho. Ele só acredita
ser aquilo que os outros dizem que ele é.

Conheço um cara que foi morar numa pensão de estudantes e passou as
primeiras 36 horas trancado no quarto, morrendo de fome, sem coragem
de descer para o refeitório. Conheço um cara que na infância passou
uma semana sendo chamado por outro nome pela professora, que o
confundiu com outro aluno, até que ela deu pelo erro e perguntou:
“Mas por que você não disse que seu nome não era esse?” Conheço um
cara que já pagou um grande mico porque não conseguia perguntar onde
era o banheiro.  Conheço um cara que bebeu até desacordar porque o
garçon não parava de trazer chopes e ele não sabia como pedir-lhe que
parasse. Conheço um cara que… bom, melhor parar por aqui, senão o
leitor irá pensar que é de mim mesmo que estou falando.

O tímido prefere a certeza do fracasso à dúvida quanto ao êxito. Kafka
era um tímido, Bertrand Russell também. Mário Quintana e Luís
Fernando Veríssimo são prova de que até os gaúchos podem ser tímidos.
Borges prova o mesmo dos argentinos. A timidez, curiosamente, é um
dos caminhos possíveis para a genialidade, porque injeta no organismo
humano doses gigantescas de uma espécie de adrenalina analítica, uma
substância hormonal que faz o sujeito pensar dez vezes mais depressa
e agir dez vezes mais devagar. O tímido está com fome, mas passa na
frente de quinze lanchonetes e em cada uma delas encontra um motivo
para pensar que não, afinal de contas não está com tanta fome assim.
Todos os seus dramas são silenciosos, todas as suas tragédias são
íntimas, e por trás daqueles olhos meio ausentes perduram os ecos de
cem gritos do Ipiranga, e as cicatrizes de duzentos Waterloos.

\chapter{O ambicioso}

Muitos anos atrás, numa preguiçosa madrugada baiana, eu estava
assistindo um programa de \textsc{tv} em que Ferreira Neto entrevistava
Gilberto Gil. Citando algum trecho de uma canção do compositor, o
jornalista perguntou-lhe se ele praticava o que pregava naqueles
versos. Gil respondeu: “Olhe, rapaz… o artista escreve muito além do
que ele é. Ele é como o alpinista, que lança sua corda lá no alto,
para que ela se prenda em algum lugar e ele possa subir. O verso do
poeta alcança os lugares onde o poeta não foi ainda, e ajuda o poeta
a chegar lá”.

Esta definição de “ambição” me parece tão boa quanto qualquer outra,
com a vantagem adicional de prescindir de julgamentos morais, que
muitas vezes colorem erradamente nossa visão das coisas. Ambicionar é
querer ir mais longe, fazer o que alguém tentou mas não conseguiu, ou
tentar o que nem sequer foi tentado. Sem ambição ninguém escreve “Os
Lusíadas”, ninguém grava o “Sergeant Pepper´s”, ninguém pinta o
“Perna de Pau”.  Sem ambição ninguém se elege presidente dos \textsc{eua},
ninguém derruba o World Trade Center.

Dou este último exemplo para reforçar a idéia de que a ambição não é
boa nem é má em si própria. Ela é um impulso de ousadia e de
vontade-de-excesso que tanto pode levar Hitler a anexar a Áustria
como pode levar Ronaldo Fenômeno a partir do grande círculo até a
marca do pênalte, cercado por um enxame de zagueiros, e estufar a
rede sem dó nem piedade. Porque também existe uma ambição-do-Bem,
para usar esta simpática expressão tão em voga. A vontade de fazer
muito. A necessidade de que Grandes Coisas aconteçam. O entusiasmo de
nos sentirmos participando de Algo Importante e de saber que com uma
palavra nossa, um gesto, uma decisão, este Algo Importante levantará
vôo e sua presença começará a fazer bip-bip em todos os radares da
História.

Em seu conto “Novelty”, John Crowley fala de um escritor pouco
ambicioso, o qual um dia percebe “…que a diferença entre ele e
Shakespeare não era propriamente talento -- mas fibra, intrepidez. A
capacidade de não se deixar amedrontar pelas suas maiores e mais
poderosas inspirações, e de simplesmente sentar-se e pô-las no papel.
E a terrível imobilidade que o acometia quando uma idéia realmente
imensa e multifacetada subitamente tornava-se clara em sua mente,
algo com as dimensões de um ‘Rei Lear’ mas com a precisão de um
soneto. Se pelo menos tais idéias não o assaltassem assim por
inteiro, de uma só vez, enormes e perfeitas, deixando-o amedrontado e
impotente diante da perspectiva de ter que articular aquilo tudo,
cada palavra, cada cena, cada página!”

O artista desambicioso é um jogador que, diante do goleiro, procura um
companheiro a quem passar a bola. Na vida só existem dois verbos:
“ficar avaliando” e “arremessar-se”. O verdadeiro ambicioso é o que
sabe cultivar estas duas virtudes tão consangüíneas: a paciência e a
audácia.

\chapter{O impessoal}

Existe gente que tem uma vocação para ser impessoal. A gente encontra
muitos deles na Europa, principalmente nos países germânicos ou
escandinavos. Deve ser uma mistura de questão genética com questão
cultural, como aliás tudo na vida. Esses indivíduos gostam de manter
contatos frios, distantes, mas não hostis. Preferem estar sós do que
acompanhados, calados do que conversando. Sentem-se mais à vontade
com um animal doméstico do que com uma pessoa da família. 

O sujeito impessoal adora hotéis. Quando viaja, às vezes um amigo ou
parente lhe oferece hospedagem; ele dá uma desculpa, e foge para o
hotel. Gosta daquele tratamento respeitoso mas distante: “Boa noite,
senhor… pois não, senhor…” Gosta dessas gentilezas superficiais, mas
detesta quem puxa conversa. É para pessoas desse tipo que os
supermercados foram inventados, porque ali ele pode entrar sem
cumprimentar ninguém, ir direto à gôndola que lhe interessa, botar os
produtos na cestinha, ir ao caixa, pagar, guardar o troco, pegar as
sacolas, sumir na escuridão aconchegante da metrópole deserta. Para
pessoas assim, uma feira-livre num sábado de manhã é uma tortura da
Gestapo.

Conheço tipos assim. Se ao chegar no prédio ele vê um vizinho dez
metros à frente, já chegando ao elevador, pára e finge que está
pegando a correspondência, para que o outro suba sozinho, e ele não
tenha que passar dez andares comentando fenômenos atmosféricos.  Ao
ver um conhecido no metrô ou no ônibus, enfia a cara no jornal que
traz sempre à mão. A conversa banal dos que se encontram por acaso
parece ser-lhe extraída a torquês. Quando tem que caminhar dois
quarteirões ao lado de alguém, sua mente se desespera em busca de
assuntos irrelevantes que possam ser prolongados sem que nada de
pessoal precise ser dito.

“Nada de pessoal”. Esta frase parece ser uma senha para o que se passa
na alma dessas criaturas. Quando criticam alguém, quando se queixam,
quando deixam escapar um tiquinho-tanto-assim de emoção, apressam-se
a avisar: “Não é nada de pessoal.”  Não têm raiva de pessoas, e sim
de princípios morais. Quanto menos envolvimento, melhor. Gostam de
táxis, de taxi-girls, de \textsc{tv} a cabo, de compras pelo correio, de
secretárias eletrônicas. Suas amizades são como aquelas amizades
inglesas ironizadas por Jorge Luís Borges, “que começam por excluir a
confidência e que muito depressa omitem o diálogo”. Aceitam a
companhia de outras pessoas, desde que estas se comportem menos como
pessoas do que como peças da mobília. 

Os impessoais fingem que não nos viram, fazem o possível para não nos
escutar. Se ao cruzar com eles no saguão os enxergássemos como são
realmente, veríamos a enorme carapaça de um caramujo, refugiando-se
num canto, tentando proteger a lesmazinha interna que nunca conseguiu
abandonar a armadura protetora. E se encostássemos o ouvido à concha,
não ouviríamos ninguém a nos responder “bom-dia”: ouviríamos apenas o
marulho de um mar sem nome numa praia deserta.

\chapter{Adeus, gringos}

Vou passando de táxi pela Rua do Catete e vejo ao longe um
ajuntamento. Penso que foi acidente ou assalto, mas quando chego mais
perto vejo o enorme ônibus de turismo parado em frente ao Museu da
República. São os gringos, outra vez: trôpegos e felizes. Tudo que
para nós é banal serve para eles de fonte de deslumbramento ou
espanto: um guri vendendo cones de papel cheios de amendoim torrado,
garotas de 10 anos dançando a boquinha-da-garrafa na calçada de um
botequim, bandeirolas juninas penduradas entre os postes elétricos,
um mendigo exibindo a perna crivada de pinos metálicos. Com os olhos
muito abertos, e sempre cochichando uns com os outros, eles tentam
perceber tudo, registrar tudo (o clique-clique inaudível das câmeras
digitais), assimilar tudo que não pára de surgir à sua frente.

Como parecem desamparados: brancos como camarões sem casca, vestindo
roupas sempre inadequadas, tentando passar despercebidos com
artifícios como enormes bonés do Flamengo ou camisas da Seleção.
Quase todos têm mais de 60 anos e parecem estar, depois de uma
existência de trabalho duro, desfrutando de uma hora-do-recreio em
que pela primeira vez se dão conta de que existe um mundo além do
trajeto entre a casa e o escritório. Por mais que os corpos estejam
vacilantes, com as juntas emperradas, percebe-se nos seus rostos uma
alegria infantil de quem na velhice consegue uma trégua momentânea na
luta pela vida, um lazer prazeroso que não colide nem com a ética
protestante nem com o espírito do capitalismo.

Às vezes andam muito próximos, ou pegados uns aos outros, como cegos
que temem se perder na multidão. Seus rostos têm de vez em quando
aquela expressão em-branco de que não apenas não entende o que vê,
mas também desconhece a necessidade de entender algo; são como Kaspar
Hausers conduzidos pelo guia, que se responsabiliza por sua segurança
entre a porta do ônibus e a porta do Museu. Parecem tão inofensivos
que chega me dá uma vontade de ir tomar conta deles, zelar para que
voltem sãos e salvos ao hotel e ao aeroporto, ajudar na pechincha com
os camelôs, conferir suas contas nos restaurantes. Quando os vejo é
que me dou conta de como nós brasileiros somos espertos, somos
ladinos, somos raposas.

Eles vêm, maravilham-se, gastam horrores, e vão embora. Adeus,
gringos! Voltem de novo. Não cobraremos de vocês os malefícios dos
seus governos ou das suas megacorporações, mesmo sabendo que devem a
elas a facilidade com que suas carteiras se abrem. Queremos manter o
fluxo desses dólares que tanto ajudam a torrar nossos amendoins.
Queremos também a chance de achar que somos parecidos uns com os
outros, e que no futuro, quando a Viga Mestra do Sistema torar no
meio e o circo vier abaixo, poderemos ser também generosos e dividir
com vocês o chão do barraco, as sardinhas esquentadas na fogueira, e
as histórias de fantasmas e espaçonaves que contaremos uns aos outros
buscando aconchego, antes que a última noite desça sobre todos nós.

\chapter{O rico, o pobre e o sábio}

O rico gasta o que quer; o pobre gasta o que pode; o sábio gasta o que
precisa. Quem disse isto foi o Budista Tibetano, entre uma baforada e
outra de seu narguilê árabe (e não me perguntem o porquê desta salada
étnica: são os mistérios do Oriente). Um grande erro que cometemos é
julgar que mais dinheiro é sempre uma coisa positiva.  Dinheiro só
resolve alguns tipos específicos de problemas, e os efeitos
colaterais que muitas vezes traz consigo não valem a pena. Dinheiro
em excesso é como açúcar em excesso, antibiótico em excesso. Tudo
demais é veneno, já dizia minha mãe, que entendia dessas coisas
melhor do que o Oriente inteiro.

Tem gente que, mal começa a ganhar dinheiro, joga seu patamar de
gastos lá pra cima. Não é um patamar 50\% maior não, é coisa de duas
ou três vezes mais.  Ocorre muito no meio artístico, no qual muitos
anos de sofrida ralação sem resultado algum parecem de repente ser
premiados com um sucesso estrondoso da-noite-para-o-dia. O sujeito
sente-se enfim recompensado de tantas noites passadas em claro,
tantos chás-de-cadeira em salas-de-espera, tanta peregrinações pelas
redações de jornal com duas fotos e um relise, tantos malabarismos
para fazer no fim do mês o rodízio entre as contas que vão ser pagas
e as que vão ser acumuladas.

Quando menos se espera, começa a entrar dinheiro a rodo! O trabalho
decola, o cara não sabe mais onde botar tanta grana. Um cara me disse
uma vez: “Abri contas em três bancos, velho, porque um banco só não
comporta”. O cara alugaoutro apartamento no mesmo andar, para
transferir seu escritório e seus cinco mil livros. Ou compra um carro
para a mulher e dois para os filhos, no espaço de três meses. Conheço
um que fez uma festa de aniversário e pagou passagem de avião e
hospedagem para uns quarenta amigos de infância.

Uns continuam a ganhar dinheiro, outros não; estes regridem para o
estágio anterior e mergulham em depressão. Acharam que as vacas
gordas tinham vindo para sempre; quando se deram conta, estavam todas
no Spa do Brejo. E não tem coisa mais sofrida do que ter
experimentado o gostinho do dinheiro e depois ficar sem ele. Me
lembra a frase de Fellini, referindo-se à época em que “A Doce Vida”
(1960) estourou no mundo inteiro: “Pensei que o sucesso tinha
finalmente chegado, que dali para a frente minha vida seria outra.
Mas nenhum filme meu voltou a dar tanto dinheiro, nenhum chegou nem
perto. Eu pensava que aquilo era o começo do meu sucesso, e acabou
sendo o ponto mais alto de minha vida”.

O alívio de quem começa a ganhar “um dinheiro legal” é tão grande que
muitas vezes não lhe ocorre que aquilo seja passageiro, e que daí a
alguns anos ele vai voltar para a boa e velha pindaíba. É o drama de
quem toda vida foi proibido de gastar muito, e de repente sentiu-se
na obrigação de gastar demais. É como dizia o Budista Tibetano: “Não
adianta dar um milhão de dólares, a um mendigo: um ano depois, ele
vai estar te pedindo dinheiro pro cafezinho”.

\chapter{Ronaldinho Gaúcho}

Se alguém tivesse me perguntado “Você prefere que Ronaldinho Gaúcho
ganhe o prêmio da Fifa de Melhor Jogador do Mundo, ou que o Flamengo
escape do rebaixamento?”, eu hesitaria um pouco, mas diria: “Rapaz,
dê logo o prêmio ao menino, e o Flamengo que aprenda.”  Nada foi tão
justo no futebol, este ano, quanto um prêmio assim para um sujeito
que não apenas joga de uma maneira bela, mas que o faz com ênfase,
com veemência, com eufórica convicção. Ronaldinho Gaúcho parece
imbuído de uma missão no mundo: a de mostrar a todos esses
cabeças-de-bagre e espíritos-de-porco que povoam o futebol brasileiro
que é possível produzir obras de arte e ganhar jogos, sem que uma
coisa prejudique a outra.

A primeira coisa que o vi fazer no futebol (eu e o Brasil inteiro) foi
um gol (se não me engano, no Pré-Olímpico de 2000) em que ele entrou
na área em velocidade, ergueu a bola meio metro com um toquinho do
calcanhar esquerdo, e desferiu um tivuco que derrubou o goleiro pela
mera deslocação do ar.  Depois disto vieram lances memoráveis: o
banho-de-cuia que ele deu em Dunga num Gre-Nal (“banho-de-cuia”,
caros leitores de além Paraíba, é o mesmo que “lençol”), o gol
espírita contra a Inglaterra na Copa de 2002, e, este ano, o rodopio
que ele deu em cima de um zagueiro do Haiti antes de marcar o gol, no
jogo da Seleção em Porto Príncipe. (Não venham com esse papo de que
“no Haiti é fácil”. Quando um repentista faz um verso genial, tanto
faz se ele está cantando com Pinto do Monteiro ou com Zezim Buchudo,é
o verso que vale.)

No último jogo Barcelona x Real Madrid, há algumas semanas, quando
Ronaldinho Gaúcho pegava na bola havia uma sensação de arrebatamento
coletivo em todo o Estádio. Já experimentei momentos assim no
futebol, momentos em que a bola chega num jogador e nosso gesto
instintivo é ficar de pé, porque sabemos que algo grandioso vai
acontecer.  É por momentos assim que o futebol se justifica, é à
espera de momentos assim que suportamos milhares de horas de
tropeções, trancos, carrinhos, cotoveladas, maltratos à bola. 
Ronaldinho nos dá esta experiência porque nele se aliam força,
elasticidade, rapidez, domínio de bola, e principalmente ousadia. A
mesma ousadia que fazia Pelé apossar-se da bola e, em vez de
esquivar-se ao combate dos zagueiros, partir na direção deles como se
quisesse afugentá-los.

Dias atrás, ao fazer no finzinho do jogo um golaço que deu a vitória
ao Barcelona, Ronaldinho Gaúcho saiu correndo pela lateral do campo,
com o estádio inteiro gritando de forma ensurdecedora; as câmaras
mostravam em close seu riso de delírio e desabafo, enquanto ele
estava gritava: “Eu-sou-fo-da!”  Alguns jornalistas criticaram esta
reação, dizendo que era arrogância, “marra”, etc.  Discordo,
coleguinhas. Quando o cara grita aquilo, sabe que ninguém está
ouvindo. É o desabafo de quem procurou o gol durante 89 minutos e
finalmente o conseguiu, e logo um gol espetacular.  Pode gritar,
Ronaldinho, porque 2004 foi seu.

\chapter{Profissionalismo}

Antigamente as pessoas aprendiam a fazer fazendo. Entrava-se numa
profissão porque o pai e o avô já faziam aquilo, de modo que o futuro
fazedor daquilo crescia num ambiente saturado de informação,
motivação, vivência. Outras vezes, descobria-se precocemente num
garoto um certo talento para alguma coisa (tocador de alaúde,
ferreiro, domador de cavalos), e o garoto promissor era encaminhado a
um Mestre no ofício, que tornava-se uma espécie de segundo pai e
ensinador de tudo. Ou então o sujeito era forçado a trabalhar para
manter a família, e não escolhia vocação: pegava o que aparecia, e
aprendia o ofício na dura lei do dia-a-dia, da tentativa-e-erro.

Isso, no entanto, era em 1900-e-cocada. Hoje em dia, inventou-se uma
palavra mágica: o Curso. Tem curso para tudo no mundo, a maioria
deles nas universidades particulares, que brotam no solo brasileiro
mais rapidamente do que as favelas. Não nego que a idéia por trás dos
cursos universitários seja uma boa idéia. De boas idéias está cheio o
mundo e a escalação da Seleção. Resta ver o que acontece na prática.
Em teoria, o candidato a aprendiz iria passar 4 ou 5 anos de sua vida
estudando, de forma concentrada e intensiva, tudo que dissesse
respeito à sua futura profissão. Uma lavagem cerebral do-Bem. Quando
saísse dali, estaria pronto para arregaçar as mangas e atender a
clientela. Mas deve ter um bug qualquer no meio do processo, porque o
Brasil está cheio de administradores que não sabem administrar,
advogados que não advogam, filósofos que jamais filosofarão. Pode ser
um problema de mercado de trabalho; mas pode ser o fato de que em
algum trecho do processo a intenção se perdeu, e o cara não passa de
um leigo com diploma.

Hoje em dia, todo futuro artista diz aos jornais: “Eu pretendo seguir
a carreira artística, e já me inscrevi em cursos de Dicção,
Empostação da Voz, Interpretação, Canto, Dança e Mímica.” Tem curso
paratudo. E o jovem artista acredita, com toda a sinceridade dos
verdes anos, que basta fazer um curso para ficar sabendo tudo que os
outros sabem. Que me perdoem os jovens, mas eu acho que estão latindo
na árvore errada. Eles acham que passar alguns meses ou anos
cochilando na última fila de um cursinho é uma maneira prática de
adquirir conhecimentos.

Eu ainda acho que futebol se aprende no campo, jornalismo na redação,
música no palco e natação na água. Não sou contra os cursos e as
escolas: sou contra a idéia simplória de que quem fêz um curso sabe
mais da profissão do que quem não o fêz.  Tem meia dúzia de
profissões (Medicina, Odontologia, Engenharia, etc.) onde vai ser
muito difícil o sujeito ser um autodidata competente, mas estas são
as exceções, não a regra. Criou-se no mundo uma idéia de
profissionalismo que não passa de um corporativismo disfarçado. O
verdadeiro profissionalismo é o que valoriza o exercício competente
da profissão, no universo da própria profissão, independentemente de
como a profissão foi aprendida.

\chapter{Baby forever}

Perto de onde eu moro há uma pet-shop. Meu lado armorial se rebela
contra essas expressões americanizadas, mas meu lado tropicalista
reconhece que é muito mais simples dizer pet-shop do que “loja de
animais de estimação”. É curtinho. Parece um acrônimo, uma sigla.
Gosto de palavras curtas, monossilábicas, e nisso a língua inglesa é
insuperável, com jatos sintéticos de som que encerram idéias
complexas: flash, clip, round, quark, blurb… Palavrasassim são
sólidas, como um pequeno receptáculo onde a significação está bem
compacta, bem socadinha. São o contrário de palavras como
“disponibilização” ou “anticonstitucionalidade”, crivadas de afixos,
frouxas como uma correntinha de clipes.

Mas, voltando à pet-shop: acho que a mesma ternura que sentimos pelas
palavras pequenas nos é despertada pelas criaturas pequenas. Toda vez
eu paro e fico olhando, nas vitrines, aqueles cachorrinhos
rechonchudos, virando bunda-canastra uns por cima dos outros,
trocando tapinhas, dando aquelas mordidinhas de mentira que eles dão,
ou simplesmente rolando pelo chão, arreganhando as patinhas pro ar e
olhando para a gente através do vidro, com aqueles olhos marrons e
líquidos, como se dissessem: “Oi! Eu tô tão feliz! E o senhor?”

Eu estaria mais feliz, companheiro, se não soubesse que você, como
todos os outros, vai crescer e transformar-se num sabujo desmedido e
malcheiroso, com aquele rosnado de maus-bofes. Devia haver um remédio
para evitar que cachorrinhos crescessem. Um genérico-de-\textsc{dna} qualquer
que bastasse a gente todo dia pingar umas gotas no leite para
garantir que nossos filhotinhos continuariam filhotando pela vida
afora, sem risco de virar um desses cérberos ameaçadores que me
espreitam todo dia quando ando pela calçada, doidos que o dono se
distraia um pentelhésimo de segundo para voarem na minha garganta e
fazerem comigo o que Bush está fazendo com o Iraque.

Pensando bem, devia ter um troço desses para os filhos também. A gente
botava na mamadeira, e algum tempo depois na Coca-Cola, e eles
ficariam a vida inteira com três anos de idade, dando aquelas
corridinhas desajeitadas no parque, aquele risinho de rosto inteiro,
e dizendo aquelas frasezinhas trôpegas de quem está fazendo suas
primeiras incursões pelos jardins da sintaxe e da semântica. Que
beleza, hem?  Não cresceriam nunca, nunca virariam esses adolescentes
rebeldes e peludos, ou esses adultos que acham que são donos do
próprio nariz mas ainda nos pedem o dinheiro do táxi. Gotas. Umas
poucas gotinhas diárias, ou uma papeleta homeopática antes do café da
manhã, e nossos filhos seriam filhotes eternos, para nossa vaidade de
pais e nossa futura ternura de avós, que é o que seríamos deles na
velhice. Ninguém devia crescer, principalmente as crianças. Eia,
cientistas! Precisamos inventar algo para que nossos serezinhos de
estimação não venham a se transformar em gente como nós. Não sei se
vale a pena.

\chapter{Os americanizados}

Sou capaz de apostar que se entrarmos numa favela em qualquer
metrópole brasileira, de São Paulo a Recife, de Belo Horizonte a
Porto Alegre, vamos encontrar um garoto preto, subnutrido, descalço,
mastigando um picolé de morango e usando uma camiseta suja onde está
escrito: “Columbia University”.  Num bairro mais adiante, se pararmos
para tomar um refrigerante diante do grupo escolar ou do Ciep local,
vamos ver sair da aula um outro garoto, branco, ou preto, ou mulato,
mas mais limpo, mais arrumado, livros embaixo do braço, vestindo o
inevitável bermudão e usando uma camiseta onde está escrito: “Chicago
Bulls”. À noite, se entrarmos num apartamento da Zona Sul, talvez
encontremos um terceiro rapaz que poderia ser irmão mais velho dos
outros dois, ouvindo um \textsc{cd} e vestindo uma camisa com o nome de uma
banda como “Wallflowers” ou “Linkin Park”.

O primeiro menino não tem a menor idéia do que seja a Universidade de
Columbia. Veste aquela camisa porque foi a única que alguém lhe deu.
O segundo sabe quem são os Chicago Bulls: é o maior time de basquete
do mundo, o time de Michael Jordan. (Não é mais, mas é a mesma coisa
do Santos de Pelé, continua existindo) O terceiro provavelmente não
apenas sabe que banda é aquela como tem os discos, conhece os
integrantes, canta as músicas. São estágios sucessivos no que o
pessoal chama “a americanização” da juventude brasileira. O curioso é
que uns acham essa americanização a primeira trombeta do Apocalipse;
para outros ela é um grito de vitória, mistura de “independência ou
morte” com “abre-te sésamo”. 

Quando alguém veste uma camisa onde está escrito um nome que não
conhece, ele está abrindo uma janela em si mesmo. Está entrando num
lugar desconhecido, tomando posição num ritual cuja finalidade
ignora, bebendo num copo sem saber o que tem dentro. Vestir uma
dessas camisas esfregar uma lâmpada mágica sem saber o que vai brotar
dali, se é um gênio que concede três desejos ou um dragão de fogo que
carboniza o descuidado. Não importa. A curiosidade é maior do que a
cautela. Ele sabe que existe um mundo lá fora, além do seu
entendimento, e que esse mundo se exprime através de palavras em
inglês. Ele até poderia dizer: “Não, obrigado, não quero, só me
interessa o que já conheço, só me interessa o que já é o meu mundo.” 
Mas ele diz: “Eu queria saber o que é isso. Eu queria experimentar.
Me mostre.”

Abrir o próprio espírito e deixar que seja invadido pela imensidão do
mundo é o gesto mais desprendido, mais indefeso e de mais grandeza
que um ser humano pode ter, e esse gesto de quase inconcebível
coragem é mais praticado por jovens do que por homens e mulheres
maduros. A única recompensa justa para essa coragem seria dar-lhes a
escolher não apenas camisetas norte-americanas, mas que pudessem
escolher também camisetas (e mitologias) chilenas, húngaras,
espanholas, mexicanas, indianas, argentinas, portuguesas, irlandesas,
russas, palestinas, judias. O mundo é grande.

\chapter{O Grito de Munch}

Os leitores devem conhecer, mesmo de relance, o famoso quadro “O
grito”, de Edward Munch. Para quem não está ligando o nome à
“pessoa”, é aquela pintura, em cores violentas e borradas, onde
aparece em primeiro plano a silhueta alongada de uma pessoa calva,
não se sabe se homem ou mulher, vestida de preto, com as mãos tapando
as orelhas e a boca alongada num uivo silencioso que quase chega a
incomodar nossos tímpanos.  Ela está numa ponte que corta
diagonalmente o quadro; na extremidade oposta vêem-se dois vultos,
que podem ser dois meros transeuntes, mas que para mentes paranóicas
como a minha assumem uma presença ominosa, ameaçadora. Por trás disto
tudo, um céu de violentos borrões avermelhados. Depressivo,
angustiado, com uma vida cheia de tragédias, Munch vivia num
permanente tumulto mental, e em seus quadros temos um vislumbre do
mundo como ele o enxergava.

Agora, astrônomos da Texas State University, depois de cuidadosas
pesquisas, anunciam que o céu vermelho pintado por Munch era vermelho
de fato. O “Grito” é de 1893 (na verdade, há vários quadros, pois
Munch, como muitos outros artistas, produzia mais de uma versão da
mesma pintura). O prof. Donald Olson lembra que de novembro de 1883 a
fevereiro de 1884, os céus da Europa ficaram cobertos pela cinza
vulcânica da erupção do Krakatoa, onde hoje é a Indonésia, inclusive
em Oslo (Noruega), onde Munch pintou o seu quadro. Jornais da época
mencionam os crepúsculos avermelhados devido à cinza vulcânica no ar.
Os astrônomos afirmam ter localizado a ponte que aparece no quadro, e
que o ponto de vista assumido pelo pintor está voltado exatamente
para sudoeste, ou seja, a direção exata da Indonésia.

Existem dois tipos de pessoas no mundo (sei disso porque pertenço a
ambos): as que acham que tudo que é atribuído à imaginação tem uma
base factual (como o Prof. Olson e seus colaboradores), e as que
acham que todos os fatos que observamos são contaminados pelo tumulto
de idéias e emoções que borbulha o tempo inteiro em nossa mente (como
era sem dúvida o caso de Edward Munch). Este não é o primeiro nem
será o último caso de cientistas pragmáticos explicando o “por quê”
de determinados aspectos de uma pintura. Já vi teorias provando que o
sorriso da Mona Lisa se devia a ela estar grávida, ou que as cores
deste ou daquele pintor se deviam ao fato dele beber absinto. 

Saber que existia no interior de Minas um vaqueiro chamado Manuelzão,
e que Guimarães Rosa o conheceu, deve provocar um suspiro de alívio
nesses pesquisadores, que até então se deparavam com a perturbadora
hipótese de que Manuelzão não passasse de uma invenção do artista.  O
mesmo se dá com o céu de Oslo, pintado por Munch. Sabemos agora que o
céu era vermelho mesmo, e que Munch não estava inventando. Ou então,
vai ver que Deus teve a idéia de um quadro, mas, como não sabe
pintar, produziu uma erupção no Krakatoa e levou Munch até a ponte
para que ele o pintasse. Deus adora uma manobra complicada.

\chapter{O fantasma da liberdade}

Peço emprestado o título de um filme de Luís Buñuel para batizar,
neste minúsculo artigo, um dos conceitos mais perigosos da arte do
século 20.  O Fantasma da Liberdade é a falsa idéia de que quanto
mais liberdade temos, melhor. Como diria Augusto dos Anjos: “Ilusão
trêda!”.  O mundo ocidental experimentou tantos séculos de
despotismo, imperialismo, tirania, fascismo, nazismo, ditaduras e
todas as suas variantes,que a Liberdade acabou parecendo, num
contexto de regimes autoritários e sanguinolentos, um valor absoluto.
Hoje em dia você pode falar mal até de Jesus Cristo, que passa. Mas
se chegar num jornal e disser que a Liberdade não é um valor
absoluto, como eu estou dizendo agora, corre o risco de ser
crucificado. Pois tragam os centuriões! Estou pronto a me sacrificar
pela Verdade. Mas primeiro deixem-me apresentar as provas da defesa.

Prova A: o Verso Livre. Um belo dia, algum Einstein da literatura teve
a idéia de dizer: “Vamos abolir essa besteira de métrica e rima,
essas coisas que é preciso estudar! Viva o verso livre, e viva o
verso branco!”  Entendo a intenção de quem fêz isto, e concordo que
foi um avanço. Mas serviu de pretexto para que milhares de
incompetentes achassem que fazer poesia era escrever qualquer coisa.

Prova B: a Pintura Abstrata. Durante séculos, só era pintor quem
soubesse pintar figuras: gente, cavalos, árvores ou navios. Aí um
belo dia, alguém disse: “Pintura não é pra mostrar nada. Bastam as
cores, as tintas, as formas.” Esta revolução nos deu Kandinsky e
Jackson Pollock, mas nos deu também uma legião de borra-botas que
acham que basta lambuzar uma tela com qualquer coisa, em nome de
liberdade.

Prova C: o Rock de Garagem. Surgiu nos anos 80 como uma resposta da
garotada à complexidade barroca e aos excessos de pretensão sinfônica
do rock progressivo. Do punk rock em diante, os garotos começaram a
dizer: “Para fazer rock nãoé preciso cantar bem, não é preciso tocar
bem. Rock é atitude.” Em poucos anos este lema tinha degenerado em:
“Para fazer rock não é preciso cantar, não é preciso tocar, rock é
qualquer coisa.”  O corolário disso todo mundo sabe: meia dúzia de
garotos pulando, soltando uivos inarticulados, espancando
instrumentos amplificados ao máximo, e dizendo que estão exercendo
sua “liberdade criativa”.

Prova D: o Cinema Experimental (e seu clone, a Video-Arte). Para
combater o comercialismo de Hollywood e o intelectualismo do
cinema-de-autor, inventou-se um cinema sem interpretação, sem
roteiro, sem fotografia, sem produção, sem direção. O seu lema parece
ser : “A idéia na câmara, e as mãos na cabeça”, ou seja, aponta-se a
câmara em qualquer direção, registra-e qualquer coisa, e o resultado
é “arte”. Por que? Porque, para muita gente hoje em dia, “tudo é
arte”.

Eu poderia juntar outros exemplos: a Escrita Automática, a Música
Aleatória, as Instalações Conceituais, etc. Mas o espaço acabou, e
continuo amanhã.

\chapter{Liberdade demais atrapalha}

Falei ontem sobre o conceito que batizei de “O fantasma da liberdade”,
pedindo emprestado o título do filme de Luís Buñuel. O século 20 foi
chamado “o século das vanguardas”. Criou-se uma ideologia de romper
com as tradições, libertar-se das fórmulas, fazer o que desse na
veneta dos artistas. “Vanguarda” e “experimentalismo” foram as
expressões mais usadas para descrever essa atitude de procurar novas
formas de expressão.  O problema é que, quando prestamos atenção em
muita coisa que se apresenta como Vanguarda ou Experimentalismo,
percebemos que o que acontece ali não é propriamente a busca de novas
formas, e sim a mera rejeição das formas antigas. O artista faz um
esforço tão grande para ser livre que acaba ficando livre até de si
mesmo.

Esse quebra-quebra de fórmulas antigas acabou redundando numa fórmula
nova: “tudo é Arte”.  O leitor certamente já terá ouvido alguma
variante desta fórmula. “Quê que tem? Tudo é poesia”, diz o poeta que
amontoa palavras sem sentido. “Tudo é cinema”, afirma o cineasta que
liga a câmara, vai para casa dormir, e volta no dia seguinte para ver
o que a câmara filmou. “Tudo é música”, diz o pretendente a músico
cuja única habilidade sonora consiste em bater-numa-lata-véia. 
Artistas assim são os maiores defensores da liberdade artística
total, porque no momento em que se criar um mínimo filtro de
qualidade, uma mínima peneira para distinguir o que presta e o que
não presta, a primeira coisa que vai para o espaço são as obras
deles.

Não sou contra o experimentalismo. Aliás, gosto mais de maluco do que
de quem é certinho demais. Bater numa lata véia não é problema, desde
que a lata véia seja uma maluquice criativa, funcione dentro de uma
maluquice maior. Quando o sujeito é músico de fato (pense Hermeto
Paschoal, pense Jaguaribe Carne, pense Tom Zé) ele pode bater em lata
véia, soltar porco e galinha no estúdio, ligar rádio de pilha durante
o show. A doidice vem num contexto de criação, de trabalho. O que me
desanima é ver gente que, em nome de uma suposta liberdade total,
abre mão daquilo que os gregos chamam “poiesis”, e que podemos chamar
de técnica, destreza, artesanato, feitura, “craft”, habilidade,
perícia.

Os artistas que querem “romper com todas as regras”, “abolir as
fórmulas”, etc. são movidos pelo impulso (tão bonito, tão elogiável)
de querer criar alguma coisa que os exprima, que seja um reflexo de
sua relação direta com o material que estão usando. Tentam afastar-se
das regras porque temem que, ficando presos a elas, seu trabalho não
passe de um papel carbono ou uma simples derivação do que já foi
feito dezenas de vezes por gente mais experimentada e mais
competente. Mas uma obra de arte é justamente o resultado de uma
tensão entre uma força que quer ir em todas as direções e um conjunto
de regras que acomprimem, a concentram, a direcionam, e lhe dizem
para onde ela deve ir. Sem essa força e sem essas regras, não existe
Arte.

\chapter{Amigo é pressas coisas}

Dias atrás cometi uma blasfêmia, falando mal da Liberdade (“O fantasma
da liberdade”, 27 de abril). Hoje, atacarei outra vaca sagrada: a
Amizade. Tenho dezenas de amigos, gosto muito de todos, e espero não
ofender nenhum deles ao afirmar que um dos maiores males do Brasil é
a amizade. Este é um dos valores morais sobre os quais se alicerça a
nossa cultura e esta maneira de ser que tanto encanta os
estrangeiros, mas onde se apóia também toda a rede de corrupções,
trambiques, maracutaias, negociatas, golpes, e todas as “tenebrosas
transações” de que falava o poeta.

Vejam nos jornais, na \textsc{tv}. Está tudo lá, nos depoimentos, nos
telefonemas grampeados, nos bilhetinhos, nos papos a meia-voz
registrados pelas microcâmeras ocultas. “Aos amigos, tudo” -- é o lema
que impera nos corredores do Poder público e privado. Se aos
políticos é vedada a nomeação de parentes para cargos públicos, este
conceito deveria, talvez, ser ampliado para incluir também os amigos.
Afinal, muitas vezes um político deixa de lado um irmão antipático ou
um primo pouco confiável, e prefere instalar no ponto-chave da
máquina estatal aquele companheirão dos velhos tempos, cuja amizade é
à prova de fogo. (Se bem que nunca se sabe. Assim como se diz que a
fidelidade feminina é solúvel no álcool, muitas lealdades masculinas
não resistem ao peso dos zeros e dos cifrões.)

Os “vampiros” da Máfia do Sangue só fazem o que fazem porque estão
cercados de amigos. Amigos às vezes honestos e bem intencionados, e
são essas boas intenções que os perdem. “Ih, rapaz, o Lalau desta vez
extrapolou… Mas não vou abandonar um amigo numa hora como essa! Vou
destruir as provas e jurar na Bíblia que não sei de nada.”

Num ensaio de 1946 (“Nosso pobre individualismo”, em “Outras
inquisições”), Jorge Luís Borges comenta o abismo ético existente
entre a moral anglo-saxônica proposta pelo cinema americano e a ética
do compadrismo cultivada pelos argentinos.  Diz ele que o argentino
só acredita em relações pessoais. Como o Estado é uma entidade
abstrata, ele não crê em sua existência, e não acha que seja um crime
roubar dinheiro público. E exemplifica: “Os filmes elaborados em
Hollywood propõem repetidamente à nossa admiração o caso de um homem
(geralmente um jornalista) que conquista a amizade de um criminoso
para entregá-lo depois à Polícia. O argentino, para quem a amizade é
uma paixão e a Polícia uma máfia, sente que este herói é um
incompreensível canalha.” Este exemplo cristalino comprova minha tese
sobre a brasilidade dos argentinos, ou a nossa própria argentinidade.
Achamo-nos devedores de favores e lealdade aos nossos amigos, não a
essa abstração chamada Brasil.  Nosso “contrato social” é no interior
de um clã. De uma família ampliada; de uma tribo; de um clube de
pessoas unidas por projetos coletivos de ascensão social e de
desfrute das boas coisas da vida.  O navio é este convés onde tomamos
drinques ao sol; o resto pode afundar, tamos nem aí.

\chapter{Jovem para sempre}

Uma canção emblemática da minha geração era “Forever Young”, de Bob
Dylan (do álbum “Planet Waves”, 1974). Quem não se deixaria seduzir
por uma canção intitulada “Jovem para sempre”?  Era tudo que a gente
queria. Aliás, não só nós, mas também aqueles alquimistas medievais e
conquistadores ibéricos que procuravam o elixir ou a fonte da Eterna
Juventude. Mal sabíamos nós que estávamos vivendo as décadas iniciais
de um processo que Joseph Epstein disseca sem pena num artigo no
“Weekly Standard” intitulado “The Perpetual Adolescent”.  Ele lembra
que antigamente a vida tinha a estrutura de uma peça aristotélica:
começo, meio e fim.  Cada uma dessas partes tinha prós e contras, mas
a parte do meio, a vida adulta, era a parte mais séria, onde as
coisas mais importantes aconteciam. A juventude era um estágio.

Hoje, no entanto, a juventude não é vista apenas como a parte bonita
da vida: é um objetivo, uma condição que a ciência, a moda e a
indústria farmacêutica procuram tornar permanente. Cinema, música, a
cultura de massas em geral, tudo pressiona o indivíduo a fazer o
possível para não se despregar dessa época em que o indivíduo
conquista a liberdade para consumir o que quiser: bebidas, cigarros,
automóveis, revistas, \textsc{cd}s, jeans, camisas, sexo, drogas,
rock-and-roll, tudo que o Mercado oferece. Por outro lado, é a fase
da vida em que a energia, a ambição e a audácia são considerados
valores absolutos. Epstein cita um empregado da Enron que, após a
falência catastrófica da empresa, comentou: “O problema é que ali não
tinha ninguém que fosse adulto”.

A publicidade nos faz crer que podemos ser sempre jovens; e mais, que
devemos, e que queremos ser sempre jovens. A juventude é definida nos
seus aspectos cosméticos: rostos sem rugas, corpos musculosos, roupas
da moda, energia para praticar esportes e consumir drogas. Voltamos,
curiosamente, ao mundo dos alquimistas medievais e suas poções
mágicas. Quer ser eternamente jovem?  Beba isto aqui. Vista isto
aqui. Use isto aqui.

A canção de Dylan, curiosamente, dizia: “Que Deus te abençoe e te
ampare sempre. Que teus desejos se realizem, e que você possa fazer
aos outros o que eles lhe fazem. Que você construa uma escada para as
estrelas, e possa pisar em cada degrau. Que você cresça para ser
justo, para ser verdadeiro. Que você sempre veja as luzes e perceba a
verdade à sua volta. Que você tenha coragem, e saiba manter-se firme
e forte. Que suas mãos estejam sempre ocupadas, e seus pés sejam
sempre ágeis. Que você tenha alicerces sólidos quando os ventos da
mudança soprarem. Que o seu coração seja sempre alegre, e sua canção
possa ser sempre cantada… e que você seja jovem para sempre.” 
Somente hoje (será tarde demais?) eu percebo que o que Dylan estava
nos desejando era que amadurecêssemos sem medo, ficássemos adultos
sem remorsos, envelhecêssemos com sabedoria, e recebêssemos a morte
com a mesma humildade com que recebemos a vida.

\chapter{A carta traz o carteiro}

Há uma expressão muito usada em inglês, “the tail that wags the dog”,
o rabo que balança o cachorro. Usa-se muito para indicar qualquer
caso em que o efeito produz a causa, ou o empregado manda no patrão,
ou algo que devia ser um mero complemento acaba ganhando mais
importância do que a parte principal. O livro “Partículas de Deus” de
Scott Adams (o criador das tirinhas de “Dilbert”, de sátira ao mundo
da informática) tem um episódio que ilustra de forma interessante
este conceito. Nele, um entregador dos Correios leva um pacote até o
endereço de destino, onde é recebido pelo dono da casa. “Trouxe este
pacote para o sr.”, diz o carteiro. O homem retruca: “Você tem
certeza de que foi você que trouxe o pacote? Pois eu acho que foi o
pacote que trouxe você até aqui, porque sem esse pacote e esse
endereço escrito nele você jamais teria vindo até a minha casa.”

É a mesma inversão que sempre me chamou a atenção nos versos iniciais
do poema “Indicações”, de Carlos Drummond: “Talvez uma sensibilidade
maior ao frio, / desejo de voltar mais cedo para casa. / Certa demora
em abrir o pacote de livros / esperado, que trouxe o correio.” Os
coleguinhas de pendores mais gramaticais podem argumentar que se
trata apenas de uma inversão, aceitável, da ordem normal da frase
(“…que o correio trouxe”), mas este verso, escrito desta forma, me
chamou a atenção para o fato indisputável de que, se os carteiros
trazem as cartas do ponto de vista físico, são as cartas que trazem
os carteiros num sentido mais amplo do termo.

Podemos visualizar essa dupla questão com mais facilidade se pensarmos
numa imagem mais simples. Um homem viaja a cavalo. É ele quem leva o
cavalo, ou o cavalo quem o leva? Mais uma vez temos as duas
respostas, ambas verdadeiras. Ou um motorista que dirige um carro -- o
que dá um nível a mais à velha frase de pára-choque: “Dirigido por
mim, guiado por Deus”.

Talvez isto possa nos ajudar a encarar a velha questão do livre
arbítrio. Somos livres para decidir, ou Deus já decidiu por nós?
Fazemos a nossa vontade, ou já está tudo escrito-nas-estrelas?
Permitam-me os coleguinhas religiosos fazer uma comparação meio
herética -- o Livre Arbítrio Cósmico e o Futebol. Digamos que o Livro
do Destino prevê tudo que vai acontecer, mas apenas numa
escala “macro”, enquanto que nós temos a ilusão de estar fazendo o que
nos dá na telha. O Livro do Destino, meus camaradinhas, diz qual vai
ser o resultado da partida. Treze 3, Campinense 1.  A correria, os
esbarrões, o esforço, o esfalfamento dos jogadores, tudo isto lhes dá
a ilusão de que são eles que estão decidindo os acontecimentos, mas
na verdade o Livro do Destino está preocupado apenas com o placar
final de cada jogo. Quem faz o gols, a que hora, de que jeito… são
bobagens irrelevantes para o Grande Plano Cósmico. E é nessa
dimensãozinha irrelevante que funciona nosso limitado livre arbítrio,
que se define nossa vida, e que acontece nossa felicidade.

\chapter{O charme de um dia nublado}

Nada tenho contra o sol, contra esta cascata dourada de raios de fogo
que parecem tornar o mundo inteiro mais colorido, mais vivo, mais
vibrante de uma energia alegre e boa. Mas pergunto: por que motivo os
Adoradores do Sol, essa multidão monoteísta que se acotovela nas
praias e nas piscinas, é incapaz de reconhecer a beleza e a poesia de
um dia nublado? O sol recorta contrastes lancinantes, fende o mundo
com suas lâminas, e o deixa fatiado em placas de luz e de sombra. No
dia nublado, a redoma de nuvens filtra e esbate esse brilho
excessivo. O mundo fica tomado por uma luminosidade leitosa, espessa,
macia. É uma luz que parece vir de todas as direções, que não projeta
sombras, uma luz democrática e onipresente, a única capaz de mostrar
o mundo como ele realmente é.

Nada tenho contra o Sol, repito. Admiro-o como admiro um leão, um
tigre: ele lá e eu cá.  Reconheço sua beleza e sua importância, mas
francamente, não preciso da companhia dele o tempo todo. Tá liberado,
companheiro!  Vá aquecer os fiordes da Escandinávia, vá dourar os
trigos da Suécia, vá bronzear Bjork. Eu por aqui vou indo muito bem,
tomando banho-de-lua, como Celly Campello. O sol é um uísque-caubói
duplo, e quem sou eu para negar seus méritos?  Tem seus momentos, sem
dúvida, mas para o correr normal dos meus dias prefiro um vinho
suave, um crepúsculo roxo-lilás com nuvens e estrelas.

Gosto de “dias brancos” como os de Geraldo Azevedo \& Renato Rocha,
como gosto das “noites brancas” de Dostoiévsky. Gosto de ver a cidade
trancada nesta caverna de claridade uniforme, ao abrigo daquela
fogueira nuclear que nos cresta a retina e nos esturrica a pele.
Gosto ainda mais do ar frio que geralmente sopra nesses dias, um
friozinho gostoso que nos faz procurar o conforto de um casaco, e o
calor aconchegante da companheira, porque tudo passa a ser pretexto
para enlaçar-lhe a cintura, colar corpo com corpo, acelerar o sangue.
É um ar fino, que clareia os pulmões; como se todos os Bancos do
mundo tivessem desaparecido e deixado atrás de si apenas o ar
condicionado, a única coisa que têm de bom.

Dêem-me dias brancos, dias nublados, dias propícios à meditação e à
paz, ao cultivo das emoções tranqüilas e dos afetos prolongados, e à
contemplação da Terra sem o clamor ensurdecedor das fornalhas do Sol.
 Dêem-me esses dias parecidos comigo, esses dias que vibram no meu
diapasão contemplativo e sereno. Podem ficar com os outros -- e isto
me alegra duplamente, porque sei o quanto farei feliz a Humanidade,
distribuindo-lhe dias ensolarados às mancheias. Mas guardarei para
mim essas moedas de modesta prata, este céu com nuvens brancas de
Chagall ou cinzentas de El Greco, este meio-dia no inverno da Serra
da Borborema onde minha alma se formou, e onde aprendi que é possível
haver no mundo beleza sem alarde, alegria sem frivolidade, e paz sem
tédio.

\chapter{O ananás de ferro}

Quando eu era pequeno passava as férias em Recife, na casa de minha
avó paterna, Vó Clotilde, uma velhinha muito esperta, parecida com
Agatha Christie. Tão parecida que foi ela mesma quem me aplicou a
obra da Dama do Crime. Aos 10 anos li “O caso dos 10 negrinhos”, para
descontentamento de minha mãe quando descobriu que Vó tinha me dado
um livro tão maquiavélico. Bobagem: pouco depois ela própria estava
com a cara enfiada no livro, e achando o máximo. Todas as minhas
leituras dessa época foram edificantes, mas poucas o terão sido tanto
quanto um conto desconhecido de um autor obscuro, numa antologia
chamada “Os Mais Belos Contos Alucinantes” (que garoto resiste a um
título assim?)

O conto era “O ananás de ferro” de Eden Philpotts, e tenho uma teoria
pessoal de que foi ele quem inspirou a Jorge Luís Borges a obra-prima
“O Zahir”, a história do objeto inesquecível, o objeto que ocupa a
mente de alguém e não pode mais ser desalojado dali. Borges
certamente leu este conto, pois em seus ensaios elogia o romance mais
famoso de Philpotts (“The Red Redmaynes”, de 1922), e publicou conto
seu na antologia “Los Mejores Cuentos Policiales” (2ª série, 1951). 
Naquele conto, o protagonista é um sujeito meio obsessivo, que se
deixa facilmente dominar por idéias fixas. Um dia ele avista, na
cerca de ferro de uma propriedade próxima, uma fileira de ananases de
ferro que servem de adorno. E ele fica obcecado por um deles. São
vários, e todos iguais; mas o que o fascina é o terceiro do lado
norte do gradil. Por que? Ele não sabe. Só sabe que aquele objeto
insignificante tornou-se a coisa mais importante de sua vida.

Diz ele: “Pensava nele como um ser sensível; considerava-o uma
criatura que podia sentir, sofrer e compreender. Nas noites úmidas
imaginava que o ananás de ferro devia sentir frio; nos dias de calor
receava que ele estivesse sofrendo com o sol de verão! Da comodidade
e conforto de minha cama, imaginava-o acorrentado ao seu pedestal
solitário no meio da escuridão. Quando caía uma trovoada, tinha medo
que um raio atingisse o ananás de ferro e o destruísse para sempre.”

O conto tem um desfecho banal (um crime é cometido), mas o seu valor
está em registrar esta curiosa emoção humana chamada paixão. Quando
nos apaixonamos por outra pessoa, justificamos esta paixão com uma
porção de explicações lógicas (identificação de espíritos, atração
sexual, admiração recíproca, auto-estima social, etc.), como se tudo
isto fosse a “causa” da paixão. Quando nos apaixonamos por um ananás
de ferro é que percebemos, sem o adorno dessas racionalizações “a
posteriori”, o quanto a paixão não tem causas racionais. É absurda e
ao mesmo tempo revestida de uma lógica inflexível; gratuita, e ao
mesmo tempo auto-justificada. Nossa alma está inexplicavelmente
acorrentada àquele ser onipresente. Não o entendemos e não o
esperávamos, mas esse obscuro objeto de desejo transformou-se na
coisa mais importante de nossa vida.

\chapter{A Larva Eletrônica}

Existe dentro de cada um de nós uma pequena larva, um embrião de
ectoplasma esperando para crescer. É minúscula: não passa de um
filete que sobe ao longo da medula espinhal, e que quando alcança o
cérebro se ramifica numa árvore fractal, ao longo das cadeias de
neurônios onde nossa consciência habita e pulsa.  Esta larva faz
parte dessa consciência, tanto quanto os sistemas automáticos que,
independentes de nossa vontade, fazem nosso coração pulsar, nosso
estômago e nossos intestinos funcionarem, nossa perna dar um pinote
quando o médico nos martela o joelho. Essa larva é a nossa
consciência imagética, e durante milhões de anos viveu em nós
adormecida.

Em mim ela deve ter despertado na primeira vez em que vi minha imagem
na televisão, andando falando, sorrindo, respondendo perguntas com
minha voz, debatendo com minhas idéias. Não era meu corpo; meu corpo
estava aqui diante da \textsc{tv}, o que eu podia confirmar, apalpando-me. Era
o corpo de quem, então? Hoje sei a resposta: era o corpo “dela”, da
minha Larva Eletrônica. Ela é uma alma que existe no meu corpo, mas o
corpo dela não é feito de carne e osso, é feito de sinais
eletrônicos, cuja vibração de som e imagem é capaz de despertá-la.
Indiferente à minha imagem no espelho, a Larva é ativada pela minha
imagem da \textsc{tv}, estremece, desperta, sente-se viva. 

Talvez venha daí essa minha fascinação em me ver na telinha, essa
sensação de alívio, de que agora sim, finalmente, graças a Deus:
despertei. Todos os meus momentos de vida meramente biológica são um
perambular de sonâmbulo. Só desperto de verdade (meu deus, é a Larva
que está escrevendo estas linhas?) quando meu corpo surge na \textsc{tv},
brilhando em seus pixels reluzentes de códigos digitais e relâmpagos
eletrônicos. Claro que Braulio Tavares continua existindo fora desses
momentos, continua a funcionar vida afora como um bom mamífero
antropóide, como um ator que cumpre seus papéis sociológicos. Mas
quando ele se vê na televisão, quando ele “me” vê, ele respira fundo
e se sente existindo por completo.

Sim, sei que é uma teoria mirabolante, mas somente ela explica que
tantas pessoas sejam capazes de tantos absurdos; que sejam capazes de
negar tudo o mais em si, que sejam capazes de inimagináveis
concessões, pactos, servidões clandestinas, auto-violências morais.
Tudo isto para que a Larva possa se ver em seu espelho colorido. Tudo
isso para que sintam em seu cérebro e sua medula espinhal o fremir da
Criatura. Ela veio embutida em nosso \textsc{dna}, talvez como uma combinação
casual, mas que começou a ser despertada pela pintura, pela
escultura, pela fotografia, pelo cinema, para finalmente brotar,
viva, inteira e completa, como se a cabeça de Júpiter se abrisse e
dali brotasse não a deusa Minerva, mas uma Górgona eletrônica que se
rejubila em sua existência híbrida e brada com voz cavernosa: “Eu
existo! Agora eu sei que existo! Eu me vi na \textsc{tv}!”

\chapter{O mundo e o computador}

Por que motivo tantas pessoas, geralmente homens, se viciam em
computador? Eis um mistério que outras tantas pessoas, geralmente
mulheres, se esforçam em vão para desvendar. Tenho um amigo que
acabou um noivado por causa do computador. Ele acordou às 8 da manhã
de um sábado, e ligou o \textsc{pc}, porque precisava pegar algo na Internet
para resolver um problema de vírus. Às 10 a noiva ligou: “Não esqueça
que hoje tem o casamento do meu irmão.” Ele: “Pode deixar.”  Ao
meio-dia ela ligou de novo: “Me pega às 4, o casamento é às 5.” Ele:
“Tá combinado.”  Às 2 da tarde, ela voltou a ligar: “É melhor eu
pegar você. Passo aí às 4.” Ele: “Tudo bem.” Ela ligou do celular às
4:30: “Estou aqui em baixo.” Ele mandou subir. Ela subiu, abriu a
porta com suachave, e o encontrou sentado diante do computador, de
cueca, em jejum, a barba por fazer, os dentes por escovar. Quando ela
começou a chorar e esbravejar, ele disse: “Mas, por que você não
avisou que o casamento era hoje?”

Mania?  Psicose? Não sei, só sei que muitas amigas já choraram
metaforicamente no meu ombro suas lamentações pelo fato do marido
passar as madrugadas pulando de saite em saite, ou esperando horas
pela chegada de um arquivo que, uma vez instalado, revela ser apenas
um protetor-de-tela com espaçonaves atirando umas nas outras.  Não
adianta dizer que muitos saites são utilíssimos: têm textos difíceis
de conseguir, letras de rock progressivo, teses de doutorado de
universidades escocesas, estatísticas sobre alpinismo ou Fórmula-1,
mini-câmaras mostrando como está o trânsito naquele momento na Quinta
Avenida. Se fosse só isto, estava explicado; mas tem gente que passa
a noite testando programas, instalando fontes, otimizando o disco
rígido.

Acho que o que mais nos seduz num computador é o fato de sabermos que,
ao contrário do Universo, tudo nele foi colocado com um propósito.
Nada num computador se deve ao Acaso; pode até se dever a um erro ou
à burrice de quem programou, mas foi posto ali por alguém. Quando
examinamos a fundo um mistério qualquer da natureza, nosso trabalho
não é o de um detetive que tenta reconstituir o raciocínio do “autor
do crime”. A Natureza é um crime sem autor. O Universo não tem
propósito: ele simplesmente aconteceu. Num computador, não. Quando a
gente tem dificuldade de executar um programa, ou de instalar um
jogo, ou de anexar um arquivo a um email, a dificuldade é nossa, mas
a coisa funciona. Foi feita para funcionar. Se a gente insistir,
acaba descobrindo.

Esta humilde esperança teleológica é o queijo que nos atrai à ratoeira
cintilante onde, uma vez presos, vemos se esvaírem as horas, os dias,
os noivados. Um computador é um labirinto, mas se encontrarmos o
caminho certo, chegaremos onde queremos.  Já o mundo, ou o coração
feminino, nada nos garantem. Ah, se as noivas, ou o Universo, nos
dessem esta mesma certeza: a de que a resposta existe, e que tantas
noites de perguntas não foram em vão.

\chapter{Música e brodagem}

Música moderna, pra mim, é Marcelo D2 mandando um \textsc{mp3} pra Fred 04. Ela
não é somente o conjunto de melodias cantadas e letras ditas, mas
toda a estrutura de tecnologias e brodagens que sustentam essa
música, e lhe dão significação e sabor.  Caso o leitor não conheça o
termo “brodagem” (o Dicionário Houaiss, por exemplo, não o registra),
basta dizer que vem do inglês “brother”, irmão, e indica o sentimento
de irmandade que impregna os grupos, geralmente jovens, envolvidos na
criação, produção e circulação dessa música.

É interessante que uma gíria “pop” como brodagem sugira termos de
ressonância tão medieval ou renascentista como Fraternidade ou
Irmandade. Grupos com estes rótulos postulam ser uma confraria de
iguais, onde os cargos hierárquicos servem apenas para simplificar as
tarefas administrativas. Fraternidades clássicas como os Rosacruzes
ou a Maçonaria se organizaram à margem do poder secular e religioso;
eram comunidades alternativas de livre-pensadores que se sentiam
incomodados com a centralização e verticalização do poder do Vaticano
e dos Imperadores. Eram grupos marginais, independentes,
alternativos: todos os adjetivos que usamos hoje para descrever
tantos movimentos culturais e artísticos de nossa própria sociedade.

Rico gosta de multiplicar, e pobre gosta de dividir. A brodagem é o
recurso dos que têm pouco mas encontram alguém que, também tendo
pouco, acha um jeito de distribuir. Uma banda que toca de graça em
todas as faixas do \textsc{cd} de um cantor amigo não é muito diferente do
grupo de vizinhos que faz um mutirão-de-domingo para “assentar a
laje” na casa de Fulano. Favores recíprocos criam laços de
camaradagem, de gratidão. Criam um crédito de ajudas que dispensam
registro contábil. No espírito da brodagem, um favor não é uma dívida
a ser cobrada no futuro: é um gesto de carinho e confiança que será
retribuído com prazer na primeira oportunidade.

A brodagem não nasceu agora. O fotógrafo Mário Carneiro disse nos anos
60: “O Cinema Novo brasileiro é acima de tudo um fenômeno de
amizade.” Ou seja: um grupo de amigos que queriam fazer as mesmas
coisas.  Amizade e criatividade alimentavam-se mutuamente, como na
Nouvelle Vague francesa, no movimento Underground americano, no atual
Dogma escandinavo. É curioso ver tais comportamentos coletivos
brotando no interior de atividades como o Cinema e a Música
Fonográfica, que são por natureza industriais, hierárquicas, sujeitas
a todas as pressões capitalistas para maximização dos lucros, etc.
Ser um artista independente num esquema como este é quase impossível.
Por que não criar, então, uma Fraternidade de Mentes Autônomas que se
ajudam entre si?  A brodagem não é um conceito novo. Ainda bem que
não o seja, porque isto talvez seja uma pista de que é no fundo a
brodagem que tem mantido viva a possibilidade de criação artística ao
longo de todos estes milênios.

\chapter{O rock nos tempos do capitalismo}

Quando o rock surgiu nos \textsc{eua}, década de 1950, a “música jovem”
norte-americana parecia a música desses programas da \textsc{tv} italiana de
hoje: artistas brancos, bem vestidinhos, bem penteadinhos,
interpretando cançonetas de amor com acompanhamento de orquestra. Uma
música de mauricinhos \& patricinhas, uma música bem comportada,
trazendo aos ouvintes a idéia de uma América mítica: limpa, decente,
bem alimentada, com dinheiro no bolso e carro na garagem, cultivando
valores religiosos e obedecendo à moral e aos bons costumes.

A verdadeira América não era nada disso, era um caldeirão de
injustiças sociais, desde a miséria dos brancos desempregados até o
apartheid racial nos Estados do Sul. Essa mistura de pólvora com
gasolina pegou fogo nos anos 1960. Por trás da fachada feliz, o país
estava profundamente doente, e o rock serviu ao mesmo tempo de
sintoma, diagnóstico e remédio.  Bebendo nas fontes da música negra e
da música rural, o rock foi veículo para a desobediência civil da
plebe rude, para o protesto esquerdista dos universitários, e para o
bundalelê geral do sexo e das drogas, que alargou a rachadura entre a
geração dos pais e a dos filhos.

Mas isso foi só no início. O apartheid oficial foi extinto, mas o
preconceito racial continua. A luta política teve algumas vitórias (a
deposição de Nixon, a retirada do Vietnam, os acordos nucleares) mas
agora está por baixo, com o recrudescimento da direita
militarista-evangélica na administração Bush. Quanto ao sexo e às
drogas, aconteceu o inevitável: foram apropriados pelo capitalismo,
viraram mercadoria. Hoje, no balcão da indústria musical, mauricinhos
e patricinhas se alternam com os que fazem pose de irreverentes,
sexy, contestadores, rebeldes… Para cada Celine Dion existe agora uma
Britney Spears ou Shakira, para atrair as garotas que têm vocação pra
doidona.

John Lennon, já grisalho, desabafou certa vez: “O rock não mudou porra
nenhuma. Tá todo mundo usando cabelo grande, mas quem manda no mundo
ainda são os mesmos caras.” Tinha razão. Depois dos anos 1960, o
principal movimento de contestação social dentro da música americana
foi o rap e o hip-hop dos anos 1990 para cá, e este também já está
sendo devidamente assimilado e faturado pelo sistema capitalista. O
Capitalismo é uma espécie de Rei Midas que transforma em ouro tudo
que toca. Melhor do que fuzilar os inimigos é suborná-los com a venda
de milhões de discos, mansão, limusine, mulher de graça, droga à
vontade, inflação do ego, bajulação da imprensa. Não tem contestador
que aguente. Toda vez que a crosta do Capitalismo é rachada por um
movimento de contestação, a lava fumegante da crise social emerge
pelas fendas; mas em pouco tempo essa lava esfria, se solidifica, se
aquieta, e passa a fazer parte da mesma crosta capitalista que tinha
ajudado a romper. Até que o bicho começa a pegar de novo, e tudo
recomeça, e o rock rola.

\chapter{O forró universitário}

O que leva rapazes nascidos e criados na Zona Sul do Rio de Janeiro
(ou em seu equivalente em qualquer metrópole), rapazes que têm acesso
a qualquer tipo de música internacional, rapazes expostos a todo tipo
de modismo da imprensa, a se voltar para o forró nordestino, uma
música esnobada por tanta gente? Parece um contra-senso, e vendo
esses grupos do que hoje se chama Forró Universitário tenho a
sensação de estar diante de um oxímoro sociológico, um paradoxo, uma
contradição. É um pouco como ver heavy-metal com letras evangélicas.

Nenhum deles faz “forró universitário” para ficar rico, ou para
atingir o mercado internacional. Acredito que eles fazem porque
gostam. Poderiam estar ouvindo e imitando qualquer banda
internacional que esteja em evidência, mas passam tardes inteiras
ouvindo \textsc{cd}s de Jackson do Pandeiro ou de Luiz Gonzaga, tirando
harmonias, e acostumando-se com o vocabulário estranho daquelas
letras. Já ouvi muitos dos \textsc{cd}s que produzem; alguns são bons, outros
são bem fraquinhos (poética e musicalmente), mas em todos eles eu
vejo uma pureza de intenções que sinceramente não consigo ver na
maioria das bandas de forró superproduzido que tocam em trios
elétricos e a rigor não se distinguem das bandas de lambada ou axé.

O Forró Universitário exprime um desejo, que tenho percebido nesse
pessoal com 20-e-poucos anos, de conhecer a vida rural, a música
rural, a visão do mundo rural. É algo que retorna ciclicamente, e
cada vez com mais força; desde que nos anos 70 Sá \& Guarabira
inventaram o conceito de “rock rural”. O forró universitário obedece
ao mesmo impulso. A rapaziada está de saco cheio do shopping, da
buate, do automóvel, do rock, da praia. Querem embrenhar-se mato
adentro, tomar banho de rio, acampar, ouvir passarinho cantando,
aprender a assar batata na fogueira, pedir um caneco dágua na
casinhola do matuto, ouvir “causos” e ponteados de viola. São rapazes
e moças com uma tendência riponga que nunca vai desaparecer; a esta
altura, o Projeto Genoma já deve ter identificado o gene responsável
pelo modo-de-ser “bicho grilo”. 

Isto tem ganho uma importância maior nas últimas décadas, quando o
Brasil deixou de ser um país rural e tornou-se urbano. Não sei os
números do \textsc{ibge}, mas fala-se que cerca de 70\% de nossa população
está nas cidades. E a neurose urbana acaba pegando. Há uma rapaziada
que vê no mundo rural um universo mais simples, mais sincero, de
valores mais humanos; e vê no forró uma expressão legítima deste
mundo. Os rapazes querem usar alpercatas, e não tênis Nike; as moças
querem saiona de pano fino, e não as mini-saias de griffe. Não querem
ir à Disney ou a Miami: querem andar de barco no rio São Francisco,
querem conhecer a Chapada dos Guimarães. Não são, a rigor, herdeiros
de Gonzagão e Jackson. São os herdeiros eternos de Janis Joplin, do
Grateful Dead e de Crosby, Sills, Nash \& Young. Que o Deus do mato
os guie e os proteja, pois estão precisando.

\chapter{A força da Tradição}

A Tradição gera a Vanguarda, e gera o Mercado. Ninguém faz trabalho
que não seja em cima de uma tradição, mesmo quando nega ou parece
ignorar que essa tradição existe. Os performáticos-de-bienal, por
exemplo, que fazem um esforço danado para dizer às pessoas que não
esperem deles uma natureza-morta pintada a óleo, estão
aproveitando-se de uma tradição de “instalações” que remonta no
mínimo aos dadaístas de depois da \textsc{ii} Guerra Mundial.

As Tradições são sempre específicas a cada atividade. Quando um garoto
de cabelo verde e piercing nas pálpebras passa o dia praticando
escalas de guitarra, ele certamente está seguindo uma tradição
qualquer, seja a de Van Halen, seja a de The Edge. Mesmo quando se
trata dessas bandas punk em que os garotos compram os instrumentos
hoje e estream amanhã sem saberem tocar, existe uma tradição punk de
fazer isto. Os gestos estéticos fundadores, originais, são raros. Os
que dão certo passam a ser a vanguarda; mas sempre existe uma
Tradição, que é o chão onde a Vanguarda pisa.

Os jovens são os mais desconfiados com a palavra Tradição, que para
eles tem um cheiro de coisa velha, arcaica, superada. O que é uma
grande bobagem, pois quando o sujeito é jovem todo o restante é mais
velho do que ele. Quando um jovem artista abre os olhos para o mundo,
para onde quer que ele olhe ele só vê a Tradição, só vê O Que Foi
Feito Antes, assim como quando olhamos para um céu estrelado não
vemos essas estrelas do jeito que elas são agora, mas do jeito que
cada uma delas era milhares de anos atrás. Sabemos que algumas dessas
constelações já se desarrumaram, que algumas das estrelas que
produziram essas luzes já se consumiram em cinza nuclear; mas para
efeitos práticos, como a navegação marítima, elas continuam servindo.
Assim é a Tradição.

A Tradição corre o perigo de ser restritiva e sufocante quando tentam
torná-la uma coisa sagrada, como ocorre às vezes com a cultura
popular, o folclore. Já que é Tradição, baixa-se uma lei dizendo que
não pode mais mexer nisso, naquilo, naquilo-outro. Sociedades
vagarosas, reacionárias, também usam a Tradição como pretexto para
boicotar novidades. O resultado é que surgem movimentos de vanguarda
radicais, violentos, que tentam enxovalhar a Tradição,
ridicularizá-la, livrar-se desse peso insuportável. É compreensível
esse niilismo, mas ele é sintoma passageiro, é distorção menor. 
Tradição é de todos, é a memória, é o Passado e o Presente. O poeta
Pablo Neruda, num poema famoso, disse do dicionário: “Dicionário, não
és tumba, sepulcro, féretro, túmulo, mausoléu, senão preservação,
fogo escondido, plantação de rubis, perpetuidade vivente da essência,
celeiro do idioma.”  A Tradição é tudo isto, e para o que fazemos
agora nada seria mais honroso do que ser um dia incorporado por ela.
A Tradição que herdamos é a soma final de tudo que era forte, de tudo
que sobreviveu.

\chapter{O funil da ciência}

Eu morava num décimo andar, numa rua do Catete. Estava escrevendo à
máquina, aí peguei uma folha de rascunhos, amassei, fui à janela e
joguei a bola de papel lá embaixo. Um gesto mal-educado, confesso,
mas o texto estava tão ruim que eu perdi a paciência. A bola
descreveu uma curva, levada pelo vento, aí ricocheteou na carroceria
de um caminhão parado em frente à loja de móveis, e entrou pela
janela de uma camionete que passava a toda velocidade. Pronto! Lá se
foi o cara, com meu artigo amassado dentro do carro.

Qual a probabilidade matemática de que isto acontecesse de propósito?
Um tanto remota. Poderíamos reconstituir a cena interditando a rua,
pedindo ao cara da camionete que ficasse passando ali àquela
velocidade, enquanto eu, lá de cima atiraria bolas e mais bolas de
papel tentando acertar o caminhão na hora exata, etc. São muitas as
variáveis envolvidas: o ângulo e a força do arremesso inicial, a
direção e a força do vento, o ângulo e o ponto exato da “tabela” na
carroceria do caminhão, a passagem da camionete no ponto certo e no
momento exato, etc. Mas se a gente dispusesse de uma verba da \textsc{nasa} ou
do \textsc{cnp}q seria possível, ao longo de semanas, de meses, ir chegando a
um controle cada vez maior de cada uma dessas fases, até conseguirmos
mais arremessos certos do que errados.

É assim que a ciência resolve problemas de ordem prática. Na
engenharia, na astronáutica, na medicina, seja lá onde fôr, o sujeito
tem pela frente um problema tão complicado quanto este (geralmente
muito mais) e é preciso ir “cercando”, fase por fase, tentando
eliminar as variáveis não-essenciais, e reduzir ao máximo a
variabilidade de outras. Por isso que muitos problemas teóricos, para
simplificar, usam expressões tipo “em condições normais de
temperatura e pressão”, ou “numa superfície ideal, com atrito zero”
ou “um corpo em movimento retilíneo e uniforme”: tudo isto serve para
eliminar variáveis e adiantar o cálculo. Mas num problema da vida
real não se pode eliminar isto por decreto. Tem que encarar.

Não sou da turma triunfalista que acha que a Ciência pode tudo, mas já
se descobriu tanta coisa do mundo físico que, com dinheiro e
know-how, pode-se fazer coisas impensáveis há 50 anos. Mas não se faz
tudo que é tecnicamente possível. Quem afunila, fiscaliza e direciona
a atividade científica é a possibilidade de lucro social (a cura de
doenças) ou de lucro econômico. Aí voltamos ao início: será que a
\textsc{nasa} ou o \textsc{cnp}q estariam dispostos a gastar essa grana toda só para
saber como acertar uma bola de papel dentro dum carro? Não. Pode-se
descobrir um método engenheirístico para isso; mas não há utilidade
social, não há interesse político em descobrir. Agora, se fôr para
mandar uma nave à Lua, ou para extrair petróleo do fundo do mar… é
muitíssimo mais difícil, vai dar muitíssimo mais trabalho, vai custar
incrivelmente mais caro, mas vale a pena. Por isso fizemos.

\chapter{A intuição matemática}

Existem certos processos criativos que parecem absurdos a quem não é
do ramo. Não que esse “ramo” da criação artística seja alguma coisa
de extraordinário, como as atividades dos deuses no Monte Olimpo. Não
vejo diferença entre a criatividade de um poeta que faz um verso
genial, a criatividade de um jogador que faz um gol de placa, a
criatividade de um administrador que recupera uma empresa caótica e
falida, a criatividade de uma dona-de-casa que abre uma geladeira
quase vazia e tira lá de dentro um almoço. Criatividade é uma função
da mente, e achar que ela só existe no âmbito da literatura é
invenção dos literatos. A gente pode ser doido, mas não é besta.

O livro “A Experiência Matemática” de Philip David e Reuben Hersh (Ed.
Francisco Alves) traz numerosos exemplos da criatividade intuitiva de
grandes matemáticos. Na Matemática (ao contrário, por exemplo, da
Física ou da Química) as intuições criativas não dependem de
confirmação através de trabalhosos experimentos de laboratório. A
Matemática é linguagem pura, e certos indivíduos têm propensão para
esse tipo de linguagem. Muitas vezes um matemático tem o vislumbre
instantâneo de um princípio matemático qualquer, mas não é capaz de
provar por quê sabe que está certo. Ao examinar o problema, a solução
parece saltar-lhe aos olhos, mesmo sem que ele consiga explicá-la.

Há matemáticos que passam anos inteiros tentando demonstrar, através
das provas e contraprovas regulamentares, um teorema ou coisa
parecida que lhes ocorreu no espaço de alguns segundos ou minutos. É
como se eles dissessem: “Cheguei lá, mas não sei o caminho.” Isso tem
muito a ver com a diferença entre o hemisfério direito do nosso
cérebro, capaz de associações de idéias instantâneas, e o caráter
minucioso e pedestre do hemisfério esquerdo, sede da linguagem, onde
tudo precisa ser explicado direitinho, passo a passo. Einstein, por
exemplo, foi um que penou para conseguir demonstrar matematicamente
as coisas que descobria.

David \& Hersh dão indicações interessantes sobre esse processo. Para
eles, muitos problemas assemelham-se a uma pergunta do tipo “Será que
este objeto cabe naquela caixa?” Uma simples olhada, em geral, nos dá
a resposta, sem que tenhamos que medir com fita métrica o objeto e a
caixa, e depois fazer as contas. Muitos problemas matemáticos têm um
grau parecido de visualidade, e repousam também num acervo de
experiências acumuladas. Quando já resolvemos centenas de problemas
que partilham uma característica comum, fica mais fácil ver
instantaneamente a solução de outro problema onde essa característica
está de certo modo embutida mas não evidente. Nós reconhecemos, antes
mesmo de pensar nisso, a presença daquele padrão, e no instante
seguinte a resposta se impõe. Depois vai ser preciso pegar a fita
métrica e provar a todo mundo que estamos certos; mas nossa resposta
de baseou naquilo que eles chamam de soluções do tipo analógico, e
efetuadas instantaneamente.

\chapter{Os campos magnéticos}

Imagine uma pequena barra de ferro imantada. A extremidade “A” é o
polo positivo, e a extremidade “B” é o polo negativo. Entre as duas,
espalham-se as linhas de força do campo magnético, invisíveis a olho
nu, mas cuja presença pode ser percebida se colocarmos a barra
embaixo de uma folha de papel e sobre a folha espalharmos uma limalha
de ferro bem fininha. O pozinho metálico irá se organizar ao longo
dessas linhas, formando um desenho que a maioria de nós viu nos
livros de Ciências do 1º Grau. 

Uma experiência interessante é serrar essa barra de ferro ao meio.
Nosso primeiro impulso é pensar que isso resultará em duas barras
menores, com os polos positivo e negativo ainda situados nas
extremidades A e B, e as novas extremidades obtidas no meio sem
nenhum polo específico. Nada disso. No momento em que a barra é
dividida em duas, o campo magnético também se divide em dois campos
menores. Mágica pura. É como se rasgássemos ao meio uma nota de 1
real e, em vez de duas metades incompletas, nos víssemos segurando
duas notas inteiras de 1 real, com metade do tamanho da nota
anterior. E, rasgando cada uma destas, nos víssemos com quatro notas
inteiras, cada uma com um quarto do tamanho da nota original.

Os campos magnéticos tendem a se recompor em torno do objeto físico
que os sustenta, e isto é uma lei interessante que pode ser
transposta, “mutatis mutandis”, para muitos aspectos da vida social.
Em algumas situações os comportamentos humanos tendem a se polarizar
em atitudes opostas: comandante e comandado, patrão e escravo, etc.
Pensamos que cada indivíduo tem um papel, e só este; mas às vezes ele
só está cumprindo este papel para que o polo que ocupa não fique
vazio. Se pegarmos 20 generais e os soltarmos numa ilha deserta,
trancarmos num cárcere, em poucos dias irão se definindo entre eles o
grupo dos que mandam e o dos que obedecem. Se pegarmos 20 monges
budistas e fizermos o mesmo, em breve uns estarão como líderes e os
demais como seguidores. Certos papéis sociais são como polos
magnéticos: não podem ficar desocupados. Alguém vai ter que fazer
aquilo.

A cultura oriental fala em termos de Yang e Yin como as duas forças
básicas da natureza física e da natureza humana. O Yang exprime uma
força centrífuga, que se expande de dentro para fora, que incita à
ação, à manifestação externa, e que tende a exercer pressão sobre o
ambiente em volta. O Yin exprime uma força centrípeta, que se
manifesta de fora para dentro, que incita à reflexão, à transformação
interior, e que tende a atrair para si o que está no ambiente em
volta.  Qualquer ser vivo, qualquer pessoa, qualquer grupo social
possui estas duas forças, assim como cada ímã possui seu polo
positivo e seu polo negativo, só que estas forças são muito mais
complexas do que o magnetismo físico. Uma das suas manifestações
sociais mais curiosas, por exemplo, é quando Oposição vira Governo e
vice-versa.

\chapter{Ler e escrever}

Sempre tive a sensação de que ler era conversar com um Mestre que mora
longe, não me conhece, mas está disposto a me ensinar tudo que sabe.
Às vezes o sujeito já morreu há duzentos anos mas as lições dele
estão todas ali,à minha espera. O único inconveniente era o fato de
ser um monólogo, não um diálogo; ele tinha muito a me dizer mas não
poderia me ouvir. Uma comunicação de mão-única, por certo, mas sempre
achei que era melhor uma comunicação de mão-única com Dostoiévski ou
com Machado de Assis do que horas de blá-blá-blá improfícuo com
certos contemporâneos.

Ler era um exercício de humildade, era o momento de me sentar em
posição de lótus diante do mestre e, mergulhado num silêncio
respeitoso, absorver suas lições da melhor maneira possível. Ler era
conversar com os mortos, com os distantes, com sujeitos importantes
que se me encontrassem pessoalmente mal dariam atenção àquele
adolescente cabeludo e mal vestido, mas que, graças ao milagre do
papel impresso, me faziam companhia durante as madrugadas, na mesa da
cozinha onde eu me sentava com o livro aberto à minha frente, tendo
ao lado o caderno-espiral, um bule de café e uma pãozeira cheia de
bolachas sete-capas. Quando a gente lê, vira discípulo, aprende a
ficar calado e escutar, aprende a aprender.

Escrever, por outro lado, era aquele momento em que o discípulo zen dá
um salto acrobático no ar e cai de pé transformado num mestre do
karatê em posição de batalha.  Escrever era a hora de mandar às favas
as lições alheias e fazer ouvir minha própria voz. Quando eu
empurrava o livro para um lado e a ponta da caneta Bic fazia contato
com a superfície mágica do caderno, desencadeava-se um Shazam cósmico
qualquer, Billy Batson virava o Capitão Marvel, e pelos poderes de
Grayskull eu me transformava nos meus super-heróis imbatíveis, Pessoa
de pince-nez, Rosa de gravatinha borboleta, Dylan de óculos rayban.

Arrufos da juventude, por certo. Porque hoje, amiguinhos, a sensação
que tenho depois de todos estes dias em Pequim é que estas duas
situações se inverteram. Resta pouco tempo, restam poucos anos. Ler a
esta altura é luxo, é egoísmo, é amealhar ainda mais moedas num cofre
já abarrotado, é pedir o adiamento do jogo para continuar treinando.
Ler está virando uma atividade egoísta, o derradeiro dos prazeres
solitários.

E escrever é agora a verdadeira lição de humildade: ter que mostrar
todo dia que o que aprendi foi só isto. É pouco, mas é o que tenho
para exibir, para oferecer. Você tem o direito de aprender na
primeira metade da vida, mas fica com a obrigação de ensinar na
segunda. Não importa se você sempre acha que se preparou mal, que
aprendeu pouco, que o que sabe é inadequado ou já-era. O que
aprendemosé um empréstimo que o mundo nos fez, e precisa ser pago.
Escrever é passar adiante aquilo que, bem ou mal, restou em nosso
juízo depois de tantas madrugadas. O Mundo é o que você aprende, mas
Você é o que você ensina.

\chapter{Educação e censura}

Em seu livro “1984”, George Orwell imaginou uma ditadura onde o
Governo seria capaz de espionar a vida de todos os cidadãos. A \textsc{tv}
seria em mão-dupla: ela mostraria imagens mas seria capaz também de
vigiar as pessoas em suas casas, em tempo real. Isaac Asimov, um dos
campeões do bom-senso e da “lógica das coisas” na ficção científica,
ironizou essa técnica dizendo que em princípio seria necessário um
grupo de umas cinco pessoas para vigiar apenas uma, uma vez que
ninguém conseguiria ficar 24 horas acompanhando o cotidiano de um
cidadão comum.

Foi Orwell quem criou a expressão “Big Brother”: o Grande Irmão era o
ditador dessa Inglaterra situada no futuro, um sujeito bigodudo e
implacável com os traidores do regime, mas de aparência paternal,
claramente inspirado em Josef Stalin. Aqui no Brasil, com o programa
da \textsc{tv}-Globo, a expressão “big brother” foi totalmente distorcida: a
imprensa chama de “big brothers” as pessoas que ficam trancadas na
casa, sendo espionadas. Na verdade, os big-brothers seríamos nós.

As verdadeiras ditaduras, no entanto, não precisam espionar. As
ditaduras mais eficientes são as que não precisam vigiar ninguém,
porque todos os cidadãos acreditam com fervor que estão no melhor dos
mundos, mesmo que vivam numa pindaíba de fazer dó (como ocorre com os
personagens de “1984”). O melhor tipo de censura não é o que tem
funcionários atarefados cortando tudo que os escritores de oposição
escrevem. O melhor tipo de censura é aquele em que durante a
madrugada os censores estão dormindo em paz, e os próprios
escritores, depois de redigirem uma frase que sabem perigosa, voltam
atrás e apagam tudo.

O final de “1984” é trágico e arrepiante porque o personagem
principal, Winston (cujo nome faz um contraste irônico com o nome de
Churchill), encerra o livro, depois de uma sessão de tortura,
proclamando sua lealdade e seu amor pelo Big Brother. Nenhum ditador
precisa espionar um sujeito que passou por uma lavagem cerebral como
esta.

É cruel, mas só posso comparar isso com a educação que damos aos
nossos filhos. Sabemos que nossos filhos estão bem educados quando
eles escovam os dentes sem que a gente mande, fazem o dever de casa
por iniciativa própria, botam a roupa suja no cesto sem que seja
preciso alguém conduzi-los pela orelha. Sabemos que estão bem
educados quando eles saem à noite dizendo que vão para um show de
rock e depois dormirão no apartamento de um amigo, e nósconfiamos que
nada de errado vai acontecer. Toda educação é uma
lavagem-cerebral-do-Bem, é algo que implantamos a ferro e fogo (ou
melhor, à base de castigo e chinelo) naquelas mentezinhas adoráveis
quando elas não parecem merecer nada além de ternura, mimos, afagos,
cheiros e mais cheiros. É nessa fasezinha dourada da existência que
cabe uma boa e velha lavagem cerebral, meus amigos. Para que depois o
Censor possa dormir em paz, sabendo que sua missão foi bem cumprida.

\chapter{A arte da narrativa}

Eu considero a Narrativa uma forma de arte. É a arte de contar
histórias, tão antiga quanto a linguagem, e que (já que ninguém pode
provar nada, podemos todos especular à vontade) pode muito bem ter
sido o impulso que, no mundo primitivo, deu origem à literatura oral,
à poesia, ao teatro, à dança. Imagino que o homem primitivo
dramatizava acontecimentos com a voz, a palavra e o corpo.  Usava
esses elementos para reproduzir acontecimentos que eram do
conhecimento da tribo (uma caçada, uma batalha, um encontro com algo
fora do comum) ou para rituais mágicos, aqueles onde encenamos algo
para fazer com que aconteça (véspera de caçada ou de batalha).

A Narrativa está presente em filmes, peças de teatro, óperas,
histórias em quadrinhos, poemas, videogames, romances, canções, balés
- desde que cada um deles conte uma história. Existe algo em comum
entre o romance “Vidas Secas” de Graciliano Ramos e o filme “Vidas
Secas” de Nelson Pereira dos Santos. Por mais diferentes que sejam,
em matéria-prima, um romance (sinais gráficos em folhas de papel) e
um filme (imagens luminosas em movimento, sonorizadas), existe ali
uma Narrativa, uma história (ou estória, como queria Guimarães Rosa),
uma sucessão de eventos que é a mesma nas duas obras, e entre as
quais é possível ir apontando correspondências. A Narrativa nunca é
exatamente a mesma quando muda de meio de expressão, mas está sempre
lá.

Existem milhares, talvez milhões de versões das narrativas
tradicionais, que geralmente são bem curtas e simples. Isto vale para
uma lenda grega ou hindu, para um conto folclórico como “Chapeuzinho
Vermelho” ou para uma anedota de português ou de bêbado. Se
comparássemos um milhão de versões da história do “Chapeuzinho”
(orais, impressas, cinematográficas, em quadrinhos, em \textsc{tv}, em teatro,
em desenho animado), não haveria duas versões iguais, mas há um
núcleo de elementos que é o “\textsc{dna}” daquela história. Há um texto de
Lévi-Strauss sobre o Mito que o descreve como esse conjunto de
elementos, nunca exatamente os mesmos, mas presentes nas diferentes
versões, e que se tornam cada vez mais nítidos quanto mais versões
são consultadas.

No meio cinematográfico existe uma máxima de que é mais fácil extrair
um bom filme de um mau romance do que de uma obra prima. Isto se
explica pelo fato de que o que chamamos de obras primas literárias
são obras que se destacam pelo estilo, pela linguagem, pela
criatividade verbal - e nada disto pode ser transposto para a tela. E
há muitos romances maus que consistem em histórias bem imaginadas mas
escritas de maneira canhestra, com má escolha de material verbal
(defeito que some na tela) e erros de estrutura (defeito que pode
perfeitamente ser corrigido por um bom roteirista). Quando se tem uma
Narrativa interessante, ela geralmente pode ser transposta sem grande
perda para qualquer outro meio. Passa por uma mutação de forma, mas
mantém o seu \textsc{dna} original, a sua essência de história, ou estória.

\chapter{A tragédia da vida}

Não existe nada mais educativo do que ouvir anedotas. Não me refiro às
piadas de sacanagem que os homens contam em mesa de bar, embora estas
também tenham seu lado propedêutico. Qualquer piada é uma pequena
cápsula de filosofia existencial. Como a história do garoto de 8 anos
que a família precisava levar ao dentista. O dente inchado, doía
muito, mas ele morria de medo. Marcaram a consulta, mas ele fincou
pé, disse que não ia. Os pais adularam, adularam, até que o pai veio
com o argumento irrespondível: “Olhe, Paulinho, você vai ter que ir.
Nós já marcamos a consulta, e esse dentista é muito caro, a consulta
dele é 100 reais.” O menino enxugou as lágrimas e concordou. Foi,
submeteu-se à tortura do tratamento. No fim, depois do “pode cuspir”,
levantou da cadeira, enxugou os olhos, e perguntou timidamente ao
dentista: “E meus 100 reais?…”

Não acho que exista parábola mais edificante para a gente contar aos
filhos. Porque isto é a vida, não é mesmo?  A gente passa a vida
inteira se submetendo às maiores provações, aos piores sacrifícios,
sempre de olho num prêmio prometido. Na hora do balanço final, a
gente descobre que não só não vai ter prêmio, vai ter um prejuízo, e
ainda vai pagar juros e correção monetária. 

Outra história (esta verídica) fala de um garoto de seus quatro anos
que a família matriculou na escola pela primeira vez. Desconfiado, o
garoto dizia que não queria ir estudar em escola nenhuma. Os pais
providenciaram tudo: uniforme, livros, lápis de cor… Tudo foi usado
como isca. Explicaram que ia ser legal, que uma professora ia ensinar
coisas interessantes; que na escola ele teria muitos amiguinhos
novos; que havia uma coisa ótima chamada “hora do recreio”, onde
todos brincariam do que quisessem; e patati, e patatá.  No primeiro
dia de aula, acordaram-no às 6 da manhã. Uniforme, café, ida à
escola. Quando ele voltou, estava feliz: tinha adorado tudo. Fêz mil
comentários, e tal. No outro dia, a mãe voltou a chamá-lo às 6:
“Joãozinho!… Tá na hora?”  Ele acordou: “Hora de quê?” A mãe: “De ir
para a escola.” E o guri, perplexo: “Oi… de novo?”

Eu sempre obriguei meus filhos a irem ao colégio. Não porque eu creia
na necessidade de sabermos extrair raiz cúbica ou de recitar de
memória os membros da Regência Trina Provisória. Os ensinamentos que
tive no colégio dissiparam-se tão rapidamente que às vezes sinto um
calafrio de horror ao folhear um livro de Oswaldo Sangiorgi ou de
Borges Hermida, e saber que perdi ali tantas tardes ensolaradas que
poderia ter dedicado ao futebol ou ao jogo de botão. A função do
colégio, no entanto, não é nos ensinar química ou botânica. É nos
mostrar que todo sofrimento na vida é pago - mas pago por nós mesmos.
E que não importa quantas vezes você tenha passado por ele: vai ter
que passar outra vez, e outra, e outra. “Está na hora.”  “De quê?”
“Ora, está na hora de fazer sua coluna do jornal.” “Oi… de novo?…”

\chapter{Uma lenda oriental}

Diz uma antiga lenda oriental que na época da dinastia T’sin, havia um
rei despótico que gastava de modo perdulário, prendia e torturava os
críticos do seu regime, e roubava o Tesouro público. O rei subira ao
trono cercado de expectativas, pois fora um príncipe inteligente,
amado pelo povo. Depois que passou a governar de modo desastroso, um
grupo de nobres reuniu-se, conspirou, e juntou exércitos para
enfrentar o tirano.  Houve uma guerra sangrenta. O tirano foi morto,
e os nobres elegeram, para substituí-lo, o nobre Li H’sien. Este era
um homem íntegro, mas, assim que subiu ao trono, começou a se
comportar de modo muito parecido com o antecessor. Perseguiu os
antigos aliados, construiu palácios para seus parentes, cercou-se de
bajuladores, e botava na cadeia quem falava mal dele.

Os nobres agüentaram isso durante alguns anos, aí juntaram-se
novamente, desencadearam outra revolução, executaram Li H’sien e
colocaram no trono o general Hsui-Pen, um homem valoroso, simples, de
julgamentos serenos e caráter firme.  Poucos anos depois, o general
tinha transformado a corte num verdadeiro bordel com orgias
intermináveis, além de promover a execução de dissidentes, e a
invasão militar das províncias vizinhas.  Os nobres, já desesperados,
sem saber o que fazer, foram queixar-se ao Budista Tibetano. O
Budista Tibetano deu uma baforada do seu narguilê, pensou, pensou, aí
disse: “Olha, eu, se fosse vocês, tocava fogo era naquele trono. Todo
mundo que se senta lá fica assim.”

Gostou da lenda oriental, caro leitor? Se gostou, obrigado, porque
acabei de inventá-la. Não, não me elogie. A imaginação e a
criatividade pouco contribuíram para a sua execução. O que mais me
valeu foi a memória, o hábito de ler jornais, e algumas décadas de
vida debruçado na janela por onde o mundo vive passando e só Carolina
não vê.  Para os que se debruçam nessa janela, o mundo traz surpresas
cíclicas. Se são cíclicas, talvez não devessem ser surpresas, porque
a repetição nos deixa prevenidos. Mas é que o mundo não se repete em
círculos, como supunha Nietzsche, mas em espirais: cada vez que uma
coisa acontece, acontece num ponto diferente da vez anterior.

O filme “Viva Zapata”, de Elia Kazan, mostra Marlon Brando no papel de
um camponês mexicano que se revolta. Um dia eles vão protestar algo
junto ao tirano local e este, enraivecendo-se, pergunta: “Você! Como
é seu nome?!”  Ele responde, intimidado: “Zapata. Emiliano Zapata.” 
Os anos se passam, Zapata entra na luta armada revolucionária, vira
líder e herói, mas descobre que é mais fácil deflagrar Revoluções do
que mantê-las. Depois que vira presidente do México, um dia uns
camponeses vão ao palácio reclamar de alguma coisa. Ele se irrita com
um dos queixosos, e diz: “Você! Como é o seu nome?!” Aí na mesma hora
o episódio antigo lhe vemà memória, e ele se cala, confuso,
percebendo a inversão dos papéis. Pois é. O problema é o trono,
presidente.

\chapter{Cordel na sala de aula}

Caros leitores, espero que não me censurem por fazer neste discreto
espaço a publicidade de minhas atividades profissionais. Na próxima
semana estarei em João Pessoa participando do Fenart, numa
mesa-redonda sobre Cultura Popular na terça (dia 4), e realizando uma
oficina sobre Cordel de quarta a sexta-feira (dias 5 a 7), sempre à
tarde. A Oficina, parece-me, será aberta ao público em geral, mas se
dirige principalmente a professores do nível fundamental e médio. Seu
título é: “Cordel: como escrever, como ensinar”. Mais informações com
a Funesc, no Espaço Cultural.

Ministrei esta oficina nos últimos anos em São Paulo, por iniciativa
de Antonio Nóbrega, meu parceiro em canções e peças teatrais. A idéia
de Nóbrega, com seu Teatro Brincante, é ministrar oficinas sobre
cultura popular brasileira para professores que lidam com crianças e
adolescentes em São Paulo. Como estes professores geralmente são
paulistas, têm dificuldade em abordar o folclore, a arte nordestina
em geral. Daí, o teatro faz oficinas diferentes a cada mês. Por
exemplo: em março os alunos estudam Frevo, em abril estudam
Artesanato em Barro, em maio estudam Bumba-meu-boi, em junho estudam
Mamulengos, etc. 

Num desses meses, estudam poesia popular nordestina: o Romanceiro e a
Literatura de Cordel. Noções elementares de métrica e rima (que
muitos poetas profissionais, acreditem, às vezes ignoram), história
do romanceiro ibérico trazido pelos colonizadores, e noções práticas
da arte da poesia. Alguém já saiu desta Oficina (ou de qualquer
outra) diplomado como poeta? Duvido. O objetivo é transmitir as
regras do cordel, as noções básicas de como escrevê-lo, e alguns
truques postos em prática por quem joga esse jogo há vários anos.

Mais do que formar poetas, oficinas deste tipo (que hoje acontecem em
muitos pontos do Brasil) querem ajudar o professor a transmitir para
crianças e adolescentes o gosto descompromissado pela poesia, pela
expressão verbal, pela brincadeira com rimas e com ritmos, pela
possibilidade de se expressar através da “linguagem enriquecida” que
é a poesia. Não estou muito a par do que as escolas de hoje ensinam
sobre poesia. Quando eu tinha 12 anos tinha que decorar o que era
écloga, ditirambo, arcadismo. Foi em casa que aprendi a contar
sílabas, a escolher uma rima, aprendi a fazer quadrinhas e
pés-quebrados, e aprendi que poesia não tem receita. Existe o verso
livre, o metro livre; e existem formas fixas, com regras claras. O
cordel é uma destas. Afora isto, temos todo o direito de escrever o
que nos dá na telha. O cordel nordestino nasceu porque um bando de
nordestinos humildes, sem títulos acadêmicos, muitas vezes
autodidatas que jamais sentaram num banco de escola, sentiram-se no
dever de aprender a fórmula, e no direito de escrever o que lhes dava
na telha. Que esse dever e esse direito sejam restaurados para os
meninos nordestinos de hoje, é o mínimo que podemos desejar.

\chapter{O poeta principiante}

Vivo cercado de poetas principiantes. Não há menosprezo nesta palavra.
Tudo no mundo tem que principiar; então, que os poetas comecem a
poetar desde cedo, porque escrever poesia é como dançar gafieira,
requer longa prática e treino constante.  Existe uma coisa que às
vezes é chamada, um tanto pomposamente, “o fazer poético”, e que é um
conjunto de técnicas. Tudo tem uma técnica. Dançar gafieira tem uma
técnica, mas saber a técnica não adianta, se você não tem jeito para
a coisa.  

Um erro típico do poeta principiante é querer publicar tudo que
escreve. O cara pega os primeiros 50 poemas que escreveu na vida,
passa a limpo e manda para uma editora. Isto equivale a um músico
mandar para uma gravadora suas aulas de violão, na esperança de que
alguém queira pagar para ouvi-lo aprendendo a fazer o tom de Dó
Maior.  Os primeiros exercícios são penosos, são constrangedores, e
mesmo quando os resultados são bons, é melhor guardá-los, e continuar
escrevendo.  Primeiros poemas são sempre regurgitações de poemas
alheios, são reciclagem de clichês, são reflexos de grandes versos
que lemos e nunca mais saíram de nossa memória, ficaram ali, moendo,
moendo, contaminando tudo que tentamos colocar em palavras.

Os poetas surrealistas dos anos 1920 diziam que para chegar um dia a
produzir grandes poemas é preciso “limpar a estrebaria intelectual”,
jogar para fora todos os detritos verbais que absorvemos nas
leituras, nas conversações do dia-a-dia, no cinema, no rádio…  O
tempo inteiro estamos registrando frases, piadas, versos, palavras
novas, trechos de canções, gírias, jargão profissional.  Nosso
vocabulário e nosso senso de sintaxe são formatados ao longo desse
bombardeio que dura a vida toda. Não temos controle sobre o que lemos
e ouvimos. O momento de ter controle sobre essa bagunça é o momento
de escrever.

É típico de poetas principiantes querer preservar a dimensão
biográfica dos próprios poemas: momentos de sua vida pessoal, ou
etapas do seu aprendizado poético. Vai daí, os poemas dos primeiros
livros de um poeta geralmente vêm todos datados, alguns com requinte
de detalhe: “14 de outubro de 2003, 16:30, praia de Copacabana.” Eu
já fiz isso, todo mundo já fêz isso, mas hoje eu acho que tais
particularizações não interessam ao leitor. O poema deve “a seco” ir
para a página. O link biográfico deve ficar para os biógrafos, se os
houver.

A maioria das pessoas começa a escrever poesias para desabafar
sentimentos ou registrar estados de espírito. São poucos, por
exemplo, os que começam a escrever poesias para contar histórias, ou
para descrever lugares e ambientes. A poesia é considerada uma forma
de olhar para dentro, de se auto examinar; e o poeta principiante
recorre a ela quando está (para usar a linguagem de hoje)
emocionalmente fragilizado. Quando se sente seguro de si, lúcido,
mente acesa, o poeta principiante acha que não precisa escrever. Ele
troca de roupa e vai beber com os amigos.

\chapter{A arte de citar}

Quando citamos uma frase, geralmente dizemos o seu autor, e mais nada:
“É como dizia Shakespeare: o resto é silêncio”.  Verdade. Mas talvez
fosse mais certo dizer: “É como dizia o príncipe Hamlet, de
Shakespeare: …”  Porque a frase é do dramaturgo, mas o contexto é do
personagem. Um dos erros mais comuns no leitor ingênuo ou
desambientado com um texto de ficção (ou dramático) é o de julgar que
todas opiniões ali expostas são as opiniões pessoais do autor. Se um
personagem faz um discurso racista, por exemplo, esses leitores
condenam o autor, afirmando que ele pensa daquela forma.  

Nelson Rodrigues disse: “O dinheiro compra tudo, até amor verdadeiro.”
Esta citação é de “Bonitinha, mas ordinária”, que nunca li. Mas quem
diz isto no livro? Não sei. Pode ser o desabafo de um pretendente
pobre, derrotado por um playboy de Copacabana que lhe arrebata a
trêfega noivinha. Pode ser um pai de arma em punho convencendo o
filho recalcitrante a casar com uma herdeira feia de dar pena. Pode
ser uma quarentona, sofrida, realista, sussurrando ao ouvido de um
banqueiro lúbrico e septuagenário.  A frase tem seu valor de origem,
como frase em si, e tem os valores que lhe vão sendo superpostos por
personagens, atores, por quem quer que as enriqueça repetindo-as.

Daria uma interessante pesquisa rastrear, por exemplo, todos os
contextos em que a frase “viver é perigoso” é repetida em “Grande
Sertão: Veredas”. Em alguns casos pode ser verdade, em outros não.
Uns a dirão com temor, outros com exultação, outros com perplexidade.
A cada ator que a diz, a frase se recria. “O nazismo,
intrinsecamente, é um fato moral, um despojar-se do velho homem, que
está viciado, para vestir o novo” (Jorge Luís Borges) Não estou
inventando a frase, nem atribuindo-a falsamente a Borges. Ele a
escreveu de fato, só que a pôs na boca de Otto Dietrich zur Linde, o
carrasco nazista de seu conto “Deutsches Requiem”. Seria absurdo
atribuir esta frase a Borges, que deixou sua visão do nazismo muito
clara nesse conto e em “Anotação ao 23 de agosto de 1944” (em “Outras
Inquisições”). Mas quando o caso não é assim tão nítido, quando do
autor sabemos apenas o nome, podemos pensar que quem diz aquilo é
ele, e não o personagem. 

Um escritor inventa frases alheias com a mesma facilidade com que um
desenhista desenha rostos alheios. Nem todo desenho é um
auto-retrato, e o mesmo vale para romance, conto, poema ou peça
teatral.  As ações e as falas dos personagens não precisam se parecer
com as do autor. E aliás nem poderiam, já que os personagens precisam
ser diferentes uns dos outros. Há escritores, mesmo grandes
escritores, que não sabem construir personagens que não sejam
parecidos com ele próprio. Outros parecem médiuns recebedores
universais, e as frases que criam para seus personagens têm às vezes
o poder de causar-lhe surpresa, ou repulsa, ou medo.

\chapter{Crianças cruéis}

Por que motivo as crianças têm fascinação por histórias de terror e
violência? Já me fiz esta pergunta antes (“Crianças e monstros”, 2 de
maio), mas acho que nunca vou acabar de respondê-la. Devo ter lido em
algum lugar que quando uma criança tem medo de um tigre e passa o
resto da infância desenhando tigres isto é uma forma de domesticar o
tigre que ficou dentro de seu cerebrozinho. Lá dentro existe a
memória ameaçadora de um tigre que um dia a assustou. Se ela desenhar
99 tigres que lhe obedecem e que se deixam transferir,
comportadinhos, para a folha de papel, o tigre-contra fica numa
tremenda duma inferioridade em relação aos tigres-a-favor.

A literatura de terror que prolifera no mundo, com todos os seus
dráculas e frankensteins, é uma extensão desse processo. Mergulhamos
a cara num livro de Lovecraft porque sabemos que, quando a barra
começar a ficar muito pesada, basta fechar o livro e olhar em volta -
o que pode haver de pior à nossa frente é nosso time levando mais uma
goleada para o país inteiro ver.  O terror literário e
cinematográfico é um terror totalmente sob controle. Basta ver a
expressão feliz dos adolescentes que se amontoam nos cinemas que
exibem as aventuras de Jason e de Freddy Kruger.

Ah, se todos os males do mundo fossem estes!… Mas não são. Existe uma
ruindadezinha embutida em cada um de nós, um instinto malévolo que
nos levou a cometer, na infância, atos que não cometeríamos hoje,
depois que a lavagem-cerebral civilizatória nos transformou nos
cidadãos exemplares que agora somos. Lembro-me ainda hoje das manhãs
que passei usando um binóculo invertido, como se fosse uma lupa,
diante do qual eu decapitava com gilete uma imensa quantidade de
saúvas vermelhas, que estoicamente ofereciam suas vidas pelo bem da
Ciência. Verdade que a única coisa que a Ciência aprendeu foi que é
possível decapitar uma saúva com uma gilete. Se a Humanidade um dia
escapar da extinção graças a este detalhe, não se esqueçam de me
agradecer. 

Talvez meu instinto carniceiro (ou científico) tenha se satisfeito com
estas experiências. Mas não me esqueço de um episódio relatado por
Stephen King. O cantor Bing Crosby deu de presente ao filho pequeno
uma tartaruguinha, e o garoto ficou louco por ela. Tempos depois a
tartaruga morreu. O menino ficou inconsolável. Crosby botou o menino
na perna, filosofou, explicou-lhe o sentido da vida, falou que a
morte é inevitável, propôs fazerem um funeral. Botaram o cadáver numa
caixa, prepararam uma sepultura no jardim, fizeram o ritual, rezaram.
Na hora de botar na cova a caixa de sapatos que servia de ataúde, o
menino pediu para dar uma última olhada. Abriram a caixa… e a
tartaruguinha estava mexendo as patas. Não tinha morrido, afinal de
contas. Houve um instante de surpresa, e aí o garoto voltou-se para o
pai e confidenciou baixinho: “Acho que vamos ter que matá-la.”  É a
vida, companheiros.

\chapter{O livro e a criança}

Como incutir o hábito de ler numa criança? Pense num problema da maior
gravidade! Um pouco ritual de ritual às vezes funciona. Fazer a
criança tomar banho, vestir o pijama, e passar meia hora lendo antes
de ir dormir; desligar a \textsc{tv}; ter uma poltrona e uma luz específica;
tudo isto é interessante, cria uma expectativa, cria um clima
propiciatório, e criança geralmente gosta disso (desde que não esteja
sendo obrigada).

O problema é que algumas crianças acabam levando a leitura tão a sério
que se distanciam dela. O livro passa a ser uma coisa especial, para
ser manuseada em ocasiões especiais. E a leitura vira às vezes uma
atividade decifratória que não pode deixar lacunas. Já vi (em casa
inclusive) inúmeros exemplos de criança que empanca num livro por
causa de uma palavra, e dali não passa. “Paaai… o que é begônia?”
“Acho que é uma flor.” “De que jeito? De que cor?” “Não tenho a
mínima idéia.” Impossibilitada de assimilar a begônia, a criança tem
a leitura travada. Como seguir adiante, deixando para trás esta
importantíssima pergunta não respondida? Vai o livro para o tapete, e
a criança para a televisão.

Minha lei é: menos respeito! Devemos aprender a dar de ombros. Não
entendeu, paciência, pula e segue em frente. É assim que eu leio, até
hoje inclusive. As palavras sempre voltam, e cada vez que voltam
trazem consigo uma pistazinha a mais sobre si próprias. Ainda hoje me
lembro dos livros de ficção científica em que vi pela primeira vez
palavras como “esporos”, “catalisador”, “mutante”. Deduzi pelo
contexto, aos poucos, mas fui em frente. 

Não entendeu um capítulo? Pula para o próximo. Deveríamos ter com os
livros a mesma relação descontraída que temos com as telenovelas,
onde mesmo tendo perdido uma dúzia de capítulos logo sabemos o
bastante para seguir em frente. Caetano Veloso comparou certa vez a
diferença de comportamento, nos cinemas, entre o público francês e o
americano. O francês, concentrado e reverente, parece estar na Ópera;
o americano, rindo e comendo pipoca, parece estar no Circo.  Acho que
um pouco de cada atitude, no momento certo, também ajuda a gente ao
longo das idas-e-vindas que a leitura de qualquer livro implica.

Gosto de ler riscando, sublinhando, anotando. Amigos bibliófilos um
dia me alertaram que desse jeito posso estar desvalorizando em
definitivo uma edição rara. Depois disso, deixo minhas primeiras
edições na estante, mas compro uma edição atual do mesmo livro, para
poder passar-lhe a caneta sem dó nem piedade. Um livro é um objeto de
trabalho, algo para ser tratado sem muita cerimônia. Como as crianças
são depredadoras por natureza, não podemos dar rédea livre, mas é
sempre bom explicar-lhes que livro é um brinquedo para se brincar com
os olhos. Deve-se entrar neles por todos os lados, frequentá-los a
qualquer hora, abri-los sem esperar nenhum milagre mas pronto para
recebê-lo caso ele aconteça. Sé é livro de verdade, acaba
acontecendo.

\chapter{Agatha Christie e o medo}

Em sua autobiografia (que é um dos seus melhores livros, se não o
melhor de todos) Agatha Christie discute de vez em quando alguns
temas ligados à literatura policial, entre eles o do medo. Embora
seja mais famosa por seus romances detetivescos (como os que têm como
protagonistas Hercule Poirot e Miss Marple), ela escreveu também
romances de crime e suspense, impecáveis, dos quais o mais conhecido
deve ser “O Caso dos Dez Negrinhos”.  O que há de mais interessante
na saudosa Mrs. Christie é que era uma mulher inteligente, intuitiva,
perspicaz, mas sem muita sofisticação conceitual. Vendo-a discutir
literatura, história da Inglaterra ou a vida de uma dona-de-casa,
estamos diante de alguém que pensa com sutileza e originalidade, mas
em momento algum transforma isto em linguajar pseudo-filosofante.

Ela relata que, na infância, uma das coisas que mais lhe causavam medo
era a brincadeira da “irmã mais velha”, uma irmã fictícia, que ela
imaginava ser louca, morar numa gruta, e ser sósia de sua irmã mais
velha, Madge.  A brincadeira consistia em Madge mudar de voz no meio
de uma conversa e dizer: “Agatha, você sabe quem eu sou, não é? Sou
Madge. Você não está pensando que eu sou outra pessoa, não é?” A
mudança na voz… a mudança no olhar… alguns pequenos gestos… e isto
bastava para que Agatha, com cinco anos, tivesse certeza de que não
era Madge que estava ali, mas A Irmã Mais Velha. E saía correndo, aos
gritos. Depois, comentava ela: “Por que gostava tanto da sensação do
medo? Será que habita em nós algo que se rebela contra uma vida com
excessiva segurança? Será que é necessária à vida humana a sensação
de perigo? Necessitamos instintivamente de algo a combater, a
superar, como se fosse uma prova que quiséssemos dar a nós próprios?
Se tirássemos o lobo da história de Chapeuzinho Vermelho, alguma
criança gostaria dessa história?”

O medo pode vir dessa capacidade de estranhamento, de distanciamento,
de olhar algo que nos é familiar e ver naquilo uma presença
ameaçadora. Este processo mental é o reverso de outro que busca nos
apaziguar, transformar o estranho ou ameaçador no familiar, no que
está sob o controle da consciência. Agatha relata também a história
divertida de um de seus netos, Matthew, que certa vez ela viu, aos
dois anos de idade, descendo uma escada sozinho. Com medo de rolar
pelos degraus, ele se agarrava à balaustrada, descia um degrau de
cada vez, murmurando baixinho: “Este é Matthew… ele está descendo a
escada…”  É uma ilustração nota-dez do nosso processo de
racionalização, de olhar-de-fora algo arriscado para assumir um
mínimo de controle sobre o que ocorre.  E ela diz que todas as vezes
que precisava participar de eventos públicos, apesar de sua timidez,
murmurava para si mesma: “Esta é Agatha… ela é uma escritora famosa…
vai dar uma palestra…” E isto a tranquilizava.  Um dos nossos maiores
medos é o medo daquilo que nossa mente não consegue dominar.

\chapter{O Homem do Fuzil}

Toda criança tem um amigo imaginário; vai ver que todas têm também um
inimigo imaginário.  Agatha Christie conta em suas memórias que um
dos seus pesadelos mais constantes durante a infância envolvia um
personagem que ela chamava O Homem do Fuzil.  Era uma espécie de
soldado francês, com chapéu de tricórnio e um mosquetão antiquado ao
ombro.  Aparecia nos momentos mais inesperados: quando a família
estava reunida para o chá, ou quando as crianças brincavam no jardim.
De repente, a pequena Agatha começava a sentir uma inquietação
crescente. Olhava em volta, e acabava vendo-o sentado à mesa, ou
caminhando na direção delas, na praia. Ele se aproximava, com os
olhos fixos nela, e eram olhos de um azul muito pálido. Ela acordava
gritando; “O Homem do Fuzil! O Homem do Fuzil!” Ela não temia que o
homem disparasse o fuzil, que era apenas uma parte de sua
indumentária, como o chapéu ou as botas.  Era a simples presença
dele, e seu olhar cruel, que a amedrontava.

Com o passar dos anos, o pesadelo foi se sofisticando, e o Homem do
Fuzil passou a fazer aparições mais sutis. Diz ela: “Algumas vezes
estávamos sentados ao redor de uma mesa de chá, eu olhava para um
amigo ou para um membro da minha família e, de repente, tinha
consciência de que não era Dorothy, ou Phyllis, ou Monty, ou minha
mãe, ou qualquer outra pessoa. Nesse rosto familiar, os pálidos olhos
azuis encontravam-se com os meus. Era o Homem do Fuzil.”

É um pesadelo notável para uma garota de cinco anos, mas é mais
notável ainda quando refletimos em quem essa garota se tornou. Agatha
Christie está entre os autores que melhor exploraram um tipo de
história que os analistas do romance policial chamam de “cozy
mystery” (“mistério aconchegante”), ou de “country house murders”.
São histórias geralmente ambientadas numa casa de campo (ou mais
raramente casa de praia) onde um grupo de pessoas amigas se reúne
para passar um feriado ou fim-de-semana, e ali, no meio daquele
ambiente tranqüilo, acaba acontecendo um assassinato.

O “cozy mystery” é tipicamente a descrição de como a harmonia num
círculo de pessoas amigas, ou numa família, é rompida subitamente por
um crime brutal. Segue-se uma investigação, no curso da qual o
detetive (geralmente Hercule Poirot) começa a desvendar segredos,
conflitos, ódios reprimidos, ressentimentos acumulados, e começa a
perceber que todo mundo ali poderia ter motivo para matar a vítima.
Ele vai reunindo as pistas, confrontando os depoimentos, e o livro
culmina com uma sessão em que todos os suspeitos são reunidos numa
sala, com a presença da polícia. Ali, Poirot faz uma reconstituição
de como o crime foi cometido, elimina de um em um os suspeitos, até
que o funil vai-se estreitando, e ele aponta o verdadeiro criminoso.
Porque mesmo num ambiente de aparente harmonia um parente nosso, ou
um amigo da família, de quem menos se suspeitava, pode ser O Homem do
Fuzil.

\chapter{Tolkien e Guimarães Rosa}

Uma vez, conversando com amigos estrangeiros, perguntaram-me quem era
o maior escritor brasileiro; respondi que era Guimarães Rosa. Ninguém
tinha ouvido falar nele; quiseram saber que tipo de escritor era. Eu
disse: “Imagine os romances de J. R. R. Tolkien escritos por James
Joyce.” Riram porque pensaram que era piada, mas não era, era um mero
exagero. A crítica literária brasileira, especialmente a que sofreu
influência do Concretismo paulistano, sempre compara Rosa com Joyce;
nunca vi ninguém compará-lo com Tolkien. E no entanto a obra dos dois
tem imensas semelhanças, que me voltam à mente ao assistir o terceiro
episódio da magnífica trilogia “O Senhor dos Anéis” de Peter Jackson.

Tanto Rosa quanto Tolkien imaginaram uma região mítica, fundada em
suas vivências pessoais e em suas fantasias metafísicas. O Sertão de
Rosa é, para usar uma linguagem meio pedante, semanticamente realista
(porque tudo ali é observado, é anotado em caderneta, é pesquisado
junto aos mais-velhos: usos, costumes, lugares, plantas, bichos) mas
sintaticamente mágico, porque os acontecimentos e os destinos dos
personagens parecem orquestrados por potestades invisíveis. Esse
sertão que na superfície é tão mineiro, tão geográfico, tem uma
escala épica que o transforma no campo de batalha entre as forças de
Deus e as do Diabo.

Quanto a Tolkien, criou a Terra Média (Middle Earth), supostamente uma
era remota no passado do nosso planeta, povoada por reis, guerreiros,
e raças fantásticas (elfos, anões, orcs, trolls, etc.), que foram
varridas da Terra depois que o Homem tornou-se o seu dono. À primeira
vista, o mundo de Tolkien é totalmente fantástico, mas basta ler uma
biografia sua (especialmente a de Humphrey Carpenter) para ver como
seu processo criativo era realista. Tolkien compunha para seus reis
árvores genealógicas inteiras, que se estendiam por milênios. Os
elfos têm uma linguagem completa, toda inventada por ele (e falada
pelos atores em trechos dos filmes). Seu cuidado ao descrever as
aventuras de Frodo o fazia calcular desde a fase da lua em
determinada noite até quanto tempo alguém levaria para ir a pé ou a
cavalo de um lugar para outro (aspecto em que autores de romances
não-fantásticos, como Walter Scott, muitas vezes se fazem de doidos).

Apesar das evidentes diferenças entre “Grande Sertão: Veredas” e “O
Senhor dos Anéis”, ambos têm um sopro épico semelhante, ambos são a
epopéia de um grupo pequeno de guerreiros do Bem enfrentando um grupo
impiedoso de guerreiros do Mal. Os “orcs” da Terra Média e os
“hermógenes” que Riobaldo enfrenta nas batalhas sertanejas são
personificações do Mal que um herói hesitante e problemático precisa
derrotar. Há muitos paralelos de detalhes que podem ser traçados
entre as duas obras, mas mais importante do que isto é o espírito de
nobre maniqueísmo medieval que os dois autores compartilhavam. O Bem
existe. O Mal também. E é preciso pegar em armas para combater o Mal.

\chapter{Aragorn e Riobaldo}

Já me referi nesta coluna aos pontos em comum entre a obra de Tolkien
e o “Grande Sertão: Veredas” de Guimarães Rosa. Em ambos se descreve
uma batalha épica do Bem contra o Mal, onde as tropas do Bem são
conduzidas por um herói problemático, cheio de dúvidas e hesitações.
O herói do “Grande Sertão” é Riobaldo, um jagunço a quem cabe liderar
o bando na jornada de vingança ao seu líder, Joca Ramiro, assassinado
à traição por um dos seus sub-chefes, Hermógenes. Joca Ramiro tem a
estatura épica e o caráter íntegro de um rei medieval (Riobaldo o
chama de “par-de-França”). Com sua morte, os jagunços ficam divididos
em bandos menores, cada qual comandado por um sub-chefe: Medeiro Vaz,
João Goanhá, Titão Passos, Sô Candelário, etc.

Riobaldo é o melhor atirador do grupo, o jagunço mais frio no gatilho,
o de melhor pontaria, talento que o torna respeitado e lhe vale
apelidos honrosos: “Tatarana”, “Urutu-Branco”. Ele junta-se por fim
ao grupo de Medeiro Vaz que, à morte, oferece-lhe a chefia. Ele
recusa. É um típico herói-em-dúvida, herói moderno, diferente dos
heróis mitológicos que em nenhum momento questionam a própria
coragem, as próprias motivações. Riobaldo pergunta-se: “Por que estou
fazendo isto tudo? Por que fazer isto? E por que logo eu?” Crivado de
dúvidas, ele recorre ao pacto com o Diabo, na encruzilhada das
Veredas Mortas, para a qual vai descrente, e de onde retorna sem ter
certeza se encontrou mesmo o Diabo ou não. Mas a partir desse
episódio ele parece mudado, imbuído de uma autoridade que não
parecera ter até então. Sob seu comando os bandos dispersos de
jagunços são unificados, e encurralam os “hermógenes” ou os “judas”,
como chamam aos inimigos, até derrotá-los na batalha final do
Paredão.

Em “O Senhor dos Anéis”, Aragorn é o legítimo herdeiro do trono, por
ser filho de Isildur, o rei que decepou a mão de Sauron, o Senhor das
Trevas, tomando dele o Anel do Poder. Aragorn, órfão, é criado pelos
elfos, que somente na idade adulta vem a saber de sua linhagem. Ele
torna-se um “Ranger”, patrulha as fronteiras da Terra Média, e
adquire não só experiência de batalhas e de privações como passa a
conhecer profundamente o território e o povo. Ele poderia repetir a
frase de Riobaldo: “Assim conheço as províncias do Estado, não há
onde eu não tenha aparecido.” Sabe que é destinado a ser rei, e que
para isso terá que unificar os diferentes reinos que se opõem a
Sauron (Gondor, Rohan, etc.). Mas sabe também que seu pai, Isildur,
cedeu à tentação do Anel e em vez de destruí-lo ficou com ele. Como
Riobaldo, ele não questiona a própria bravura ou sua competência como
guerreiro, mas, até ser arrastado pelos acontecimentos, hesita diante
da missão que lhe cabe. Ambos pertencem a uma estirpe de heróis (como
o Paul Atreides de “Duna”) que enfrentam a Morte sem medo, mas que
hesitam diante do Poder, por saberem que nenhum Poder é conquistado
com mãos limpas, por melhores que sejam as intenções do Herói. 

\chapter{Frodo e Riobaldo}

Na coluna de ontem, comparei o perfil do Riobaldo de “Grande Sertão:
Veredas” com o de Aragorn em “O Senhor dos Anéis”. Chamo isto de
“recorrência arquetípica”. Todo autor, ao recriar um tipo clássico de
personagem (no caso, o Guerreiro Heróico) dá-lhe (mesmo sem perceber)
traços que pertencem a uma tradição literária. Aragorn e Riobaldo são
grandes guerreiros, merecedores do posto mais alto do Poder, mas
cheios de dúvidas e de hesitações. A literatura está cheia de
riobaldos, mas nenhum como o de Rosa, e de aragorns, mas nenhum como
o de Tolkien. 

Há outro herói no “Senhor dos Anéis”, contudo, que tem muitos traços
em comum com Riobaldo: é Frodo Baggins (na tradução brasileira,
“Frodo Bolseiro”). Se Aragorn é o Herói Guerreiro a quem cabe
derrotar os exércitos do Mal e unificar sob um poder central os clãs
rivais, Frodo é, como Riobaldo, o sujeito pacífico sem vocação para
herói mas que, é jogado pela circunstâncias no meio de uma batalha, e
vira guerreiro a contragosto. É, assim como Riobaldo, um herói
problemático, pouco à vontade com este papel (Riobaldo: “Sou de ser e
executar, não me ajusto de produzir ordens”). E, como Riobaldo, é
também um herói atormentado pela presença do Diabo.

Riobaldo, embora se torne grande guerreiro, foi criado para uma vida
pacífica, e entrou na jagunçagem por acaso. Muito apegado aos livros,
é contratado para ser professor e secretário de Zé Bebelo, o capitão
das tropas que perseguiam os jagunços. Ao ver o primeiro combate de
verdade, desgosta-se daquilo e abandona o patrão. Uma série de acasos
o faz entrar em contato com o bando de jagunços de Joca Ramiro, entre
os quais reconhece o “Reinaldo”, ou Diadorim, um menino que conhecera
anos antes. E é por essa amizade, depois transformada em amor, que
Riobaldo se junta ao grupo, e de intelectual vira guerreiro. 

Riobaldo vai à encruzilhada das Veredas Mortas para fazer um pacto com
o Diabo. O Diabo não aparece, mas o jagunço volta de lá transformado,
mais seguro, mais ambicioso, e pela primeira vez disposto a tornar-se
líder do bando. Nos “Anéis”, Frodo, tendo colocado o Anel, entra em
contato direto com Sauron, o Senhor das Trevas, fica daí em diante
sob a mira deste, e acaba oferecendo-se a contragosto para destruir o
Anel e fazer desmoronar o império do Mal. Tanto Frodo quanto Riobaldo
são heróis contaminados por esse contato com o Mal. Frodo diz, após o
fim da aventura: “Estou ferido, Sam, ferido, e nunca vou me curar.”
Riobaldo, na encruzilhada: “Nunca em minha vida eu não tinha sentido
a solidão duma friagem assim. E se aquele gelado inteiriço não me
largasse mais.” Cada frase poderia ter sido dita pelo outro.

Note-se também que ambos exorcizam o Mal virando “escritores”:
Riobaldo constrói oralmente sua epopéia, ditando-a a um interlocutor
invisível (implicitamente o próprio Guimarães Rosa); Frodo é o
cronista que passa para o papel a Guerra dos Anéis, finalizando o
manuscrito iniciado por Bilbo.

\chapter{Da Terra Média ao Sertão}

A trilogia “O Senhor dos Anéis” foi trazida para o cinema por Peter
Jackson com a fidelidade possível quando se trata se adaptar um livro
tão imenso - na edição de bolso que possuo, ele ultrapassa as 1.500
páginas. A comunidade internacional de fãs de J. R. R. Tolkien teve
um papel importante nisto, pressionando diretor, roteiristas e
produtores, e impedindo as catástrofes dramatúrgicas típicas das
adaptações dos clássicos feitas em Hollywood. Assim, grande parte da
substância do livro acabou tendo um equivalente aceitável na tela.

Tolkien era um sujeito introvertido, ascético. Teve uma terrível
experiência nas trincheiras durante a I Guerra Mundial, quando perdeu
vários amigos. “O Senhor dos Anéis” foi escrito entre 1936 e 1949,
durante a \textsc{ii} Guerra, portanto. É comovente (e educativo) nos dias de
hoje, ver Tolkien afirmar que o manuscrito inteiro foi duas vezes
datilografado por ele próprio, porque mesmo sendo professor em Oxford
não podia pagar um datilógrafo. Era um conservador, apaixonado pela
Idade Média, sobre a qual falava aos seus alunos com entusiasmo;
conta-se que costumava encerrar essas descrições dizendo: “E aí veio
a Renascença, e estragou tudo.” Era profundamete católico,
misoginista como muitos britânicos de sua geração, detestava a
tecnologia e a modernização. O “Condado” (Shire) onde vivem os
hobbits é sua utopia pessoal, uma visão idealizada de uma Inglaterra
rural, pacífica mas resoluta, amante do sossego e dos livros, mas
capaz de ganhar uma guerra se ameaçada de invasão. 

“O Senhor dos Anéis”, apesar de ser aquele catatau, é apenas a ponta
do iceberg ficcional de Tolkien, que imaginou uma história-do-mundo
completa, desenhou mapas, criou genealogias, idiomas e alfabetos.
Tolkien fundou o que podemos chamar de “ficção catalográfica”, onde o
autor inventa todos os detalhes de um mundo. Ele inventou primeiro o
idioma dos elfos, e ao imaginar sua História começou a inventar as
narrativas que hoje conhecemos. Neste aspecto, sua obra tem uma
coerência e um mapeamento interno muito maior que a obra de Guimarães
Rosa. Rosa era meticuloso ao pesquisar, totalmente intuitivo e
instintivo ao escrever. Seus arquivos guardam uma quantidade
impressionante de material, mas ele não lhes deu a coerência
catalográfica que existe na obra de Tolkien. 

Extrovertido, vaidoso, bem-humorado, cosmopolita, Guimarães Rosa era
em muitos aspectos o avesso de Tolkien. Seus heróis (Riobaldo,
Diadorim, Augusto Matraga) são tomados muitas vezes por uma alegre
ferocidade, uma euforia-de-batalha que está ausente nos heróis dos
“Anéis”. Por outro lado, o romance de Tolkien é otimista (dos nove
membros da Irmandade do Anel, apenas Boromir morre), enquanto que o
“Grande Sertão” é no fundo a história de um fracasso, ou de uma
vitória de Pirro: no final, Riobaldo é feliz no jogo (na guerra) e
infeliz no amor. Tolkien era um homem triste que sonhava com finais
felizes, e Rosa um homem alegre que temia o Final.

\chapter{A eternidade dos pássaros}

Um dos meus contos preferidos sobre Realidade Virtual (mundos criados
em computador) é “In the Upper Room” de Terry Bisson (“Playboy”,
abril 1996), cujo texto completo pode ser obtido em:
http://www.freesfonline.de/authors/bisson.html. É a história de um
cara que se perde no interior de um catálogo virtual da Victoria´s
Secret, a famosa loja de lingerie. Nesse catálogo virtual, o cliente,
em vez de folhear uma revista com fotos das mulheres usando aqueles
trajes provocantes, “entra” numa mansão e percorre quartos onde
encontra simulações de belas modelos trajando coisas mais provocantes
ainda. Um crítico chamou a atenção para um detalhe que revela o
caráter serial, repetitivo, mecânico daquele mundo. Diz o narrador:
“I stood beside her at the window watching the robins arrive and
depart on the grass. It was the same robin over and over.” (“Fiquei
ao lado dela, observando os tordos chegarem e partirem do gramado.
Era o mesmo tordo, que ia e voltava, ia e voltava.”) Esse passarinho,
sempre o mesmo, revela a natureza artificial daquela paisagem; e o
escritor destaca isto com sutileza, com o uso de verbos ( “arrive”,
“depart”) que usamos normalmente para aviões, não para aves.

Que frio na espinha, que calafrio na alma não sentiríamos se
percebêssemos, em nosso mundo real, que certos elementos se repetem
em “loop” interminável, como os figurantes de filmes como “Cidade das
Trevas” ou “O 13º andar”? As pessoas acostumadas a jogar jogos em
\textsc{cd-rom} (de “The Sims” a “Zoo Tycoon” ou a “Great Theft Auto”) estão
acostumadas à presença desses figurantes cibernéticos: pessoas,
carros ou animais que estão sempre passando ao fundo, sempre os
mesmos, cumprindo as mesmas ações e os mesmos gestos, para nos dar a
ilusão de Vida Real.

O que não deixa de me trazer à memória a famosa “Ode to a Nightingale”
de John Keats (1819), em que o grande poeta romântico sente-se
desprendido da realidade terrena ao escutar o canto de um rouxinol,
cuja beleza o liberta por alguns instantes das tristezas da vida e da
fragilidade do corpo (Keats morreria de tuberculose dois anos depois,
aos 26 anos). Ele se sente transportado para um plano fora do espaço
e do tempo ao escutar aquela canção que, sem dúvida, é a mesma que os
rouxinóis cantam desde que o mundo é mundo. Keats percebeu (embora
não nos termos que aqui coloco) que um pássaro não passa de um corpo
físico descartável que executa um software musical repetitivo, sempre
o mesmo, e que nunca se extingue: “Thou wast not born for death,
immortal bird!” O pássaro não morre, porque é um figurante virtual em
nosso mundo; cada rouxinol de hoje é o mesmo que cantou na
Antiguidade remota. O poeta percebeu que era o mesmo rouxinol que ia
e voltava, cantando para indivíduos únicos, efêmeros, mortais,
conscientes da existência do Tempo, e de que só deixariam na Terra a
sua canção. O rouxinol de Keats continua cantando, mas me consola
pensar que Keats também.

\chapter{O Amor e a Fé}

Um dos textos cruciais da nossa literatura é o “Cântico dos Cânticos”
de Salomão. Digo “cruciais” ao pé da letra, no sentido de cruzamento,
encruzilhada, ponto onde dois caminhos divergentes se tocam por um
segundo. Esses caminhos são a comunhão afetiva com outro ser humano e
a comunhão mística com Deus -- ou seja, o Amor e a Fé.  Desde a
infância eu relia fascinado (sob o casto pretexto de “estar lendo a
Bíblia”) aqueles versos onde o poeta sai descrevendo sua Amada, fala
dos cabelos, dos olhos, do pescoço, dos peitos…  Ai de mim, depois
dos peitos o texto dava uma guinada de 90 graus, mudava de assunto.
Mas versos assim ainda ecoam como uma sextilha de cantador inspirado:
“Eu disse: subirei à palmeira, e colherei os seus frutos; e os teus
peitos serão como dois cachos de uvas; e o cheiro de tua boca como o
dos pomos…”

Estes versos me vêm à mente quando leio agora, numa antologia de
poesia espanhola do Século de Ouro, estes belos tercetos finais de um
sonetista anônimo, onde o “Tu” a quem se dirige é o próprio Deus:
“Muéveme, al fin, tu amor, y en tal manera, / que aunque no hubiera
cielo, yo te amara, / y aunque no hubiera infierno, te temiera. // No
me tienes que dar por que te quiera; / pues aunque lo que espero no
esperara, / lo mismo que te quiero te quisiera.”

Os grandes poetas místicos são justamente estes, que usam para se
dirigir a Deus a mesma retórica de intensa paixão dos grandes poetas
eróticos. E por que isto? Porque, por mais afastadas que pareçam, não
existem duas condições psicológicas mais assemelhadas do que o amor
por uma mulher e a fé em Deus. São os dois exemplos mais cabais do
caso em que Desejo é convertido em Certeza por um simples gesto da
Vontade. Esta certeza muitas vezes dá com os burros nágua: a paixão
não é correspondida, como a de Dante por Beatriz. Mas ainda assim o
poeta dá um jeito de escrever um catatau de milhares de versos
provando que Deus permitiu sua entrada no Paraíso e que lá estava
Beatriz à sua espera.

Certeza é certeza, e não existe certeza maior do que a de um
apaixonado, seja ele o poeta Neruda cantando os atributos físicos de
sua musa, seja San Juan de la Cruz desmanchando-se em estrofes de
ardor pelo seu Amado.  Nada garante a um apaixonado que seu amor é
correspondido ou mesmo que tem razão de ser, e nada garante ao
místico que Deus existe de fato.  Isto, no entanto, não abala essa
Certeza Absoluta, que não tem outro aval que o de si própria.  Nossos
poetas religiosos modernos (penso em Jorge de Lima, em Murilo Mendes)
já não se dirigem a Deus em termos diretamente eróticos. A
modernidade trouxe consigo essa cisão entre corpo e alma. Mas os
poetas místicos católicos ficarão para sempre como o melhor exemplo
de uma poesia movida a Fé, e de uma Fé movida pelos mesmos motores
hormonais que movem o Amor: a Certeza do próprio desejo físico, o
mais intenso (e aqui pra nós, o menos dispendioso) dos estimulantes
químicos.

\chapter{O velho e o mar}

A coleção de livros do Globo e da Folha de São Paulo pôs nas bancas “O
velho e o mar” de Hemingway. O livro poderia intitular-se “O menino,
o velho, o mar e o peixe”, porque são quatro seus protagonistas. Li
esse livro uma vez quando era garoto, com olhos de garoto para quem o
velho não poderia ser outro senão meu pai, que me deu o livro de
presente. Relê-lo agora, quarenta anos depois, é trocar de olhos; e
fechar um ciclo.

Hemingway era mais truculento e machista na vida real do que nos
livros. Talvez porque na vida real ele fosse um personagem de
Hemingway, e nos livros fosse apenas ele mesmo. “O velho e o mar” é
um livro de enorme aspereza e enorme doçura. Compõe um tríptico com
“Moby Dick” de Melville, onde o “peixe” é o Mistério, e com o
“Tubarão” de Spielberg, onde o peixe é o Mal. Em Hemingway, o marlim
é quase humano, e ao mesmo tempo uma imagem de indescritível beleza e
altivez.  O Velho pede-lhe perdão por matá-lo, e diz: “Nunca vi nada
mais bonito, mais sereno ou mais nobre do que você, meu irmão. Venha
daí e mate-me. Para mim tanto faz quem mate quem, por aqui.”

Há um verso de Oscar Wilde que diz “todo homem mata a coisa que ama”.
Pode até ser, mas este livro nos dá o caso, mais notável, do homem
que ama a coisa que mata. Lembro a cena final do “Augusto Matraga” de
Guimarães Rosa, depois que Matraga e Joãozinho Bem-Bem se esfaqueiam
mortalmente um ao outro. Caído, agonizando, Matraga vê a multidão a
gritar e debochar do jagunço que estertora ao lado, aí ergue-se e
diz: “Pára com essa matinada, cambada de gente herege!… E depois
enterrem bem direitinho o corpo, com muito respeito e em chão
sagrado, que esse aí é meu parente seu Joãozinho Bem-Bem!”

É um aspecto curioso da mentalidade masculina essa admiração guerreira
pelo inimigo cuja nobreza impõe respeito. A luta de morte não é
sempre um esforço para exterminar algo maligno, algo que desperta
apenas medo e asco. A luta às vezes se dá por causa de forças muito
acima dos dois lutadores, que estão ali, naquele momento, apenas
cumprindo um ritual cósmico. Eles são os dois pontos através dos
quais dois mundos entram em choque; o fato de um delesprecisar ser
destruído nesse choque não exclui o respeito, a admiração recíproca,
como num combate sem quartel entre dois samurais.

São tantas as interpretações sobre “O velho e o mar” que me arrisco a
somar mais uma. O velho é um Escritor. O menino é um Leitor. O peixe
é um Livro. O mar é o lugar onde os escritores vão buscar os livros,
seja este lugar o que fôr. Depois de toda aquela luta, o que o
Escritor consegue trazer ao mundo parece-lhe um monte de despojos, de
destroços sem sentido. Os outros o elogiam, mas ele sabe que foi o
único a ter a visão do Livro-como-era-para-ter-sido. Ele viu um
clarão, tentou transmitir seu reflexo. O que trouxe é pouco; mas pelo
menos ele teve o privilégio de ver o Livro como o Livro era antes de
ser trazido à terra.

\chapter{Notas}

\begin{description}%\setlength{\itemindent}{17em}
\item[1. Agatha Christie / O Homem do Fuzil]\hfill\\
Dizem que Agatha Christie é a personalidade literária inglesa mais
conhecida depois de Shakespeare.  Aqui no Brasil, suas obras vêm
sendo traduzidas continuamente desde a década de 1930, já tendo sido
lançadas pela Companhia Editora Nacional (“Série Negra”, “Coleção
Paratodos”), Editora Globo, de Porto Alegre (“Coleção Amarela”),
Edameris, José Olympio, Edições de Ouro...  São edições hoje raras
(mas felizmente não são caras) e que podem ser encontradas em sebos. 
No momento, as obras da Dama do Crime estão sendo reeditadas
principalmente pela Nova Fronteira, L \& PM, Record e Globo. 

Meus livros preferidos, entre as aventuras do detetive Hercule Poitor,
são \textit{O assassinato de Roger Ackroyd}, \textit{Assassinato no
Orient Express}, \textit{Morte no Nilo}, \textit{Os Crimes do ABC} e
a coletânea de contos \textit{Os Trabalhos de Hércules}, em que o
detetive resolve doze casos simétricos aos doze trabalhos do herói
mitológico.

Entre as aventuras de Miss Marple, a solteirona que resolve crimes
baseada em seu conhecimento da natureza humana, meu preferido é
\textit{A Maldição do Espelho}.  Um personagem curioso é Parker Pyne,
cujos contos aparecem num volume com variados títulos (mas sempre com
seu nome).  É um ex-estatístico aposentado que, como Miss Marple, usa
sua experiência de vida para prever o comportamento das pessoas e
resolver situações de crise.

Ms. Christie também tem livros notáveis de crime e suspense que não
envolvem seus personagens principais.  O mais famoso deles é
\textit{O Caso dos Dez Negrinhos}, cuja premissa básica (um grupo de
pessoas numa ilha, ameaçadas por um anfitrião invisível que as atraiu
para lá) influenciou desde séries de TV como \textit{Lost} até filmes
como a série \textit{Jogos Mortais (Saw)}. 

Veja também este site bastante informativo:\\
<http://us.agathachristie.com/site/home/>

Os comentários de Michael Grost sobre literatura policial em geral são
sempre bem fundamentados, e cheios de observações interessantes. 
Suas notas a respeito da Ms. Christie estão em:\\
<http://members.aol.com/MG4273/chris1.htm>

\item[2. A eternidade dos pássaros]\hfill\\
A “Ode a um Rouxinol” de Keats pode ser encontrada na maioria das
antologias de poesia inglesa, e na Internet em diversos saites,
como:\\
<http://www.bartleby.com/101/624.html>

Uma tradução de Augusto de Campos pode ser lida aqui: \\
<http://www.bartleby.com/101/624.html>

E aqui, outra tradução, por Raymundo Silveira:\\
<http://www.raymundosilveira.hpg.ig.com.br/pdf/poetas.pdf>

\item[3. O Fantasma da Liberdade / Liberdade demais atrapalha.]

“Ilusão trêda” significa mais ou menos, “ilusão vã”.  A citação de
Augusto dos Anjos é do poema “Versos de amor”: \textit{“Parece muito
doce aquela cana. / Descasco-a, provo-a, chupo-a... Ilusão trêda! / O
amor, poeta, é como a cana azeda, / a toda boca que o não prova
engana”.  } É uma imagem típica de um poeta da Zona da Mata
nordestina, criado num engenho de cana de açúcar.  Algumas canas
azedas podem ser identificadas porque têm veios arroxeados ou
marrons, mas muitas vezes só é possível distinguir a cana doce da
azeda provando-as.  

Alguns leitores talvez estranhem a citação do grupo Jaguaribe Carne ao
lado de Hermeto Paschoal e Tom Zé.  O grupo atua em João Pessoa (PB)
desde a década de 1970, liderado pelos irmãos Pedro Osmar e Paulo Ró.
 Seus discos e shows envolvem poesia de vanguarda, ritmos populares,
inovação gráfica, “happenings” no palco, virtuosismo instrumental. 
Artistas paraibanos como Chico César e Totonho (do grupo Totonho \&
Os Cabras) foram formados no “laboratório” do Jaguaribe Carne. 

\item[8. O tamanho do tempo]

Os “relógios moles” de Salvador Dali aparecem principalmente no quadro
“A persistência da memória”, que pode ser visto, por exemlo, aqui:\\
<http://fragmentosculturais.wordpress.com/2007/12/20/the-persistence-of-memory/>
 
\item[9. O amor e a fé]

O soneto espanhol pode ser lido, por exemplo, aqui:\\
<http://jardimdeluz.blogspot.com/2005/12/no-me-mueve-mi-dios-para-quererte.html>

\item[10. Eu vou estar enviando]

A revista \textit{Língua Portuguesa} (Ano II, número 21, de 2007) tem
um artigo de John Robert Scchmitz, intitulado “Repensando o
gerúndio”, em que avalia os prós e os contras desse modo de falar.

\item[12. Adeus, gringos]

Este artigo, traduzido por Tom Moore sob o título “Watching the
Tourists in Brazil”, foi publicado pelo jornalista Bill Hinchberger
em seu websaite \textit{BrazilMax}, onde despertou comentários de
natureza variada:\\
<http://www.brazilmax.com/columnist.cfm/idcolumn/68>

\item[14. Ler e escrever]

\textit{Bolacha sete-capas} é o nome de um tipo de bolacha muito
popular no Nordeste.  São várias camadas quebradiças, superpostas,
muito finas, como nos salgadinhos a que se chama “folhados”.

\item[19. Ronaldinho Gaúcho]

\textit{Tivuco} -- No futebol, chute muito forte; canhão. 

\item[20. A carta traz o carteiro]

Treze e Campinense (conhecidos respectivamente como o Galo e a Raposa)
são os dois principais clubes de futebol de Campina Grande e da
Paraíba.  Um jogo entre os dois equivale a um Fla-Flu, um Ba-Vi, um
Gre-Nal...

\item[21. O charme de um dia nublado]

Celly Campello foi provavelmente a primeira cantora de rock do Brasil,
e um dos seus primeiros sucessos foi uma versão de uma canção
italiana, que no Brasil se chamou “Banho de Lua” (1960):
\textit{“Tomo um banho de lua / fico branca como a neve / se o luar é
meu amigo / censurar ninguém se atreve / como é bom sonhar contigo /
oh, luar tão cândido!”}.  A canção foi regravada pelos Mutantes em
seu segundo álbum, de 1969.  

“Dia Branco” é uma canção de amor de Geraldo Azevedo e Renato Rocha:
\textit{“Se você vier / pro que der e vier / comigo...”}

\item[23. Eu era feliz e não sabia]

Congresso de Violeiros é um festival competitivo que se trava, no
Nordeste, entre duplas de poetas que cantam de improviso.  Quando
cada dupla sobe ao palco, faz-se um sorteio, num envelope com
assuntos previamente escolhidos, e ele têm que cantar de improviso
sobre o tema sorteado.  Um “mote” são duas linhas fornecidas aos
improvisadores para que eles criem uma estrofe inteira a partir
delas, que devem ser as duas últimas.

\item[26. A tragédia da vida]

Oswaldo Sangiorgi era autor de livros de Matemática muito usados nos
colégios na década de 1960; Borges Hermida era autor de livros de
História.

\end{description}

