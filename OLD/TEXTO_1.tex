\pagestyle{plain}
\part{Livro dos dragões}

\chapter{Perseu e Andrômeda\subtitulo{Ovídio}}

Perseu pegou suas sandálias aladas e as pôs nos pés. Prendeu sua
espada curva à cintura, e com o movimento das asas em suas sandálias
alçou vôo e atravessou os céus. Era já o segundo dia em que estava
retornando de seu combate com a Medusa, trazendo como troféu a cabeça
daquela monstra de cabelos de serpente. Voou sobre inúmeros povos,
cujas terras se estendiam em todas as direções, até avistar as tribos
etíopes e o reino de Cepheus. Lá, a jovem e inocente Andrômeda estava
a ponto de pagar injustamente pela insensata vaidade de sua mãe, a
rainha Cassiopeia, que se gabara de serem ela e sua filha muito mais
belas que as Nereidas, despertando a ira de Netuno, que enviara uma
inundação e um monstro marinho para devastar a costa do reino;
Cepheus consultara então o oráculo de Ammon, que lhe respondera que
seu reino só seria salvo se ele entregasse sua filha ao monstro
marinho.

Quando Perseu viu a princesa, com seus braços acorrentados à firme
rocha, teria achado que era uma estátua de mármore, não fosse pela
brisa que agitava os cabelos dela, e as cálidas lágrimas que
escorriam de seus olhos. Sem se dar conta, ele imediatamente se
apaixonou por ela. Encantado de ver tamanha beleza, deteve-se
maravilhado, e quase esqueceu de manter suas asas se movendo no ar.
Assim que pousou, disse:

-- Você não devia estar assim acorrentada; os laços que lhe cabem são
aqueles que unem os corações daqueles que se amam! Diga-me seu nome,
eu lhe peço, e o nome de seu país, e por que está acorrentada.

A princípio, ela nada disse, pois sendo uma virgem, não ousava
dirigir-se a um homem. Teria escondido por pudor sua face com as
mãos, se elas não estivessem acorrentadas. O que podia fazer, ela
fez; e era deixar seus olhos se encherem ainda mais de lágrimas. Ante
a insistência de Perseu, que repetia suas perguntas, ficou com medo
que sua recusa em falar fosse entendida como admissão de culpa; então
ela lhe disse o nome do país, o dela, e também como sua mãe, uma bela
mulher, deixara sua beleza subir-lhe à cabeça.

Antes que ela terminasse, as águas se agitaram em ruidosos vagalhães,
e das profundezas do oceano veio o ameaçador monstro, tão enorme que
cobria toda a extensão das ondas. A jovem gritou. Seu desconsolado
pai estava bem perto, e sua mãe também. Os dois estavam tomados por
um profundo desespero, embora sua mãe ainda com mais razão. Nenhuma
ajuda podiam lhe oferecer, apenas as lágrimas e os lamentos que
cabiam nas circunstâncias. E assim ficaram, até que o recém-chegado
forasteiro, Perseu, se dirigisse a eles, dizendo:

-- Haverá tempo de sobra para as lágrimas depois, mas para ajudá-la o
tempo é curto. Meu nome é Perseu, sou filho de Júpiter e Danae, a
qual era prisioneira numa torre quando Júpiter a fecundou sob a forma
de uma chuva de ouro. Sou aquele Perseu que venceu a Górgona de
cabelos de serpente, e que ousa viajar pelas brisas do ar com asas em
seus pés. Já seria o bastante para vocês me quererem como seu genro
se eu pedisse a mão de sua filha; mas se os deuses me ajudarem, vou
tentar acrescentar ainda outro serviço aos que já fiz. Quero esse
compromisso de vocês: que ela seja minha, se com a minha coragem eu
puder salvá-la.

Os pais dela concordaram -- quem, de fato teria hesitado? Imploraram
pela ajuda dele, prometendo que além da filha deles, teria também o
reino como dote. 

Mas eis que, como um navio veloz cortando as ondas com sua proa aguda,
impulsionado pelos braços fortes da suada tripulação em seus remos, o
monstro avançou, rompendo as ondas com seu peito. Não estava mais
longe dos rochedos do que o quanto alcança o projétil de uma fronda
baleárica, quando de repente o herói, saltando da terra, lançou-se
muito alto nas nuvens. A sombra dele se projetou na superfície do
oceano, e o monstro a atacou com toda sua fúria. Então Perseu
investiu lá do alto; como age a águia de Júpiter ao ver do céu uma
serpente ao sol num deserto, e a agarra por trás, cravando suas
garras ávidas nas escamas do pescoço do réptil, para evitar o ataque
de suas cruéis presas, foi o que Perseu velozmente vindo do ar fez,
atacando o monstro pelas costas e, ao som de seus urros, enterrando
fundo no ombro direito dele sua espada curva. Atormentado por tão
profunda ferida, o monstro arqueou para o alto seu corpo, voltou a
afundá-lo nas águas, e começou a se virar de um lado para o outro
como um feroz porco selvagem cercado e aterrorizado por uma matilha
de cães ladrando. O herói, valendo-se de suas velozes asas,
desviava-se das terríveis mandíbulas do monstro, desferindo golpes
com sua espada curva sempre que tinha oportunidade; acertando ou as
costas cobertas de conchas ocas da criatura, ou suas costelas, ou
ainda o ponto em que sua cauda se transformava no rabo de um peixe.
Da sua boca o monstro lançava golfadas de água do mar tingidas com o
vermelho do sangue, e logo as asas de Perseu ficaram úmidas e pesadas
com a espuma. Sem se atrever a continuar confiando em suas enxarcadas
asas, ele percebeu uma pedra cujo topo ficava acima da superfície
quando as águas estavam paradas, mas que o movimento das ondas cobria
inteira. Nela ele se apoiou, e segurando firme nas escarpas com a mão
esquerda, com a outra cravou sua espada duas, três, quatro vezes nos
flancos do monstro; tantas foram as vezes em que repetidamente o
atingiu, que logo das costas do mar até a morada dos deuses no céu só
o que se ouvia eram aplausos e gritos de júbilo.

Cassiopeia e Cepheus ficaram felicíssimos, e acolheram Perseu como
genro, dizendo que a casa deles lhe devia sua salvação e
continuidade. A jovem, causa e recompensa desse feito heróico, foi
libertada de suas correntes. O herói lavou suas mãos vitoriosas na
água do mar, e para que a áspera areia não prejudicasse a cabeça
cheia de serpentes da Medusa, antes de colocá-la no chão forrou-o com
folhas e depois com algas-marinhas. As algas, colhidas frescas e
ainda vivas, mostraram-se sensíveis ao poder da monstruosa cabeça, e
se endureceram, adquirindo uma nova e estranha rigidez em suas folhas
e ramos. As ninfas do mar testaram então esse milagre, aplicando-o a
vários ramos, e ficaram encantadas ao ver que ele sempre se repetia;
e espalharam pelas ondas suas sementes, produzindo mais e mais dessa
nova substância. E até hoje o coral mantém essa mesma característica,
enrijecendo-se em contato com o ar; o que é uma planta debaixo da
água se torna rocha acima da superfície.

Perseu erigiu então três altares de turfa em honra a três deuses: o da
esquerda para Mercúrio, o da direita para Minerva, e no centro, entre
os dois, um para Júpiter. Para a deusa sacrificou uma vaca, para
Mercúrio um bezerro, e para o mais poderoso dos deuses um touro. E
sem mais demora pediu Andrômeda como recompensa por sua grande
proeza, e a aceitou sem dote algum. Cupido e Hymen agitaram em frente
ao par as torchas nupciais; incenso em abundância tornou perfurmadas
as chamas, guirlandas foram penduradas no teto, e por toda a parte se
ouviram as liras e as flautas, e os cantos testemunhando e celebrando
a felicidade dos corações. Os grandes portões foram por inteiro
abertos, revelando o dourado luxo do palácio, e todos os nobres da
corte de Cepheus participaram do fausto banquete que foi servido.

\chapter{Stan Bolovan\subtitulo{Andrew Lang}}

O que aconteceu uma vez, aconteceu mesmo; pois se não tivesse
acontecido essa história nunca teria sido contada.

Nos arredores de uma aldeia, bem onde os bois são deixados para pastar
e os porcos perambulam enfiando seus focinhos nas raízes das árvores,
havia uma casinha. 

Nela morava um homem com sua mulher, e a mulher passava o dia todo
triste.

-- Cara esposa, o que há de errado para você ficar com sua cabeça
pendendo como uma rosa murcha? -- perguntou o marido uma manhã. -- Você
tem tudo o que quer; por que não pode viver contente como as outras
mulheres?

-- Deixe-me em paz, e não tente descobrir a razão -- ela respondeu,
explodindo em lágrimas; o homem achou que não era uma boa hora para
ficar fazendo perguntas a ela, e saiu para trabalhar.

Ele não conseguia, entretanto, esquecer o assunto, e alguns dias
depois voltou a perguntar a razão da tristeza dela, mas recebeu
apenas a mesma resposta. Por fim ele não podia mais aguentar, e
tentou uma terceira vez, e então sua mulher voltou-se para ele e
respondeu:

-- Por Deus! -- ela exclamou -- Por que você não pode deixar as coisas
como estão? Se eu lhe contasse, você iria ficar tão infeliz quanto
eu. Se ao menos você acreditasse que é melhor não saber de nada!

Mas nenhum homem jamais ficaria contente com uma resposta assim.
Quanto mais se implora para não perguntar, maior fica a curiosidade
de saber tudo.

-- Bom, se você precisa tanto saber -- disse enfim sua mulher -- eu vou
dizer. Não temos sorte nessa casa, nenhuma sorte!

-- Mas não é nossa vaca a que mais dá leite da aldeia? Não ficam nossas
árvores tão cheias de frutos quanto as colmeias de abelhas? Alguém
tem um milharal tão bom quanto o nosso? Olhe, você está falando
bobagem ao dizer isso.

-- Sim, tudo o que você disse é verdade, mas não temos filhos.

Então Stan comprendeu, e quando um homem comprende e tem seus olhos
abertos não há mais o que fazer. Desse dia em diante a casinha
continha um homem infeliz além de uma mulher infeliz. E ver a
tristeza de seu marido deixou a mulher mais desconsolada que nunca.

E assim ficaram as coisas por um tempo.

Algumas semanas se passaram, e Stan resolveu consultar um sábio que
vivia a um dia de viagem de sua casa. O sábio estava sentado em
frente à porta de sua casa quando ele chegou, e Stan se ajoelhou
diante dele.

-- Dê-me filhos, senhor, dê-me filhos.

-- Cuidado com o que você está pedindo -- respondeu o sábio. -- E se
filhos forem um fardo para você? Você é rico o suficiente para
alimentá-los e vesti-los? 

-- Apenas faça com que eu os tenha, e eu me virarei de algum jeito! -- e
o sábio lhe fez um sinal para ele partir.

Ele voltou para casa naquela noite cansado e sujo da viagem, mas com
esperança no coração. Quando estava se aproximando de sua casa, o som
de vozes chegou até seus ouvidos. Ele olhou donde vinham, e percebeu
o lugar todo cheio de crianças. Crianças no jardim, crianças no
quintal, crianças olhando de todas as janelas -- pareceu ao homem que
todas as crianças do mundo haviam se juntado ali. E nenhuma era maior
que a outra, mas cada uma era menor que a outra, e cada uma era mais
barulhenta, mais petulante e mais atrevida que as outras, e Stan
parou ali e gelou de horror ao se dar conta que eram todas dele.

-- Meu Deus! Quantas que há! Quantas! -- murmurou para si mesmo.

-- Ah, mas nem uma demais -- sua mulher sorriu, aparecendo com mais uma
multidão de crianças nas barras da saia.

Mas mesmo ela descobriu que não era tão fácil cuidar de cem crianças,
e depois de alguns dias, quando elas tinham comido toda a comida que
havia na casa, começaram a gritar:

-- Pai! Estou com fome! Estou com fome! 

E Stan coçou a cabeça e se perguntou o que ia fazer agora. Não que ele
achasse que havia crianças demais, pois sua vida parecia mais cheia
de alegria desde que elas apareceram, mas tinham chegado ao ponto em
que não sabia mais como alimentá-las. A vaca parara de dar leite, e
não estava ainda na época das árvores darem frutos.

-- Sabe, minha velha -- ele disse um dia a sua mulher -- preciso correr
mundo e tentar trazer comida de algum jeito, embora não saiba dizer
donde vou tirá-la.

Para o homem com fome toda estrada é comprida, e ainda havia sempre a
lembrança de que tinha a fome de cem crianças para satisfazer, além
da sua própria.

Stan andou, andou e andou, até chegar ao fim do mundo, onde o que
existe se mistura com o que não existe, e lá ele viu, a uma pequena
distância, um ovil, com sete ovelhas dentro. Na sombra de umas
árvores estava o resto do rebanho.

Stan se escondeu, esperando que conseguiria fazer algumas escapulirem
com ele discretamente, para levá-las para alimentar sua família, mas
logo descobriu que isso não ia dar certo. Pois à meia-noite ouviu um
barulho: era um dragão que viera voando e levou embora um carneiro,
uma ovelha e um cordeiro, e três vacas que estavam ali por perto. E
além disso tirou também o leite de 77 ovelhas, que levou para sua
velha mãe nele se banhar para recuperar sua juventude. E isso
acontecia toda noite.

O pastor se lamentava em vão: o dragão apenas ria, e Stan viu que
aquele não era um bom lugar para conseguir comida para sua família.

Mas apesar de entender que era quase impossível lutar contra um
monstro tão poderoso, a lembrança de suas crianças famintas em casa
se agarrava a ele feito um carrapicho, do qual não dá para se livrar,
e por fim ele disse ao pastor:

-- O que você me daria se eu o livrasse do dragão?

-- Um de cada três carneiros, uma de cada três ovelhas, um de cada três
cordeiros -- respondeu o pastor.

-- É uma barganha -- disse Stan, embora naquele momento não soubesse
como, supondo que ele vencesse o dragão, iria conseguir levar um
rebanho tão grande para casa.

Todavia, esse problema podia ser resolvido depois. No momento, a noite
estava chegando, e ele precisava pensar como seria o melhor jeito de
lutar com o dragão.

Precisamente à meia-noite, Stan foi tomado por uma sensação horrível,
nova e estranha para ele -- uma sensação que ele não encontrou
palavras para descrever a si mesmo, mas que quase o forçou a desistir
da batalha e pegar o caminho mais curto para casa. Ele se virou para
partir; mas aí lembrou das crianças, e se virou de novo.

-- É você ou eu -- disse Stan para si mesmo, e se posicionou junto ao
rebanho.

-- Pare! -- ele gritou de repente, quando o ar se encheu com o barulho
das asas do dragão descendo.

-- Ora essa! -- exclamou o dragão, olhando em volta. -- Quem é você, e de
onde vem? 

-- Eu sou Stan Bolovan, que come pedras toda noite, e durante o dia as
flores da montanha; e se você bulir com essas ovelhas eu vou entalhar
uma cruz em seu lombo.

Ao ouvir essas palavras o dargão parou bem quieto no meio da estrada,
pois sabia que havia encontrado um adversário a sua altura.

-- Mas você vai ter antes que lutar comigo -- ele disse com voz trêmula,
pois quando alguém o enfrentava de verdade ele não era nem um pouco
corajoso.

-- Lutar com você? -- retrucou Stan. -- Ora, se posso matá-lo com um
sopro…

Então, pegando um grande queijo que estava a seus pés, acrescentou:

-- Vá buscar uma pedra como essa no rio, para que não percamos tempo em
descobrir quem é o melhor.

O dragão fez o que Stan lhe pediu, e trouxe uma pedra do riacho.

-- Você consegue tirar leite da sua pedra? -- Stan perguntou.

O dragão catou sua pedra com uma mão, e espremeu-a até virar pó, mas
nenhum leite escorreu dela.

-- Claro que não! -- ele disse, com uma certa raiva.

-- Bom, se você não consegue, eu consigo -- respondeu Stan, e apertou o
queijo até o leite escorrer entre seus dedos.

Quando o dragão viu aquilo, achou que já era mais que hora de voltar
para sua casa, mas Stan ficou no caminho dele.

-- Ainda temos contas a acertar -- ele disse -- sobre o que você anda
fazendo por aqui.

E o pobre dragão estava com muito medo para se mexer, temendo que Stan
o matasse com um sopro e o enterrasse no meio das flores dos pastos
das montanhas.

-- Escute -- ele disse enfim. -- Posso ver que você é uma pessoa muito
útil, e minha mãe está precisando de alguém como você. Digamos que
você preste serviços a ela por três dias, que duram tanto quanto um
de seus anos, e ela lhe pague com sete sacos cheio de ducados por
cada dia.

Três vezes sete sacos cheios de ducados! A oferta era muito tentadora,
e Stan não conseguiu resistir a ela. Não gastou palavras, apenas
acenou que concordava para o dragão, e os dois partiram pela estrada.

Foi uma jornada muito, muito longa, mas quando chegaram ao fim dela
encontraram a mãe do dragão, que era tão velha quanto o próprio
tempo, esperando por eles. Stan viu de longe os olhos dela brilhando
como lanternas, e quando entraram na casa viram uma enorme chaleira
no fogo, cheia de leite. Quando a velha mãe viu que seu filho voltara
de mãos vazias ficou muito brava, e fogo e chamas saíram de suas
narinas, mas antes de ela dizer qualquer coisa o dragão voltou-se
para Stan.

-- Fique aqui -- disse -- e me espere; vou explicar as coisas para minha
mãe.

Stan já estava se arrependendo amargamente de ter vindo para tal
lugar, mas como já estava lá, não havia nada a fazer senão enfrentar
tudo calmamente, e não demonstrar que estava com medo.

-- Olhe, mãe -- disse o dragão assim que ficaram sozinhos. -- Eu trouxe
esse homem para me livrar dele. Ele é um sujeito terrível, que come
rochas e pode tirar leite de pedras -- e lhe contou o que acontecera
na noite passada.

-- Ah, deixe-o comigo! -- ela disse. -- Nunca deixei um homem escapar por
entre meus dedos.

Então Stan teve que ficar prestando serviços à velha mãe.

No dia seguinte ela lhe disse que ele e seu filho deviam ver quem era
mais forte, e catou uma enorme clava, envolta sete vezes com ferro.

O dragão pegou-a como se fosse uma pena, e depois de girá-la sobre sua
cabeça, atrirou-a a três milhas de distância, dizendo a Stan para
jogá-la mais longe se pudesse.

Andaram até o lugar onde caíra a clava. Stan se inclinou e
experimentou pegá-la; e então um grande medo lhe assomou, pois sabia
que ele e todas suas crianças juntas jamais conseguiriam levantar do
chão aquela clava. 

-- O que você está fazendo? -- perguntou o dragão.

-- Eu estava pensando o quanto é bela essa clava, e me deu pena de que
seja ela a causar a sua morte.

-- Minha morte? O que você quer dizer? -- perguntou o dragão.

-- Apenas que receio que, se atirá-la, você nunca mais verá outro
amanhecer. Você não faz ideia do quanto eu sou forte!

-- Ah, não se preocupe, atire-a de uma vez.

-- Se você quer isso mesmo, vamos festejar por três dias; ao menos
assim você terá três dias a mais de vida.

Stan falou com tanta calma que dessa vez o dragão ficou com um pouco
de medo, embora não acreditasse que a coisa ia ser tão ruim quanto
Stan dissera.

Eles voltaram para a casa, pegaram toda a comida que acharam na
despensa da velha mãe, e voltaram para onde estava a clava. Então
Stan sentou-se no saco de mantimentos, e ficou tranquilamente
observando a lua que se punha.

-- O que você está fazendo? -- perguntou o dragão.

-- Esperando a lua sair do meu caminho.

-- Como assim? Não entendi.

-- Você não está vendo que a lua está exatamente no meu caminho? Mas
claro, se você quiser, posso atirar a clava na lua.

Essas palavras deixaram o dragão inquieto pela segunda vez.

Ele tinha grande estima por aquela clava, que havia sido herança de
seu avô, e não tinha a menor vontade de que ela fosse parar na lua.

-- Vou lhe dizer uma coisa -- ele disse, depois de pensar um pouco. --
Não precisa atirar a clava. Eu a atiro uma segunda vez, e fica por
isso mesmo.

-- Não, de jeito nenhum! -- respondeu Stan. -- Basta esperar a lua se
pôr.

Mas o dragão, temendo que Stan cumprisse suas ameaças, tentou toda
espécie de suborno para evitá-las, e no fim teve que prometer a Stan
sete sacos de ducados antes de enfim conseguir jogar de volta a clava
ele mesmo.

-- Ah, nossa, ele é mesmo um homem forte -- disse o dragão para sua mãe.
-- Você acredita que foi a maior dificuldade evitar que ele atirasse a
clava na lua? 

Então a velha ficou inquieta também, só de pensar nisso! Atirar coisas
na lua não é brincadeira! Então não se falou mais na clava, e no diz
seguinte todos tinham outro assunto com que se preocupar.

-- Vão buscar água! -- disse a mãe, assim que amanheceu, e deu a eles
doze odres de pele de búfalo com a ordem de enchê-los até de noite.

Eles partiram na mesma hora para o riacho, e num piscar de olhos o
dragão enchera todos os doze, levou-os até a casa, e os trouxe de
volta para Stan. Stan estava cansado: mal conseguia levantar os odres
vazios, e teve um arrepio só de pensar o que iria acontecer qaundo
estivessem cheios. Mas a única coisa que fez foi tirar uma faca de
seu bolso e começar a cavar a terra perto do riacho.

-- O que você está fazendo aí? Não vai carregar a água para casa? --
perguntou o dragão.

-- O quê? Ora, isso é fácil demais! Eu vou é levar o riacho inteiro!

Essas palavras fizeram o queixo do dragão cair. Era a última coisa que
passaria por sua cabeça, pois o riacho sempre estivera ali, desde os
tempos de seu avô.

-- Vou lhe dizer uma coisa -- disse. -- Deixe que eu carrego os odres
para você.

-- De jeito nenhum -- respondeu Stan, continuando a cavar, e o dragão,
temendo que ele cumprisse sua ameaça, tentou toda espécie de suborno,
e no fim teve que prometer de novo sete sacos de ducados para fazer
Stan concordar em largar o riacho ali em paz e deixar o próprio
dragão levar a água para a casa.

No terceiro dia a velha mãe mandou Stan buscar lenha na floresta e,
como sempre, o dragão foi junto com ele.

Antes de você contar até três ele já tinha derrubado mais árvores que
Stan poderia ter cortado em toda a sua vida, e as arrumara
devidamente em fileiras. Quando o dragão tinha terminado, Stan
começou a olhar a sua volta e, escolhendo a maior das árvores, subiu
nela, onde cortou um longo cipó e com ele amarrou a ponta dela à
ponta da árvore seguinte ao lado. E assim ele fez com toda uma
fileira de árvores.

-- O que você está fazendo aí? -- perguntou o dragão.

-- Basta você olhar para saber -- repsondeu Stan, continuando calmamente
seu trabalho.

-- Por que você está amarrando as árvores juntas?

-- Para não ter trabalho à toa; quando eu arrancar uma, todas as outras
já virão junto. 

-- Mas como você vai carregá-las para casa?

-- Ora essa! Você ainda não entendeu que eu vou levar a floresta
inteira de volta comigo? -- disse Stan, amarrando mais duas árvores
enquanto falava.

-- Vou lhe dizer uma coisa -- exclamou o dragão, tremendo de medo só de
pensar nisso -- deixe que eu carrego a lenha para você, e eu lhe dou
sete vezes sete sacos cheios de ducados.

-- Você é um bom sujeito, e eu concordo com sua proposta -- Stan
respondeu, e o dragão carregou a lenha.

Então os três dias de serviço que era para ser contados como um ano já
haviam terminado, e a única coisa que preocupava Stan era: como levar
todos aqueles ducados de volta para casa?

Naquela noite o dragão e sua mãe tiveram uma longa conversa, mas Stan
ouviu-a inteirinha por um buraco no teto.

-- Que infelicidade a nossa, mãe -- disse o dragão -- esse homem logo vai
nos dominar. Dê a ele o dinheiro, para podermos nos livrar dele.

Mas a velha mãe gostava de dinheiro, e a ideia não a agradava.

-- Escute -- ela disse -- você precisa matá-lo esta noite.

-- Tenho medo -- disse ele.

-- Não há nada a temer -- retrucou a velha mãe. -- Quando ele estiver
dormindo, pegue a clava e acerte-o na cabeça com ela. Vai ser fácil.

E teria sido mesmo, se Stan não tivesse ouvido tudo. Quando o dragão e
sua mãe apagaram as luzes, ele pegou a gamela dos porcos, encheu-a de
terra e a pôs em sua cama, cobrindo-a com suas roupas. Então ele se
escondeu sob a cama, e se pôs a roncar bem alto.

Logo o dragão entrou silenciosamente no quarto e deu um tremendo golpe
no lugar onde a cabeça de Stan deveria estar. Stan gemeu bem alto
debaixo da cama, e o dragão saiu tão silenciosamente quanto entrara.
Assim que ele fechou a porta, Stan tirou dali a gamela dos porcos, e
deitou-se em seu lugar, depois de ter deixado tudo limpo e arrumado;
mas foi esperto o bastante para não pregar os olhos naquela noite.

Na manhã seguinte ele veio para a sala quando o dragão e sua mãe
estavam tomando café da manhã.

-- Bom dia -- disse.

-- Bom dia. Dormiu bem?

-- Ah, muito bem, mas sonhei que uma pulga havia me mordido, e ainda a
estou sentindo.

O dragão e sua mãe se entreolharam.

-- Você ouviu só? -- ele sussurrou. -- Ele falou de uma pulga. E eu
quebrei minha clava na cabeça dele.

Dessa vez a mãe ficou com tanto medo quanto seu filho. Não havia nada
a fazer com um homem como esse, e ela se apressou a encher os sacos
com os ducados, para se livrar de Stan o mais rápido possível. Mas,
por sua parte, Stan estava tremendo feito vara verde, pois não
conseguia levantar nem mesmo um saco do chão. Então ele ficou ali
parado olhando para eles.

-- O que você está esperando aí? -- perguntou o dragão.

-- Ah, eu estava esperando aqui porque acabou de me ocorrer que
gostaria de ficar a serviço de vocês mais um ano. Tenho vergonha de
chegar em casa e verem que eu trouxe tão pouco. Tenho certeza que vão
dizer “olha só o Stan Bolovan, que em um ano acabou ficando tão fraco
quanto um dragão”.

Nesse momento uma exclamação de pasmo escapou tanto do dragão quanto
de sua mãe, que declarou que ela ia lhe dar sete vezes ou mesmo sete
vezes sete vezes o número de sacos se ele fosse embora.

-- Vou lhe dizer uma coisa -- enfim Stan disse. -- Estou vendo que não
querem que eu fique, e não quero incomodá-los. Eu irei embora
imediatamente, mas sob a condição do dragão carregar para mim o
dinheiro até em casa; assim não passarei vergonha frente a meus
amigos.

Mal acabara de falar e o dragão já agarrara os sacos e os empilhara em
suas costas. Então ele e Stan partiram.

O caminho de volta, se na verdade não era muito comprido, ainda assim
demorou um bocado para Stan, mas enfim ele ouviu as vozes de suas
crianças, e parou de repente. Não queria que o dragão ficasse sabendo
onde ele morava, temendo que algum dia ele viesse recuperar seu
tesouro. Não haveria nada que ele pudesse dizer para se livrar do
monstro? De repente uma ideia lhe veio à cabeça, e ele se virou.

-- Não sei bem o que fazer -- disse. -- Tenho cem filhos, e tenho medo
que eles possam machucá-lo, pois estão sempre prontos a entrar numa
briga. Em todo o caso, vou fazer todo o possível para protegê-lo.

Cem crianças! Isso não era brincadeira mesmo! O dragão até deixou cair
os sacos, tanto o seu terror, mas tratou de catá-los de novo. Foi
então que as crianças, que não tinham comido nada desde que seu pai
partira, vieram correndo na direção dele, brandindo facas na mão
direita e garfos na esquerda, e gritando “Papai! Dê-nos carne de
dragão! Queremos carne de dragão!”

Ao ver essa terrível cena o dragão não esperou nem mais um instante:
largou os sacos e saiu voando o mais rápido que podia, e tão
aterrorizado ficou com o destino do qual escapou por pouco que desse
dia em diante nunca mais teve coragem de mostrar a cara pelo mundo de
novo.

\chapter{São Jorge e o dragão\subtitulo{Jacobus de Voragine e William Caxton}}

São Jorge era um cavaleiro que nascera na Capadócia. Um dia ele foi
para a província da Líbia, onde havia uma cidade chamada Sylene. E
perto dessa cidade havia uma lagoa feito um mar, e nela um dragão que
ameaçava envenenar todo o país. Um dia o povo se juntou para matá-lo,
mas ao vê-lo se apavoraram e fugiram. E quando chegou a noite, o
dragão ia envenenar o povo com o ar que expirava, de modo que eles
decidiram dar todo dia duas ovelhas para alimentá-lo, para que assim
ele não fizesse mal ao povo. 

Quando as ovelhas acabaram, chegou a vez dos homens, e por um decreto
na cidade se estabeleceu que seriam entregues primeiro as crianças e
os jovens, e quem se recusasse a entregar seus filhos iria em lugar
deles. E assim foi que muitas crianças e jovens foram entregues ao
dragão, tantas que acabou chegando a vez da filha do rei. O rei
entristecido disse a seu povo:

-- Pelo amor dos deuses, levem ouro e prata e tudo o que tenho, mas
deixem eu ficar com minha filha. 

Eles responderam:

-- Como assim? Foi Vossa Majestade quem fez a lei e mandou cumprir, de
tal modo que agora todos nossos filhos estão mortos, e agora quer
fazer diferente? Sua filha deverá ser entregue, ou então levaremos
Vossa Majestade. 

Quando o rei viu que não havia nada a fazer começou a chorar e disse
para a filha dele que jamais iria ver as bodas dela. Então ele voltou
ao povo e pediu oito dias de prorrogação, e o povo lhe concedeu. E
quando os oito dias se passaram voltaram a ele dizendo que já era a
hora, pois ele podia ver que a cidade estava perecendo.

Então o rei fez vestirem sua filha como se ela fosse se casar, e a
abraçou e beijou, e a abençoou, e a conduziu para o lugar em que
estava o Dragão. 

Quando ela estava lá, São Jorge passou por ali. Vendo a dama,
perguntou o que ela fazia ali, e ela disse:

-- Siga seu caminho, belo jovem, para que não morra você também.

Então ele disse:

-- Diga-me o que a aflige e por que você chora, e nada tema.

Quando ela viu que ele tanto queria saber, ela disse que havia sido
deixada para o dragão. São Jorge disse:

-- Bela jovem, nada tema, pois eu vou ajudá-la em nome de Jesus Cristo.


-- Pelos deuses -- ela disse -- siga seu caminho, não fique aqui comigo,
pois você pode não conseguir me salvar.

Enquanto eles falavam o dragão apareceu e veio correndo até eles. São
Jorge montou em seu cavalo, desembainhou a espada, fez o sinal da
cruz, cavalgou bravamente em direção ao Dragão e o acertou com sua
lança e o feriu gravemente, jogando-o por terra. Então ele disse para
a jovem:

-- Pegue seu cinto e amarre-o em volta do pescoço do dragão. Não tenha
medo.Quando ela fez isso o Dragão se pôs a segui-la como se fosse um
animal domesticado e inofensivo. Ela levou-o até a cidade, e o povo
fugiu para as montanhas e vales, dizendo que seriam todos mortos.
Então São Jorge disse: 

-- Nada temam, creiam em Jesus Cristo, e não tardem a ser batizados,
que eu matarei o dragão. 

Então o rei e todo seu povo foi batizado, e São Jorge matou o dragão
cortando-lhe a cabeça, e ordenou que ele fosse jogado no campo, e foi
preciso três carros de boi para tirá-lo da cidade. 

Haviam sido batizados quinze mil homens, sem contar mulheres e
crianças, e o rei mandou erguer uma igreja para Nossa Senhora e São
Jorge, onde ainda hoje brota uma fonte de água da vida que cura os
doentes que dela bebem.

Depois disso o rei ofereceu a São Jorge tanto dinheiro quanto tinha,
mas ele recusou tudo e pediu que fosse dado aos pobres em nome de
Deus, e exigiu do rei quatro coisas, a saber, que ele se encarregasse
das igrejas, que ele honrasse os padres, que ouvisse diligentemente
seus sermões, e que ele tivesse piedade dos pobres; e depois de
beijar o rei ele partiu.

\chapter{A história de Sigurd\subtitulo{Andrew Lang}}

Havia uma vez um rei no Norte que ganhara muitas guerras, e que agora
já estava velho. Mas ele se casou com uma outra mulher, e então um
certo príncipe, que também queria casar com ela, veio atacá-lo com um
grande exército. O velho rei lutou bravamente, mas sua espada acabou
se quebrando, ele se feriu e todos os seus homens fugiram. De noite,
depois que a batalha tinha terminado, sua jovem esposa veio
procurá-lo entre os mortos e feridos. Enfim o encontrou, e perguntou
se ele podia ser curado. Mas ele respondeu que não, sua sorte
acabara, sua espada quebrara, e ele ia morrer. E disse ainda que ela
ia ter um filho, que esse filho seria um grande guerreiro e iria
vingar-se do outro rei seu inimigo. E pediu a ela que guardasse os
pedaços quebrados da espada, para que com eles uma nova espada fosse
feita para o seu filho; e essa nova arma deveria ser chamada de Gram.


Então ele morreu. E sua mulher chamou a criada e disse:

-- Vamos trocar de roupa; e você será chamada pelo meu nome, e eu pelo
seu, caso os inimigos nos encontrem.

Foi o que fizeram, e se esconderam na floresta, mas então uns
forasteiros as encontraram e as levaram num navio para a Dinamarca. E
quando elas foram levadas até o rei, este achou que a criada parecia
uma rainha, e a rainha uma criada. Então ele perguntou à rainha:

-- Como você sabe no meio da noite que falta pouco para amanhecer? 

E ela disse:

-- Eu sempre sei porque, quando era mais jovem, costumava levantar para
acender os fogos, e ainda acordo na mesma hora.

“Estranho, uma rainha que acende os fogos”, o rei pensou.

Então ele perguntou à rainha que estava vestida de criada:

-- Como você sabe no meio da noite que a aurora se aproxima? 

-- Meu pai me deu um anel de ouro -- ela disse -- e sempre antes do
amanhecer ele esfria em meu dedo.

-- Rica essa casa em que as criadas usam ouro -- disse o rei. -- Na
verdade você não é nenhuma criada, mas a filha de um rei. 

Então ele a tratou como convinha a uma rainha, e com o passar do tempo
ela teve um filho que chamou de Sigurd, um menino belo e muito forte.
Ele tinha um tutor a acompanhá-lo, e um dia o tutor disse a ele para
ir pedir um cavalo para o rei. 

-- Escolha você mesmo o seu cavalo -- disse o rei; e Sigurd foi até a
floresta, onde encontrou um velho com uma barba branca, e disse a
ele:

-- Venha! Ajude-me a escolher um cavalo.

O velho disse:

-- Leve todos os cavalos para o rio, e escolha aquele que o atravessar
nadando. 

Então Sigurd levou-os até o rio, e só um deles o atravessou. Sigurd o
escolheu: seu nome era Grani, ele vinha da linhagem de Sleipnir e era
o melhor cavalo do mundo. Pois Sleipnir era o cavalo de Odin, o Deus
do Norte, e era tão rápido quanto o vento. 

Um dia ou dois depois disso, o tutor disse a Sigurd:

-- Há um grande tesouro em ouro escondido não muito longe daqui, e
seria apropriado para você se o encontrasse. 

Mas Sigurd respondeu:

-- Já ouvi falar desse tesouro, e sei que o dragão Fafnir o guarda, e
ele é tão enorme e terrível que nenhum homem ousa se aproximar dele.

-- Ele não é maior que outros dragões -- disse o tutor -- e se você fosse
tão corajoso quanto seu pai não teria medo dele. 

-- Não sou um covarde -- disse Sigurd. -- Por que você quer que eu lute
com esse dragão? 

Então seu tutor, que se chamava Regin, contou a ele que todo aquele
tesouro de ouro vermelho pertencera antes a seu pai. E seu pai teve
três filhos: o primeiro foi Fafnir, o dragão; o segundo foi Lontra,
que podia tomar a forma de uma lontra sempre que quisesse; e o último
foi ele, Regin, que era um grande ferreiro e fabricante de espadas. 

Havia então um anão chamado Andvari, que morava junto a uma lagoa sob
uma cachoeira, e lá ele escondera um tesouro. E um dia Lontra estava
pescando ali, e tinha pego um salmão e o comido, e estava dormindo
numa pedra na forma de lontra. Alguém que passava por ali jogou uma
pedra na lontra e a matou; e tirou a pele dela, e a levou para a casa
do pai de Lontra. Então ele soube que seu filho estava morto, e para
punir a pessoa que o matara exigiu que a pele da lontra fosse enchida
com ouro, e coberta com ouro, ou senão teria que se ver com ele. A
pessoa que matara Lontra foi até a cachoeira e capturou o anão que
tinha o tesouro e o tirou dele. 

Só sobrara um anel, que o anão estava usando; mas até este foi tirado
dele. 

O pobre anão ficou muito bravo, e rogou uma praga: aquele ouro só iria
trazer má sorte para quem o possuísse, para sempre. 

Então a pele da lontra foi enchida com ouro e coberta com ouro
inteira, exceto por um pelo, e nele foi enfiado o último anel do
pobre anão. 

Mas não trouxe sorte para ninguém. Primeiro Fafnir, o dragão, matou
seu próprio pai; e então ele foi até o ouro e sobre ele se estendeu,
não deixando seu irmão ficar com nem um pouco; e nenhum homem ousava
se aproximar dele. 

Ao ouvir a história Sigurd disse a Regin:

-- Faça-me uma boa espada para eu matar esse dragão. 

Regin fez uma espada para ele, mas ele a testou com um golpe num bloco
de ferro, e ela se quebrou. 

Outra espada foi feita, e Sigurd também a quebrou. 

Então Sigurd foi ter com sua mãe, e pediu os pedaços da espada de seu
pai, e os deu para Regin. E ele os forjou e martelou numa espada
nova, tão afiada que as bordas de sua lâmina pareciam incandescentes.

Sigurd experimentou essa espada num bloco de ferro e ela não quebrou,
mas partiu ao meio o ferro. Então ele jogou um floco de lã no rio, e
quando ele flutuou até a lâmina, cortou-se em dois. Sigurd disse que
aquela espada servia. Mas antes de ir lutar com o dragão ele liderou
um exército para lutar com os homens que haviam matado seu pai. Ele
executou o Rei deles, ficou com toda sua fortuna, e voltou para casa.


Depois de alguns dias, ele foi a cavalo com Regin até a charneca onde
o dragão costumava ficar. Então viu o rastro que o dragão deixava
quando ia num rochedo beber; e o rastro era como se um grande rio
tivesse passado e deixado um vale profundo. 

Então Sigurd foi até esse valo profundo, e cavou muitos buracos nele,
e num deles se escondeu com sua espada na mão. Lá ficou esperando, e
logo a terra começou a tremer com o peso do dragão rastejando até a
água. E uma nuvem de veneno era lançada à sua frente quando ele
bufava e urrava, de modo que seria morte certa ficar diante dele. 

Mas Sigurd esperou até a metade do corpo do dragão ter rastejado sobre
o buraco, e então enfiou a espada Gram bem no coração dele. 

O dragão vergastou sua cauda em volta, despedaçando rochas e
derrubando árvores. E quando viu que ia morrer, disse:

-- Quem quer que seja você que me matou, esse ouro será sua ruína, e a
ruína de todos aqueles que o possuírem. 

Sigurd disse:

-- Eu não o tocaria se isso me fizesse nunca morrer. Mas todos os
homens morrem, e nenhum homem corajoso deixa a morte amedrontá-lo de
ter aquilo que deseja. Morra, Fafnir! 

E Fafnir morreu. 

E Sigurd passou a ser chamado de o Algoz de Fafnir, e Matador de
Dragões. 

Então Sigurd voltou, e encontrou com Regin, que disse a ele para assar
o coração de Fafnir e deixá-lo provar. 

Sigurd pôs o coração de Fafnir num espeto, e o pôs para assar. Mas
aconteceu de ele tocá-lo por acidente, queimando o dedo. Ele pôs o
dedo na boca, e assim provou o coração de Fafnir. 

E imediatamente ele compreendeu a linguagem dos pássaros, e ouviu o
que diziam os pica-paus:

-- Eis Sigurd, assando o coração de Fafnir para outro, quando devia ser
ele mesmo a prová-lo e aprender toda a sabedoria. 

O pássaro ao lado disse:

-- Eis Regin, pronto para trair Sigurd, que confia nele. 

O terceiro pássaro disse:

-- Que ele corte a cabeça de Regin, e fique com todo o ouro para si
mesmo.

O quarto pássaro disse:

-- Que ele faça isso, e então vá para Hindfell, o lugar onde Brynhild
dorme. 

Ao ouvir essas palavras, e como Regin tramava traí-lo, Sigurd cortou a
cabeça dele com um só golpe da espada Gram.

Então todos os pássaros cantaram, e era uma canção sobre uma bela
jovem que dormia num lugar cercado por fogo, à espera dele para
acordá-la. 

E Sigurd lembrou-se que havia uma história sobre uma bela dama
enfeitiçada num lugar muito distante dali. Ela estava sob um encanto,
que a fazia dormir num castelo cercado por chamas; lá ela dormiria
até que chegasse um cavaleiro capaz de atravessar o fogo para
acordá-la. Ele decidiu ir até lá, mas antes seguiu o horrível rastro
de Fafnir. E encontrou sua caverna, que era bem profunda e tinha
portas de ferro, e estava cheia de braceletes, coroas e anéis de
ouro; e lá Sigurd achou também o Elmo do Pavor, que era de ouro e
deixava invisível quem o usava. Ele carregou tudo isso no bom cavalo
Grani, e então partiu para Hindfell, ao sul. 

Já era de noite, e no cimo de uma colina Sigurd viu um fogo brilhando
vermelho no céu, e dentro das chamas um castelo, com um estandarte na
torre mais alta. Então ele arremeteu no fogo com seu cavalo Grani,
que o saltou com leveza, como se não passasse de um arbusto. Sigurd
passou pelo portão do castelo e viu alguém dormindo, usando uma
armadura. Ele tirou o elmo da pessoa adormecida, e para sua surpresa
viu que era uma dama das mais belas. Ela acordou e disse:

-- Ah, é Sigurd, o filho de Sigmund, que quebrou a maldição, e veio
enfim me acordar?

A maldição tinha sido posta nela quando o espinho da árvore do sono
picou sua mão, tempos atrás, como uma punição por ter desagradado ao
deus Odin. E tempos atrás também ela tinha jurado nunca casar com um
homem que tivesse medo, e não ousasse atravessar a cerca de chamas.
Pois ela era uma guerreira, e ia para as batalhas armada como um
homem. Mas agora ela e Sigurd se apaixonaram, e prometeram ser fiéis
uma ao outro. Ele deu a ela um anel, que era o último anel do anão
Andvari. Então Sigurd partiu, e encontrou o castelo de um rei que
tinha uma bela filha. O nome dela era Gudrun, e sua mãe era uma
feiticeira. Gudrun se apaixonou por Sigurd, mas ele só falava em
Brynhild, em como ela era bela e o quanto ele a amava. Um dia então a
mãe feiticeira de Gudrun pôs sementes de papoula e outras drogas de
esquecimento numa taça mágica, e fez Sigurd beber à saúde dela. No
mesmo instante ele esqueceu a pobre Brynhild e se apaixonou por
Gudrun, e eles se casaram com grandes festejos.

A feiticeira, mãe de Gudrun, quis então que seu filho Gunnar casasse
com Brynhild, e disse a ele para ir com Sigurd fazer a corte a ela.
Eles partiram para o castelo do pai dela, pois Brynhild tinha saído
completamente da cabeça de Sigurd por causa da poção da feiticeira,
mas ela ainda lembrava dele e o amava. O pai de Brynhild disse a
Gunnar que ela só casaria com alguém capaz de atravessar o fogo em
frente a sua torre encantada, e para lá eles foram. Gunnar tentou
passar pelas chamas, mas seu cavalo não quis enfrentá-las. Então
Gunnar tentou usar o cavalo de Sigurd, mas montado por ele, Grani não
se movia. Daí Gunnar lembrou-se dos feitiços que sua mãe havia lhe
ensinado, e com eles fez Sigurd ficar exatamente igual a ele, e ele
exatamente igual a Sigurd. Então Sigurd, sob a forma de Gunnar,
montou em Grani, que saltou a cerca de fogo, e o rapaz foi até
Brynhild, mas ainda nada lembrava dela, por causa da poção de
esquecimento da feiticeira.

E Brynhild não teve alternativa a não ser prometer casar com ele, ser
a esposa de Gunnar, pois Sigurd estava sob a forma de Gunnar, e ela
prometera casar com quem quer que atravessasse o fogo. E ele deu a
ela um anel, e ela devolveu a ele o anel que ele tinha lhe dado antes
sob sua própria forma de Sigurd, aquele que era o último anel do
pobre anão Andvari. Ele voltou, e trocou de aparência com Gunnar, e
cada um sendo ele mesmo de novo, voltaram ao castelo da rainha
feiticeira, e Sigurd deu o anel do anão a sua esposa Gudrun. Brynhild
foi ter com seu pai e disse que um príncipe chamado Gunnar havia
atravessado o fogo, e ela teria de se casar com ele. 

-- E no entanto eu achava -- ela disse -- que nenhum homem conseguiria
tal proeza a não ser Sigurd, o Algoz de Fafnir, que era o meu
verdadeiro amor. Mas ele se esqueceu de mim, e tenho de manter minha
promessa. 

Então Gunnar e Brynhild se casaram, embora não tivesse sido Gunnar que
atravessara o fogo, mas Sigurd disfarçado. 

Quando acabou o casamento e todos os festejos, a mágica da poção da
feiticeira deixou a mente de Sigurd, e ele se lembrou de tudo.
Lembrou que libertara Brynhild do encanto, e que era ela seu
verdadeiro amor, como a esquecera e casara com outra mulher, e
conquistara Brynhild para ser a mulher de outro homem. 

Mas ele era corajoso, e nada disse aos outros, para não fazê-los
infelizes. Não conseguia escapar, entretanto, da maldição sobre todo
aquele que possuísse o tesouro do anão Andavari, e seu anel de ouro
fatal. 

E logo a maldição voltou a se abater sobre todos eles. Pois um dia,
quando Brynhild e Gudrun estavam se banhando num rio, Brynhild foi
mais fundo nas águas, e disse que fizera isso para provar sua
superioridade sobre Gudrun, porque o marido dela, disse, tinha
atravessado o fogo que nenhum outro homem ousara enfrentar. Gudrun
ficou muito brava, e disse que tinha sido Sigurd, e não Gunnar, que
atravessara o fogo, e recebera de volta de Brynhild aquele anel
fatal, o anel do anão Andvari. 

Então Brynhild viu o anel que Sigurd dera a Gudrun e soube de tudo,
ficou tão pálida como morta, e foi para casa. Durante aquela noite
ficou em silêncio. Na manhã seguinte, ela disse a Gunnar, seu marido,
que ele era um covarde e um mentiroso, pois jamais atravessara o
fogo, mas mandara Sigurd no lugar dele, e fingira que tinha sido ele.
E disse que ele jamais voltaria a vê-la feliz em sua casa, bebendo
vinho, jogando xadrez, bordando com fios de ouro ou trocando palavras
carinhosas. Então ela rasgou todos seus bordados e chorou bem alto,
para que todos na casa a ouvissem. Pois seu coração se partira, e no
mesmo instante também seu orgulho. Perdera seu verdadeiro amor,
Sigurd, o Algoz de Fafnir, e casara com um homem mentiroso. 

Sigurd então veio tentar consolá-la, mas ela não quis ouvi-lo, e disse
que queria uma espada atravessando o coração dela. 

-- Não terá que esperar muito -- ele disse -- até a espada atravessar o
meu coração, pois você não viverá muito depois que eu morrer. Mas,
Brynhild querida, console-se e siga vivendo, e ame Gunnar, seu
marido, e eu lhe darei todo o ouro, o tesouro do dragão Fafnir. 

Brynhild disse:

-- Tarde demais.

Sigurd se viu tomado de tamanha tristeza que seu coração inchou em seu
peito e os anéis de sua malha de ferro se romperam.

Ele então se foi, e Brynhild decidiu matá-lo. Misturou veneno de
serpente com carne de lobo, e ofereceu-os num prato para o irmão mais
novo de seu marido; quando ele comeu ficou enlouquecido, foi até o
quarto de Sigurd enquanto ele dormia e o prendeu a cama
atravessando-o com uma espada. Mas Sigurd acordou, agarrou a espada
Gram e atirou-a no homem que fugia, partindo-o ao meio. E assim
morreu Sigurd, o Algoz de Fafnir, que nem dez homens seriam capazes
de matar num combate justo. Então Gudrun acordou, viu-o morto e ficou
aos prantos; Brynhild a ouviu e riu, mas o bom cavalo Grani se deitou
e morreu de tanto pesar. Depois Brynhild se pôs a chorar amarga e
desesperadamente, até que seu coração se partiu de vez. Então os
outros puseram em Sigurd sua armadura dourada, acenderam uma fogueira
a bordo de seu navio, e de noite nele deitaram os mortos Sigurd e
Brynhild, e o bom cavalo Grani, atearam fogo e o lançaram na água. E
o vento levou-o ardendo para o mar, as chamas brilhando na escuridão.
Sygurd e Brynhild foram cremados juntos, e a maldição do anão Andvari
se cumpriu. 

\chapter{Os salvadores da pátria\subtitulo{Edith Nesbit}}

Tudo começou quando caiu um cisco no olho de Effie. Doía muito mesmo,
dava a sensação de ter uma fagulha incandescente no olho -- só que
parecia ter pernas e asas também, feito uma mosca. Effie esfregou os
olhos e chorou -- não um choro de verdade, mas do tipo a que o olho se
entrega por conta própria, sem você precisar se sentir horrível por
dentro -- e então ela foi atrás de seu pai para que ele tirasse o
cisco de seu olho. O pai de Effie era médico, de modo que ele sabia
como tirar ciscos dos olhos -- e ele o fez muito habilmente, usando um
pincel macio embebido em óleo de rícino.

Quando tirou o cisco, ele disse:

-- Isso é muito curioso.

Effie frequentemente tivera ciscos no olho antes, e seu pai sempre
pareceu achar normal -- um tanto aborrecido e descuidado, talvez, mas
ainda assim normal. Ele nunca tinha achado curioso.

Effie ficou lá com seu lenço no olho, dizendo:

-- Não acredito que enfim saiu. -- As pessoas sempre dizem isso quando
tiram um cisco dos olhos.

-- Ah sim, saiu -- disse o doutor. -- Está aqui, no pincel. E é muito
interessante.

Effie nunca ouvira seu pai falar isso sobre qualquer coisa em que ela
tivesse parte. Ela disse:

-- O quê?

O doutor levou o pincel com muito cuidado para o outro lado da sala, e
pôs a ponta dele sob seu microscópio, e então ajustou os botões, e
espiou pela parte de cima do microscópio com um olho só.

-- Minha nossa -- ele disse. -- Minha nossa! Quatro membros bem
desenvolvidos; um longo apêndice caudal; cinco dedos, de comprimento
desigual; quase igual a um dos Lacertidae, e no entanto há traços de
asas. 

A criatura retorceu-se um pouco no óleo de rícino, e ele continuou:

-- Sim, uma asa similar à dos morcegos. Um novo espécime, sem a menor
dúvida. Effie, corra até o professor e peça para ele fazer a
gentileza de vir aqui por alguns minutos.

-- Você podia me dar seis pence, papai -- disse Effie -- porque fui eu
que trouxe para você o novo espécime. Eu tomei bastante cuidado com
ele dentro de meu olho; e meu olho está mesmo doendo.

O doutor estava tão contente com o novo espécime que ele deu a Effie
um xelim; e logo o professor apareceu. Ele ficou para o almoço, e ele
e o doutor discutiram todos contentes a tarde inteira sobre o nome e
a família da coisa que saíra do olho de Effie.

Mas na hora do jantar outra coisa aconteceu. Harry, o irmão de Effie,
pescou alguma coisa dentro de seu chá, que ele a princípio achou que
era uma pequena centopeia. Ele estava a ponto de deixá-la cair no
chão, e acabar com a vida dela da maneira usual, quando ela se
sacudiu na colher -- abriu duas asas molhadas, e se deixou cair na
toalha da mesa. E lá ficou, se esfregando com suas patas e esticando
as asas, e Harry disse:

-- Ora, é uma salamandra minúscula!

O professor debruçou-se antes que o doutor pudesse dizer uma palavra.

-- Dou-lhe meia coroa por ele, Harry, meu rapaz -- disse bem rápido; e
então catou-o cuidadosamente em seu lenço.

-- É um novo espécime -- disse. -- E superior ao seu, doutor.

Era um lagarto minúsculo, de um centímetro e meio de comprimento, com
escamas e asas.

De modo que agora tanto o doutor quanto o professor tinham o seu
espécime, e estavam ambos muito satisfeitos. Mas não demorou muito
para esses espécimes ficarem bem menos valiosos, porque na manhã
seguinte, quando o engraxate lustrava as botas do doutor, ele, de
repente, largou a escova e a graxa, gritando que tinha sido queimado.

E de dentro da bota saiu rastejando um lagarto do tamanho de um gato,
com asas grandes e brilhantes.

-- Ora -- disse Effie -- eu sei o que é isso. É um dragão, igual ao que
São Jorge matou.

E Effie estava certa. Naquela tarde Towser foi mordido no jardim por
um dragão do tamanho de um coelho, que ele tentara caçar, e na manhã
seguinte todos os jornais só falavam dos “lagartos com asas” que
estavam aparecendo por todo o país. Os jornais não os chamavam de
dragões porque, claro, ninguém acredita em dragões hoje em dia -- e
obviamente os jornais não seriam tolos a ponto de acreditar em contos
de fadas. A príncipio, havia apenas uns poucos, mas em uma ou duas
semanas o país estava simplesmente infestado de dragões de todos os
tamanhos, e era possível às vezes ver muitos deles no ar, feito um
enxame de abelhas. Eram todos iguais, exceto no tamanho. Eram verdes
com escamas, tinham quatro patas, uma cauda comprida, e grandes asas
parecidas com as dos morcegos, exceto pelo fato de que eram de um
amarelo claro, semitransparente, como as caixas de câmbio das
bicicletas.

Exalavam fogo e fumaça, como convém a todo dragão legítimo, mas ainda
assim os jornais continuaram fingindo que eram lagartos, até que o
editor do Standard foi pego por um deles e levado embora, e então o
resto do pessoal do jornal ficou sem quem lhes dissesse no que deviam
ou não acreditar. De modo que, quando o maior elefante do zoológico
foi carregado embora por um dragão, os jornais desistiram de fingir,
e saíram com a manchete “Alarmante Praga de Dragões” na primeira
página.

Você nem imagina o quanto era alarmante, e ao mesmo tempo o quanto era
irritante. Os dragões de tamanho grande eram certamente terríveis,
mas depois que se descobriu que eles iam para a cama cedo porque
tinham medo do ar frio da noite, bastava passar o dia inteiro dentro
de casa, para ficar a salvo deles. Mas os de tamanhos menores eram um
perfeito incômodo. Os do tamanho de centopeias ficavam caindo na sopa
ou na manteiga. Os do tamanho de cachorros pulavam nas banheiras, e o
fogo e a fumaça dentro deles fazia com que a água fria da torneira
virasse vapor instantaneamente, de modo que pessoas descuidadas
podiam se escaldar seriamente. Os do tamanho de pombos entravam nas
cestas de costura e nas gavetas e mordiam quem estava com pressa de
pegar uma agulha ou um lenço. Os do tamanho de uma ovelha eram mais
fáceis de se evitar, porque era possível vê-los chegando; mas quando
eles voavam pelas janelas e se aninhavam debaixo das cobertas, e não
se notava antes de entrar na cama, era sempre um choque. Os desse
tamanho não comiam gente, só alface, mas eles sempre chamuscavam os
lençóis e as fronhas horrivelmente. 

Claro, o Conselho do Condado e polícia fizeram tudo o que havia para
fazer; mas oferecer a mão da princesa a quem matasse o dragão não
adiantava. Essa solução era muito boa nos velhos tempos, quando havia
apenas um dragão e uma princesa, mas agora havia bem mais dragões do
que princesas, mesmo a Família Real sendo bem grande. Além disso,
seria um desperdício de princesas oferecê-las como recompensa a quem
matasse dragões, porque todo mundo matava tantos dragões quanto
podia, inteiramente por conta própria e sem pensar em recompensas,
apenas para tirar da frente aqueles bichos tão desagradáveis. O
Conselho do Condado encarregara-se de cremar todos os dragões
entregues entre as dez e as catorze horas, e carrinhos, carroças e
caminhões cheios de dragões mortos podiam ser vistos todos os dias da
semana fazendo uma longa fila na rua em que ficava o prédio do
Conselho. Meninos traziam carrinhos de mão cheios de dragões, e
crianças na volta da escola, no fim da manhã, paravam para deixar um
ou dois punhados de dragões que traziam em suas malas, ou no bolso,
embrulhados em seus lenços. E no entanto parecia continuar a haver
tantos dragões quanto antes. Então a polícia ergueu torres de pano e
madeira cobertas de cola. Quando os dragões ao voar batiam nessas
torres, ficavam grudados como moscas e vespas no papel mata-moscas da
cozinha; quando as torres ficavam todas cobertas de dragões, o
inspetor da polícia punha fogo nelas, queimando os dragões e o resto
junto.

E no entanto parecia haver ainda mais dragões que antes. As lojas
estavam cheias de veneno para dragão, e sabão antidragão, e cortinas
a prova de dragão para as janelas; e de fato, tudo o que era possível
foi feito.

E no entanto parecia haver ainda mais dragões que antes.

Não era muito fácil descobrir o que envenava um dragão, porque eles
comiam as coisas mais variadas. Os maiores comiam elefantes, enquanto
havia elefantes, e depois passaram a comer cavalos e vacas. Um outro
tamanho não comia nada a não ser lírios do vale, e um terceiro
tamanho comia apenas primeiros ministros se os havia disponíveis e,
em não havendo, se alimentavam generosamente de meninos que
trabalhavam de uniforme como criados. Outro tamanho vivia de tijolos,
e três deles comeram dois terços da enfermaria de South Lambeth numa
tarde.

Mas aqueles de que Effie tinha mais medo eram os tão grandes quanto a
sala de jantar; os desse tamanho comiam menininhas e menininhos.

A princípio Effie e seu irmão ficaram muito satisfeitos com as
mudanças na vida deles. Era tão divertido ficar acordado a noite
inteira em vez de ir dormir, e brincar no jardim iluminado com luz
elétrica. E soava tão engraçado ouvir a mãe dizer, quando iam para a
cama:

-- Boa noite, meus queridos, durmam bem o dia todo, e não levantem
muito cedo. Vocês não podem sair da cama até ficar bem escuro. Não
vão querer que os horríveis dragões os peguem.

Mas depois de um tempo cansaram de tudo aquilo. Queriam ver as flores
e as árvores crescendo no campo e o sol brilhando do lado de fora, em
vez de através do vidro e da cortina a prova de dragão das janelas. E
queriam brincar na grama, o que não era permitido no jardim iluminado
com luz elétrica por causa da umidade do orvalho.

E eles queriam tanto sair lá fora, uma vez que fosse, na bela,
brilhante e perigosa luz do dia, que começaram a tentar achar alguma
razão para terem de sair. Só que não gostavam de desobedecer a mãe
deles.

Mas uma manhã a mãe deles estava ocupada preparando algum novo veneno
para dragão para pôr na adega, e o pai estava fazendo um curativo na
mão do engraxate que fora arranhado por um dos dragões que gostava de
comer primeiros ministros quando disponíveis, de modo que ninguém
lembrou de dizer às crianças “não levantem até ficar escuro”.

-- Vamos agora -- disse Harry. -- Não vai ser desobediência. E eu sei
exatamente o que temos que fazer, só não sei como vamos fazer.

-- O que temos que fazer? -- disse Effie.

-- Temos que acordar São Jorge, claro -- disse Harry. -- Ele era a única
pessoa da cidade que sabia como lidar com dragões; o pessoal dos
contos de fadas não conta. Mas São Jorge é uma pessoa de verdade, ele
só está dormindo, à espera de ser acordado. Só que ninguém acredita
mais em São Jorge. Ouvi papai dizer isso.

-- Nós acreditamos -- disse Effie.

-- Claro que sim. E você não percebe, Ef, qual a razão para eles não
conseguirem acordá-lo? Não dá para acordar alguém em quem não se
acredita, dá?

Effie disse que não, mas onde eles iam encontrar São Jorge?

-- Precisamos procurá-lo -- disse Harry decididamente. -- Você vai usar
um vestido a prova de dragão, feito do mesmo pano que as cortinas. E
eu vou passar no corpo o melhor veneno para dragão, e…

Effie apertou as mãos e pulou de alegria dizendo:

-- Oh, Harry! Eu sei onde podemos achar São Jorge! Na igreja de São
Jorge, claro.

-- Hum -- disse Harry, querendo que tivesse sido ele a pensar nisso. --
Você às vezes até que é inteligente para uma menina.

Então na tarde seguinte, bem cedo, muito antes que os raios do
pôr-do-sol anunciassem a noite chegando, quando todo mundo iria
acordar para ir trabalhar, as duas crianças saíram da cama. Effie
enrolou em volta dela um xale de musselina a prova de dragões -- não
havia tempo para fazer um vestido -- e Harry fez dele uma meleca só
com veneno para dragão. Era seguramente inofensivo para crianças e
inválidos, de modo que ele não precisava se preocupar.

Eles se deram as mãos e saíram para ir até a igreja de São Jorge. Como
você sabe, há muitas igrejas de São Jorge, mas por sorte eles viraram
a esquina que levava à certa, e lá se foram sob o sol brilhante,
sentindo-se muito corajosos e aventureiros.

Não havia ninguém nas ruas a não ser dragões, e a cidade estava
simplesmente infestada deles. Por sorte nenhum era do tamanho certo
para comer menininhos e menininhas, se não talvez esta história
terminasse aqui. Havia dragões na calçada, e dragões na rua, e
dragões tomando sol nas escadarias dos prédios públicos, e dragões
alisando as asas nos telhados. A cidade estava toda verde deles.
Mesmo quando as crianças saíram da cidade e seguiram pela estrada,
elas perceberam que os campos dos dois lados estavam mais verdes que
o normal, com todas aquelas escamas e caudas; e alguns dos menores
haviam feito ninhos de asbestos nas cercas-vivas de espiriteiro
florido.

Effie segurava a mão de seu irmão apertando muito, e quando um dragão
gordo bateu as asas perto de sua orelha ela deu um berro, fazendo com
que uma revoada de dragões verdes levantasse vôo do campo, se
espalhando pelo céu. As crianças podiam ouvir o ruído das asas deles
no ar.

-- Oh, quero ir para casa -- disse Effie.

-- Não seja boba -- disse Harry. -- Com certeza você não esqueceu dos
Sete Campeões e suas princesas. As pessoas que vão ser os salvadores
da pátria nunca berram e dizem que querem ir para casa. 

-- E nós… somos? -- Effie perguntou. -- Salvadores, quero dizer.

-- Você vai ver -- disse o irmão dela, e eles seguiram em frente.

Quando chegaram à igreja de São Jorge encontraram a porta aberta e
entraram; mas São Jorge não estava lá dentro, então eles saíram para
o cemitério do lado de fora da igreja, e logo acharam a grande tumba
de pedra de São Jorge, com a figura dele esculpida em mármore do lado
de fora, com sua armadura e capacete, e as mãos cruzadas sobre o
peito.

-- Como vamos acordá-lo? -- disseram. Então Harry falou com São Jorge,
mas ele não respondia; e aí ele tentou acordar o grande matador de
dragões chacoalhando seus ombros de mármore. Mas São Jorge nem notou.

Então Effie começou a chorar, e pôs os braços em volta do pescoço de
São Jorge o melhor que pôde no mármore, que ficava muito no caminho
nas costas; ela beijou o rosto de mármore e disse:

-- Oh, caro, bom, gentil São Jorge, por favor acorde e nos ajude.

E com isso São Jorge abriu os olhos sonolentamente, se espreguiçou e
disse:

-- Qual é o problema, menininha?

Então as crianças contaram tudo o que havia para contar; ele se virou
em seu mármore e apoiou-se num cotovelo para escutar. Mas quando
soube que havia tantos dragões balançou a cabeça.

-- Assim não dá -- ele disse. -- São dragões demais para o velho Jorge.
Vocês deviam ter me acordado antes. Sempre fui a favor de lutas
justas: um homem, um dragão, era meu lema.

Bem naquele momento uma revoada de dragões passou por cima deles, e
São Jorge começou a desembainhar a espada. Mas ele balançou a cabeça
de novo e empurrou a espada de volta para o lugar dela, enquanto os
dragões iam ficando pequenos ao se a distanciarem.

-- Não posso fazer nada -- ele disse. -- As coisas mudaram desde a minha
época. Santo André me contou. Ele foi acordado na greve dos
maquinistas, e veio conversar comigo. Disse que hoje em dia tudo se
faz com máquinas; deve haver algum maneira de dar um jeito nesses
dragões. Falando nisso, como tem estado o tempo ultimamente?

A pergunta soou tão descabida e indelicada que Harry se recusou a
responder, mas Effie disse pacientemente:

-- Muito bom. Papai disse que é o verão mais quente que já houve nesse
país.

-- Ah, foi o que imaginei -- disse o campeão, pensativo. -- Bom, a única
coisa que podia ajudar… Dragões não suportam o frio e a umidade, essa
é a única coisa. Se ao menos vocês conseguissem achar as torneiras…

São Jorge estava começando a se ajeitar de novo em sua lápide de
pedra.

-- Boa noite, sinto muito não poder ajudá-los -- disse, bocejando por
trás de sua mão de mármore.

-- Ah, mas você pode -- exclamou Effie. -- Diga: que torneiras?

-- Ah, igual no banheiro -- disse São Jorge, ainda mais sonolento. -- E
tem um espelho, também: mostra o mundo todo e o que acontece nele.
São Dionísio que me contou; disse que era uma coisa muito bonita.
Sinto muito não… Boa noite.

Ele voltou a seu mármore e num instante dormia profundamente.

-- Nós nunca vamos achar as torneiras -- disse Harry. -- Escuta, não ia
ser terrível se São Jorge acordasse justo quando houvesse um dragão
por perto, um do tamanho dos que comem campeões?

Effie tirou seu xale a prova de dragão.

-- Não encontramos nenhum dos do tamanho da sala de jantar -- ela disse.
-- Acho que estamos seguros.

Então ela cobriu São Jorge com o pano, e Harry esfregou o máximo que
conseguiu de veneno para dragão na armadura de São Jorge, de modo a
deixá-lo bem seguro.

-- Podemos nos esconder na igreja até escurecer -- ele disse -- e então…

Mas naquele momento uma sombra escura se abateu sobre eles, e eles
viram que era um dragão exatamente do tamanho da sala de jantar de
casa. E viram que tudo estava perdido. O dragão arremeteu para o solo
e pegou os dois com suas garras, Effie pelo cinto verde de seda dela,
e Harry pela pontinha da parte de trás de sua jaqueta de Eton -- e
então, abrindo suas enormes asas amarelas, alçou vôo, fazendo um
barulhão igual a um vagão de terceira classe com o breque puxado.

-- Oh, Harry -- disse Effie -- me pergunto quando ele vai nos comer!

O dragão estava voando por cima de florestas e campos com o lento
bater de suas enormes asas, atravessando um quarto de milha a cada
batida delas.

Harry e Effie podiam ver o campo lá embaixo, cercas e rios e igrejas e
casas de fazendas ficando para trás embaixo deles, muito mais rápido
do que passavam no mais rápido dos trens expressos.

E o dragão continuava a voar. As crianças viram outros dragões no ar
por onde passavam, mas o dragão que era do tamanho da sala de jantar
nunca parou para falar com nenhum deles, apenas continuou voando
adiante num ritmo constante.

-- Ele sabe aonde está indo -- disse Harry. -- Ah, se a menos ele nos
largasse antes de chegar lá!

Mas o dragão segurava-os firme, e ele voou e voou e voou, até que
finalmente, quando as crianças estavam já bem tontas, aterrisou no
topo de uma montanha, com todas suas escamas retinindo. Ele então
deitou seu corpo escamoso e verde, ofegando, completamente sem fôlego
de tanto que voara. Mas suas garras estavam firmes no cinto de Effie
e na jaqueta de Harry.

Então Effie catou o canivete que Harry lhe dera de presente de
aniversário. Custara só seis pence, e ela já o tinha há um mês e até
agora tudo o que conseguira fazer com ele foi apontar lápis de
ardósia, mas de algum jeito ela fez com que aquele canivete cortasse
o cinto dela, e ela se livrou dele, deixando o dragão apenas com uma
tira de seda verde em suas garras. Mas aquele canivete jamais
cortaria a jaqueta de Harry. Depois de tentar um bocado Effie viu que
não tinha jeito e desistiu. Mas com a ajuda dela Harry conseguiu
esgueirar-se para fora das mangas, de modo que o dragão ficou apenas
com uma jaqueta de Eton na sua outra garra. Então as crianças foram
na ponta dos pés até uma rachadura nas pedras e entraram nela. Era
estreita demais para o dragão entrar também, e lá elas ficaram,
esperando para fazer caretas para o dragão quando ele tivesse
descansado o bastante para se sentar e começar a pensar em comer
eles. Ele ficou muito bravo mesmo quando eles fizeram caretas, e
exalou fogo e fumaça, mas eles correram mais para dentro da caverna
para não serem alcançados, e o dragão acabou cansando e indo embora.

Mas os dois estavam com medo de sair da caverna, então eles
continuaram andando para dentro, e a caverna foi se alargando e
ficando maior, e o chão dela era de areia macia, e quando eles
chegaram ao fim dela havia uma porta, na qual estava escrito: Sala
Universal das Torneiras. Privativo. Entrada Proibida a Todos.

Eles abriram a porta na mesma hora para espiar dentro, lembrando do
que São Jorge dissera.

-- A nossa situação não pode ficar pior do que já está -- disse Harry --
com um dragão esperando do lado de fora. Vamos entrar.

Entraram então decididamente na sala das torneiras, e fecharam a porta
atrás de si.

E agora estavam numa espécie de uma sala cavada na rocha sólida, e
havia torneiras ao longo de toda uma das paredes, e todas as
torneiras tinham rótulos feitos de plaquinhas de porcelana como os
das estâncias de águas mineirais. E como ambos eram capazes de ler
palavras de duas sílabas e às vezes até mesmo de três, eles
entenderam no mesmo instante que tinham achado o lugar onde se liga o
tempo. Havia três torneiras grandes com os rótulos: “sol”, “vento”,
“chuva”, “neve”, “granizo”, “geada”; e várias menores, dizendo: “leve
a moderada”, “chuvoso”, “brisa do sul”, “bom para as plantas
crescerem”, “patinação”, “vento sul”, “vento leste”, e por aí afora.
E a torneira grande com o rótulo “sol” estava inteira aberta. Não
dava para ver nenhuma luz do sol -- a caverna era iluminada por uma
claraboia de vidro azul -- de modo que eles acharam que a luz do sol
devia estar saindo por algum outro lugar, como acontece com a
torneira que lava as partes de baixo de certas pias de cozinha.

Então eles viram que do outro lado da sala havia apenas um grande
espelho, e olhando por ele podia se ver tudo que estava acontecendo
no mundo -- e tudo de uma vez só, também, bem diferente da maioria dos
outros espelhos. Eles viram as carroças entregando dragões mortos no
prédio do Conselho do Condado, e viram São Jorge dormindo sob o
tecido à prova de dragão. E eles viram a mãe deles em casa chorando
porque seus filhos tinham saído na terrível e perigosa luz do dia, e
estava com medo de um dragão tê-las comido. E eles viram a Inglaterra
inteira, feito um grande mapa de quebra-cabeça, verde nas partes do
campo, marrom nas cidades, e preto nos lugares onde se faz carvão e
cerâmica e facas e substâncias químicas. Cobrindo tudo, as partes
pretas, verdes e marrons, havia uma rede de dragões verdes. E eles
puderam ver que ainda era dia, e os dragões ainda não tinham ido para
a cama. Effie disse:

-- Dragões não gostam do frio.

E ela tentou fechar o sol, mas a torneira estava com defeito, e era
essa a razão de estar fazendo tanto calor, e dos dragões terem sido
chocados a ponto de quebrarem seus ovos. Então eles deixaram a
torneira do sol em paz, mas abriram a de neve e a deixaram inteira
aberta enquanto iam olhar no espelho. Lá eles viram os dragões
correndo para todos os lados como as formigas fazem se você é cruel o
suficiente para derramar água num formigueiro, o que você nunca é,
claro. E cada vez caía mais neve.

Então Effie abriu inteira a torneira da chuva, e logo os dragões
estavam se mexendo menos, e aos poucos alguns deles foram ficando
completamente imóveis, e as crianças sabiam que a água tinha apagado
o fogo dentro deles e eles estavam mortos. Eles abriram então a de
granizo -- só até a metade, com medo de quebrar as janelas das pessoas
-- e depois de um tempo não havia mais dragões se movendo para ver.

As crianças souberam então que eram de fato os salvadores da pátria.

-- Vão erguer um monumento para a gente -- disse Harry -- tão alto quanto
o do Almirante Nelson! Todos os dragões morreram.

-- Espero que o que estava esperando pela gente do lado de fora esteja
morto! -- disse Effie. -- E quanto ao monumento, Harry, não sei não. O
que eles vão fazer com esse monte de dragões mortos? Iria levar anos
e anos para queimar todos, e nem dá para queimá-los agora que eles
estão todos ensopados. Gostaria que a chuva levasse todos eles para o
mar.

Mas isso não aconteceu, e as crianças começaram a achar que não tinham
sido tão terrivelmente espertas assim, no final das contas.

-- Para que será que essa coisa velha serve? -- disse Harry. Tinha
achado uma torneira velha e enferrujada, que parecia não ser usada há
um monte de tempo. O rótulo de porcelana dela estava coberto de
poeira e teias de aranha. Quando Effie a limpou com a ponta de sua
saia (pois curiosamente ambas as crianças haviam saído sem seus
lenços) ela descobriu que estava escrito “ralo”.

-- Vamos abri-la -- ela disse. -- Quem sabe leva os dragões.

A torneira estava emperrada por não ter sido usada durante tanto
tempo, mas juntos eles conseguiram abri-la, e correram para o espelho
para ver o que acontecera.

Um enorme e redondo buraco negro já havia se aberto bem no meio do
mapa da Inglaterra, e os lados do mapa se inclinaram, de modo que a
água da chuva corria toda para o buraco.

-- Viva, viva, viva! -- gritou Effie, e correu de volta para as
torneiras para abrir tudo o que parecia molhado: “chuvoso”, “bom para
as plantas crescerem”, e até “vento sul” e “vento sudeste”, pois
ouvira seu pai dizer que esses ventos traziam chuva.

E então a chuva caía torrencial em todo o país, e grandes vagas de
água corriam para o centro do mapa, e cataratas caíam no grande
buraco redondo no meio do mapa, e os dragões estavam sendo levados
para desaparecerem cano abaixo pelo ralo, em compactas massas verdes
e espalhados grumos verdes, dragões sozinhos e dragões às dezenas,
dos que carregavam elefantes aos que caíam no chá.

Logo não havia nenhum dragão sobrando. Então eles fecharam a torneira
“ralo”, e fecharam até a metade a que estava marcada “sol” -- estava
quebrada, de modo que não puderam fechá-la de vez. Abriram “leve a
moderada” e “chuvoso”, e as duas torneiras emperraram, e não dava
mais para fechá-las, o que explica o clima do país.


\bigskip

Como eles voltaram para casa? Pela estrada de ferro de Snowdon, claro.

E a pátria ficou agradecida? Bom, a pátria estava muito molhada. E
quando enfim a pátria ficou seca de novo, estava mais interessda numa
nova invenção que usava eletricidade para assar muffins, e todos os
dragões já tinham sido quase esquecidos. Dragões não parecem assim
tão importantes depois de terem morrido e sumido todos e, você sabe,
nunca foi oferecida uma recompensa.

E o que disseram o pai e a mãe quando Effie e Harry chegaram em casa?
É o tipo da pergunta boba que vocês crianças sempre fazem. No
entanto, só dessa vez não vou me importar de responder. A mãe disse:

-- Oh meus queridos, meus queridos, vocês estão salvos! Suas crianças
travessas, como puderam ser tão desobedientes? Já para a cama!

E o pai deles, o doutor, disse:

-- Se eu soubesse o que vocês iam fazer! Queria de ter preservado um
espécime. Joguei fora o que tirei do olho da Effie. Pretendia pegar
um espécime em melhores condições. Não previ essa tão imediata
extinção dos dragões.

O professor nada disse, mas esfregou as mãos. Tinha guardado o seu
espécime -- o do tamanho de uma pequena centopeia, pelo qual dera meia
coroa a Harry -- e o tem até hoje. Você precisa conseguir que ele o
mostre para você!

\chapter{O dragão relutante\subtitulo{Kenneth Grahame }}

Tempos atrás -- pode ter sido centenas de anos atrás --, num chalé a
meio caminho entre uma aldeia e as encostas dos Downs, no sul da
Inglaterra, vivia um pastor com sua mulher e seu filho. O pastor
passava os dias -- e em certas épocas do ano as noites também -- lá em
cima nas vastas encostas, tendo apenas o sol, as estrelas e as
ovelhas de companhia, bem longe da cordial tagarelice do mundo dos
homens e das mulheres. Mas seu filho, quando não estava ajudando o
pai, e às vezes quando estava também, passava muito de seu tempo
imerso em grossos volumes que ele emprestava da gente culta afável e
dos vigários letrados da região. E seus pais gostavam muito dele, e
também tinham bastante orgulho dele, embora não o dissessem na frente
dele, de modo que deixavam-no viver sua vida e ler o quanto quisesse;
e em vez de frequentemente levar uns sopapos na cabeça, como bem
poderia ter acontecido, ele era tratado mais ou menos como um igual
por seus pais, os quais achavam sensatamente que era uma divisão de
trabalho muito justa aquela: eles forneciam o conhecimento prático, e
ele o que se encontrava nos livros. Sabiam que o conhecimento dos
livros não raro podia se revelar muito útil numa emergência, apesar
do que os seus vizinhos diziam. O que sobretudo interessava o menino
era história natural e contos de fadas, e ele ia lendo os dois
assuntos conforme apareciam, de um jeito meio ensanduichado, sem
fazer quaisquer distinções; e realmente esse seu modo de ler parece
bastante sensato.

Um dia, ao entardecer, o pastor, que fazia algumas noites andava
incomodado e preocupado, fora de seu usual equilíbrio mental, voltou
para casa tremendo todo e, sentando à mesa onde sua mulher e filho
estavam tranquilamente ocupados, ela com sua costura, ele com as
aventuras do Gigante sem coração em seu corpo, exclamou muito
agitado:

-- Está tudo acabado para mim, Maria! Nunca mais de jeito nenhum eu vou
poder subir lá nas encostas, nem uma vez mais!

-- Não fique assim -- disse sua mulher, que era muito sensata. --
Primeiro conte-nos tudo o que aconteceu, o que foi que lhe causou
toda essa agitação, e então comigo e com o filho aqui, talvez a gente
consiga esclarecer o assunto.

-- Começou algumas noites atrás -- disse o pastor. -- Sabe aquela caverna
que tem lá em cima? Eu nunca gostei dela, por algum motivo, e as
ovelhas também não, e quando as ovelhas não gostam de alguma coisa em
geral há uma razão. Bom, já faz algum tempo que dela têm vindo uns
ruídos fracos, ruídos como suspiros profundos, com uns grunhidos no
meio, e às vezes um ronco, bem lá do fundo, ronco de verdade, mas de
algum jeito não um ronco honesto, como o meu e o seu de noite, você
sabe como é.

-- Eu sei -- observou o menino, discretamente.

-- Claro, fiquei com muito medo -- o pastor continuou -- e no entanto não
consegui ficar longe. Então hoje, no fim da tarde, antes de vir
embora, fui de mansinho dar uma olhada ali pela caverna. E lá, meu
Deus!, lá estava ele, enfim o vi, tão bem quanto te vejo agora!

-- Viu quem? -- disse sua mulher, começando a se contagiar com o
aterrorizado nervosismo do marido.

-- Ora, ele, estou lhe dizendo! -- disse o pastor. -- Ele estava parado,
metade fora da caverna, e parecia estar apreciando o friozinho do fim
da tarde de um jeito meio poético. Era tão grande quanto quatro
cavalos de puxar carroça, e todo coberto de escamas brilhantes;
escamas azul-escuras na parte de cima dele, passando para um verde
suave embaixo. Quando ele respirava, sobre suas narinas havia esse
tipo de ar tremido que se vê sobre as estradas num dia de sol quente
e sem vento no verão. Ele estava com o queixo apoiado nas patas, e eu
diria que estava meditando sobre as coisas. Oh, sim, era um tipo
bastante pacífico de animal, sem ameaçar atacar ou aprontar ou fazer
qualquer coisa que não fosse certa ou direita. Admito tudo isso.
Ainda assim, o que posso fazer? Escamas, sabe, e garras, e com
certeza uma cauda, embora essa parte dele eu não tenha visto; não
estou acostumado com essas coisas e não me dou bem com elas e isso é
um fato.

O menino, que aparentemente ficara concentrado em seu livro durante a
récita de seu pai, fechou o volume, bocejou, cruzou as mãos atrás da
cabeça e disse sonolento:

-- Está tudo certo, pai. Não precisa se preocupar. É só um dragão.

-- Só um dragão? -- seu pai gritou. -- O que você quer dizer, sentado aí,
você e seus dragões? Só um dragão, essa é boa! E o que você sabe
disso?

-- Porque é, e porque eu sei -- respondeu o menino, calmamente. --
Escute, pai, você sabe que cada um de nós tem a sua
especialidade.Você sabe sobre ovelhas, e o tempo, e coisas assim; eu
sei sobre dragões. Eu sempre disse, você sabe, que aquela caverna lá
em cima era uma caverna de dragão. Eu sempre disse que em alguma
época ela devia ter sido de um dragão, e deveria ser de um dragão
agora, se as regras valem alguma coisa. Bom, agora você me diz que
ela tem um dragão, e isso que é o certo. Eu fiquei bem menos surpreso
agora do que quando você me disse que não tinha um dragão nela. As
regras sempre dão certo se você espera com calma. Agora, por favor,
deixe que eu cuide disso. Vou dar um passeio por lá amanhã de manhã,
não, de manhã não dá, tenho uma pilha de coisas para fazer, bom,
talvez no fim da tarde, se eu tiver tempo, vou até lá ter uma
conversa com ele, e você verá que vai dar tudo certo. Só o que peço,
por favor, é que você não fique se preocupando por ali sem mim. Você
não entende nada deles, e eles são muito sensíveis, sabe?

-- Ele está muito certo, pai -- disse a mãe sensata. -- Como ele disse,
dragões são a especialidade dele e não a nossa. Ele é ótimo nisso de
animais de livros, como todo mundo admite. E para falar a verdade, eu
mesma não fico nem um pouco contente, pensando no pobre daquele
animal sozinho lá em cima, sem um jantar quentinho ou alguém com quem
trocar ideias; e quem sabe poderemos fazer alguma coisa por ele; e se
ele não for respeitável, nosso menino vai logo descobrir. Ele tem um
certo jeito simpático que faz todo mundo contar tudo para ele.

No dia seguinte, depois do jantar, o menino subiu a estrada de
cascalho que leva até o topo dos Downs; e lá, claro, encontrou o
dragão, deitado preguiçosamente na relva na frente da caverna dele. A
vista dali é das mais magníficas. Para a direita e esquerda, léguas e
léguas das ondulantes encostas sem árvores dos Downs; à frente, o
vale, com as aglomerações das fazendas, e a trama de estradinhas
brancas passando por pomares e terras bem aradas e, bem longe no
horizonte, sinais das velhas cidades cinzentas. Uma brisa fresca
brincava com a superfície da relva e um pedaço prateado de uma enorme
lua estava aparecendo sobre distantes zimbros. Não era nenhuma
surpresa que o dragão parecesse estar num estado de espírito
tranquilo e satisfeito; e de fato, ao chegar perto o menino ouviu-o
ronronando com uma feliz regularidade. “Bom, vivendo e aprendendo!”
ele disse para si mesmo. “Nenhum dos meus livros jamais disse que
dragões ronronavam!” 

-- Olá, dragão -- disse o menino tranquilamente, quando chegou até onde
ele estava.

O dragão, ao ouvir alguém se aproximando, começara a fazer um esforço
para se levantar, por cortesia. Mas quando viu que era um menino,
franziu as sobrancelhas com severidade.

-- Você não venha me bater -- ele disse -- ou jogar pedras, ou espirrar
água, ou qualquer coisa assim. Não vou tolerar, já vou logo avisando.

-- Não vou jogar nada em você -- disse o menino, aborrecido, largando-se
sentado na grama perto do bicho. -- E, pelo amor de Deus, não fique me
dizendo “não isso” e “não aquilo”; eu já ouço tanto disso, e é
monótono, enjoa. Eu só passei por aqui para perguntar como vai você e
esse tipo de coisa; mas se estou atrapalhando, eu posso muito bem ir
embora. Tenho um monte de amigos, e nenhum deles pode dizer que eu
tenho o hábito de ficar insistindo quando não querem a minha
presença!

-- Não, não, não vá embora bravo -- o dragão apressou-se a dizer. -- O
fato é que eu estou muito bem mesmo aqui em cima; nunca sem alguma
ocupação, meu camarada, nunca sem alguma ocupação! Ainda assim, cá
entre nós, é um pouquinho chato às vezes.

O menino mordeu um talo de grama e ficou chupando-o. 

-- Pretende ficar muito tempo por aqui? -- perguntou, polidamente. 

-- No momento, não saberia dizer -- respondeu o dragão. -- Parece um
lugar bom o bastante; mas faz pouco tempo que estou aqui, é preciso
ver mais e refletir e considerar bem antes de se decidir por ficar
num lugar. É uma coisa muito séria, escolher onde morar. Além disso
(e agora eu vou lhe contar uma coisa que você jamais teria
adivinhado, mesmo se tivesse tentado) o fato é que eu sou um
vagabundo danado de preguiçoso!

-- Fico muito surpreso -- disse o menino, educadamente.

-- É a triste verdade -- o dragão continuou, se acomodando entre suas
patas e evidentemente encantado de ter enfim achado um ouvinte. -- E
imagino que seja isso que me fez vir parar aqui, na verdade. Veja
você, todos os outros camaradas são tão ativos e sérios e todo esse
tipo de coisa, sempre agressivos, arranjando briga, varrendo as
areias do deserto, rondando as margens do mar, e perseguindo
cavaleiros por toda a parte, devorando donzelas, e aprontando em
geral… já eu sempre gostei de fazer minhas refeições na hora certa, e
então deitar as costas em alguma rocha e tirar uma sonequinha, e daí
acordar e pensar como as coisas estão e como elas sempre estão
iguais, sabe? De modo que quando aconteceu eu fui pego de surpresa.

-- Quando o quê aconteceu, posso saber? -- perguntou o menino.

-- É precisamente isso que eu não sei -- disse o dragão. -- Suponho que a
terra espirrou, ou chacoalhou, ou o fundo caiu de algum lugar. De
qualquer modo, houve um tremor, um estrondo e uma total barafunda, e
eu fui parar milhas debaixo da terra, e fiquei completamente preso.
Bom, graças a Deus, não preciso de muito, e de qualquer maneira tinha
paz e tranquilidade, e não era sempre chamado para ir junto fazer
alguma coisa. E eu tenho uma mente tão ativa: sempre ocupada, eu lhe
asseguro! Mas o tempo foi passando, e minha vida começou a ficar numa
certa mesmice, e afinal achei que ia ser divertido abrir um caminho
para o andar de cima e ver o que os outros andavam fazendo. Então eu
cavei e escavei, e trabalhei assim e assado, e enfim consegui sair
através dessa caverna aqui. E eu gostei da região, e da vista, e das
pessoas (as poucas que eu vi) e no geral estou inclinado a ficar por
aqui.

-- Com o que sua mente está sempre ocupada? -- perguntou o menino. --
Gostaria de saber. 

O dragão ficou ligeiramente vermelho e desviou os olhos. Enfim disse,
envergonhado:

-- Você alguma vez, só de farra, já tentou fazer poesia… versos, sabe?

-- Claro que sim -- disse o menino. -- Pilhas e pilhas. E algumas delas
são bem boas, tenho certeza, só que não há ninguém por aqui que ligue
para essas coisas. Minha mãe é muito simpática e etcétera, quando eu
as leio para ela, e também meu pai, quanto a isso. Mas de algum jeito
eles não…

-- Exatamente -- exclamou o dragão -- exatamente o meu caso. Eles não… e
não há o que você possa fazer com isso. Agora, você eu vejo que tem
cultura, vi no mesmo instante, e eu gostaria de ouvir sua opinião
sincera sobre algumas coisinhas que eu fui pondo no papel, enquanto
estava por aqui. Fiquei incrivelmente satisfeito em conhecê-lo, e
espero que os outros vizinhos sejam igualmente agradáveis. Ontem à
noite mesmo esteve aqui um velho senhor muito simpático, mas ele
pareceu não querer incomodar.

-- Era o meu pai -- disse o menino. Ele é um velho senhor simpático, e
qualquer dia eu posso apresentá-lo a ele se você quiser.

-- Será que vocês dois não poderiam subir aqui amanhã para um jantar ou
algo assim? -- perguntou o dragão ansiosamente. E acrescentou
polidamente -- Claro, se vocês não tiverem nada melhor para fazer. 

-- Muito obrigado mesmo -- disse o menino -- mas não saímos para ir em
lugar nenhum sem minha mãe e, para falar a verdade, receio que ela
possa não aprová-lo muito. Veja você, não há como escapar da dura
realidade de que você é um dragão, há? E quando você fala de ficar
morando aqui, e os vizinhos, e por aí afora, não consigo evitar a
impressão de que você não percebe exatamente a sua situação. Você é
um inimigo da raça humana, afinal!

-- Não tenho um inimigo no mundo -- disse o dragão, alegremente. -- Sou
muito preguiçoso para fazer algum, para começo de conversa. E se eu
insisto em ler minhas poesias para os outros, estou sempre pronto
para ouvir as deles!

-- Ora essa! -- exclamou o menino. -- Gostaria que você tentasse
realmente entender a sua situação. Quando as outras pessoas ficarem
sabendo sobre você, vão vir todas atrás de você com lanças e espadas
e todo tipo de coisa. Você terá que ser exterminado, de acordo com a
maneira deles de ver o mundo! Você é um flagelo, uma praga, um
monstro assassino!

-- Não há uma palavra de verdade nisso tudo -- disse o dragão,
balançando a cabeça solenemente. -- Meu caráter provar-se-á íntegro
mesmo sob a mais estrita das investigações. E agora, eis um pequeno
sonetinho no qual eu estava trabalhando quando você apareceu…

-- Ih, se você se recusa a ser sensato -- exclamou o menino,
levantando-se -- eu vou embora para casa. Não, não posso dar só uma
olhadinha no soneto; minha mãe está me esperando. Vou vir vê-lo
amanhã, alguma hora, e você pelo amor de Deus veja se tenta entender
que é um flagelo pestilencial, ou vai acabar entrando na pior fria.
Boa noite!

Foi fácil para o menino deixar seus pais tranquilos em relação a seu
novo amigo. Eles sempre tinham confiado ao menino esse ramo, e
aceitaram o que ele disse sem um murmúrio. O pastor foi formalmente
apresentado, e muitos elogios e informações foram gentilmente
trocados. Sua mulher, no entanto, mesmo se dizendo disposta a fazer o
que estivesse a seu alcance -- remendar coisas, pôr ordem na caverna
ou cozinhar alguma coisinha quando o dragão ficasse absorto em seus
sonetos e esquecesse das refeições, como os machos sempre fazem -- não
pôde ser convencida a aceitá-lo completamente. O fato de que ele era
um dragão e que “eles não sabiam quem ele era” parecia contar mais
que tudo para ela. No entanto, ela não fez objeção a que seu filhinho
passasse tranquilamente todo começo de noite com o dragão, desde que
voltasse para casa até as nove; e muitas noites agradáveis eles
tiveram, sentados na relva, enquanto o dragão contava histórias dos
velhos, bem velhos tempos, quando havia dragões de monte e o mundo
era um lugar mais animado que agora, e a vida cheia de frêmitos,
sobressaltos e surpresas.

Mas o que o menino temia, todavia, não demorou a acontecer. Mesmo o
mais reservado e discreto dos dragões do mundo, se é tão grande
quanto quatro cavalos e coberto de escamas azuis, acaba não tendo
como escapar do conhecimento público. De modo que nas noites na
taverna da aldeia, o fato de que um dragão de verdade ficava cismando
numa caverna nos Downs era naturalmente o assunto das conversas.
Embora os aldeões estivessem com muito medo, estavam também bastante
orgulhosos. Era uma marca de distinção ter um dragão próprio,
acrescentava um charme à aldeia. Ainda assim, todos concordavam que
era o tipo da coisa que não se podia permitir que continuasse. O
terrível monstro devia ser exterminado, o campo devia ficar livre
daquela praga, daquele terror, daquele flagelo destruidor. O fato de
que nem mesmo um único poleiro de galinhas havia sido prejudicado
desde que o dragão chegara não era levado em conta, nem se permitia
que tivesse algo a ver com o assunto. Ele era um dragão, não se podia
negar, e se preferia não se comportar como um, isso era lá problema
dele. Mas apesar de muita valente falação, nenhum herói aparecia
disposto a pegar espada e lança para libertar a aldeia de seu
padecimento e adquirir fama imortal; e a inflamada discussão de todas
as noites acabava em nada. Enquanto isso, o dragão, um boêmio feliz,
refestelava-se na relva, apreciava os pôres-do-sol, contava anedotas
antediluvianas para o menino, e polia seus velhos versos ou cogitava
novos.

Um dia o menino, ao entrar na aldeia, percebeu que estava tudo com uma
aparência de festa, embora nenhuma constasse do calendário. Tapetes e
panos de cores alegres estavam pendurados nas janelas, os sinos da
igreja repicavam ruidosamente, a rua estreita estava cheia de flores,
e a população inteira estava se empurrando dos dois lados dela,
tagarelando, empurrando, e mandando uns aos outros ficarem mais para
trás. O menino viu um amigo da mesma idade dele e o chamou. 

-- O que está acontecendo? -- gritou. -- São atores, ou ursos, ou o
circo, ou o quê?

-- Está tudo bem -- seu amigo gritou de volta. -- Ele está vindo.

-- Quem está vindo? -- quis saber o menino, sendo empurrado na multidão.

-- Ora, quem; São Jorge, claro -- respondeu seu amigo. -- Ele ouviu falar
do dragão, e está vindo com o propósito de matar o mortífero monstro,
e nos libertar de seu horrível jugo. Nossa vai ser uma luta daquelas!

Aquela era uma novidade e tanto! O menino achou que devia se
certificar ele mesmo, e se insinuou pelo meio das pernas dos mais
velhos, xingando-os o tempo todo pelo seu mal-educado hábito de ficar
empurrando. Quando conseguiu chegar na fila da frente, ficou
esperando ansiosamente a chegada.

Logo veio da ponta mais distante da multidão o som das ovações. Em
seguida, o barulho compassado dos cascos do cavalo de batalha fez seu
coração bater mais rápido, e logo ele se descobriu gritando junto com
o resto quando, em meio aos brados de boas-vindas, os gritinhos
agudos das mulheres, o erguimento de bebês e a agitação de lenços,
São Jorge avançou lentamente pela rua. O coração do menino parou
quieto e ele respirava aos soluços, a beleza e a graça do herói iam
muito além de qualquer coisa que ele já tinha visto. Sua armadura era
incrustada em ouro, seu elmo pendia da sela, e seus densos cabelos
louros emolduravam um rosto de uma delicadeza inexprimível, até que
se via a firmeza dos olhos. Ele puxou as rédeas em frente à estalagem
e os aldeões se aglomeraram em volta com saudações, agradecimentos e
enumerações loquazes de injúrias, injustiças e opressões. O menino
ouviu a voz séria e gentil do Santo, assegurando a todos que tudo
ficaria bem agora, e que ele ia lutar por eles, reparar as injustiças
e livrá-los de seu inimigo; então ele desceu do cavalo e entrou pela
porta, e a multidão se amontoou atrás dele. Mas o menino saiu em
disparada para o morro o mais rápido que suas pernas conseguiam.

-- Está tudo acabado, dragão! -- ele gritou assim que viu o bicho. -- Ele
está vindo! Ele já está aqui! Você vai ter que criar vergonha e enfim
fazer alguma coisa!

O dragão estava lambendo suas escamas e esfregando-as com um pedaço de
flanela que a mãe do menino lhe emprestara, até ficar brilhante feito
uma grande turquesa.

-- Não seja violento, menino -- ele disse sem se virar. -- Sente e
recupere o fôlego, e tente lembrar que o sujeito antecede o
predicado, e então talvez você podessa me fazer a gentileza de dizer
quem está vindo.

-- Está certo, fique frio -- disse o menino. -- Só quero ver você
conseguir ficar na metade dessa tranquilidade depois que eu contar a
novidade. É só o São Jorge que está vindo, não passa disso; ele
chegou na aldeia faz uma meia hora. Claro que você pode dar uma surra
nele, um sujeito grandão como você! Mas achei que eu devia
preveni-lo, porque com certeza ele vai vir para cá logo cedo, e ele
tem a lança mais comprida e mais mal-encarada que você já viu!

E o menino ficou de pé e se pôs a pular para lá e para cá de puro
prazer com a perespectiva da batalha.

-- Essa não -- gemeu o dragão. -- Isso é muito desagradável. Eu não quero
vê-lo, e isso não se discute. Não tenho o menor interesse em conhecer
o sujeito. Tenho certeza que ele não é boa gente. Você precisa ir lá
dizer a ele para ir embora imediatamente, por favor. Diga a ele que
pode escrever se quiser, mas eu não vou recebê-lo. Não estou para
ninguém, no momento. 

-- Ah, dragão, dragão -- disse o menino, implorando -- não seja insensato
e teimoso. Você tem que lutar com ele alguma hora, você sabe, porque
ele é o São Jorge e você é o dragão. Melhor se livrar disso logo, e
então você pode continuar com seus sonetos. E você precisava levar um
pouco em consideração os outros também. Se por aqui anda meio chato
para você, imagine só o quanto tem sido chato para mim!

-- Meu caro homenzinho -- disse o dragão solenemente --, faça o favor de
entender, de uma vez por todas, que eu não posso lutar e eu não vou
lutar. Eu jamais lutei em toda a minha vida, e não vou começar agora,
só para você ter um espetáculo de feira. Nos velhos tempos eu sempre
deixava os outros, os que eram sérios, se encarregarem das lutas
todas, e sem dúvida essa é a razão de eu ter o prazer de estar aqui
agora.

-- Mas se você não lutar ele vai cortar fora sua cabeça! -- o menino
disse quase num soluço, desconsolado de perder tanto a luta quanto
seu amigo.

-- Ah, acho que não -- disse o dragão, com seu jeito indolente. -- Você
vai ser capaz de dar um jeito. Tenho inteira confiança em você, você
é tão bom para cuidar das coisas. Seja um bom sujeito, desça até lá e
resolva tudo. Deixo inteiramente por sua conta.

O menino percorreu todo o caminho de volta à aldeia num estado de
grande desânimo. Antes de mais nada, não haveria combate algum;
depois, seu bom e honrado amigo, o dragão, não se mostrou nem um
pouco heróico como ele teria gostado; e por fim, fosse o dragão no
fundo um herói ou não, não fazia a menor diferença, pois sem a menor
dúvida São Jorge iria cortar fora a cabeça dele. “Dar um jeito,
sei!”, ele disse amargamente para si mesmo. “O dragão trata a coisa
toda como se tivesse sido convidado para um chá e uma partida de
croqué.”

Os aldeões estavam indo para suas casas quando ele passou pela rua,
todos muito animados, e alegremente discutindo a esplêndida luta que
estava para acontecer. O menino seguiu até a estalagem e foi até o
salão principal, onde São Jorge estava sentado sozinho, considerando
suas chances na luta, e as tristes histórias de saques e violências
tão recentemente despejadas em seus ouvidos solidários.

-- Posso entrar, São Jorge? -- o menino perguntou polidamente, da porta
onde parara. -- Eu gostaria de conversar com o senhor sobre esse
assunto do dragão, se o senhor não estiver cansado dele a essa
altura.

-- Sim, pode entrar, menino -- disse o Santo gentilmente. -- Outra
história de maldade e infortúnio, receio. Sua boa mãe ou seu honrado
pai, foi algo assim, de que o tirano o privou? Ou de alguma terna
irmãzinha, ou irmão caçula? Bom, logo você será vingado.

-- Nada assim -- disse o menino. -- Há um mal-entendido em algum lugar, e
eu queria esclarecê-lo. O fato é que se trata de um bom dragão.

-- Exatamente -- disse São Jorge, sorrindo satisfeito. -- Entendo
perfeitamente. Um bom dragão. Creia-me, eu nem por um momento lamento
que ele seja um adversário à altura da minha espada, em vez de um
espécime fraco da sua tribo nociva.

-- Mas ele não é uma tribo nociva -- disse o menino todo aflito. --
Caramba, como ficam estúpidas as pessoas quando metem uma ideia na
cabeça! O que eu estou dizendo é que ele é um bom dragão, e meu
amigo, e conta as histórias mais bonitas que você já ouviu, todas
sobre os velhos tempos, quando ele era pequeno. E ele é muito gentil
com a minha mãe, e ela faria qualquer coisa por ele. E meu pai gosta
dele também, embora não tenha lá muito interesse em arte e poesia, e
sempre durma quando o dragão começa a falar sobre estilo. Mas o fato
é que ninguém consegue deixar de gostar dele assim que o conhece. Ele
é tão cativante e confiável, tão simples quanto uma criança!

-- Sente-se, e aproxime sua cadeira -- disse São Jorge. -- Gosto de
alguém que defende seus amigos, e tenho certeza que o dragão tem seus
pontos positivos, se tem um amigo feito você. Mas essa não é a
questão. Durante toda esta noite eu fiquei ouvindo, com inexprimível
angústia e pesar, histórias de assassinatos, roubos e demais
maldades; talvez um pouco exageradas demais, nem sempre inteiramente
convincentes, mas no geral constituindo uma lista de crimes dos mais
sérios. A história nos ensina que o pior dos malfeitores não raro se
mostra todo virtuoso em seu lar; e eu receio que seu culto amigo,
apesar das qualidades pelas quais ele mereceu (e devidamente) sua
afeição, tem que ser exterminado o quanto antes.

-- Ah, o senhor engoliu todas as lorotas que esse fulanos ficaram lhe
contando -- disse o menino, sem paciência. -- Ora, os aldeões daqui são
os maiores contadores de lorota de toda a região. É um fato sabido e
notório. O senhor é um forasteiro por aqui, se não já teria ouvido
falar. Tudo o que eles querem é uma luta. Eles são capazes de
qualquer coisa para conseguir lutas; é do que eles mais gostam.
Cachorros, touros, dragões; qualquer coisa, desde que seja uma luta.
Pois nesse momento mesmo há um pobre de um inocente texugo no
estábulo aqui atrás; eles iam se divertir com ele hoje, mas
resolveram guardá-lo para depois que o seu caso terminar. E não tenho
a menor dúvida que eles ficaram lhe dizendo que grande herói o senhor
é, e como tem tudo para vencer, em defesa do que é certo e justo, e
por aí afora; mas posso lhe assegurar, acabo de vir da rua, e eles
estavam apostando seis contra quatro que o dragão ganha!

-- Seis contra quatro no dragão -- São Jorge murmurou tristemente,
apoiando o rosto na mão. -- É um mundo perverso este, e às vezes eu me
ponho a achar que toda a maldade dele não fica inteiramente
engarrafada dentro dos dragões. Ainda assim, será que esse ardiloso
animal não o iludiu quanto ao verdadeiro caráter dele, para que a sua
boa impressão dele servisse de cobertura às malvadas façanhas dele?
Não, poderia até mesmo haver, neste preciso instante, alguma
desafortunada princesa aprisionada na lugúbre caverna dele?

Mas São Jorge se arrependeu do que disse assim que acabou de falar, ao
ver o quanto o menino ficara genuinamente incomodado.

-- Posso lhe assegurar, São Jorge -- ele disse --, que não há nada nem
sequer parecido na caverna dele. O dragão é um cavalheiro de verdade,
cada centímetro dele, e posso dizer que ninguém ficaria mais chocado
ou magoado do que ele, se ouvisse o senhor falando… falando desse
jeito de assuntos em relação aos quais ele tem os mais elevados
princípios!

-- Bom, talvez eu tenha sido exageradamente crédulo -- disse São Jorge.
-- Talvez eu tenha me enganado quanto ao animal. Mas o que podemos
fazer? Aqui estamos, o dragão e eu, quase frente a frente,
supostamente ávidos pelo sangue um do outro. Não consigo ver uma
saída. O que você sugere? Você não tem como dar um jeito, de algum
modo?

-- Foi exatamente o que o dragão disse -- retrucou o menino, bastante
exasperado. -- Sério, isso de vocês deixarem para eu resolver tudo…
Bom, suponho que não dá para convencê-lo a ir embora discretamente,
dá?

-- Impossível, receio -- disse o Santo. -- Completamente contra as
regras. Você sabe disso tanto quanto eu. 

-- Bom, então, escute aqui -- disse o menino. -- Ainda está cedo, será
que não daria para o senhor vir comigo encontrar com o dragão, e
tentar resolver com ele? Não é muito longe, e qualquer amigo meu será
muito bem recebido. 

-- Bem, é inteiramente contra as regras -- disse São Jorge, se
levantando -- mas realmente parece ser a coisa mais sensata a fazer.
Você está se dando a um trabalhão por conta de seu amigo -- ele
acrescentou afavelmente, ao passarem juntos pela porta. -- Mas
anime-se! Talvez afinal não precise haver luta alguma.

-- É, mas eu espero que haja sim -- disse o menino, nostalgicamente. 

-- Trouxe um amigo que quer conhecê-lo, dragão -- disse o menino, um
tanto alto.

O dragão acordou num sobressalto.

-- Eu estava só, hum, meditando sobre as coisas -- disse, do seu jeito
simples. -- Muito prazer em ser apresentado ao senhor. O tempo está
excelente, não?

-- Esse é São Jorge -- disse o menino, secamente. -- São Jorge, deixe-me
apresentá-lo ao dragão. Subimos até aqui para resolver discretamente
as coisas, dragão, e agora, pelo amor de Deus, veja se consegue ter
um pouco de simples bom senso, para ver se chegamos a alguma solução
prática e eficiente, porque já estou enjoado de opiniões e teorias
sobre a vida e tendências pessoais, e todo esse tipo de coisa. Talvez
eu deva acrescentar que minha mãe está me esperando.

-- Fico muito feliz em conhecê-lo, São Jorge -- começou o dragão um
tanto nervoso --, porque ouvi dizer que é um grande viajante, e eu
sempre fui mais do tipo caseiro. Mas posso lhe mostrar muitas
relíquias, muitos aspectos interessantes de nossa região, se fôr
ficar por algum tempo…

-- Eu acho -- disse São Jorge, de seu jeito franco e simpático -- que o
melhor a fazer é seguirmos o conselho de nosso jovem amigo, e tentar
chegar a algum entendimento, um acordo sério, sobre esse nosso
pequeno problema. Agora, você não acha que o plano mais simples,
afinal, seria apenas lutarmos, de acordo com as regras, e que vença o
melhor? Estão apostando em você, devo lhe dizer, lá na aldeia, mas eu
não ligo para isso.

-- Oh, sim, dragão, lute -- disse o menino. -- Iria nos poupar tantos
aborrecimentos!

-- Meu jovem amigo, você cale a boca -- disse o dragão, severamente. E
prosseguiu: -- Creia-me, São Jorge, não há ninguém no mundo que eu
gostaria de contentar mais do que este jovem cavalheiro aqui. Mas a
coisa toda é bobagem, e convencionalismo, e burrice popular. Não há
absolutamente nada pelo que lutar, do começo ao fim. E de qualquer
forma eu não vou, e isso encerra o assunto.

-- Mas supondo que eu o obrigue a lutar ---- disse São Jorge, um tanto
exasperado.

-- Não há como -- disse o dragão, triunfantemente. -- Eu simplesmente
entraria na minha caverna e voltaria por um tempo ao buraco donde
vim. E você logo enjoaria de ficar sentado do lado de fora esperando
eu sair para lutar. E assim que tivesse ido embora, eu sairia todo
contente, pois, para falar a verdade, gosto desse lugar e pretendo
ficar aqui!

São Jorge observou por um instante a bela paisagem em torno deles.

-- Mas esse seria um lugar maravilhoso para uma luta -- ele recomeçou,
persuasivo. -- Essas encostas do Downs de arena, eu em minha armadura
dourada fazendo contraste com suas escamas azuis! Pense só que bela
pintura daria!

-- Agora você está tentando me pegar pela minha sensibilidade artística
-- disse o dragão. -- Mas não vai funcionar. Não que não desse uma
pintura muito bela, como você disse -- acrescentou, cedendo um pouco.

-- Parece que estamos chegando mais perto de tratar de negócios -- o
menino opinou. -- Você precisa entender, dragão, que vai ser
necessário haver uma luta de algum tipo, porque você não vai querer
se enfiar de novo naquele velho buraco sujo e lá ficar até sabe-se lá
quando.

-- Podia ser arranjada -- disse São Jorge, pensativo. -- Eu tenho que
enfiar minha lança em algum lugar, claro, mas não precisa ser um
lugar que doa muito. Você é tão grande que com certeza deve haver
alguns pedaços de sobra em alguma parte. Aqui, por exemplo, bem atrás
da coxa. Não há de doer muito, só aí!

-- Assim você me faz cócegas, Jorge -- disse o dragão,
envergonhadamente. -- Não, esse lugar não vai dar de jeito nenhum.
Mesmo se não doesse, e tenho certeza que vai, e muito, iria me fazer
rir, e ia estragar tudo.

-- Vamos tentar outro lugar, então -- disse São Jorge, pacientemente. --
Debaixo do pescoço, por exemplo; todas essas dobras de pele grossa,
se eu enfiar minha lança aí você nem vai notar…

-- Sim, mas você tem certeza que consegue enfiá-la bem no lugar certo?
-- o dragão perguntou, ansiosamente.

-- Claro que tenho -- disse São Jorge, com confiança. -- Deixe comigo! 

-- É exatamente porque eu tenho que deixar com você que estou
perguntando -- disse o dragão, com uma certa irritação. -- Sem dúvida
você vai lamentar profundamente qualquer erro que fizer no calor da
hora; mas não vai lamentar nem a metade do que eu vou! No entanto,
suponho que é preciso confiar nos outros, ao longo da vida, e seu
plano parece, no geral, o melhor que há.

-- Escute aqui, dragão -- interrompeu o menino, um pouco ciumento por
seu amigo, que parecia estar ficando com a pior parte da barganha. --
Eu não estou entendendo direito como você entra nisso! Vai haver uma
luta, aparentemente, e você vai levar uma surra; o que eu quero saber
é: o que você vai ganhar com isso?

-- São Jorge -- disse o dragão --, conte a ele, por favor, o que irá
acontecer depois de eu ter sido vencido no mortal combate.

-- Bom, de acordo com as regras suponho que eu vou levá-lo em triunfo
até a praça do mercado ou o que servir como tal -- disse São Jorge.

-- Precisamente -- disse o dragão. -- E então?

-- E então haverá ovações e discursos e o resto -- continuou São Jorge.
-- E eu explicarei que você foi convertido, e viu o erro em sua
conduta, e assim por diante.

-- Certo -- disse o dragão. -- E aí?

-- Ah, e aí… -- disse São Jorge -- ora, suponho que vai haver o usual
banquete.

-- Exatamente -- disse o dragão. -- E é aí que eu entro. Olhe aqui --
continuou, dirigindo-se ao menino. -- Eu morro de tédio aqui em cima,
e ninguém realmente gosta de mim. Vou ser apresentado à sociedade,
graças à ajuda gentil de nosso amigo aqui, que está se dando a tanto
trabalho por mim; e você vai ver que tenho todas as qualidades para
que gostem de mim as pessoas que realmente contam! Bom, agora que
está tudo resolvido, e se vocês não se importarem, sou um sujeito de
modos antiquados, não gostaria de mandá-los embora, mas…

-- Lembre-se, você vai ter que fazer direito a sua parte na luta,
dragão! -- disse São Jorge, ao perceber a insinuação e se levantando
para partir. -- Quero dizer, dar investidas, exalar fogo, e por aí
afora!

-- Sei investir muito bem -- respondeu o dragão, todo confiante. -- Já
quanto a exalar fogo, é surpreendente como se perde fácil a prática;
mas farei o melhor que puder. Boa noite!

Já haviam descido o morro e estavam quase de volta na aldeia quando
São Jorge parou de repente.

-- Sabia que tinha esquecido de alguma coisa! -- disse. -- Precisa haver
uma princesa. Aterrorizada e acorrentada a uma rocha, essa coisa
toda. Menino, será que você não podia arranjar uma princesa?

O menino estava no meio de um bocejo tremendo.

-- Estou morto de cansaço -- ele reclamou -- e não posso arranjar uma
princesa, ou qualquer outra coisa, a essa hora da noite. E minha mãe
está me esperando, e faça o favor de parar de ficar me pedindo para
arranjar coisas até amanhã!


\bigskip

Na manhã seguinte as pessoas começaram a afluir para o morro bem cedo,
com suas roupas de domingo e carregando cestas com garrafas
aparecendo, todos querendo assegurar bons lugares para o combate. Não
era exatamente uma questão simples, porque era bem possível, claro,
que o dragão ganhasse, e nesse caso nem mesmo os que tinham apostado
seu dinheiro nele podiam esperar que ele tratasse os seus apoiadores
de maneira muito diferente do que o resto. Os lugares eram
escolhidos, portanto, com muito critério e com a garantia de uma fuga
rápida em caso de emergência; e a fila da frente era composta na
maioria de meninos que haviam escapado ao controle dos pais e agora
se espalhavam e rolavam na relva, ignorando as estridentes ameaças e
avisos a eles disparados por suas ansiosas mães lá atrás.

O menino tinha garantido um bom lugar na frente, bastante perto da
caverna, e estava tão ansioso quanto um diretor de palco na estreia.
Seria possível contar com o dragão? Ele podia mudar de ideia e
arruinar completamente a performance; ou então, percebendo que a
coisa fora planejada muito às pressas, sem nem mesmo um ensaio, podia
estar muito nervoso para aparecer. O menino olhou atentamente a
caverna, mas não havia nela sinal de vida ou ocupação. Teria o dragão
escapado no meio da noite?

As partes mais altas do morro estavam agora cobertas de espectadores,
e logo um som de aplausos e os acenos de lenços indicaram que havia
algo a vista que o menino, do tanto que estava para cima no lado do
dragão, não podia ainda ver. Um minuto mais e as plumas vermelhas de
São Jorge apareceram no topo do morro, quando ele foi chegando
lentamente à grande parte plana que ia até a soturna entrada da
caverna. Muito galante e belo ele estava, em seu alto cavalo de
batalha, sua armadura dourada refletindo o sol, sua enorme lança
ereta, a pequena flâmula branca, com uma cruz carmesim, tremulando na
ponta dela. Ele puxou as rédeas e ficou imóvel. As filas de
espectadores recuaram um pouco, nervosamente; e mesmo os meninos na
frente pararam de puxar o cabelo e se cutucar uns aos outros, e se
inclinaram para a frente, na expectativa.

-- Agora, dragão -- murmurou o menino impaciente, sem conseguir ficar
parado no lugar. Ele não precisava ter-se preocupado, se soubesse. As
possibilidades dramáticas da coisa haviam interessado imensamente o
dragão, e ele estava acordado desde a madrugada, preparando-se para
sua primeira aparição em público com inteira dedicação, como se os
anos tivessem voltado para trás e ele ainda fosse um pequeno
dragãozinho brincando de santo-e-dragão no chão da caverna de sua mãe
com suas irmãs; uma brincadeira em que o dragão sempre ganhava.

Um surdo múrmurio e um intermitente resfolegar fizeram-se ouvir então;
aumentando para se tornar um ensurdecedor rugido que pareceu encobrir
todo o platô. Em seguida uma nuvem de fumaça encobriu a entrada da
caverna, e do meio dela o dragão em pessoa, brilhante, azul-marinho,
magnífico, avançou solene e esplêndido; e todo mundo fez “oooh!” como
se tivessem visto um potente fogo de artifício! Suas escamas
resplandeciam, sua longa e espinhosa cauda serpenteava no chão, suas
garras arrancavam tufos de relva e os lançavam por cima de suas
costas, e fumaça e fogo eram incessantemente expelidos de suas iradas
narinas. 

-- Oh, muito bem, dragão -- o menino gritou, entusiasmado. E para si
mesmo acrescentou: “não achava que ele tivesse tanto jeito para a
coisa”.

São Jorge abaixou sua lança, inclinou a cabeça, cravou os calcanhares
no cavalo e investiu tronitruante pela relva. O dragão atacou com um
rugido e um guincho -- uma enorme e azul mistura serpenteante e
resfolegante de escamas, garras, espinhos e fogo.

-- Errou! -- gritou a multidão. Houve um instante de emaranhamento de
armadura dourada com escamas azul-turquesa e cauda espinhuda, e então
o grande cavalo, puxando por seu lado, carregou o Santo, sua lança
balançando no ar, quase até a entrada da caverna.

O dragão sentou e rosnou malevolamente, enquanto São Jorge com
dificuldade manobrava seu cavalo de volta à posição.

“Fim do primeiro round!”, pensou o menino. “Como eles se saíram bem!
Mas espero que o Santo não se entusiasme demais. No dragão dá para
confiar. Que excelente ator ele é!”

São Jorge enfim conseguiu fazer com que seu cavalo ficasse firme, e
enquanto enxugava a testa deu uma olhada em volta. Vendo o menino,
sorriu e fez um gesto afirmativo com a cabeça, e mostrou três dedos
por um instante.

“Parece que está tudo planejado”, disse o menino para si mesmo. “O
terceiro round vai ser o final, evidentemente. Gostaria que durasse
um pouco mais. E o que diabos aquele tonto do dragão está aprontando
agora?”

O dragão estava aproveitando o intervalo para para fazer uma exibição
de sua investida para a multidão. A investida dele, é necessário
explicar, consistia de correr num amplo círculo, mandando vagas e
ondas de movimento por toda a extensão de sua espinha, de suas
orelhas pontudas até o último espinho na ponta de sua comprida cauda.
Quando se é recoberto de escamas azuis, o efeito é particularmente
notável; e o menino lembrou o desejo recentemente expresso do dragão
de se tornar um sucesso na sociedade. 

São Jorge então juntou suas rédeas e começou a se mover para a frente,
abaixando a ponta de sua lança e se firmando na sela.

-- Segundo round! -- todo mundo gritou, entusiasmadamente; e o dragão
parou sua exibição, sentou-se e começou a pular de um lado para outro
com enormes saltos desajeitados, bradando feito um pele-vermelha.
Isso naturalmente desconcertou o cavalo, e ele deu uma brusca
guinada, o Santo só não caiu por ter se agarrado à crina. Quando eles
passaram a toda por ele, o dragão deu uma safada mordida na cauda do
cavalo que fez o pobre animal desembestar ensandecido encosta abaixo,
de modo que as palavras do Santo, com um dos pés fora do estribo, por
sorte ficaram inaudíveis para a plateia.

O segundo round produziu uma sonora prova de um sentimento favorável
ao dragão. Os espectadores não tardaram em apreciar um combatente que
mantinha tão bem sua posição e que claramente queria uma exibição
limpa; e muitos comentários encorajadores chegaram aos ouvidos de
nosso amigo enquanto ele se pavoneava para lá e para cá, o peito
empinado e a cauda no ar, apreciando enormemente sua nova
popularidade.

São Jorge tinha apeado e estava apertando as cilhas, e dizendo ao seu
cavalo, com um fluxo bastante oriental de metáforas, exatamente o que
pensava dele, sua família, e sua conduta na presente situação; então
o menino foi até o lado do Santo, e segurou a lança para ele.

-- Está sendo uma ótima luta, São Jorge! -- disse com um suspiro. -- Será
que não daria para fazê-la durar mais? 

-- Bem, acho melhor não -- respondeu o Santo. -- O fato é que seu velho
amigo simplório está ficando metido, agora que começaram a
aplaudi-lo, e ele bem pode esquecer o combinado e se meter a besta, e
aí não há como saber onde isso vai dar. Eu vou terminar com ele neste
round.

Ele subiu na sela e pegou a lança que o menino lhe deu.

-- Mas não fique com medo -- ele acrescentou gentilmente. -- Marquei o
lugar certo, e ele com certeza vai me ajudar o melhor que puder,
porque sabe que é sua única chance de ser convidado para o banquete.

São Jorge então encurtou sua lança, trazendo a sua empunhadura bem
debaixo do braço; e, em vez de galopar como antes, foi trotando em
direção dragão, que se agachou com a aproximação dele, sacudindo sua
cauda até fazê-la estalar no ar como um chicote. O Santo foi virando
ao chegar perto dele e circundou-o cautelosamente, mantendo os olhos
fixos no lugar marcado, enquanto o dragão, adotando tática similar,
moveu-se com cuidado no mesmo círculo, ocasionalmente fintando com a
cabeça. Os dois ficaram assim medindo o adversário, enquanto os
espectadores acompanhavam com a respiração em suspenso.

Embora o round tenha durado alguns minutos, o fim foi tão veloz que
tudo que o menino viu foi um movimento rápido como um relâmpago do
braço do Santo, e então um rodamoinho e uma confusão de espinhas,
garras, cauda, e tufos de grama arrancados. A poeira se assentou, os
espectadores correram para lá dando vivas e aplaudindo, e o menino
conseguiu ver que o dragão estava caído, preso ao solo pela lança.
São Jorge apeara e estava com um pé sobre ele.

Tudo parecia tão genuíno que o menino correu esbaforido para lá,
esperando que o bom e velho dragão não estivesse ferido de verdade.
Quando ele se aproximava, o dragão ergueu uma de suas enormes
pálpebras, piscou solenemente, e fechou-a de novo. Ele estava
firmemente preso à terra pela lança, mas o Santo a acertara no lugar
combinado, onde havia sobra, e não parecia nem fazer cócegas.

-- Não vai cortar a cabeça, Santo? -- perguntou um fulano da multidão
que aplaudia. Tinha apostado no dragão, e naturalmente estava um
pouco ressentido.

-- Bom, acho que hoje não -- respondeu São Jorge, todo simpático. --
Veja, isso pode ser feito qualquer dia. Não há a menor pressa. Acho
que é melhor voltarmos à aldeia antes, para os comes e bebes, e então
eu vou ter uma conversa séria com ele, e vocês verão que ele vai se
tornar um dragão muito diferente.

Com as palavras mágicas comes e bebes, a multidão formou uma procissão
e aguardou silenciosamente o sinal de partida. A hora de falar,
aplaudir e apostar já tinha passado, chegava a hora de agir. São
Jorge, puxando a lança com as duas mãos, soltou o dragão, que se
levantou e se sacudiu e passou os olhos por suas escamas, espinhos e
o demais, para ver se estava tudo em ordem. Então o Santo montou em
seu cavalo e tomou a dianteira da procissão, com o dragão seguindo
humildemente ao lado do menino, e com os sedentos espectadores
mantendo um respeitoso intervalo atrás dele.

Houve momentosos eventos quando todos chegaram à aldeia e se
organizaram em frente à estalagem. Depois dos comes e bebes, São
Jorge fez um discurso, no qual informou à plateia que havia removido
o terrível flagelo, com muito trabalho e sacrifício de sua parte, e
que agora não era mais para eles ficarem resmungando e imaginando que
tinham infortúnios, porque não tinham. E que eles não deviam gostar
tanto assim de lutas, porque da próxima vez podiam bem ter que
lutarem eles mesmos, que de jeito nenhum ia ser a mesma coisa. E que
havia um certo texugo no estábulo da estalagem que deveria ser
imediatamente libertado, e que ele pessoalmente ia garantir que isso
fosse feito. Então ele disse que o dragão andara refletindo sobre as
coisas, e vira que há dois lados em toda questão, e não ia mais fazer
isso, e se eles fossem bons talvez ele ficasse morando ali. De modo
que eles deviam ficar amigos, não serem preconceituosos, nem ficarem
por aí achando que sabem tudo o que há para saber, porque não sabem,
nem de longe. E ele advertiu-os quanto ao pecado de fantasiar e
inventar histórias e fazer os outros acreditarem só por serem
plausíveis e cheias de detalhes. Então ele sentou-se de novo, em meio
a muitos vivas arrependidos, e o dragão cutucou o menino e cochichou
que nem ele mesmo teria dito melhor. Então todo mundo foi embora se
arrumar para o banquete.

Banquetes são sempre agradáveis, consistindo em sua maior parte em
comer e beber; mas a melhor coisa de um banquete é que ele acontece
no fim de alguma coisa, e não é preciso mais se preocupar, e o dia
seguinte parece ficar muito longe. São Jorge estava feliz porque
tinha havido uma luta e ele não precisara matar ninguém; porque ele
realmente não gostava de matar, embora em geral tivesse que fazê-lo.
O dragão estava feliz porque tinha havido uma luta, e além de não ter
se machucado nem um pouco nela, ganhara popularidade e uma posição
garantida na sociedade. O menino estava feliz porque tinha havido uma
luta e apesar disso seus dois amigos estavam no maior dos
entendimentos. E todos os outros estavam felizes porque tinha havido
uma luta e… bem, eles não precsiavam de nenhuma outra razão para
estarem felizes. O dragão esforçou-se para dizer a coisa certa para
todo mundo, e provou-se a alma da festa; enquanto o Santo e o menino
olhavam, com a impressão de estarem assistindo a uma festa em que a
honra e a glória eram inteiramente do dragão. Mas eles não se
incomodaram, sendo boa gente, e o dragão não tinha ficado orgulhoso
ou ingrato. Ao contrário, a cada dez minutos ele se inclinava na
direção do menino e dizia, comovido:

-- Olhe, você vai me levar para casa, não vai? -- E o menino sempre
fazia que sim, embora tivesse prometido a sua mãe não voltar muito
tarde.

Por fim o banquete terminou, os convidados foram embora com muitos
boas-noites, congratulações e convites, e o dragão, que tinha ficado
para se despedir até o último deles, emergiu na rua seguido pelo
menino, enxugou a testa, suspirou, sentou na calçada e olhou para as
estrelas.

-- Foi uma noite excelente! -- murmurou. -- Excelentes estrelas!
Excelente lugarzinho! Acho que vou ficar aqui mesmo. Não estou com a
menor vontade de subir aquele maldito morro. O menino prometeu me
levar para casa. Melhor o menino fazer isso! Responsabilidade não é
minha. Responsabilidade é do menino!

E seu queixo afundou no peito largo e ele pegou tranquilamente no
sono.

-- Ah, levanta, dragão -- gritou o menino, desconsolado. -- Vocês sabe
que minha mãe está me esperando, e eu estou tão cansado. Você me fez
prometer que iria levá-lo para casa, mas eu não sabia o que você
queria dizer, senão não teria prometido! 

E o menino sentou na calçada ao lado do dragão dormindo e começou a
chorar.

A porta atrás deles se abriu, um recorte de luz iluminou a rua, e São
Jorge, que havia saído para dar uma caminhada no ar frio da noite,
viu as duas figuras sentadas ali, o enorme e imóvel dragão e o
menininho chorando.

-- Qual é o problema, menino? -- ele perguntou gentilmente, do lado
dele.

-- Ah, é esse grandessíssimo desse porco dorminhoco desse dragão! --
soluçou o menino. -- Primeiro me fez prometer levá-lo para casa, aí
disse que era melhor eu fazer isso mesmo, e então se pôs a dormir!
Mais fácil levar um monte de feno para casa! E estou tão cansado, e
minha mãe…

E ele começou a chorar de novo.

-- Não fique assim -- disse São Jorge. -- Eu vou ajudá-lo, e nós dois
vamos levá-lo para casa. Acorda, dragão! -- ele disse incisivamente,
chacoalhando o dragão pelo ombro.

O dragão abriu os olhos todo sonolento.

-- Que noite, Jorge! -- murmurou -- Que…

-- Escute aqui, dragão -- disse o Santo, firmemente. -- Aqui está esse
camaradinha esperando para levá-lo para casa, e você sabe muito bem
que ele já devia estar na cama há duas horas atrás, e o que a mãe
dele vai dizer eu nem imagino, e qualquer um menos porco e egoísta
teria feito ele ir para cama faz tempo…

-- E ele vai para a cama! -- exclamou o dragão, levantando-se. --
Pobrezinho, imagine só, ainda acordado a essa hora! É uma vergonha, é
isso o que é, e eu não acho, São Jorge, que você teve muita
consideração… Mas vamos embora imediatamente, e chega de discussões
ou papo furado. Dê-me sua mão, menino; e obrigado, Jorge, um braço
amigo morro acima é só o que eu queria!

E lá se foram os três abraçados morro acima, o Santo, o dragão e o
menino. As luzes na aldeia começaram a se apagar; mas havia as
estrelas, e uma lua no fim da noite, enquanto eles subiam os Downs
juntos. E, quando eles viraram a última esquina e desapareceram de
vista, pedaços de uma velha canção chegaram trazidos pela brisa da
noite. Não sei com certeza quem estava cantando, mas eu acho que era
o dragão!

\chapter{O último dragão\subtitulo{Edith Nesbit}}

Claro que você sabe que houve uma época em que os dragões eram tão
comuns quanto os ônibus são hoje, e quase tão perigosos. Mas como se
esperava de todo príncipe que tivesse tido uma boa educação, que
matasse um dragão e salvasse uma princesa, começou a haver cada vez
menos dragões, até ficar frequentemente bem difícil para uma princesa
achar um dragão do qual ser salva. E por fim não havia mais dragões
na França, nem na Alemanha, na Espanha, na Itália ou na Rússia.
Alguns sobraram na China, e ainda estão lá, mas são frios e de
bronze, e nunca houve nenhum, claro, na América. Mas o último dragão
de verdade vivo que sobrou estava na Inglaterra, e claro que isso foi
há muito tempo atrás, antes do que se chama História da Inglaterra
ter começado. Esse dragão vivia na Cornualha, em cavernas enormes no
meio dos rochedos, e era um excelente e enorme dragão, com nada menos
que 21 metros da ponta de seu temível focinho ao fim de sua terrível
cauda. Exalava fogo e fumaça, e fazia barulho ao andar, pois suas
escamas eram feitas de ferro. Suas asas eram iguais a meios
guarda-chuvas -- ou como asas de morcego, só que milhares de vezes
maiores. Todo mundo tinha muito medo dele, e com razão.

Acontece que o rei da Cornualha tinha uma filha, e claro que quando
ela fizesse 16 anos teria que ir enfrentar o dragão. Essas histórias
sempre se contam ao anoitecer nos quartos das crianças das famílias
reais, de modo que a Princesa sabia o que lhe esperava. O dragão não
iria comê-la, claro -- porque o príncipe ia chegar para salvá-la. Mas
ela não conseguia deixar de pensar que seria muito mais agradável não
se relacionar com o dragão, de jeito nenhum -- nem mesmo ser salva
dele.

-- Todos os príncipes que eu conheço são uns menininhos completamente
bobos -- ela disse a seu pai. -- Por que eu tenho de ser salva por um
príncipe?

-- É como sempre se faz, minha querida -- disse o Rei, tirando a coroa e
pondo-a na grama, pois estavam sozinhos no jardim, e mesmo reis
precisam relaxar às vezes.

-- Papai querido, -- disse a Princesa, quando terminou de fazer uma
coroa de margaridas e a pôs na cabeça do Rei, onde devia estar a
outra. -- Papai querido, não podíamos amarrar um dos principezinhos
bobos para o dragão ir comer, e então seria eu a matar o dragão e
salvar o Príncipe? Eu esgrimo muito melhor que qualquer um dos
príncipes que conhecemos. 

-- Que ideia inapropriada para uma dama! -- disse o Rei, e pôs de volta
a coroa, ao ver o primeiro ministro vindo com uma cesta de decretos
novinhos para ele assinar. -- Tire essa ideia da cabeça, minha filha.
Eu salvei sua mãe de um dragão, e você não pretende se achar melhor
que ela, espero.

-- Mas esse é o último dragão. É diferente de todos os outros.

-- Como assim? -- perguntou o Rei.

-- Porque ele é o último -- disse a Princesa, e foi para sua aula de
esgrima, a qual ela levava muito a sério. Ela levava a sério todas as
aulas dela; porque não conseguia desistir da ideia de lutar com o
dragão. Ela as levou tão a sério que se tornou a princesa mais forte,
corajosa, hábil e sensata da Europa. A mais bonita e a mais simpática
ela sempre fora. 

E os dias e os anos se passaram, até que enfim chegou a véspera do dia
em que a Princesa deveria ser salva do dragão. O príncipe encarregado
dessa valorosa façanha era um príncipe pálido, de olhos grandes e com
a cabeça cheia de matemática e filosofia, mas que infelizmente
negligenciara suas aulas de esgrima. Ele ia passar a noite no
palácio, e houve um banquete.

Depois do jantar a Princesa mandou seu papagaio de estimação para o
príncipe com um bilhete. Dizia: “Por favor, Príncipe, venha até o
terraço. Quero falar com você sem ninguém ouvindo. -- A Princesa.”

E lá foi ele, claro, e viu de longe o vestido prateado dela brilhando
entre as sombras das árvores, como água sob a luz das estrelas. E
quando chegou bem perto, disse:

-- Princesa, a seu serviço -- e dobrou seu joelho coberto de
tecido-bordado-em-ouro e pôs a mão em seu coração coberto de
tecido-bordado-em-ouro.

-- Você acha -- disse a Princesa sinceramente -- que você vai ser capaz
de matar o dragão?

-- Matarei o dragão -- disse o Príncipe firmemente -- ou sucumbirei
tentando. 

-- Você sucumbir não vai servir para nada -- disse a Princesa. 

-- É o mínimo que posso fazer -- disse o Príncipe.

-- O meu medo é que acabe sendo o máximo que você vai poder fazer --
disse a Princesa.

-- É a única coisa que posso fazer -- disse ele --, a menos que eu mate o
dragão.

-- Por que você deveria fazer alguma coisa por mim é o que eu não
entendo -- ela disse.

-- Mas eu quero -- ele disse. -- Você precisa saber que eu a amo mais do
que qualquer outra coisa no mundo. 

Ao dizer isso, ele pareceu tão gentil que a Princesa começou a gostar
um pouquinho dele.

-- Escute aqui -- ela disse --, ninguém mais vai sair amanhã. Você sabe
que eles me amarram numa rocha, me deixam lá; e então todo mundo sai
em disparada para casa e fecha as venezianas e as deixa fechadas até
você entrar na cidade triunfante, gritando que matou o dragão, e eu
vou atrás de você no cavalo chorando de alegria.

-- Ouvi dizer que é assim que se faz -- ele disse.

-- Bom, você me ama o suficiente para vir muito rápido e me soltar, e
então lutamos juntos contra o dragão?

-- Não seria seguro para você.

-- É muito mais seguro para nós dois se eu estiver solta com uma espada
na mão do que amarrada e indefesa. Concorda, vai.

Ele não podia negar nada a ela. Então concordou. E no dia seguinte
tudo aconteceu como ela dissera. 

Quando ele cortou as cordas que a amarravam à rocha, os dois ficaram
na montanha deserta olhando um para o outro.

-- Tenho a impressão -- o Príncipe disse -- que essa cerimônia podia ter
sido arranjada sem o dragão.

-- Sim, -- disse a Princesa -- mas como ela foi arranjada com o dragão…

-- É uma pena matar o dragão, o último do mundo -- disse o Príncipe.

-- Bom, então, não vamos -- disse a Princesa. -- Vamos ensiná-lo a comer
na nossa mão, em vez de comer princesas. Dizem que tudo pode ser
amansado através da bondade.

-- Amansá-lo assim implica dar-lhe algo para comer -- disse o Príncipe.
-- Você tem alguma coisa para comer?

Ela não tinha, mas o príncipe admitiu que tinha alguns biscoitos. 

-- O café da manhã foi tão cedo -- ele disse -- que eu achei que você ia
ficar com fome depois da luta.

-- Bem pensado -- disse a princesa, e eles pegaram um biscoito com cada
mão. E olharam aqui e olharam acolá, mas nada de dragão à vista.

-- Mas eis o rastro dele -- disse o Príncipe, apontando para onde a
rocha estava marcada e arranhada de modo a fazer uma pista até a
entrada de uma caverna escura. Era como as marcas de uma carroça numa
estrada de Sussex, misturadas com as pegadas de uma gaivota na areia
da praia. -- Veja, aqui é ele arrastando sua cauda de metal e essas
são as marcas de suas garras de aço.

-- Melhor não pensar quão duras são a cauda e as garras dele -- disse a
Princesa --, senão vou começar a ficar com medo, e você sabe que não
dá para amansar nada, não importa a sua bondade, se você estiver com
medo. Vamos. É agora ou nunca.

Ela pegou o Príncipe pela mão e os dois correram pela pista que levava
à entrada escura da caverna. Mas não correram até dentro dela. Era de
fato escura demais.

Então eles ficaram do lado de fora, e o Príncipe gritou: 

-- Ó de casa! Ei, dragão! Tem alguém em casa?

E da caverna eles ouviram uma voz respondendo e um monte de estrondos
e estalos. Soava como se um cotonifício estivesse se espreguiçando e
se levantando de seu sono.

O Príncipe e a Princesa tremeram, mas fiacaram ali firmes.

-- Dragão, dragão! -- disse a Princesa -- Por favor saia e fale com a
gente. Trouxemos um presente para você.

-- Ah, sei; conheço os presentes de vocês -- grunhiu o dragão com uma
voz trovejante. -- Uma dessas preciosas princesas, eu suponho. E eu
tenho que sair e lutar por ela. Bom, vou logo lhes dizendo, eu não
pretendo fazer isso. A um combate justo eu não diria não, uma luta
justa e sem favorecimentos; mas essas lutas arranjadas que você tem
de perder, não. É o que eu lhes digo. Se eu quisesse uma princesa, eu
iria atrás de uma, quando me desse vontade; mas eu não quero. O que
vocês acham que eu iria fazer com uma princesa, se eu pegasse uma?

-- Comê-la… não? -- a Princesa disse com um ligeiro tremor na voz. 

-- Eca! Comer uma gororoba dessas? -- disse o dragão rudemente. -- Eu nem
tocaria nesta horrorosa coisa. 

A voz da Princesa ficou mais firme.

-- Você gosta de biscoitos? -- ela perguntou.

-- Não -- grunhiu o dragão.

-- Nem mesmo aqueles caros com cobertura de chocolate?

-- Não -- grunhiu o dragão.

-- Mas então do que você gosta? -- perguntou o Príncipe. 

-- Vão embora e não me encham mais -- grunhiu o dragão, e eles puderam
ouvir ele voltando para dentro, e os rangidos e estalos de seus
movimentos ecoaram pela caverna como o barulho dos martelos a vapor
[?] do arsenal de Woolwich.

O Príncipe e a Princesa olharam um para o outro. O que eles iam fazer?
Claro que não adiantaria voltar para casa e dizer ao Rei que o dragão
não queria princesas -- porque Sua Majestade era muito antiquada e
jamais acreditaria  que um dragão moderno pudesse de alguma forma ser
diferente de um dragão à moda antiga. Eles não podiam entrar na
caverna e matar o dragão. De fato, a menos que ele atacasse a
Princesa, não ia ser nada justo matá-lo.

-- De alguma coisa ele deve gostar -- a Princesa sussurrou, e chamou-o
de novo com uma voz tão doce quanto mel e cana-de-açúcar: -- Dragão!
Dragãzinho querido!

-- O quê? -- berrou o dragão. -- Repita isso!

E eles oviram o dragão vindo na direção deles no escuro da caverna. A
princesa teve um arrepio, e disse num fiapo de voz:

-- Dragão! Dragãozinho querido!

E então o dragão saiu. O Príncipe puxou sua espada e a Princesa a dela
-- a bonita espada com cabo de prata que o Príncipe trouxera em seu
automóvel. Mas eles não atacaram; recuaram lentamente enquanto o
dragão saía, com todo seu imenso e escamoso tamanho, e se estendia
sobre a rocha, suas enormes asas meio abertas e os reflexos prateados
dele resplandescendo feito diamantes no sol. Por fim não tinham mais
para onde recuar -- o rochedo escuro atrás dele barrava a passagem -- e
com as costas contra a pedra eles ficaram com as espadas em punho,
esperando.

O dragão foi chegando cada vez mais perto, e agora eles podiam ver que
ele não estava exalando fogo e fumaça como esperavam; veio rastejando
lentamente na direção deles, meneando um pouco a cabeça feito um
filhote de cachorro que quer brincar mas não tem muita certeza se
você está bravo ou não com ele.

E então eles viram que grandes lágrimas escorriam nas bochechas
metálicas dele.

-- Qual é o problema? -- disse o Príncipe

-- Ninguém -- o dragão soluçou -- jamais me chamara de “querido” antes.

-- Não chore, querido dragão -- disse a Princesa. -- Vamos chamá-lo de
“querido” o quanto você quiser. Queremos amansá-lo.

-- Eu sou manso -- disse o dragão. -- Essa é a questão. É o que ninguém a
não ser vocês jamais descobriu. Sou tão manso que comeria de suas
mãos.

-- Comeria o que, dragão querido? -- disse a Princesa. -- Biscoitos,
talvez?

O dragão balançou devagar sua enorme cabeça.

-- Nada de biscoitos -- disse a Princesa carinhosamente. -- O que então,
querido dragão?

-- Sua gentileza me desdragona um bocado -- ele disse. -- Ninguém jamais
perguntou a nenhum de nós o que gostaríamos de comer. Sempre nos
oferecendo princesas, e aí as salvando, e nem uma vez “o que você
gostaria de beber para brindar à saúde do Rei?” Cruel e injusto, é o
que acho -- e começou a chorar de novo.

-- Mas o que você gostaria de beber para brindar à nossa saúde? -- disse
o Príncipe. -- Nós vamos nos casar hoje, não vamos, Princesa?

Ela disse que supunha que sim.

-- O que eu gostaria de beber à sua saúde? -- perguntou o dragão. -- Ah,
o senhor é decididamente um cavalheiro, é sim. Faço questão de
dizê-lo, sim senhor. Terei orgulho de beber à sua saúde e à da sua
bondosa dama só um pequeno golinho de -- a voz dele falhou. 

-- …e pensar que o senhor me pergunta assim todo amistoso -- continuou
ele. -- Sim, senhor, só um pequeno golinho de ga-ga-ga-ga-ga-gasolina,
isso que… faria bem a um dragão, senhor…

-- Tenho de monte no carro -- disse o Príncipe, e saiu descendo a
montanha feito um relâmpago. Era bom em julgar caráter, e sabia que
com aquele dragão a Princesa estaria segura.

-- Se me permite a ousadia -- disse o dragão -- enquanto o cavalheiro não
volta… talvez só para passar o tempo você poderia ter a gentileza de
me chamar de “querido” de novo, e se consentir em trocar um aperto de
garras com um pobre dragão velho que nunca foi inimigo de ninguém a
não ser dele mesmo… Seria o que faria do último dos dragões também o
mais orgulhoso dos dragões, desde o primeiro deles.

Ele estendeu uma enorme pata, e os grandes ganchos de aço que eram
suas garras se fecharam em volta da mão da Princesa tão suavemente
quanto as garras do urso do Himalaia em volta do pedaço de bolo que
você dá a ele através das grades do zoológico.

E então o Príncipe e a Princesa voltaram triunfantes ao palácio, com o
dragão seguindo-os como um cachorro de estimação. E durante todas as
festividades do casamento ninguém brindou à felicidade dos noivos com
mais sinceridade que o dragão de estimação da Princesa, o qual ela
logo batizara de Fido.

E quando o feliz casal estava instalado em seu próprio reino, Fido foi
até eles implorar que lhe permitissem ser útil.

-- Deve ter alguma coisinha que eu possa fazer -- ele disse, retinindo
suas asas e esticando suas garras. -- Minhas asas, garras e etc.
poderiam ter algum uso; isso sem nem falar em meu coração eternamente
agradecido.

De modo que o Príncipe providenciou que lhe fizessem uma sela ou um
assento especial para suas costas: enormemente comprido, era feito de
várias carrocerias de bonde soldadas juntas. 150 poltronas foram ali
instaladas, e o dragão, cujo maior prazer agora era dar prazer aos
outros, se deliciava em levar grupos de crianças para a praia. Voava
pelo ar com bastante facilidade carregando seus 150 passageirozinhos,
e ficava deitado na areia esperando pacientemente eles se disporem a
voltar. As crianças gostavam muito dele e costumavam-no chamá-lo de
“dragão querido”, uma palavra que nunca deixava de produzir lágrimas
de afeição e gratidão em seus olhos. E assim ele viveu, útil e
respeitado, até o outro dia mesmo -- quando aconteceu de alguém dizer,
ao alcance de seus ouvidos, que dragões estavam fora de moda, agora
que haviam aparecido tantas máquinas novas. Isso o incomodou tanto
que ele foi pedir ao Rei que o transformasse em alguma coisa menos
fora de moda, e o bondoso monarca no mesmo instante o transformou
numa invenção mecânica. O dragão, de fato, tornou-se o primeiro
avião.

